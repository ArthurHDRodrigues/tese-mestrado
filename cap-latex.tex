\chapter{Usando o \LaTeX{} e este modelo}

Não é necessário que o texto seja redigido usando \LaTeX{}, mas é fortemente
recomendado o uso dessa ferramenta, pois ela facilita diversas etapas do
trabalho e o resultado final é muito bom. Este modelo inclui vários
comentários explicativos e pacotes interessantes para auxiliá-lo com ele.

O modelo é composto por estes arquivos:

\begin{itemize}
  \item Arquivo principal:
  \begin{itemize}
	  \item \texttt{tese-exemplo.tex} (leia os comentários neste arquivo!)
  \end{itemize}

  \item Arquivos dos capítulos e apêndice:
  \begin{itemize}
    \item \texttt{cap-introducao.tex}
    \item \texttt{cap-latex.tex}
    \item \texttt{cap-exemplos.tex}
    \item \texttt{cap-conclusoes.tex}
    \item \texttt{ape-conjuntos.tex}
  \end{itemize}

  \item Arquivo de bibliografia:
  \begin{itemize}
    \item \texttt{bibliografia.bib}
  \end{itemize}

  \item Diretório de figuras:
  \begin{itemize}
    \item \texttt{./figuras/}
  \end{itemize}

  \item Arquivo com formato sugerido de capa, resumo e outros elementos:
  \begin{itemize}
    \item \texttt{imeusp.sty}
  \end{itemize}

  \item Arquivos com formatos sugeridos de referências e citações:
  \begin{itemize}
    \item \texttt{plainnat-ime.bbx} (estilo plainnat com biblatex -- bibliografia)\index{biblatex}
    \item \texttt{plainnat-ime.cbx} (estilo plainnat com biblatex -- citações)
    \item \texttt{plainnat-ime.bst} (estilo plainnat com bibtex)
    \item \texttt{alpha-ime.bst} (estilo alpha com bibtex)
    \item \texttt{natbib.sty} (tradução do pacote padrão natbib)
  \end{itemize}
\end{itemize}

Para compilar o documento, basta executar o comando ``make''. O comando ``make
clean'' apaga todos os arquivos gerados durante a compilação.

\section{Instalação do \LaTeX{}}

Embora uma instalação completa do \TeX{}/\LaTeX{} seja relativamente grande (perto
de 5GB), em geral vale a pena instalar a maior parte dos pacotes incluídos
na sua distribuição. No caso do debian ou ubuntu, por exemplo, o pacote
``texlive-full'' instala praticamente tudo do \LaTeX{} que pode interessar (incluindo
suporte a línguas como árabe, japonês etc.). Se você preferir não usá-lo,
observe que este template sugere alguns pacotes que normalmente não estão
incluídos em uma instalação ``econômica'' do \LaTeX{}; por exemplo, no debian:

\begin{description}
  \item[inconsolata] -- está incluído em ``texlive-fonts-extra''
  \item [siunitx] -- está incluído em ``texlive-science''
  \item [biblatex] -- está incluído em ``texlive-bibtex-extra''
  \item [biber] -- é um pacote separado
  \item [xindy] -- é um pacote separado
\end{description}

\section{Bibliografia}

Você pode usar referências bibliográficas nos formatos ``alpha'' ou ``plainnat''.
Se estiver usando natbib+bibtex, use os arquivos .bst ``alpha-ime.bst'' ou
``plainnat-ime.bst'', que são versões desses dois formatos traduzidas para o
português. Se estiver usando biblatex\index{biblatex} (recomendado), escolha o estilo
``alphabetic'' (que é um dos estilos padrão do biblatex) ou ``plainnat-ime''.
O arquivo de exemplo inclui todas essas opções; basta des-comentar as linhas
correspondentes e, se necessário, modificar o arquivo Makefile para chamar
o bibtex ao invés do biber (este último é usado em conjunto com o biblatex).

\section{Editores}

Você pode usar qualquer editor de textos para trabalhar com o \LaTeX{}, mas
talvez prefira usar algum editor específico. Essas ferramentas podem compilar
o documento automaticamente quando há mudanças, exibir a versão compilada
do documento ao lado do texto sendo editado etc. Os programas mais comumente
usados são o TeXmaker, TeXstudio e TeXworks; os três são software livre e
funcionam em Windows, MacOS e Linux. TeXnicCenter é outra opção livre, mas
funciona apenas em Windows. Você também pode querer usar editores online,
como o overleaf (\url{www.overleaf.com}) e o sharelatex (\url{www.sharelatex.com}).

\section{Perguntas Frequentes sobre o modelo}

\begin{itemize}

\item \textbf{Posso usar pacotes \LaTeX{} adicionais aos sugeridos, como por exemplo: pstricks, pst-all, etc?}\\
Com certeza! Você pode modificar o arquivo o quanto desejar. O modelo \LaTeX{} serve só como uma ajuda inicial para o seu trabalho.

\item \textbf{As figuras podem ser colocadas no meio do texto ou devem ficar no final dos capítulos?}\\
Em geral as figuras devem ser apresentadas assim que forem referenciadas. Colocá-las no final dos capítulos dificultaria um pouco a leitura, mas isso depende do estilo do autor, orientador, ou lugar de publicação. Converse com seu orientador!

\item \textbf{Existe algo específico para citações de páginas web?}\\
Biblatex define o tipo ``online''; bibtex, por padrão, não tem um tipo específico. Se o que você está citando não é um texto específico mas sim um sítio, como por exemplo o sítio de uma empresa ou de um produto, pode ser mais adequado colocar a referência como nota de rodapé e não na lista de referências; nesses casos, algumas pessoas acrescentam uma segunda lista de referências especificamente para recursos online (biblatex \index{biblatex} permite criar múltiplas bibliografias). Se, no entanto, trata-se de um texto específico, como uma postagem em um blog, uma matéria jornalística ou mesmo uma mensagem de email para uma lista de discussão, a citação deve seguir o formato de outros tipos de documento e informar título, autor etc. Normalmente usa-se o campo ``howpublished'' para especificar que se trata de um recurso online. Observe também que artigos que fazem parte de uma publicação, como os anais de um congresso, e que estão disponíveis online devem ser citados por seu tipo verdadeiro e apenas incluir o campo ``url'', aceito por todos os tipos de documento do bibtex/biblatex.

\item \textbf{A bibliografia está sendo impressa em inglês (usa ``and'' ao invés de ``e'' para separar os nomes dos autores)}\\
Você deve estar usando um estilo de bibliografia bibtex diferente dos sugeridos. Uma simples solução é copiar o arquivo de estilo correspondente da sua instalação \LaTeX{} para o diretório onde seus arquivos estão e mudar ``and'' por ``e'' (ou ``\&'' se preferir) na função format.names. Biblatex tem pleno suporte a diferentes línguas e é possível personalizar as traduções (há um exemplo no modelo).

\item \textbf{Aparece uma folha em branco entre os capítulos}\\
Essa característica foi colocada propositalmente, dado que todo capítulo deve (ou deveria) começar em uma página de numeração ímpar (lado direito do documento). Acrescente ``openany'' como opção da classe, i.e., \texttt{\textbackslash{}documentclass[openany,11pt,twoside,a4paper]\{book\}}.

\item \textbf{É possível resumir o nome das seções/capítulos que aparece no topo das páginas?}\\
	Sim, usando a sintaxe \texttt{\textbackslash{}section[mini-titulo]\{titulo enorme\}}. Isso é especialmente útil nos \textit{captions} das figuras e tabelas, que muitas vezes são demasiadamente longos para a lista de figuras/tabelas.

\item \textbf{Existe algum programa para gerenciar referências em formato bibtex?}\\
Sim, há vários. Uma opção bem comum é o JabRef; outra é usar Zotero ou Mendeley e exportar os dados deles no formato .bib.

\item \textbf{Como faço para usar o MakeFile (comando make) no Windows?}\\
Se você instalou o \LaTeX{} usando o Cygwin, você já deve ter o comando make instalado. Se você pretende usar algum dos editores sugeridos, é possível deixar a compilação a cargo deles, dispensando o uso do make.
\end{itemize}
