%%%%%%%%%%%%%%%%%%%%%%%%%%%%%%%%%%%%%%%%%%%%%%%%%%%%%%%%%%%%%%%%%%%%%%%%%%%%%%%%
%%%%%%%%%%%%%%%%%%%%%%%%%%%%%%%%%%% LÍNGUAS %%%%%%%%%%%%%%%%%%%%%%%%%%%%%%%%%%%%
%%%%%%%%%%%%%%%%%%%%%%%%%%%%%%%%%%%%%%%%%%%%%%%%%%%%%%%%%%%%%%%%%%%%%%%%%%%%%%%%

\makeatletter
\ExplSyntaxOn

% We need to have at least some variant of Portuguese and of English
% loaded to generate the abstract/resumo, palavras-chave/keywords etc.
% We will make sure that both languages are present in the class options
% list by adding them if needed. With this, these options become global
% and therefore are seen by all packages (among them, babel).
%
% babel traditionally uses "portuguese", "brazilian", "portuges", or
% "brazil" to support the Portuguese language, using .ldf files. babel
% is also in the process of implementing a new scheme, using .ini
% files, based on the concept of "locales" instead of "languages". This
% mechanism uses the names "portuguese-portugal", "portuguese-brazil",
% "portuguese-pt", "portuguese-br", "portuguese", "brazilian", "pt",
% "pt-PT", and "pt-BR" (i.e., neither "portuges" nor "brazil"). To avoid
% compatibility problems, let's stick with "brazilian" or "portuguese"
% by substituting portuges and brazil if necessary.

\NewDocumentCommand\@IMEportugueseAndEnglish{m}{

  % Make sure any instances of "portuges" and "brazil" are replaced
  % by "portuguese" e "brazilian"; other options are unchanged.
  \seq_gclear_new:N \l_tmpa_seq
  \seq_gclear_new:N \l_tmpb_seq
  \seq_gset_from_clist:Nc \l_tmpa_seq {#1}

  \seq_map_inline:Nn \l_tmpa_seq{
    \def\@tempa{##1}
    \ifstrequal{portuges}{##1}
      {
        \GenericInfo{sbc2019}{}{Substituting~language~portuges~->~portuguese}
        \def\@tempa{portuguese}
      }
      {}
    \ifstrequal{brazil}{##1}
      {
        \GenericInfo{}{Substituting~language~brazil~->~brazilian}
        \def\@tempa{brazilian}
      }
      {}
    \seq_gput_right:NV \l_tmpb_seq {\@tempa}
  }

  % Remove the leftmost duplicates (default is to remove the rightmost ones).
  % Necessary in case the user did "portuges,portuguese", "brazil,brazilian"
  % or some variation: When we substitute the language, we end up with the
  % exact same language twice, which may mess up the main language selection.
  \seq_greverse:N \l_tmpb_seq
  \seq_gremove_duplicates:N \l_tmpb_seq
  \seq_greverse:N \l_tmpb_seq

  % If the user failed to select some variation of English and Portuguese,
  % we add them here. We also remember which ones of portuguese/brazilian,
  % english/american/british etc. were selected.
  \exp_args:Nnx \regex_extract_all:nnNTF
    {\b(portuguese|brazilian)\b}
    {\seq_use:Nn \l_tmpb_seq {,}}
    \l_tmpa_tl
    {
      \tl_reverse:N \l_tmpa_tl
      \xdef\@IMEpt{\tl_head:N \l_tmpa_tl}
    }
    {
      \seq_gput_left:Nn \l_tmpb_seq {brazilian}
      \gdef\@IMEpt{brazilian}
    }

  \exp_args:Nnx \regex_extract_all:nnNTF
    {\b(english|american|USenglish|canadian|british|UKenglish|australian|newzealand)\b}
    {\seq_use:Nn \l_tmpb_seq {,}}
    \l_tmpa_tl
    {
      \tl_reverse:N \l_tmpa_tl
      \xdef\@IMEen{\tl_head:N \l_tmpa_tl}
    }
    {
      \seq_gput_left:Nn \l_tmpb_seq {english}
      \gdef\@IMEen{english}
    }

  \exp_args:Nc \xdef {#1} {\seq_use:Nn \l_tmpb_seq {,}}
}


% https://tex.stackexchange.com/a/43541
% This message is part of a larger thread that discusses some
% limitations of this method, but it is enough for us here.
\def\@getcl@ss#1.cls#2\relax{\def\@currentclass{#1}}
\def\@getclass{\expandafter\@getcl@ss\@filelist\relax}
\@getclass

% The three class option lists we need to update: \@unusedoptionlist,
% \@classoptionslist and one of \opt@book.cls, \opt@article.cls etc.
% according to the current class. Note that beamer.cls (and maybe
% others) does not use \@unusedoptionlist; with it, we incorrectly
% add "english,brazilian" to \@unusedoptionlist, but that does not
% cause problems.
\@IMEportugueseAndEnglish{@unusedoptionlist}
\@IMEportugueseAndEnglish{@classoptionslist}
\@IMEportugueseAndEnglish{opt@\@currentclass .cls}

\ExplSyntaxOff
\makeatother

% Babel permite definir a língua ou línguas usadas no documento e deve
% ser um dos primeiros pacotes a serem carregados. É possível definir
% as línguas como parâmetro aqui, mas já fizemos isso ao carregar a
% classe, no início do documento.
%
% A escolha da língua afeta quatro coisas:
%
% 1. A internacionalização das palavras criadas automaticamente, como
%    "Capítulo/Chapter", "Sumário/Table of Contents" etc. - babel chama
%    essas palavras de "captions";
%
% 2. A hifenização das palavras;
%
% 3. Algumas convenções tipográficas. Por exemplo, em francês é usual
%    acrescentar um espaço antes de caracteres como "?" e "!"; línguas
%    diferentes usam caracteres diferentes para as aspas tipográficas;
%    com algumas línguas asiáticas, pode ser necessário utilizar uma
%    fonte diferente etc.;
%
% 4. Atalhos (shorthands) - algumas línguas definem "atalhos" (shorthands"),
%    ou seja, tratam alguns caracteres como comandos especiais; por exemplo,
%    em francês o espaço que é colocado antes da exclamação funciona porque
%    o caracter "!" é, na verdade, um comando.
%
%%%% MUDANDO A LÍNGUA E HIFENIZAÇÃO %%%%
%
% Cada documento tem uma língua padrão; quando usamos pequenos trechos em
% outra língua, como por exemplo em citações, queremos alterar apenas os
% aspectos 2, 3 e 4; nesse caso, a troca da língua deve ser feita com
% \foreignlanguage{língua}{texto} ou com \begin{otherlanguage*}{língua}.
% Para alterar todos os quatro aspectos, deve-se usar \selectlanguage{língua}
% (que altera a língua padrão a partir desse ponto) ou
% \begin{otherlanguage}{língua}{texto}, que faz a alteração apenas até
% \end{otherlanguage}. Se você quiser apenas desabilitar a hifenização de
% um trecho de texto, pode usar \begin{hyphenrules}{nohyphenation}.
% Finalmente, com \babeltags é possível definir comandos curtos como
% "\textbr" (para "brazilian") que são equivalentes a \foreignlanguage.
%
%%%% PERSONALIZANDO CAPTIONS %%%%
%
% É possível personalizar os captions. Para versões de babel a partir
% de 3.51 (lançada em outubro de 2020), basta fazer
% \setlocalecaption{english}{contents}{Table of Contents}. Com versões
% anteriores, por razões históricas há dois mecanismos para fazer isso,
% então é preciso checar qual deve ser usado para cada língua (veja a
% documentação de babel ou faça um teste). São eles:
%
%   1. \renewcommand\spanishchaptername{Capítulo}
%
%   2. \addto\captionsenglish{\renewcommand\contentsname{Table of Contents}}
%      (este é o mais comum)
%
% Esses métodos valem também para a bibliografia, mas apenas se você
% estiver usando bibtex; com biblatex, que é o padrão neste modelo, é
% melhor usar o comando "\DefineBibliographyStrings" (veja a documentação
% de biblatex).
%
% Quando babel faz uma troca de língua, ele executa \extraslíngua e, se for
% necessário trocar os "captions", \captionslíngua (ou seja, os comandos
% acima modificam \captionslíngua). Então, se você quiser executar algo a
% mais quando uma língua é selecionada, faça \addto\extrasenglish{\blah}.
\usepackage{babel}
\usepackage{iflang}

% Por padrão, LaTeX utiliza maiúsculas no início de cada palavra nestas
% expressões ("Lista de Figuras"); vamos usar maiúsculas apenas na primeira
% palavra.
\addto\captionsbrazilian{%
  \renewcommand\listfigurename{Lista de figuras}%
  \renewcommand\listtablename{Lista de tabelas}%
  \renewcommand\indexname{Índice remissivo}%
}

% Alguns pacotes (espeficicamente, tikz) usam, além de babel, este pacote
% como auxiliar para a tradução de palavras-chave, como os meses do ano.
\usepackage{translator}
