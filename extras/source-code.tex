% Para formatar código-fonte (ex. em Java). listings funciona bem mas
% tem algumas limitações (https://tex.stackexchange.com/a/153915 ).
% Se isso for um problema, a package minted pode oferecer resultados
% melhores; a desvantagem é que ela depende de um programa externo,
% o pygments (escrito em python).
%
% listings também não tem suporte específico a pseudo-código, mas
% incluímos uma configuração para isso que deve ser suficiente.
% Caso contrário, há diversas packages específicas para a criação
% de pseudocódigo:
%
%  * a mais comum é algorithmicx ("\usepackage{algpseudocode}");
%  * algorithm2e é bastante flexível, mas um tanto complexa;
%  * clrscode3e foi usada no livro "Introduction to Algorithms"
%    de Cormen, Leiserson, Rivert e Stein.
%  * pseudocode foi usada no livro "Combinatorial Algorithms"
%    de Kreher e Stinson.

\usepackage{listings}
% Carrega a "linguagem" pseudocode para listings
\appto{\lstaspectfiles}{,lstpseudocode.sty}
\appto{\lstlanguagefiles}{,lstpseudocode.sty}
\dowithsubdir{extras/}{\lstloadaspects{simulatex,fixexcludedelims,pseudocode}}
\dowithsubdir{extras/}{\lstloadlanguages{[base]pseudocode,[english]pseudocode,[brazil]pseudocode}}

% O pacote listings não lida bem com acentos! No caso dos caracteres acentuados
% usados em português é possível contornar o problema com a tabela abaixo.
% From https://en.wikibooks.org/wiki/LaTeX/Source_Code_Listings#Encoding_issue
\lstset{literate=
  {á}{{\'a}}1 {é}{{\'e}}1 {í}{{\'i}}1 {ó}{{\'o}}1 {ú}{{\'u}}1
  {Á}{{\'A}}1 {É}{{\'E}}1 {Í}{{\'I}}1 {Ó}{{\'O}}1 {Ú}{{\'U}}1
  {à}{{\`a}}1 {è}{{\`e}}1 {ì}{{\`i}}1 {ò}{{\`o}}1 {ù}{{\`u}}1
  {À}{{\`A}}1 {È}{{\'E}}1 {Ì}{{\`I}}1 {Ò}{{\`O}}1 {Ù}{{\`U}}1
  {ä}{{\"a}}1 {ë}{{\"e}}1 {ï}{{\"i}}1 {ö}{{\"o}}1 {ü}{{\"u}}1
  {Ä}{{\"A}}1 {Ë}{{\"E}}1 {Ï}{{\"I}}1 {Ö}{{\"O}}1 {Ü}{{\"U}}1
  {â}{{\^a}}1 {ê}{{\^e}}1 {î}{{\^i}}1 {ô}{{\^o}}1 {û}{{\^u}}1
  {Â}{{\^A}}1 {Ê}{{\^E}}1 {Î}{{\^I}}1 {Ô}{{\^O}}1 {Û}{{\^U}}1
  {œ}{{\oe}}1 {Œ}{{\OE}}1 {æ}{{\ae}}1 {Æ}{{\AE}}1 {ß}{{\ss}}1
  {ű}{{\H{u}}}1 {Ű}{{\H{U}}}1 {ő}{{\H{o}}}1 {Ő}{{\H{O}}}1
  {ç}{{\c c}}1 {Ç}{{\c C}}1 {ø}{{\o}}1 {å}{{\r a}}1 {Å}{{\r A}}1
  {€}{{\euro}}1 {£}{{\pounds}}1 {«}{{\guillemotleft}}1
  {»}{{\guillemotright}}1 {ñ}{{\~n}}1 {Ñ}{{\~N}}1 {¿}{{?`}}1
}

% Opções default para o pacote listings
% Ref: http://en.wikibooks.org/wiki/LaTeX/Packages/Listings
\lstset{
  columns=[l]fullflexible,            % do not try to align text with proportional fonts
  basicstyle=\footnotesize\ttfamily,  % the font that is used for the code
  numbers=left,                       % where to put the line-numbers
  numberstyle=\footnotesize\ttfamily, % the font that is used for the line-numbers
  stepnumber=1,                       % the step between two line-numbers. If it's 1 each line will be numbered
  numbersep=20pt,                     % how far the line-numbers are from the code
  commentstyle=\color{Brown}\upshape,
  stringstyle=\color{black},
  identifierstyle=\color{DarkBlue},
  keywordstyle=\color{cyan},
  showspaces=false,                   % show spaces adding particular underscores
  showstringspaces=false,             % underline spaces within strings
  showtabs=false,                     % show tabs within strings adding particular underscores
  %frame=single,                       % adds a frame around the code
  framerule=0.6pt,
  tabsize=2,                          % sets default tabsize to 2 spaces
  captionpos=b,                       % sets the caption-position to bottom
  breaklines=true,                    % sets automatic line breaking
  breakatwhitespace=false,            % sets if automatic breaks should only happen at whitespace
  escapeinside={\%*}{*)},             % if you want to add a comment within your code
  backgroundcolor=\color[rgb]{1.0,1.0,1.0}, % choose the background color.
  rulecolor=\color{darkgray},
  extendedchars=true,
  inputencoding=utf8,
  xleftmargin=30pt,
  xrightmargin=10pt,
  framexleftmargin=25pt,
  framexrightmargin=5pt,
  framesep=5pt,
}

% Um exemplo de estilo personalizado para listings (tabulações maiores)
\lstdefinestyle{wider} {
  tabsize = 4,
  numbersep=15pt,
  xleftmargin=25pt,
  framexleftmargin=20pt,
}

% Outro exemplo de estilo personalizado para listings (sem cores)
\lstdefinestyle{nocolor} {
  commentstyle=\color{darkgray}\upshape,
  stringstyle=\color{black},
  identifierstyle=\color{black},
  keywordstyle=\color{black}\bfseries,
}

% Um exemplo de definição de linguagem para listings (XML)
\lstdefinelanguage{XML}{
  morecomment=[s]{<!--}{-->},
  morecomment=[s]{<!-- }{ -->},
  morecomment=[n]{<!--}{-->},
  morecomment=[n]{<!-- }{ -->},
  morestring=[b]",
  morestring=[s]{>}{<},
  morecomment=[s]{<?}{?>},
  morekeywords={xmlns,version,type}% list your attributes here
}

% Estilo padrão para a "linguagem" pseudocode
\lstdefinestyle{pseudocode}{
  basicstyle=\rmfamily\small,
  commentstyle=\itshape,
  keywordstyle=\bfseries,
  identifierstyle=\itshape,
  % as palavras "function" e "procedure"
  procnamekeystyle=\bfseries\scshape,
  % funções precedidas por function/procedure ou com \func{}
  procnamestyle=\ttfamily,
  specialidentifierstyle=\ttfamily\bfseries,
}
\lstset{defaultdialect=[english]{pseudocode}}

% No arquivo "floats.tex" definimos o tipo de float "program".

% A package listings tem seu próprio mecanismo para a criação de
% captions, lista de programas etc. Neste modelo não usamos esses
% recursos mas, se necessário, lembre-se de traduzir o caption e
% o título da lista de programas usados por ele:
%\addto\extrasbrazil{
%  \gdef\lstlistlistingname{Lista de Programas}
%  \gdef\lstlistingname{Programa}
%}
%\addto\extrasenglish{
%  \gdef\lstlistlistingname{List of Programs}
%  \gdef\lstlistingname{Program}
%}
