%%%%%%%%%%%%%%%%%%%%%%%%%%%%%%%%%%%%%%%%%%%%%%%%%%%%%%%%%%%%%%%%%%%%%%%%%%%%%%%%
%%%%%%%%%%%%%%%%%%%%%%%%%%%%% FIGURAS / FLOATS %%%%%%%%%%%%%%%%%%%%%%%%%%%%%%%%%
%%%%%%%%%%%%%%%%%%%%%%%%%%%%%%%%%%%%%%%%%%%%%%%%%%%%%%%%%%%%%%%%%%%%%%%%%%%%%%%%

% Permite importar figuras. LaTeX "tradicional" só é capaz de trabalhar com
% figuras EPS. Hoje em dia não há nenhuma boa razão para usar essa versão;
% pdfTeX, XeTeX, e LuaTeX podem usar figuras nos formatos PDF, JPG e PNG; EPS
% também pode funcionar em algumas instalações mas não é garantido, então é
% melhor evitar.
\usepackage{graphicx}

% A package float é amplamente utilizada; ela permite definir novos tipos
% de float e também acrescenta a possibilidade de definir "H" como opção de
% posicionamento do float, que significa "aqui, incondicionalmente". No
% entanto, ela tem algumas fragilidades e não é atualizada desde 2001.
% floatrow é uma versão aprimorada e com mais recursos da package "float",
% mas também não é atualizada desde 2009. Aqui utilizamos alguns recursos
% disponibilizados por ambas e é possível escolher qual delas utilizar.
% Um dos principais recursos dessas packages é permitir a criação de novos
% tipos de float; veja o arquivo source-code.tex para um exemplo.
%\usepackage{float}
\usepackage{floatrow}

% Em documentos com duas colunas, floats normalmente são colocados como
% parte de uma das colunas. No entanto, é possível usar "\begin{figure*}"
% ou "\begin{table*}" para criar floats que ocupam as duas colunas. Floats
% "duplos" desse tipo têm algumas limitações:
%
% 1. Mesmo que haja espaço disponível na página atual, eles são sempre
%    inseridos na página seguinte ao lugar em que foram definidos (então
%    é comum defini-los antes do lugar "certo" para compensar isso)
%
% 2. Eles só podem aparecer no topo da página ou em uma página de floats,
%    ou seja, nunca "here" nem no pé da página.
%
% 3. Em alguns casos, eles podem aparecer fora da ordem em relação aos
%    demais floats do mesmo tipo (o que não acontece com floats "normais")
%
% Esta package:
%
% 1. Soluciona parcialmente o primeiro problema: floats "duplos" podem
%    aparecer na página em que são definidos se sua definição está contida
%    no texto da coluna da esquerda;
%
% 2. Soluciona o segundo problema: floats "duplos" podem aparecer tanto no
%    topo quanto no pé da página. Observe que eles *não* podem aparecer
%    "here" porque isso não faz sentido: a figura interromperia o fluxo
%    do texto da "outra" coluna.
%
% 3. Soluciona o terceiro problema.
%
\usepackage{stfloats}

% Às vezes é interessante utilizar uma imagem mais larga que o texto.
% Por padrão, \centering *não* vai centralizar a imagem corretamente
% nesse caso. Com esta package, podemos acrescentar a opção "center"
% ao comando \includegraphics para resolver esse problema
% (ou seja, \includegraphics[width=1.2\textwidth,center]{imagem}.
% A package tem muitos outros recursos também
\usepackage[export]{adjustbox}

% Por padrão, LaTeX prefere colocar floats no topo da página que
% onde eles foram definidos; vamos mudar isso. Este comando depende
% do pacote "floatrow", carregado logo acima.
\floatplacement{table}{htbp}
\floatplacement{figure}{htbp}

% Garante que floats (tabelas e figuras) só apareçam após as seções a que
% pertencem. Por padrão, se a seção começa no meio da página, LaTeX pode
% colocar a figura no topo dessa página
\usepackage{flafter}

% Às vezes um float pode ser adiado por muitas páginas; é possível forçar
% LaTeX a imprimir todos os floats pendentes com o comando \clearpage mas,
% para isso, o usuário deve identificar os casos problemáticos e inserir
% \clearpage manualmente. Esta package acrescenta o comando \FloatBarrier,
% que executa \clearpage automaticamente se necessário no local em que é
% chamado. A opção "section" faz o comando ser aplicado automaticamente
% a cada nova seção. "above" e "below" desabilitam a barreira quando os
% floats estão na mesma página. A desvantagem de placeins é que, para
% funcionar, ela gera quebras de página que muitas vezes são inesperadas.
%\usepackage[section,above,below]{placeins}

% LaTeX escolhe automaticamente o "melhor" lugar para colocar cada float.
% Por padrão, ele tenta colocá-los no topo da página e depois no pé da
% página; se não tiver sucesso, vai para a página seguinte e recomeça.
% Se esse algoritmo não tiver sucesso "logo", LaTeX cria uma página só
% com floats. É possível modificar esse comportamento com as opções de
% posicionamento: "tp", por exemplo, instrui LaTeX a não colocar floats
% no pé da página, e "htbp" o instrui para tentar "aqui" como a primeira
% opção. O pacote "floatrow" acrescenta a opção "H", que significa "aqui,
% incondicionalmente".
%
% A escolha do "melhor" lugar leva em conta os parâmetros abaixo, mas é
% possível ignorá-los com a opção de posicionamento "!". Dado que os
% valores default não são muito bons para floats "grandes" ou documentos
% com muitos floats, é muito comum usar "!" ou "H". No entanto, modificando
% esses parâmetros o algoritmo automático tende a funcionar melhor. Ainda
% assim, vale ler a discussão a respeito na seção "Limitações do LaTeX"
% deste modelo.

% Fração da página que pode ser ocupada por floats no topo. Default: 0.7
\renewcommand{\topfraction}{.85}
% Idem para documentos em colunas e floats que tomam as 2 colunas. Default: 0.7
\renewcommand{\dbltopfraction}{.66}
% Fração da página que pode ser ocupada por floats no pé. Default: 0.3
\renewcommand{\bottomfraction}{.7}
% Fração mínima da página que deve conter texto. Default: 0.2
\renewcommand{\textfraction}{.15}
% Numa página só de floats, fração mínima que deve ser ocupada. Default: 0.5
% floatpagefraction *deve* ser menor que topfraction.
\renewcommand{\floatpagefraction}{.66}
% Idem para documentos em colunas e floats que tomam as 2 colunas. Default: 0.5
\renewcommand{\dblfloatpagefraction}{.66}
% Máximo de floats no topo da página. Default: 2
\setcounter{topnumber}{9}
% Idem para documentos em colunas e floats que tomam as 2 colunas. Default: 2
\setcounter{dbltopnumber}{9}
% Máximo de floats no pé da página. Default: 1
\setcounter{bottomnumber}{9}
% Máximo de floats por página. Default: 3
\setcounter{totalnumber}{20}

% Define o ambiente "\begin{landscape} -- \end{landscape}"; o texto entre
% esses comandos é impresso em modo paisagem, podendo se estender por várias
% páginas. A rotação não inclui os cabeçalhos e rodapés das páginas.
% O principal uso desta package é em conjunto com a package longtable: se
% você precisa mostrar uma tabela muito larga (que precisa ser impressa em
% modo paisagem) e longa (que se estende por várias páginas), use
% "\begin{landscape}" e "\begin{longtable}" em conjunto. Note que o modo
% landscape entra em ação imediatamente, ou seja, "\begin{landscape}" gera
% uma quebra de página no local em que é chamado. Na maioria dos casos, o
% que se quer não é isso, mas sim um "float paisagem"; isso é o que a
% package rotating oferece (veja abaixo).
\usepackage{pdflscape}

% Define dois novos tipos de float: sidewaystable e sidewaysfigure, que
% imprimem a figura ou tabela sozinha em uma página em modo paisagem. Além
% disso, permite girar elementos na página de diversas outras maneiras.
\usepackage[figuresright,clockwise]{rotating}

% Captions com fonte menor, indentação normal, corpo do texto
% negrito e nome do caption itálico
\usepackage[
  font=small,
  format=plain,
  labelfont=bf,up,
  textfont=it,up]{caption}

% Em geral, a package caption é capaz de "adivinhar" se o caption
% está acima ou abaixo da figura/tabela, mas isso não funciona
% corretamente com longtable. Aqui, forçamos a package a considerar
% que os captions ficam abaixo das tabelas.
\captionsetup*[longtable]{position=bottom}

% Sub-figuras (e seus captions) - observe que existe uma package chamada
% "subfigure", mas ela é obsoleta; use esta no seu lugar.
\usepackage{subcaption}

% Permite criar imagens com texto ao redor
\usepackage{wrapfig}

% Permite incorporar um arquivo PDF como uma página adicional. Útil se
% for necessário importar uma imagem ou tabela muito grande ou ainda
% para definir uma capa personalizada.
\usepackage{pdfpages}

% Caixas de texto coloridas
%\usepackage{tcolorbox}


%%%%%%%%%%%%%%%%%%%%%%%%%%%%%%%%%%%%%%%%%%%%%%%%%%%%%%%%%%%%%%%%%%%%%%%%%%%%%%%%
%%%%%%%%%%%%%%%%%%%%%%%%%%%%%%%%%% TABELAS %%%%%%%%%%%%%%%%%%%%%%%%%%%%%%%%%%%%%
%%%%%%%%%%%%%%%%%%%%%%%%%%%%%%%%%%%%%%%%%%%%%%%%%%%%%%%%%%%%%%%%%%%%%%%%%%%%%%%%

% Tabelas simples são fáceis de fazer em LaTeX; tabelas com alguma sofisticação
% são trabalhosas, pois é difícil controlar alinhamento, largura das colunas,
% distância entre células etc. Ou seja, é muito comum que a tabela final fique
% "torta". Por isso, em muitos casos, vale mais a pena gerar a tabela em uma
% planilha, como LibreOffice calc ou excel, transformar em PDF e importar como
% figura, especialmente se você quer controlar largura/altura das células
% manualmente etc. No entanto, se você quiser fazer as tabelas em LaTeX para
% garantir a consistência com o tipo e o tamanho das fontes, é possível e o
% resultado é muito bom. Aqui há alguns pacotes que incrementam os recursos de
% tabelas do LaTeX e alguns comandos pré-prontos que podem facilitar um pouco
% seu uso.

% LaTeX por padrão não permite notas de rodapé dentro de tabelas. De maneira
% geral, notas de rodapé em tabelas são consideradas "ruins" em termos de
% tipografia, mas às vezes são necessárias. Se esse é o caso, o recomendado
% é que as notas de rodapé apareçam no "rodapé" da tabela, com numeração
% própria, e não no rodapé da página. Você pode fazer isso com esta package:
\usepackage{threeparttable}
% Formatação personalizada das notas de threeparttable:
\appto{\TPTnoteSettings}{\footnotesize\itshape}
\def\TPTtagStyle{\textit}

% Se você realmente quer notas de rodapé em tabelas que aparecem como as
% demais notas de rodapé (no final da página e mantendo a sequência numérica),
% você pode usar a package abaixo. No entanto, ela não funciona com floats
% duplos (floats que ocupam toda a largura da página em um documento de duas
% colunas) e, em alguns casos, a nota pode desaparecer ou aparecer em uma
% página diferente da tabela (mova o lugar do texto em que ela é definida
% para resolver esse problema).
\usepackage{tablefootnote}

% Por padrão, cada coluna de uma tabela tem a largura do maior texto contido
% nela, ou seja, se uma coluna contém uma célula muito larga, LaTeX não
% força nenhuma quebra de linha e a tabela "estoura" a largura do papel. A
% solução simples, nesses casos, é inserir uma ou mais quebras de linha
% manualmente, o que além de deselegante não é totalmente trivial (é preciso
% usar \makecell).
% Esta package estende o ambiente tabular para permitir definir um tamanho
% fixo para uma ou mais colunas; nesse caso, LaTeX quebra as linhas se uma
% célula é larga demais para a largura definida. Encontrar valores "bons"
% para as larguras das colunas, no entanto, também é um trabalho manual
% um tanto penoso. As packages tabularx e tabulary permitem configurar
% algumas colunas como "largura automática", evitando a necessidade da
% definição manual. Finalmente, ltxtable permite utilizar tabularx e
% longtable juntas. Neste modelo, não usamos tabularx/tabulary, mas você
% pode carregá-las se quiser.
\usepackage{array}

% Se você quer ter um pouco mais de controle sobre o tamanho de cada coluna da
% tabela, utilize estes tipos de coluna (criados com base nos recursos do pacote
% array). É só usar algo como M{número}, onde "número" (por exemplo, 0.4) é a
% fração de \textwidth que aquela coluna deve ocupar. "M" significa que o
% conteúdo da célula é centralizado; "L", alinhado à esquerda; "J", justificado;
% "R", alinhado à direita. Obviamente, a soma de todas as frações não pode ser
% maior que 1, senão a tabela vai ultrapassar a linha da margem.
\newcolumntype{M}[1]{>{\centering}m{#1\textwidth}}
\newcolumntype{L}[1]{>{\RaggedRight}m{#1\textwidth}}
\newcolumntype{R}[1]{>{\RaggedLeft}m{#1\textwidth}}
\newcolumntype{J}[1]{m{#1\textwidth}}

% Permite alinhar os elementos de uma coluna pelo ponto decimal; dê
% preferência à package siunitx (carregada em utils.tex), que também
% oferece esse recurso e muitos outros.
\usepackage{dcolumn}

% Define tabelas do tipo "longtable", similares a "tabular" mas que podem ser
% divididas em várias páginas. "longtable" também funciona corretamente com
% notas de rodapé. Note que, como uma longtable pode se estender por várias
% páginas, não faz sentido colocá-las em um float "table". Por conta disso,
% longtable define o comando "\caption" internamente.
\usepackage{longtable}

% Permite agregar linhas de tabelas, fazendo colunas "compridas"
\usepackage{multirow}

% Cria comando adicional para possibilitar a inserção de quebras de linha
% em uma célula de tabela, entre outros
\usepackage{makecell}

% Às vezes a tabela é muito larga e não cabe na página. Se os cabeçalhos da
% tabela é que são demasiadamente largos, uma solução é inclinar o texto das
% células do cabeçalho. Para fazer isso, use o comando "\rothead".
\renewcommand{\rothead}[2][60]{\makebox[11mm][l]{\rotatebox{#1}{\makecell[c]{#2}}}}

% Se quiser criar uma linha mais grossa no meio de uma tabela, use
% o comando "\thickhline".
\newlength\savedwidth
\newcommand\thickhline{
  \noalign{
    \global\savedwidth\arrayrulewidth
    \global\arrayrulewidth 1.5pt
  }
  \hline
  \noalign{\global\arrayrulewidth\savedwidth}
}

% Modifica (melhora) o layout default das tabelas e acrescenta os comandos
% \toprule, \bottomrule, \midrule e \cmidrule
\usepackage{booktabs}

% Permite colorir linhas, colunas ou células
\usepackage{colortbl}

% Ao invés de digitar os dados de uma tabela dentro do seu documento,
% você pode fazer LaTeX ler os dados de um arquivo CSV e criar uma
% tabela automaticamente com uma destas duas packages:
%\usepackage{csvsimple}     % mais simples
%\usepackage{pgfplotstable} % mais complexa

% Você também pode se interessar pelo ambiente "tabbing", que permite
% criar tabelas simples com algumas vantagens em relação a "tabular",
% ou por esta package, que permite criar tabulações.
%\usepackage{tabto-ltx}
