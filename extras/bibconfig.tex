%%%%%%%%%%%%%%%%%%%%%%%%%%%%%%%%%%%%%%%%%%%%%%%%%%%%%%%%%%%%%%%%%%%%%%%%%%%%%%%%
%%%%%%%%%%%%%%%%%%%%%%%%%%%%%%% BIBLIOGRAFIA %%%%%%%%%%%%%%%%%%%%%%%%%%%%%%%%%%%
%%%%%%%%%%%%%%%%%%%%%%%%%%%%%%%%%%%%%%%%%%%%%%%%%%%%%%%%%%%%%%%%%%%%%%%%%%%%%%%%

% Tradicionalmente, bibliografias no LaTeX são geradas com uma combinação entre
% LaTeX (muitas vezes usando o pacote natbib) e um programa auxiliar chamado
% bibtex. Nesse esquema, LaTeX e natbib são responsáveis por formatar as
% referências ao longo do texto e a formatação da bibliografia fica por conta
% do programa bibtex. A configuração dessa formatação é feita através de um
% arquivo auxiliar de "estilo", com extensão ".bst". Vários journals etc.
% fornecem o aqruivo .bst que corresponde ao formato esperado da bibliografia.
%
% bibtex e natbib funcionam bem e, se você tiver alguma boa razão para usá-los,
% obterá bons resultados. No entanto, bibtex tem dois problemas: não lida
% corretamente com caracteres acentuados (embora, na prática, funcione com
% os caracteres usados em português) e o formato .bst, que define a formatação
% da bibliografia, é complexo e pouco flexível.
%
% Por conta disso, a comunidade está migrando para um novo sistema chamado
% biblatex. No biblatex, a formatação da bibliografia e das citações são feitas
% pelo próprio pacote biblatex, dentro do LaTeX. Assim, é bem mais fácil
% modificar e personalizar o estilo da bibliografia. biblatex usa o mesmo
% formato de arquivo de dados do bibtex (".bib") e, portanto, não é difícil
% migrar de um para o outro. biblatex também usa um programa auxiliar (biber),
% mas não para realizar a formatação da bibliografia. A maior desvantagem de
% biblatex é que ele é significativamente mais lento que bibtex.
%
% Observe que biblatex pode criar bibliografias independentes por capítulo
% ou outras divisões do texto. Normalmente é preciso indicar essas seções
% manualmente, mas as opções "refsection" e "refsegment" fazem biblatex
% identificar cada capítulo/seção/etc como uma nova divisão desse tipo.
% No entanto, refsection e refsegment são incompatíveis com o pacote
% titlesec, mencionado em miolo-preambulo. Se você pretende criar
% bibliografias independentes por seções, há duas soluções: (1) desabilitar
% o pacote titlesec; (2) indicar as seções manualmente.

%%%%%%%%%%% Usando bibtex: %%%%%%%%%%%%
%\usepackage[hyperpageref]{backref}
%\renewcommand*{\backref}[1]{}
%\renewcommand*{\backrefalt}[4]{%
%  \scriptsize%
%  \ifcase #1\relax%
%  \or(Citado na pg. #2)%
%  \else(Citado nas pgs. #2)%
%  \fi%
%}
%\usepackagefromsubdir[square,sort,nonamebreak,comma]{extras/}{natbib-ime}  % citação bibliográfica alpha (alpha-ime.bst)
%\usepackagefromsubdir[round,sort,nonamebreak]{extras/}{natbib-ime} % citação bibliográfica textual(plainnat-ime.bst)

%%%%%%%%%%% Usando biblatex: %%%%%%%%%%%
% https://tex.stackexchange.com/questions/12806/guidelines-for-customizing-biblatex-styles
% https://github.com/PaulStanley/biblatex-tutorial/releases
\usepackage[
  % Ativa o suporte ao pacote hyperref
  hyperref=true,
  % Reconhece comandos no estilo do pacote natbib (\citet, \citep)
  natbib=true,
  % Se um item da bibliografia tem língua definida (com langid), permite
  % hifenizar com base na língua selecionada.
  autolang=hyphen,
  % Inclui, em cada item da bibliografia, links para as páginas onde o
  % item foi citado
  backref=true,
  % Com mais de 5 nomes, usa "et. al." na bibliografia
  maxbibnames=5,
  % Com mais de 2 nomes, usa "et. al." nas citações (só faz
  % diferença nos estilos autor-data, como plainnat-ime)
  maxcitenames=2,
  % Estilo similar a plainnat
  bibstyle=extras/plainnat-ime,
  citestyle=extras/plainnat-ime,
  % Estilo similar a alpha
  %bibstyle=alphabetic,
  %citestyle=alphabetic,
  % O estilo numérico é comum em artigos
  %bibstyle=numeric,
  %citestyle=numeric,
  % Um estilo que busca ser compatível com a ABNT:
  %bibstyle=abnt,
  %citestyle=abnt,
]{biblatex}

% Sobrenomes nas citações e na bibliografia em Small Caps
\renewcommand{\mkbibnamefamily}[1]{\textsc{#1}}

% Se desejável, o label na bibliografia no formato
% plainnat-ime pode ser feito em negrito
%\renewcommand*{\labelhighlight}[1]{\textbf{#1}}

% Autores no formato "nome sobrenome"
\DeclareNameAlias{sortname}{given-family}
\DeclareNameAlias{default}{given-family}

% Autores no formato "sobrenome, nome"
%\DeclareNameAlias{sortname}{family-given}
%\DeclareNameAlias{default}{family-given}

% Vamos deixar um pequeno espaço entre cada item da bibliografia
\setlength{\bibitemsep}{1em}

% A primeira linha de cada item da bibliografia pode ter margem menor
% que as demais; aqui definimos essa diferença:
\setlength{\bibhang}{2em}

\AtBeginBibliography{
  % No estilo alfabético, biblatex coloca o "label" de cada item
  % alinhado à esquerda; bibtex alinha à direita e isso faz
  % mais sentido, então redefinimos o alinhamento aqui.
  \renewcommand*{\makelabel}[1]{\hss#1}
}

\DefineBibliographyStrings{brazilian}{
  % Na bibliografia, criamos links para as páginas onde uma
  % determinada obra foi citada. O texto padrão para indicar
  % isso é "ver...", vamos trocar.
  backrefpage  = {citado na pg\adddot},
  backrefpages = {citado nas pgs\adddot},
  % "et al." em itálico
  andothers    = {\textit{et\addabbrvspace al}\adddot},
  page         = {pg\adddot},
  pages        = {pgs\adddot},
}

\DefineBibliographyStrings{american}{
  % "et al." em itálico
  andothers    = {\textit{et\addabbrvspace al}\adddot},
}

% biblatex redefine os cabeçalhos das páginas na bibliografia;
% vamos deixar essa tarefa para fancyhdr.
\defbibheading{bibliography}[\bibname]{%
  \chapter*{#1}%
  %\section*{#1}% Se for a classe "article"
  \chaptermark{#1}%
  %\sectionmark{#1}% Se for a classe "article"
}

\defbibheading{bibintoc}[\bibname]{%
  \chapter*{#1}%
  %\section*{#1}% Se for a classe "article"
  \addcontentsline{toc}{chapter}{#1}%
  \chaptermark{#1}%
  %\sectionmark{#1}% Se for a classe "article"
}

% bibtex assume que os títulos no arquivo .bib seguem o formato "title case",
% ou seja, "A Maioria das Palavras É Iniciada por Maiúsculas". Se o estilo
% bibliográfico usa esse formato, o título é impresso sem mudanças; se o
% estilo bibliográfico não usa esse formato, bibtex transforma o título
% em caixa baixa.
%
% Por padrão, biblatex não faz nada e mantém o título como digitado no
% arquivo bib, mas fornece a possibilidade de modificar esse comportamento.
% Isso depende (1) da língua do documento (ou a língua de entrada do
% item na bibliografia) e (2) da definição do comando que deve realizar
% o processamento.
% Aqui, ativamos esse mecanismo para a língua portuguesa:
\DeclareCaseLangs{brazilian,brazil,portuges}

% Três opções de processamento:
%
% Não faz nada (default):
%\DeclareFieldFormat{titlecase}{#1}
%
% Transforma todos os títulos em caixa baixa, que é o usual em português.
%\DeclareFieldFormat{titlecase}{\MakeSentenceCase*{#1}}
%
% Mantém maiúsculas/minúsculas como no arquivo .bib, exceto para artigos
% ou capítulos de livro (o comando \NoChangeOrSentenceCase está definido
% no arquivo miolo-preambulo.tex):
\DeclareFieldFormat{titlecase}{\NoChangeOrSentenceCase{#1}}

% Esta macro implementa o mecanismo descrito acima
% https://tex.stackexchange.com/questions/22980/sentence-case-for-titles-in-biblatex
\newrobustcmd{\NoChangeOrSentenceCase}[1]{%
  \ifthenelse{\ifcurrentfield{booktitle}\OR\ifcurrentfield{booksubtitle}%
    \OR\ifcurrentfield{maintitle}\OR\ifcurrentfield{mainsubtitle}%
    \OR\ifcurrentfield{journaltitle}\OR\ifcurrentfield{journalsubtitle}%
    \OR\ifcurrentfield{issuetitle}\OR\ifcurrentfield{issuesubtitle}%
    \OR\ifentrytype{book}\OR\ifentrytype{mvbook}\OR\ifentrytype{bookinbook}%
    \OR\ifentrytype{booklet}\OR\ifentrytype{suppbook}%
    \OR\ifentrytype{collection}\OR\ifentrytype{mvcollection}%
    \OR\ifentrytype{suppcollection}\OR\ifentrytype{manual}%
    \OR\ifentrytype{periodical}\OR\ifentrytype{suppperiodical}%
    \OR\ifentrytype{proceedings}\OR\ifentrytype{mvproceedings}%
    \OR\ifentrytype{reference}\OR\ifentrytype{mvreference}%
    \OR\ifentrytype{report}\OR\ifentrytype{thesis}%
    \OR\ifentrytype{online}\OR\ifentrytype{misc}}
    {#1}
    {\MakeSentenceCase*{#1}}}
