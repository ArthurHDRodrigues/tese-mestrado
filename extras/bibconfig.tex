%%%%%%%%%%%%%%%%%%%%%%%%%%%%%%%%%%%%%%%%%%%%%%%%%%%%%%%%%%%%%%%%%%%%%%%%%%%%%%%%
%%%%%%%%%%%%%%%%%%%%%%%%%%%%%%% BIBLIOGRAFIA %%%%%%%%%%%%%%%%%%%%%%%%%%%%%%%%%%%
%%%%%%%%%%%%%%%%%%%%%%%%%%%%%%%%%%%%%%%%%%%%%%%%%%%%%%%%%%%%%%%%%%%%%%%%%%%%%%%%

% Tradicionalmente, bibliografias no LaTeX são geradas com uma combinação entre
% LaTeX (muitas vezes usando o pacote natbib) e um programa auxiliar chamado
% bibtex. Nesse esquema, LaTeX e natbib são responsáveis por formatar as
% referências ao longo do texto e a formatação da bibliografia fica por conta
% do programa bibtex. A configuração dessa formatação é feita através de um
% arquivo auxiliar de "estilo", com extensão ".bst". Vários journals etc.
% fornecem o arquivo .bst que corresponde ao formato esperado da bibliografia.
%
% bibtex e natbib funcionam bem e, se você tiver alguma boa razão para usá-los,
% obterá bons resultados. No entanto, bibtex tem dois problemas: não lida
% corretamente com caracteres acentuados (embora, na prática, funcione com
% os caracteres usados em português) e o formato .bst, que define a formatação
% da bibliografia, é complexo e pouco flexível.
%
% Por conta disso, a comunidade está migrando para um novo sistema chamado
% biblatex. No biblatex, as formatações da bibliografia e das citações são
% feitas pelo próprio pacote biblatex, dentro do LaTeX. Assim, é bem mais fácil
% modificar e personalizar o estilo da bibliografia. biblatex usa o mesmo
% formato de arquivo de dados do bibtex (".bib") e, portanto, não é difícil
% migrar de um para o outro. biblatex também usa um programa auxiliar (biber),
% mas não para realizar a formatação da bibliografia. A maior desvantagem de
% biblatex é que ele é significativamente mais lento que bibtex.
%
% Observe que biblatex pode criar bibliografias independentes por capítulo
% ou outras divisões do texto. Normalmente é preciso indicar essas seções
% manualmente, mas as opções "refsection" e "refsegment" fazem biblatex
% identificar cada capítulo/seção/etc como uma nova divisão desse tipo.
% No entanto, refsection e refsegment são incompatíveis com o pacote
% titlesec, mencionado em imeusp-formatting.tex. Se você pretende criar
% bibliografias independentes por seções, há duas soluções: (1) desabilitar
% o pacote titlesec; (2) indicar as seções manualmente.

\makeatletter
\@ifpackageloaded{biblatex}
{
  %%%%%%%%%%% Usando biblatex: %%%%%%%%%%%
  % https://tex.stackexchange.com/questions/12806/guidelines-for-customizing-biblatex-styles
  % https://github.com/PaulStanley/biblatex-tutorial/releases

  \ExecuteBibliographyOptions{
    % Ativa o suporte ao pacote hyperref
    hyperref=true,
    % Se um item da bibliografia tem língua definida (com langid), permite
    % hifenizar com base na língua selecionada.
    autolang=hyphen,
    % Inclui, em cada item da bibliografia, links para as páginas onde o
    % item foi citado
    backref=true,
    % Com mais de 5 nomes, usa "et. al." na bibliografia
    maxbibnames=5,
    % Com mais de 2 nomes, usa "et. al." nas citações (só faz
    % diferença nos estilos autor-data, como plainnat-ime)
    maxcitenames=2,
  }

  % Sobrenomes nas citações e na bibliografia em Small Caps
  \renewcommand{\mkbibnamefamily}[1]{\textsc{#1}}

  % Autores no formato "nome sobrenome"
  \DeclareNameAlias{sortname}{given-family}
  \DeclareNameAlias{default}{given-family}

  % Autores no formato "sobrenome, nome"
  %\DeclareNameAlias{sortname}{family-given}
  %\DeclareNameAlias{default}{family-given}

  % Vamos deixar um pequeno espaço entre cada item da bibliografia
  \setlength{\bibitemsep}{1em}

  % A primeira linha de cada item da bibliografia pode ter margem menor
  % que as demais; aqui definimos essa diferença:
  \setlength{\bibhang}{2em}

  % Para mudar o tamanho da fonte na bibliografia:
  %\renewcommand*{\bibfont}{\footnotesize}

  \DefineBibliographyStrings{brazilian}{
    % Na bibliografia, criamos links para as páginas onde uma
    % determinada obra foi citada. O texto padrão para indicar
    % isso é "ver...", vamos trocar.
    backrefpage  = {citado na pg\adddot},
    backrefpages = {citado nas pgs\adddot},
    page         = {pg\adddot},
    pages        = {pgs\adddot},
  }

  % Por padrão, biblatex mantém a caixa alta/baixa dos títulos como
  % digitados no arquivo bib, mas fornece a possibilidade de modificar
  % esse comportamento em função da língua. Se a opção abaixo for
  % ativada, plainnat-ime mantém maiúsculas/minúsculas como no arquivo
  % .bib, exceto para artigos ou capítulos de livro (essa é uma convenção
  % comum). Para mais opções, leia os comentários em plainnat-ime.bbx.
  %
  % Aqui, ativamos esse mecanismo para a língua portuguesa:
  \DeclareCaseLangs{brazilian,brazil,portuges,portuguese}

  % Impede que um item da bibliografia seja dividido em duas páginas.
  % À parte a estética, isso contorna este bug, que afeta links na
  % úlima página do trabalho, ou seja, pode afetar a bibliografia
  % (atenddvi pode ser carregada por hyperxmp):
  % https://github.com/ho-tex/atenddvi/issues/1
  \AtBeginBibliography{\interlinepenalty=10000\raggedbottom}
}
{
  %%%%%%%%%%% Usando bibtex: %%%%%%%%%%%%
  \usepackage[hyperpageref]{backref}
  \renewcommand*{\backref}[1]{}
  \renewcommand*{\backrefalt}[4]{%
    \scriptsize%
    \ifcase #1\relax%
    \or(Citado na pg. #2)%
    \else(Citado nas pgs. #2)%
    \fi%
  }
}
\makeatother
