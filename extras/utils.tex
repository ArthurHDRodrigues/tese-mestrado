%%%%%%%%%%%%%%%%%%%%%%%%%%%%%%%%%%%%%%%%%%%%%%%%%%%%%%%%%%%%%%%%%%%%%%%%%%%%%%%%
%%%%%%%%%%%%%%%%%%%%%%%%%%%% OUTROS PACOTES ÚTEIS %%%%%%%%%%%%%%%%%%%%%%%%%%%%%%
%%%%%%%%%%%%%%%%%%%%%%%%%%%%%%%%%%%%%%%%%%%%%%%%%%%%%%%%%%%%%%%%%%%%%%%%%%%%%%%%

% Você provavelmente vai querer ler a documentação de alguns destes pacotes
% para personalizar algum aspecto do trabalho ou usar algum recurso específico.

% A classe Book inclui o comando \appendix, que (obviamente) permite inserir
% apêndices no documento. No entanto, não há suporte similar para anexos. Esta
% package (que não é padrão do LaTeX, foi criada para este modelo) define o
% comando \annex. Ela deve ser carregada depois de hyperref.
\dowithsubdir{extras/}{\usepackage{annex}}

% Formatação personalizada das listas "itemize", "enumerate" e
% "description", além de permitir criar novos tipos de listas
%\usepackage{paralist}

% Trechos de texto "puro" (tabs, quebras de linha etc. não são modificados)
\usepackage{verbatim}

% Sublinhado e outras formas de realce de texto
\usepackage{soul}
\usepackage{soulutf8}

% Recursos e símbolos adicionais para o modo matemático
% para evitar problemas de compatibilidade com algumas fontes, o pacote
% amsthm já foi carregado mais acima
\usepackage{latexsym}
\usepackage{amsmath}
\usepackage{amssymb}
\usepackage{mathtools}
%\usepackage{MnSymbol}
%\usepackage{stmaryrd}

% Notação bra-ket
%\usepackage{braket}

%\num \SI and \SIrange. For example, \SI{10}{\hertz} \SIrange{10}{100}{\hertz}
%\usepackage[binary-units]{siunitx}

% Citações melhores; se você pretende fazer citações de textos
% relativamente extensos, vale a pena ler a documentação. biblatex
% utiliza recursos deste pacote.
\usepackage{csquotes}

\usepackage{url}
% URL com fonte sem serifa ao invés de teletype
\urlstyle{sf}

% Permite inserir comentários, muito bom durante a escrita do texto
\usepackage[textsize=scriptsize,colorinlistoftodos,textwidth=2.5cm]{todonotes}
\presetkeys{todonotes}{color=orange!40!white}{}

% Comando para fazer notas com highlight no texto correspondente:
% \hltodo[texto][opções]{comentário}
\makeatletter
\if@todonotes@disabled
  \NewDocumentCommand{\hltodo}{O{} O{} +m}{#1}
\else
  \NewDocumentCommand{\hltodo}{O{} O{} +m}{
    \ifstrempty{#1}{}{\texthl{#1}}%
    \todo[#2]{#3}{}%
  }
\fi
\makeatother

% Símbolos adicionais, como \celsius, \ohm, \perthousand etc.
%\usepackage{gensymb}

% Símbolos adicionais, como \textrightarrow, \texteuro etc.
\usepackage{textcomp}

% Permite criar listas como glossários, listas de abreviaturas etc.
% https://en.wikibooks.org/wiki/LaTeX/Glossary
%\usepackage{glossaries}

% Permite formatar texto em colunas
\usepackage{multicol}

% Gantt charts; útil para fazer o cronograma para o exame de
% qualificação, por exemplo.
\usepackage{pgfgantt}

% Na versão 5 do pacote pgfgantt, a opção "compress calendar"
% deixou de existir, sendo substituída por "time slot unit=month".
% Aqui, um truque para funcionar com ambas as versões.
\makeatletter
\@ifpackagelater{pgfgantt}{2018/01/01}
  {\ganttset{time slot unit=month}}
  {\ganttset{compress calendar}}
\makeatother

% Estes parâmetros definem a aparência das gantt charts e variam
% em função da fonte do documento.
\ganttset{
    time slot format=isodate-yearmonth,
    vgrid,
    x unit=1.7em,
    y unit title=3ex,
    y unit chart=4ex,
    % O "strut" é necessário para alinhar o baseline dos nomes dos meses
    title label font=\strut\footnotesize,
    group label font=\footnotesize\bfseries,
    bar label font=\footnotesize,
    milestone label font=\footnotesize\itshape,
    % "align=right" é necessário para \ganttalignnewline funcionar
    group label node/.append style={align=right},
    bar label node/.append style={align=right},
    milestone label node/.append style={align=right},
    group incomplete/.append style={fill=black!50},
    bar/.append style={fill=black!25,draw=black},
    bar incomplete/.append style={fill=white,draw=black},
    % Não é preciso imprimir "0%"
    progress label text=\ifnumequal{#1}{0}{}{(#1\%)},
    % Formato e tamanho dos elementos
    title height=.9,
    group top shift=.4,
    group left shift=0,
    group right shift=0,
    group peaks tip position=0,
    group peaks width=.2,
    group peaks height=.3,
    milestone height=.4,
    milestone top shift=.4,
    milestone left shift=.8,
    milestone right shift=.2,
}

% Em inglês, tanto o nome completo quanto a abreviação do mês de maio
% são "May"; por conta disso, na tradução em português LaTeX erra a
% abreviação. Como talvez usemos o nome inteiro do mês em outro lugar,
% ao invés de forçar a tradução para "Mai" globalmente, fazemos isso
% apenas em ganttchart.
\AtBeginEnvironment{ganttchart}{\deftranslation[to=Portuguese]{May}{Mai}}

% Os comandos \TeX e \LaTeX são nativos do LaTeX; esta package acrescenta os
% comandos \XeLaTeX e \LuaLaTeX. Você provavelmente não precisa desse recurso
% e, portanto, pode removê-la.
\usepackage{metalogo}
