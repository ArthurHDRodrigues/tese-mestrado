%%%%%%%%%%%%%%%%%%%%%%%%%%%%%%%%%%%%%%%%%%%%%%%%%%%%%%%%%%%%%%%%%%%%%%%%%%%%%%%%
%%%%%%%%%%%%%%%%%%%%%%% CONFIGURAÇÕES E PACOTES BÁSICOS %%%%%%%%%%%%%%%%%%%%%%%%
%%%%%%%%%%%%%%%%%%%%%%%%%%%%%%%%%%%%%%%%%%%%%%%%%%%%%%%%%%%%%%%%%%%%%%%%%%%%%%%%

% Vários comandos auxiliares para o desenvolvimento de packages e classes;
% aqui, usamos em alguns comandos de formatação e condicionais.
\usepackage{etoolbox}
\usepackage{xstring}
\usepackage{xparse}
\usepackage{regexpatch}
% Esta não está em uso mas pode ser útil.
%\usepackage{ltxcmds}

% Esta package permite detectar XeTeX, LuaTeX e pdfTeX, mas pode não estar
% disponível em todas as instalações de TeX.
%\usepackage{iftex}
% Por conta disso, usaremos estas (que não detectam pdfTeX):
\usepackage{ifxetex}
\usepackage{ifluatex}

\newbool{unicodeengine}
\ifboolexpr{bool{xetex} or bool{luatex}}
  {\booltrue{unicodeengine}}
  {\boolfalse{unicodeengine}}

% Detecta se estamos produzindo um arquivo PDF ou DVI (lembrando que tanto
% pdfTeX quanto LuaTeX podem gerar ambos)
\usepackage{ifpdf}

% Algumas packages "padrão" da AMS, que são praticamente  obrigatórias.
% Algumas delas devem ser carregadas antes de unicode-math ou das
% definições das fontes do documento.
\usepackage{amssymb}
\usepackage{amsthm}
\usepackage{amsmath}
\usepackage{mathtools}

\ifunicodeengine
  % Não é preciso carregar fontenc e inputenc com LuaTeX e XeTeX!
  \usepackage{fontspec}
  \usepackage{unicode-math}
\else
  % "fontenc" é um parâmetro interno do LaTeX. Com pdfLaTeX, o fontenc
  % default é OT1, mas ele tem algumas limitações; a mais importante é
  % que, com ele, palavras acentuadas não podem ser hifenizadas. Por
  % conta disso, quase todos os documentos LaTeX utilizam o fontenc T1.
  % A escolha do fontenc tem consequências para as fontes que podem ser
  % usadas no documento; hoje em dia T1 tem mais opções de qualidade,
  % então não se perde nada.
  \usepackage[T1]{fontenc}
  \usepackage[utf8]{inputenc}
  \usepackage{fontaxes}
\fi

% Internacionalização dos nomes das seções ("chapter" X "capítulo" etc.),
% hifenização e outras convenções tipográficas. babel deve ser um dos
% primeiros pacotes carregados. É possível passar a língua do documento
% como parâmetro aqui, mas já fizemos isso ao carregar a classe, no início
% do documento.
\usepackage{babel}

% É possível personalizar as palavras-chave que babel utiliza, por exemplo:
%\addto\extrasbrazil{\renewcommand{\chaptername}{Chap.}}
% Com BibTeX, isso vale também para a bibliografia; com BibLaTeX, é melhor
% usar o comando "DefineBibliographyStrings".

% Para línguas baseadas no alfabeto latino, como o inglês e o português,
% o pacote babel funciona muito bem, mas com outros alfabetos ele às vezes
% falha. Por conta disso, o pacote polyglossia foi criado para substituí-lo.
% polyglossia só funciona com LuaTeX e XeTeX; como babel também funciona com
% esses sistemas, provavelmente não há razão para usar polyglossia, mas é
% possível que no futuro esse pacote se torne o padrão.
%\usepackage{polyglossia}
%\setdefaultlanguage{brazil}
%\setotherlanguage{english}

% Alguns pacotes (espeficicamente, tikz) usam, além de babel, este pacote
% como auxiliar para a tradução de palavras-chave, como os meses do ano.
\usepackage{translator}

% microajustes no tamanho das letras, espaçamento etc. para melhorar
% a qualidade visual do resultado. LaTeX tradicional não dá suporte a
% nenhum tipo de microajuste; pdfLaTeX dá suporte a todos. LuaLaTeX
% e XeLaTeX dão suporte a alguns:
%
% * expansion não funciona com XeLaTeX
% * tracking não funciona com XeLaTeX; é possível obter o mesmo resultado
%   com a opção "LetterSpace" do pacote fontspec, mas a configuração é
%   totalmente manual. Por padrão, aumenta o afastamento entre caracteres
%   nas fontes "small caps"; o resultado não se presta ao uso na
%   bibliografia ou citações, então melhor desabilitar.
% * kerning e spacing só funcionam com pdfLaTex; ambas são funções
%   consideradas experimentais e nem sempre produzem resultados vantajosos.

\newcommand\microtypeopts{
  protrusion=true,
  tracking=false,
  kerning=false,
  spacing=false
}

% TeXLive 2018 inclui a versão 2.7a da package microtype e a versão
% 1.07 de luatex. Essa combinação faz aparecer um bug:
% https://tex.stackexchange.com/questions/476740/microtype-error-with-lualatex-attempt-to-call-field-warning-a-nil-value
% Aqui, aplicamos a solução sugerida, que não tem "contra-indicações".
\ifluatex
  \usepackage{luatexbase}
\fi

\ifxetex
  \usepackage[expansion=false,\microtypeopts]{microtype}
\else
  \usepackage[expansion=true,\microtypeopts]{microtype}
\fi

% Alguns "truques" (sujos?) para minimizar over/underfull boxes.
%
% Para fazer um texto justificado, é preciso modificar o tamanho dos espaços
% em cada linha para mais ou para menos em relação ao seu tamanho ideal. Para
% escolher as quebras de linha, TeX vai percorrendo o texto procurando lugares
% possíveis para quebrar as linhas considerando essa flexibilidade mas dentro
% de um certo limite mínimo/máximo. Nesse processo, ele associa a cada possível
% linha o valor badness, que é o nível de distorção do tamanho dos espaços
% daquela linha em relação ao ideal, e ignora opções que tenham badness muito
% grande (esse limite é dado por \tolerance). Depois de encontradas todas as
% possíveis quebras de linha e a badness de cada uma, TeX calcula as penalties
% e demerits de cada possibilidade e escolhe a solução que minimiza o demerit
% total do parágrafo.
%
% Para cada fonte, o espaço em TeX tem um tamanho ideal, um tamanho mínimo e um
% tamanho máximo. TeX nunca reduz um espaço para menos que o mínimo da fonte,
% mas pode aumentá-lo para mais que o máximo. Se os espaços de uma linha ficam
% com o tamanho ideal, a badness da linha é 0; se o tamanho é
% reduzido/aumentado 50% do mínimo/máximo, a badness da linha é 12; se o
% tamanho é reduzido/aumentado para o mínimo/máximo, a badness é 100. Se esse
% aumento for de 30% além do máximo, a badness da linha é 200; se for de 45%
% além do máximo, a badness é 300; se for de 60% além do máximo, a badness é
% 400; se for de 100% além do máximo, a badness é 800.
%
% A \tolerance default é 200; aumentar para, digamos, 300 ou 400, permite que
% TeX escolha parágrafos com maior variação no espaçamento. No entanto, no
% cálculo de demerits, a badness de cada linha é elevada ao quadrado, então TeX
% geralmente prefere escolher outras opções no lugar de uma linha ruim. Por
% exemplo, órfãs/viúvas têm demerit de 22.500 e dois hífens seguidos têm
% demerit de 10.000 e, portanto, não é surpreendente que a maioria dos
% parágrafos tenha demerits abaixo de 40.000, quase todos abaixo de 100.000 e
% praticamente nenhum acima de 1.000.000. Isso significa que, para a grande
% maioria dos parágrafos, aumentar \tolerance não faz diferença: uma linha com
% badness 400 (correspondendo a demerit de 160.000) nunca será efetivamente
% escolhida se houver qualquer outra opção com badness menor. Também fica claro
% que não há muita diferença real entre definir \tolerance como 800 ou 9.999.
%
% O problema muda de figura se TeX não consegue encontrar uma solução. Isso
% pode acontecer em dois casos: (1) o parágrafo tem ao menos uma linha que não
% pode ser quebrada com badness < 10.000 e (2) o parágrafo tem ao menos uma
% linha que não pode ser quebrada com badness < tolerance (mas essa badness é
% menor que 10.000).
%
% No primeiro caso, se houver várias possibilidades de linhas que não podem ser
% quebradas, TeX não vai ser capaz de compará-las e escolher a melhor: todas
% têm a badness máxima (10.000) e, portanto, a que gerar menos deméritos no
% restante do parágrafo será a escolhida. Na realidade, no entanto, essas
% linhas *não* são igualmente ruins entre si, o que pode levar TeX a fazer uma
% má escolha. Para evitar isso, TeX tenta novamente aplicando
% \emergencystretch, que "faz de conta" que o tamanho máximo ideal dos espaços
% da linha é maior que o definido na fonte. Isso reduz a badness de todas as
% linhas, o que soa parecido com aumentar \tolerance. Há três diferenças, no
% entanto: (1) essa mudança só afeta os parágrafos que falharam; (2) soluções
% que originalmente teriam badness = 10.000 (e, portanto, seriam vistas como
% equivalentes) podem ser avaliadas e comparadas entre si; e (3) como a badness
% de todas as linhas diminui, a possibilidade de outras linhas que
% originalmente tinham badness alta serem escolhidas aumenta. Esse último ponto
% significa que \emergencystretch pode fazer TeX escolher linhas mais
% espaçadas, fazendo o espaçamento do parágrafo inteiro aumentar e, portanto,
% tornando o resultado mais homogêneo mesmo com uma linha particularmente ruim.
%
% É esse último ponto que justifica o uso de \emergencystretch no segundo caso
% também: apenas aumentar a tolerância, nesse caso, poderia levar TeX a
% diagramar uma linha ruim em meio a um parágrafo bom, enquanto
% \emergencystretch pode fazer TeX aumentar o espaçamento de maneira geral no
% parágrafo, minimizando o contraste da linha problemática com as demais.
% Colocando a questão de outra maneira, aumentar \tolerance para lidar com
% esses parágrafos problemáticos pode fazê-los ter uma linha especialmente
% ruim, enquanto \emergencystretch pode dividir o erro entre várias linhas.
% Assim, definir \tolerance em torno de 800 parece razoável: no caso geral,
% não há diferença e, se um desses casos difíceis não pode ser resolvido com
% uma linha de badness até 800, \emergencystretch deve ser capaz de gerar um
% resultado igual ou melhor.
%
% Penalties & demerits: https://tex.stackexchange.com/a/51264
% Definições (fussy, sloppy etc.): https://tex.stackexchange.com/a/241355
% Mais definições (hfuzz, hbadness etc.): https://tex.stackexchange.com/a/50850
% Donald Arseneau defendendo o uso de \sloppy: https://groups.google.com/d/msg/comp.text.tex/Dhf0xxuQ66E/QTZ7aLYrdQUJ
% Artigo detalhado sobre \emergencystretch: https://www.tug.org/TUGboat/tb38-1/tb118wermuth.pdf
% Esse artigo me leva a crer que algo em torno de 1.5em é suficiente

\tolerance=800
\hyphenpenalty=100 % Default 50; se o texto é em 2 colunas, 50 é melhor
\setlength{\emergencystretch}{2.5em}

% Não gera warnings para Overfull menor que 0.5pt
\hfuzz=.5pt
\vfuzz\hfuzz

% Não gera warnings para Underfull com badness < 1000
\hbadness=1000
\vbadness=1000

% Por padrão, o algoritmo LaTeX para textos não-justificados é (muito) ruim;
% este pacote implementa um algoritmo bem melhor
\usepackage[newcommands]{ragged2e}

% Com ragged2e e a opção "newcommands", textos curtos não-justificados
% podem gerar warnings sobre "underfull \hbox". Não há razão para pensar
% muito nesses warnings, então melhor desabilitá-los.
% https://tex.stackexchange.com/questions/17659/ragged2e-newcommands-option-produces-underfull-hbox-warnings
\makeatletter
\g@addto@macro{\centering}{\hbadness=\@M}
\g@addto@macro{\Centering}{\hbadness=\@M}
\g@addto@macro{\raggedright}{\hbadness=\@M}
\g@addto@macro{\RaggedRight}{\hbadness=\@M}
\g@addto@macro{\raggedleft}{\hbadness=\@M}
\g@addto@macro{\RaggedLeft}{\hbadness=\@M}
\g@addto@macro{\center}{\hbadness=\@M}
\g@addto@macro{\Center}{\hbadness=\@M}
\g@addto@macro{\flushleft}{\hbadness=\@M}
\g@addto@macro{\FlushLeft}{\hbadness=\@M}
\g@addto@macro{\flushright}{\hbadness=\@M}
\g@addto@macro{\FlushRight}{\hbadness=\@M}
\makeatother

% LaTeX às vezes coloca notas de rodapé logo após o final do texto da
% página ao invés de no final da página; este pacote evita isso.
\usepackage[bottom]{footmisc}

% Se uma página está vazia, não imprime número de página ou cabeçalho
\usepackage{emptypage}

% Espaçamento entre linhas configurável (\singlespacing, \onehalfspacing etc.)
\usepackage{setspace}

% Carrega nomes de cores disponíveis (podem ser usados com hyperref e listings)
\usepackage[hyperref,svgnames,x11names,table]{xcolor}

% LaTeX define os comandos "MakeUppercase" e "MakeLowercase", mas eles têm
% algumas limitações; esta package define os comandos MakeTextUppercase e
% MakeTextLowercase que resolvem isso.
\usepackage{textcase}

% Normalmente, LaTeX faz o final da página terminar sempre no mesmo lugar
% (exceto no final dos capítulos). Esse padrão pode ser ativado explicitamente
% com o comando "\flushbottom". Mas se, por alguma razão, o volume de texto na
% página é "pequeno", essa página vai ter espaços verticais artificialmente
% grandes. Uma solução para esse problema é modificar o padrão para
% "\raggedbottom"; isso permite que as páginas terminem em lugares diferentes.
% Outra opção é corrigir manualmente cada página problemática, por exemplo
% com o comando "\enlargethispage".
%\raggedbottom

% Por padrão, LaTeX coloca uma espaço aumentado após sinais de pontuação;
% Isso não é tão bom quanto alguns TeX-eiros defendem :) .
% Esta opção desabilita isso e, consequentemente, evita problemas com
% "id est" (i.e.) e "exempli gratia" (e.g.)
\frenchspacing

% Trechos de texto "puro" (tabs, quebras de linha etc. não são modificados)
\usepackage{verbatim}

% LaTeX procura por arquivos adicionais no diretório atual e nos diretórios
% padrão do sistema. Assim, é preciso usar caminhos relativos para incluir
% arquivos de subdiretórios: "\input{diretorio/arquivo}". No entanto, há
% duas limitações:
%
% 1. É necessário dizer "\input{diretorio/arquivo} mesmo quando o arquivo
%    que contém esse comando já está dentro do subdiretório.
%
% 2. Isso não deve ser usado para packages ("\usepackage{diretorio/package}"),
%    embora na prática funcione.
%
% O modo recomendado de resolver esses problemas é modificando o arquivo
% texmf.cnf ou a variável de ambiente TEXINPUTS ou colocando os arquivos
% compartilhados na árvore TEXMF (geralmente, no diretório texmf dentro do
% diretório do usuário), o que é um tanto complicado para usuários menos
% experientes.
%
% O primeiro problema pode ser solucionado também com a package import,
% mas não há muita vantagem pois é preciso usar outro comando no lugar de
% "\input". O segundo problema é mais importante, pois torna muito difícil
% colocar packages adicionais em um diretório separado. Para contorná-lo,
% vamos usar um truque que é suficiente para nossa necessidade, embora
% *não* seja normalmente recomendado.
%\usepackage{import}

\newcommand\dowithsubdir[2]{
    \csletcs{@oldinput@path}{input@path}
    \csappto{input@path}{{#1}}
    #2
    \csletcs{input@path}{@oldinput@path}
}
