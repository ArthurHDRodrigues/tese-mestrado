%%%%%%%%%%%%%%%%%%%%%%%%%%%%%%%%%%%%%%%%%%%%%%%%%%%%%%%%%%%%%%%%%%%%%%%%%%%%%%%%
%%%%%%%%%%%%%%%%%%%%%%% CONFIGURAÇÕES E PACOTES BÁSICOS %%%%%%%%%%%%%%%%%%%%%%%%
%%%%%%%%%%%%%%%%%%%%%%%%%%%%%%%%%%%%%%%%%%%%%%%%%%%%%%%%%%%%%%%%%%%%%%%%%%%%%%%%

% Vários comandos auxiliares para o desenvolvimento de packages e classes;
% aqui, usamos em alguns comandos de formatação e condicionais.
\usepackage{etoolbox}
\usepackage{xstring}
\usepackage{expl3}
\usepackage{xparse}
\usepackage{letltxmacro}
\usepackage{regexpatch}
% Algumas packages dependem de xpatch e tentam carregá-la, causando conflitos
% com regexpatch. Como regexpatch oferece todos os recursos de xpatch (ela
% é uma versão estendida de xpatch, mas ainda considerada experimental), vamos
% fazê-las acreditar que xpatch já foi carregada.
\expandafter\xdef\csname ver@xpatch.sty\endcsname{2012/10/02}

% Arithmetic expressions in \set{length,counter} & \addto{length,counter};
% commands \widthof, \heightof, \depthof, \totalheightof, \settototalheight
\usepackage{calc}

% Sempre que possível, é melhor usar os recursos de etoolbox ao invés de
% ifthen; no entanto, várias packages dependem dela.
%\usepackage{ifthen}

% Esta não está em uso mas pode ser útil.
%\usepackage{ltxcmds}

%\usepackage{xfp} % Floating-point calculations


%%%%%%%%%%%%%%%%%%%%%%%%%%%%%%%%%%%%%%%%%%%%%%%%%%%%%%%%%%%%%%%%%%%%%%%%%%%%%%%%
%%%%%%%%%%%%%%%%%%%%%%%%%%%%% PORTUGUÊS E INGLÊS %%%%%%%%%%%%%%%%%%%%%%%%%%%%%%%
%%%%%%%%%%%%%%%%%%%%%%%%%%%%%%%%%%%%%%%%%%%%%%%%%%%%%%%%%%%%%%%%%%%%%%%%%%%%%%%%

\makeatletter
\ExplSyntaxOn

% We need to have at least some variant of Portuguese and of English
% loaded to generate the abstract/resumo, palavras-chave/keywords etc.
% We will make sure that both languages are present in the class options
% list by adding them if needed. With this, these options become global
% and therefore are seen by all packages (among them, babel).
%
% babel traditionally uses "portuguese", "brazilian", "portuges", or
% "brazil" to support the Portuguese language, using .ldf files. babel
% is also in the process of implementing a new scheme, using .ini
% files, based on the concept of "locales" instead of "languages". This
% mechanism uses the names "portuguese-portugal", "portuguese-brazil",
% "portuguese-pt", "portuguese-br", "portuguese", "brazilian", "pt",
% "pt-PT", and "pt-BR" (i.e., neither "portuges" nor "brazil"). To avoid
% compatibility problems, let's stick with "brazilian" or "portuguese"
% by substituting portugese and brazil if necessary.

\NewDocumentCommand\@IMEportugueseAndEnglish{m}{

  \seq_gclear_new:N \l_tmpa_seq
  \seq_gclear_new:N \l_tmpb_seq
  \seq_gset_from_clist:Nc \l_tmpa_seq {#1}

  \seq_map_inline:Nn \l_tmpa_seq{
    \def\@tempa{##1}
    \ifstrequal{portuges}{##1}
      {
        \GenericInfo{sbc2019}{}{Substituting~language~portuges~->~portuguese}
        \def\@tempa{portuguese}
      }
      {}
    \ifstrequal{brazil}{##1}
      {
        \GenericInfo{}{Substituting~language~brazil~->~brazilian}
        \def\@tempa{brazilian}
      }
      {}
    \seq_gput_right:NV \l_tmpb_seq {\@tempa}
  }

  % Remove the leftmost duplicates (default is to remove the rightmost ones).
  % Necessary in case the user did "portuges,portuguese" etc. and we substitute
  % portuges -> portuguese, because having the exact same language twice may
  % mess up the main language selection.
  \seq_greverse:N \l_tmpb_seq
  \seq_gremove_duplicates:N \l_tmpb_seq
  \seq_greverse:N \l_tmpb_seq

  % If the user failed to select some variation of English and Portuguese,
  % we add them here. We also remember which ones of portuguese/brazilian,
  % english/american/british etc. were selected.
  \exp_args:Nnx \regex_extract_all:nnNTF
    {\b(portuguese|brazilian)\b}
    {\seq_use:Nn \l_tmpb_seq {,}}
    \l_tmpa_tl
    {
      \tl_reverse:N \l_tmpa_tl
      \xdef\@IMEpt{\tl_head:N \l_tmpa_tl}
    }
    {
      \seq_gput_left:Nn \l_tmpb_seq {brazilian}
      \gdef\@IMEpt{brazilian}
    }

  \exp_args:Nnx \regex_extract_all:nnNTF
    {\b(english|american|USenglish|canadian|british|UKenglish|australian|newzealand)\b}
    {\seq_use:Nn \l_tmpb_seq {,}}
    \l_tmpa_tl
    {
      \tl_reverse:N \l_tmpa_tl
      \xdef\@IMEen{\tl_head:N \l_tmpa_tl}
    }
    {
      \seq_gput_left:Nn \l_tmpb_seq {english}
      \gdef\@IMEen{english}
    }

  \exp_args:Nc \xdef {#1} {\seq_use:Nn \l_tmpb_seq {,}}
}


% https://tex.stackexchange.com/a/43541/217608
% This message is part of a larger thread that discusses some
% limitations of this method, but it is enough for us here.
\def\@getcl@ss#1.cls#2\relax{\def\@currentclass{#1}}
\def\@getclass{\expandafter\@getcl@ss\@filelist\relax}
\@getclass

% The three class option lists we need to update: \@unusedoptionlist,
% \@classoptionslist and one of \opt@book.cls, \opt@article.cls etc.
% according to the current class. Note that beamer.cls (and maybe
% others) does not use \@unusedoptionlist; with it, we incorrectly
% add "english,brazilian" to \@unusedoptionlist, but that does not
% cause problems.
\@IMEportugueseAndEnglish{@unusedoptionlist}
\@IMEportugueseAndEnglish{@classoptionslist}
\@IMEportugueseAndEnglish{opt@\@currentclass .cls}

\ExplSyntaxOff
\makeatother

% Esta package permite detectar XeTeX, LuaTeX e pdfTeX, mas pode não estar
% disponível em todas as instalações de TeX.
%\usepackage{iftex}
% Por conta disso, usaremos estas (que não detectam pdfTeX):
\usepackage{ifxetex}
\usepackage{ifluatex}

\newbool{unicodeengine}
\ifboolexpr{bool{xetex} or bool{luatex}}
  {\booltrue{unicodeengine}}
  {\boolfalse{unicodeengine}}

% Detecta se estamos produzindo um arquivo PDF ou DVI (lembrando que tanto
% pdfTeX quanto LuaTeX podem gerar ambos)
\usepackage{ifpdf}

% Algumas packages "padrão" da AMS, que são praticamente  obrigatórias.
% Algumas delas devem ser carregadas antes de unicode-math ou das
% definições das fontes do documento.
\usepackage{amssymb}
\usepackage{amsthm}
\usepackage{amsmath}
\usepackage{mathtools}

% "fontenc" é um parâmetro do NFSS (sistema de gestão de fontes do
% LaTeX; consulte "texdoc fntguide" e "texdoc fontenc"). O default
% é OT1, mas ele tem algumas limitações; a mais importante é que,
% com ele, palavras acentuadas não podem ser hifenizadas. Por
% conta disso, quase todos os documentos LaTeX utilizam o fontenc
% T1. A escolha do fontenc tem consequências para as fontes que
% podem ser usadas com NFSS; hoje em dia T1 tem mais opções de
% qualidade, então não se perde nada em usá-lo. A package fontspec
% (para gestão de fontes usando outro mecanismo, compatível apenas
% com lualatex e xelatex) carrega fontenc automaticamente, mas
% usando outra codificação ("TU" e não "T1"). Ainda assim, é útil
% carregar o fontenc T1 (antes de carregar fontspec!) para o caso
% de alguma fonte "antiga" ser utilizada no documento (embora isso
% não seja recomendado: lualatex e xelatex só são capazes de
% hifenizar palavras acentuadas com o fontenc TU).
\usepackage[T1]{fontenc}

\ifunicodeengine
  % Não é preciso carregar inputenc com LuaTeX e XeTeX, pois
  % com eles utf8 é obrigatório.
  \usepackage{fontspec}

  % Ao invés de usar o sistema tradicional de LaTeX para gerir
  % as fontes matemáticas, utiliza as extensões matemáticas do
  % formato otf definidas pela microsoft. Ao ativar esta package
  % o mecanismo tradicional não funciona mais! Há poucas fontes
  % com suporte a unicode-math.
  \usepackage{unicode-math}
\else
  % O texto está escrito em utf8.
  \usepackage[utf8]{inputenc}

  % Permitem utilizar small caps + itálico (e outras pequenas
  % melhorias). Em geral, desnecessário com fontspec, a menos
  % que alguma package utilize especificamente. Algumas raras
  % packages de fontes podem causar conflitos com fontaxes, em
  % geral por utilizarem a package "concorrente" nfssext-cfr.
  \usepackage{fontaxes}
  \usepackage{mweights}
\fi

% Acesso a símbolos adicionais, como \textrightarrow, \texteuro etc.,
% disponíveis na maioria das fontes através do fontenc TS1 ou mudando
% momentaneamente para computer modern/latin modern. Raramente útil
% com lualatex/xelatex, mas não causa problemas. Várias packages de
% fontes carregam textcomp, às vezes com opções específicas; assim,
% para evitar problemas, vamos carregá-la no final do preâmbulo para
% o caso de ela não ter sido carregada antes.
\AtBeginDocument{\usepackage{textcomp}}

% Internacionalização dos nomes das seções ("chapter" X "capítulo" etc.),
% hifenização e outras convenções tipográficas. babel deve ser um dos
% primeiros pacotes carregados. É possível passar a língua do documento
% como parâmetro aqui, mas já fizemos isso ao carregar a classe, no início
% do documento.
\usepackage{babel}
\usepackage{iflang}

% É possível personalizar as palavras-chave que babel utiliza, por exemplo:
%\addto\extrasbrazilian{\renewcommand{\chaptername}{Chap.}}
% Com BibTeX, isso vale também para a bibliografia; com BibLaTeX, é melhor
% usar o comando "DefineBibliographyStrings".

% Para línguas baseadas no alfabeto latino, como o inglês e o português,
% o pacote babel funciona muito bem, mas com outros alfabetos ele às vezes
% falha. Por conta disso, o pacote polyglossia foi criado para substituí-lo.
% polyglossia só funciona com LuaTeX e XeTeX; como babel também funciona com
% esses sistemas, provavelmente não há razão para usar polyglossia, mas é
% possível que no futuro esse pacote se torne o padrão.
%\usepackage{polyglossia}
%\setdefaultlanguage{brazilian}
%\setotherlanguage{english}

% Alguns pacotes (espeficicamente, tikz) usam, além de babel, este pacote
% como auxiliar para a tradução de palavras-chave, como os meses do ano.
\usepackage{translator}

% microajustes no tamanho das letras, espaçamento etc. para melhorar
% a qualidade visual do resultado. LaTeX tradicional não dá suporte a
% nenhum tipo de microajuste; pdfLaTeX dá suporte a todos. LuaLaTeX
% e XeLaTeX dão suporte a alguns:
%
% * expansion não funciona com XeLaTeX
% * tracking não funciona com XeLaTeX; é possível obter o mesmo resultado
%   com a opção "LetterSpace" do pacote fontspec, mas a configuração é
%   totalmente manual. Por padrão, aumenta o afastamento entre caracteres
%   nas fontes "small caps"; o resultado não se presta ao uso na
%   bibliografia ou citações, então melhor desabilitar.
% * kerning e spacing só funcionam com pdfLaTex; ambas são funções
%   consideradas experimentais e nem sempre produzem resultados vantajosos.

\newcommand\microtypeopts{
  protrusion=true,
  tracking=false,
  kerning=false,
  spacing=false
}

% TeXLive 2018 inclui a versão 2.7a da package microtype e a versão
% 1.07 de luatex. Essa combinação faz aparecer um bug:
% https://tex.stackexchange.com/questions/476740/microtype-error-with-lualatex-attempt-to-call-field-warning-a-nil-value
% Aqui, aplicamos a solução sugerida, que não tem "contra-indicações".
\ifluatex
  \usepackage{luatexbase}
\fi

\ifxetex
  \usepackage[expansion=false,\microtypeopts]{microtype}
\else
  \usepackage[expansion=true,\microtypeopts]{microtype}
\fi

% Alguns "truques" (sujos?) para minimizar over/underfull boxes.
%
% Para fazer um texto justificado, é preciso modificar o tamanho dos espaços
% em cada linha para mais ou para menos em relação ao seu tamanho ideal. Para
% escolher as quebras de linha, TeX vai percorrendo o texto procurando lugares
% possíveis para quebrar as linhas considerando essa flexibilidade mas dentro
% de um certo limite mínimo/máximo. Nesse processo, ele associa a cada possível
% linha o valor *badness*, que é o nível de distorção do tamanho dos espaços
% daquela linha em relação ao ideal, e ignora opções que tenham badness muito
% grande (esse limite é dado por \tolerance). Depois de encontradas todas
% as possíveis quebras de linha e a badness de cada uma, TeX calcula as
% *penalties* das quebras encontradas, que são uma medida de quebras "ruins".
% Por exemplo, na configuração padrão, quebrar uma linha hifenizando uma
% palavra gera uma penalty de 50; já uma quebra que faça a última linha
% do parágrafo ficar sozinha na página seguinte gera uma penalty de 150.
% Finalmente, TeX calcula a "feiúra" de cada possível linha (demerits)
% com base na badness e nas penalties e escolhe a solução que minimiza os
% demerits totais do parágrafo. Os comandos \linebreak e \pagebreak funcionam
% simplesmente acrescentando uma penalty negativa ao lugar desejado para a
% quebra.
%
% Para cada fonte, o espaço em TeX tem um tamanho ideal, um tamanho mínimo e um
% tamanho máximo. TeX nunca reduz um espaço para menos que o mínimo da fonte,
% mas pode aumentá-lo para mais que o máximo. Se os espaços de uma linha ficam
% com o tamanho ideal, a badness da linha é 0; se o tamanho é
% reduzido/aumentado 50% do mínimo/máximo, a badness da linha é 12; se o
% tamanho é reduzido/aumentado para o mínimo/máximo, a badness é 100. Se esse
% aumento for de 30% além do máximo, a badness da linha é 200; se for de 45%
% além do máximo, a badness é 300; se for de 60% além do máximo, a badness é
% 400; se for de 100% além do máximo, a badness é 800. O valor máximo possível
% para badness é 10.000, que significa "badness infinita".
%
% \tolerance indica a badness máxima que TeX aceita para uma linha; seu valor
% default é 200. Assim, aumentar para, digamos, 300 ou 400, permite que
% TeX escolha parágrafos com maior variação no espaçamento entre as linhas.
% No entanto, no cálculo de demerits, a badness e as penalties de cada linha
% são elevadas ao quadrado, então TeX geralmente prefere escolher outras
% opções no lugar de uma linha ruim. Por exemplo, órfãs/viúvas têm demerit
% de 22.500 e dois hífens seguidos têm demerit de 10.000; já uma linha com
% badness 400 tem demerit 160.000. Portanto, não é surpreendente que a maioria
% dos parágrafos tenha demerits abaixo de 40.000, quase todos abaixo de 100.000
% e praticamente nenhum acima de 1.000.000. Isso significa que, para a grande
% maioria dos parágrafos, aumentar \tolerance não faz diferença: uma linha com
% badness 400 nunca será efetivamente escolhida se houver qualquer outra opção
% com badness menor. Também fica claro que não há muita diferença real entre
% definir \tolerance como 800 ou 9.999.
%
% O problema muda de figura se TeX não consegue encontrar uma solução. Isso
% pode acontecer em dois casos: (1) o parágrafo tem ao menos uma linha que não
% pode ser quebrada com badness < 10.000 ou (2) o parágrafo tem ao menos uma
% linha que não pode ser quebrada com badness < tolerance (mas essa badness é
% menor que 10.000).
%
% No primeiro caso, se houver várias possibilidades de linhas que não podem ser
% quebradas, TeX não vai ser capaz de compará-las e escolher a melhor: todas
% têm a badness máxima (10.000) e, portanto, a que gerar menos deméritos no
% restante do parágrafo será a escolhida. Na realidade, no entanto, essas
% linhas *não* são igualmente ruins entre si, o que pode levar TeX a fazer uma
% má escolha. Para evitar isso, TeX tenta novamente aplicando
% \emergencystretch, que "faz de conta" que o tamanho máximo ideal dos espaços
% da linha é maior que o definido na fonte. Isso reduz a badness de todas as
% linhas, o que soa parecido com aumentar \tolerance. Há três diferenças, no
% entanto: (1) essa mudança só afeta os parágrafos que falharam; (2) soluções
% que originalmente teriam badness = 10.000 (e, portanto, seriam vistas como
% equivalentes) podem ser avaliadas e comparadas entre si; e (3) como a badness
% de todas as linhas diminui, a possibilidade de outras linhas que
% originalmente tinham badness alta serem escolhidas aumenta. Esse último ponto
% significa que \emergencystretch pode fazer TeX escolher linhas mais
% espaçadas, fazendo o espaçamento do parágrafo inteiro aumentar e, portanto,
% tornando o resultado mais homogêneo mesmo com uma linha particularmente ruim.
%
% É esse último ponto que justifica o uso de \emergencystretch no segundo caso
% também: apenas aumentar a tolerância, nesse caso, poderia levar TeX a
% diagramar uma linha ruim em meio a um parágrafo bom, enquanto
% \emergencystretch pode fazer TeX aumentar o espaçamento de maneira geral no
% parágrafo, minimizando o contraste da linha problemática com as demais.
% Colocando a questão de outra maneira, aumentar \tolerance para lidar com
% esses parágrafos problemáticos pode fazê-los ter uma linha especialmente
% ruim, enquanto \emergencystretch pode dividir o erro entre várias linhas.
% Assim, definir \tolerance em torno de 800 parece razoável: no caso geral,
% não há diferença e, se um desses casos difíceis não pode ser resolvido com
% uma linha de badness até 800, \emergencystretch deve ser capaz de gerar um
% resultado igual ou melhor.
%
% Penalties & demerits: https://tex.stackexchange.com/a/51264
% Definições (fussy, sloppy etc.): https://tex.stackexchange.com/a/241355
% Mais definições (hfuzz, hbadness etc.): https://tex.stackexchange.com/a/50850
% Donald Arseneau defendendo o uso de \sloppy: https://groups.google.com/d/msg/comp.text.tex/Dhf0xxuQ66E/QTZ7aLYrdQUJ
% Artigo detalhado sobre \emergencystretch: https://www.tug.org/TUGboat/tb38-1/tb118wermuth.pdf
% Esse artigo me leva a crer que algo em torno de 1.5em é suficiente

\tolerance=800
\hyphenpenalty=100 % Default 50; se o texto é em 2 colunas, 50 é melhor
\setlength{\emergencystretch}{1.5em}

% Não gera warnings para Overfull menor que 1pt
\hfuzz=1pt
\vfuzz\hfuzz

% Não gera warnings para Underfull com badness < 1000
\hbadness=1000
\vbadness=1000

% Por padrão, o algoritmo LaTeX para textos não-justificados é (muito) ruim;
% este pacote implementa um algoritmo bem melhor
\usepackage[newcommands]{ragged2e}

% ragged2e funciona porque permite que LaTeX hifenize palavras em textos
% não-justificados quando necessário. No caso de textos centralizados,
% no entanto, isso em geral não é desejável. Assim, newcommands não é
% interessante para \centering e \begin{center}. newcommands também
% causa problemas com legendas se o float correspondente usa \centering
% (o que é muito comum). Assim, vamos voltar \centering e \begin{center}
% à definição padrão.
\let\centering\LaTeXcentering
\let\center\LaTeXcenter

% Com ragged2e e a opção "newcommands", textos curtos não-justificados
% podem gerar warnings sobre "underfull \hbox". Não há razão para pensar
% muito nesses warnings, então melhor desabilitá-los.
% https://tex.stackexchange.com/questions/17659/ragged2e-newcommands-option-produces-underfull-hbox-warnings
\makeatletter
\g@addto@macro{\raggedright}{\hbadness=\@M}
\g@addto@macro{\RaggedRight}{\hbadness=\@M}
\g@addto@macro{\raggedleft}{\hbadness=\@M}
\g@addto@macro{\RaggedLeft}{\hbadness=\@M}
\g@addto@macro{\flushleft}{\hbadness=\@M}
\g@addto@macro{\FlushLeft}{\hbadness=\@M}
\g@addto@macro{\flushright}{\hbadness=\@M}
\g@addto@macro{\FlushRight}{\hbadness=\@M}
\makeatother

% Espaçamento entre linhas configurável (\singlespacing, \onehalfspacing etc.)
\usepackage{setspace}

% LaTeX às vezes coloca notas de rodapé logo após o final do texto da
% página ao invés de no final da página; este pacote evita isso e faz
% notas de rodapé funcionarem corretamente em títulos de seções.
% Esta package deve ser carregada depois de setspace.
\usepackage[stable,bottom]{footmisc}

% Se uma página está vazia, não imprime número de página ou cabeçalho
\usepackage{emptypage}

% hyperref deve preferencialmente ser carregada próximo ao final
% do preâmbulo mas, para o caso de alguma package forçar a sua
% carga antes de executarmos \usepackage explicitamente, vamos
% garantir que estas opções estejam ativas.
\PassOptionsToPackage{
  unicode=true,
  pdfencoding=unicode,
  plainpages=false,
  pdfpagelabels,
  bookmarksopen=true,
  breaklinks=true,
  %hyperfootnotes=false, % polui desnecessariamente com bordercolor
}{hyperref}

% Carrega nomes de cores disponíveis (podem ser usados com hyperref e listings)
\usepackage[hyperref,svgnames,x11names,table]{xcolor}

% LaTeX define os comandos "MakeUppercase" e "MakeLowercase", mas eles têm
% algumas limitações; esta package define os comandos MakeTextUppercase e
% MakeTextLowercase que resolvem isso.
\usepackage{textcase}

% Em documentos frente-e-verso, LaTeX faz o final da página terminar sempre
% no mesmo lugar (exceto no final dos capítulos). Esse comportamento pode ser
% ativado explicitamente com o comando "\flushbottom". Mas se, por alguma
% razão, o volume de texto na página é "pequeno", essa página vai ter espaços
% verticais artificialmente grandes. Uma solução para esse problema é utilizar
% "\raggedbottom" (padrão em documentos que não são frente-e-verso): com essa
% opção, as páginas podem terminar em alturas ligeiramente diferentes. Outra
% opção é corrigir manualmente cada página problemática, por exemplo com o
% comando "\enlargethispage".
%\raggedbottom
\flushbottom

% Por padrão, LaTeX coloca uma espaço aumentado após sinais de pontuação;
% Isso não é tão bom quanto alguns TeX-eiros defendem :) .
% Esta opção desabilita isso e, consequentemente, evita problemas com
% "id est" (i.e.) e "exempli gratia" (e.g.)
\frenchspacing

% Trechos de texto "puro" (tabs, quebras de linha etc. não são modificados)
\usepackage{verbatim}

% LaTeX procura por arquivos adicionais no diretório atual e nos diretórios
% padrão do sistema. Assim, é preciso usar caminhos relativos para incluir
% arquivos de subdiretórios: "\input{diretorio/arquivo}". No entanto, há
% duas limitações:
%
% 1. É necessário dizer "\input{diretorio/arquivo} mesmo quando o arquivo
%    que contém esse comando já está dentro do subdiretório.
%
% 2. Isso não deve ser usado para packages ("\usepackage{diretorio/package}"),
%    embora na prática funcione.
%
% O modo recomendado de resolver esses problemas é modificando o arquivo
% texmf.cnf ou a variável de ambiente TEXINPUTS ou colocando os arquivos
% compartilhados na árvore TEXMF (geralmente, no diretório texmf dentro do
% diretório do usuário), o que é um tanto complicado para usuários menos
% experientes.
%
% O primeiro problema pode ser solucionado também com a package import,
% mas não há muita vantagem pois é preciso usar outro comando no lugar de
% "\input". O segundo problema é mais importante, pois torna muito difícil
% colocar packages adicionais em um diretório separado. Para contorná-lo,
% vamos usar um truque que é suficiente para nossa necessidade, embora
% *não* seja normalmente recomendado.
%\usepackage{import}

\newcommand\dowithsubdir[2]{
    \csletcs{@oldinput@path}{input@path}
    \csappto{input@path}{{#1}}
    #2
    \csletcs{input@path}{@oldinput@path}
}

% Às vezes é necessário forçar quebras de linha no título para a capa, mas
% essas quebras não devem aparecer em outros lugares. Estas macros "apagam"
% esses comandos; basta usá-las em um grupo para que a mudança não seja
% permanente.

\makeatletter
\ExplSyntaxOn

% Primeiro definimos a regex que vamos usar. O que estamos fazendo aqui é
% "(regex1)|(regex2)|(regex3)|(regex4)"
\regex_const:Nn \@IMEbreakRegex
  {
    ( \c{break} ) % Encontra "\break"
    |
    ( \c{newline} ) % Encontra "\newline"
    |
      % Isto encontra:
      % 1. A control sequence "\linebreak" -> "\c{linebreak}"
      % 2. Opcionalmente seguida de "[NÚMERO]" -> "( \s*\x{5B}\s*\d\s*\x{5D} )?"
    ( \c{linebreak} (\s*\x{5B}\s*\d\s*\x{5D})? )
    |
      % Isto encontra:
      % 1. A control sequence "\\" (quebra de linha) -> "\c{\\}"
      % 2. Optionalmente seguida de "[...]" -> "( \s*\x{5B}\s*...\s*\x{5D} )?"
      % 3. Onde "..." é qualquer dimensão TeX válida
      %    (veja o exemplo de regex em texdoc interface3)
    ( \c{\\}
      ( \s*\x{5B}\s*
          [\+\-]?(\d+|\d*\.\d+)\s*((?i)pt|in|[cem]m|ex|[bs]p|[dn]d|[pcn]c)
        \s*\x{5D}
      )?
    )
  }

\newcommand{\@IMEremoveLinebreaksEtc}[1]{
    % Primeiro, substitui todas as quebras de linha por espaços
    \regex_replace_all:NnN \@IMEbreakRegex {\ } #1

    % Depois elimina eventuais notas de rodapé
    \regex_replace_all:nnN {\s*\c{footnote}} {\c{@gobble}} #1

    % Depois transforma URLs em texto simples
    \regex_replace_all:nnN {\c{url}} {\c{@firstofone}} #1

    % Depois elimina espaços repetidos
    \regex_replace_all:nnN {\s+} {\ } #1
}

\ExplSyntaxOff
\makeatother
