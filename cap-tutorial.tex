\chapter{Do zero ao mínimo com \LaTeX{}}

Preparar um texto para impressão envolve duas coisas:

\begin{description}
\item[Escrever:] digitar, recortar/colar trechos, revisar etc.
\item[Formatar:] definir o tamanho da fonte, o
espaçamento entre parágrafos etc.
\end{description}

Hoje é comum fazer essas duas coisas ao mesmo tempo, graças à visualização
imediata que o computador oferece. No entanto, imagine como era o processo de
produção de um livro nos anos 1970: o autor escrevia seu texto em uma máquina
de escrever e enviava esse material para o editor, que era responsável pela
tarefa de formatá-lo para impressão. O autor muitas vezes inseria anotações
para o editor explicando coisas como ``este parágrafo é uma citação'', e o
editor criava algum mecanismo visual para representar isso.

Não é de se surpreender que, com o surgimento do microcomputador, os primeiros
programas para criação de textos seguissem um funcionamento similar: o autor
digitava e editava seu texto sem formatá-lo visualmente, apenas inserindo
alguns comandos correspondentes a aspectos da formatação que ele depois
revisava na versão impressa. \LaTeX{} é uma ferramenta baseada nesse processo:
você prepara seu texto no editor de sua preferência, insere comandos no texto
que indicam a estrutura do documento e o processa com o \LaTeX{}, que gera um
arquivo PDF formatado. Embora seja um estilo ``antigo'' de trabalhar, ele é
muito eficiente em vários casos. Ou seja, dependendo da situação, pode ser
mais adequado trabalhar fazendo tudo ao mesmo tempo ou dividindo o trabalho
nessas duas fases. De maneira geral:

\begin{itemize}
\item Se você precisa criar páginas diferentes entre si com \emph{layout}
definido manualmente, é melhor usar uma ferramenta que permita trabalhar
visualmente, como LibreOffice Writer, MS-Word, Google Docs etc.;

\item Se você precisa fazer um documento relativamente longo com estrutura
regular (capítulos, seções etc.), é melhor usar ferramentas que formalizam
essa estrutura (como \LaTeX{}) ao invés de usar ferramentas visuais;

\item Se você precisa fazer um documento envolvendo referências cruzadas,
bibliografia relativamente extensa ou fórmulas matemáticas, é difícil
encontrar outra ferramenta tão eficiente quanto \LaTeX{};

\item Se você precisa criar um documento simples, ambas as abordagens
funcionam bem; cada um escolhe esta ou aquela em função da familiaridade
com as ferramentas;

\item Se você quer que a qualidade tipográfica do resultado seja realmente
excelente, é necessário usar uma ferramenta profissional, como \LaTeX{},
Scribus, Adobe InDesign ou outras; processadores de texto convencionais não
oferecem o mesmo nível de qualidade dessas ferramentas.
\end{itemize}

\section{Visão Geral}

Com \LaTeX{}, você prepara o texto (incluindo as indicações de estrutura) em
um editor de textos qualquer, salva como arquivo de texto puro (``.txt'',
mas é comum usar a extensão ``.tex'' ao invés de ``.txt'') e processa esse
arquivo com o comando ``pdflatex'' (``compila'' o documento) para obter o
PDF correspondente. Qualquer editor capaz de salvar arquivos em formato
texto puro, como o bloco de notas do windows, vim, emacs etc. pode ser usado.
Programas como LibreOffice Writer, MS-Word etc. também funcionam, mas
possivelmente vão gerar dores de cabeça porque vão tentar formatar algumas
coisas automaticamente (e de maneira incompatível com \LaTeX{}). Se você
preferir, existem editores projetados especificamente para trabalhar com
\LaTeX{}; eles em geral utilizam cores para distinguir o texto dos comandos
de formatação, automatizam o processo de compilação do documento e oferecem
outras comodidades. Os mais comumente usados são TeXmaker, TeXstudio e
TeXworks; os três são software livre e funcionam em Windows, MacOS e Linux.
TeXnicCenter é outra opção livre, mas funciona apenas em Windows. Você também
pode querer usar editores \emph{online}, como overleaf (\url{www.overleaf.com})
e sharelatex (\url{www.sharelatex.com}).

\LaTeX{} ignora quebras de linha e trata sequências de vários espaços como
se fossem apenas um. Isso significa que você pode usar quebras de linha e
espaços no texto que está digitando como ``dicas visuais'' da estrutura do
texto. É muito comum fazer isso com listas de itens, por exemplo. Uma ou
mais linhas em branco sinalizam o fim de um parágrafo e o início de outro.
O caractere ``\%'' indica que o restante da linha é um comentário,
ou seja, um trecho de texto que não tem nenhum efeito sobre o resultado final
do documento. Comentários podem ser usados como lembrete sobre alguma decisão,
para indicar um parágrafo que ainda precisa de revisão etc. Por conta desse
significado especial, para inserir um caractere \% ``normal'' no
texto é preciso digitar ``\textsf{\textbackslash\%}''.

Um documento \LaTeX{} é dividido em duas partes: o \emph{preâmbulo}, onde
você coloca comandos de configuração para o documento, e o \emph{corpo} do
documento em si, que contém o texto propriamente dito. O preâmbulo é onde
você define as características do resultado tipográfico esperado: tipo e
tamanho da fonte a usar, posição dos títulos e subtítulos na página etc.
Na prática, definir todas as configurações de impressão desejadas em
\LaTeX{} não é simples. Por conta disso, \LaTeX{} tem alguns modelos
prontos, que você escolhe com o comando
\textsf{\textbackslash{}documentclass\{modelo\}} no preâmbulo. Os
principais modelos são \textsf{book}, \textsf{report} e \textsf{article};
você pode saber mais sobre eles (e outros) em qualquer texto introdutório
sobre \LaTeX{} na Internet. \textsf{book} e \textsf{report} são os mais
adequados para a escrita de teses ou dissertações acadêmicas.

\LaTeX{} também tem \textit{packages} (``\textit{plugins}'') que acrescentam
funcionalidades ou modificam os modelos padrão e também são carregadas no
preâmbulo, com o comando \textsf{\textbackslash{}usepackage\{nome-da-package\}}.
Várias delas podem receber opções adicionais no formato
\textsf{\textbackslash{}usepackage[opção1,opção2...]\{nome-da-package\}};
a documentação de cada package detalha as opções disponíveis.

Qualquer documento \LaTeX{} utiliza várias packages, portanto é preciso
conhecê-las. Isso às vezes é trabalhoso porque algumas delas podem se
tornar obsoletas e, com isso, sítios web com ``dicas'' podem estar
desatualizados. O sítio \url{www.ctan.org} é um índice com praticamente
todas as packages disponíveis, incluindo sua documentação. Além dessas,
é comum que revistas científicas ofereçam packages que pré-definem a
formatação esperada para os artigos. Finalmente, o sítio
\url{tex.stackexchange.com} é um fórum de perguntas e respostas sobre
\LaTeX{} que é muito útil.

Usar algum documento existente como base para criar seu texto em geral é
uma boa ideia; o IME/USP oferece um modelo adequado para teses e
dissertações (\url{github.com/LSS-USP/modelo-latex}) que pode ser
adaptado para outros usos e outras instituições. Há também um modelo
(\url{www.abntex.net.br}) que procura seguir as normas da ABNT para
documentos científicos.

\section{Comandos Básicos}

Um exemplo de documento \LaTeX{} simples:

\begin{verbatim}
        % O documento começa com o preâmbulo
        % Vamos usar o modelo "book" com fonte no tamanho 11pt
        \documentclass[11pt]{book}
        % Vamos usar caracteres acentuados
        \usepackage[utf8]{inputenc}
        % Vamos escrever em português do Brasil
        \usepackage[brazil]{babel}
        % Estas linhas não imprimem nada, apenas definem
        % os valores que serão usados por "\maketitle"
        \author{Fulano de Tal}
        \title{Começando a usar o \LaTeX{}}
        % Finaliza o preâmbulo e inicia o conteúdo:
        \begin{document}
        % Cria uma página de título com os dados definidos acima
        \maketitle
        % Capítulos, seções etc. são numerados automaticamente
        \chapter{Cheguei!}
        Oi Galera!
        % É preciso sinalizar o final do documento
        \end{document}
\end{verbatim}

Esse exemplo mostra como definir o nome de um capítulo. Existem também os
comandos \textsf{\textbackslash{}section}, \textsf{\textbackslash{}subsection},
\textsf{\textbackslash{}subsubsection} e \textsf{\textbackslash{}paragraph}.
Usar o nome do comando seguido de um asterisco
(\textsf{\textbackslash{}chapter*} etc.) faz o capítulo/seção não ser incluído
no sumário (nem considerado na contagem de capítulos, seções etc.).

Para criar listas de itens, você pode fazer\footnote{Observe o uso de
espaços no início das linhas com \textsf{\textbackslash{}item} para deixar a
estrutura visualmente mais clara durante a edição.}:

\begin{verbatim}
        \begin{itemize}
            \item Primeiro item
            \item Segundo item
            \item Terceiro item
        \end{itemize}
\end{verbatim}

Além de ``itemize'', há também ``enumerate'' (auto-explicativo) e ``description'':

\begin{verbatim}
        \begin{description}
            \item[O primeiro item] é o primeiro;
            \item[O segundo item] é o segundo;
            \item[O terceiro item] é o terceiro.
        \end{description}
\end{verbatim}

Citações curtas normalmente são incluídas no fluxo normal do texto e colocadas
entre aspas; para citações mais longas, use \textsf{\textbackslash{}begin\{quote\}}
ou \textsf{\textbackslash{}begin\{quotation\}} (este último é mais adequado
para citações com vários parágrafos). Para poesia, use \textsf{verse}. A
package \textsf{csquotes} acrescenta recursos sofisticados para citações.

Para inserir uma nota de rodapé, use o comando
\textsf{\textbackslash{}footnote\{texto da nota\}}\index{Notas de rodapé}.
Um espaço não-separável é indicado pelo caractere til (``\textasciitilde{}'')
e é possível forçar uma quebra de linha com
``\textsf{\textbackslash{}\textbackslash{}}''. Aspas tipográficas (``'' e `')
são inseridas com
\textasciigrave\textasciigrave\textquotesingle\textquotesingle{} e
\textasciigrave\textquotesingle. Você pode consultar a lista completa de
símbolos em \url{www.ctan.org/tex-archive/info/symbols/comprehensive/symbols-a4.pdf}.
Uma outra maneira de encontrar símbolos é usar este sítio: \url{detexify.kirelabs.org/classify.html}.

\section{Referências Cruzadas e \emph{Floats}}

É comum que um trecho do texto faça referência a outro trecho (``como discutimos
no capítulo~X\ldots''). Isso é feito com o par de comandos
\textsf{\textbackslash{}label\{nome-sugestivo\}} e
\textsf{\textbackslash{}ref\{nome-sugestivo\}}.

É claro que o ideal seria que tabelas e figuras sempre aparecessem junto ao
texto a que se referem. No entanto, isso é impossível por conta da divisão
do texto em páginas. Em \LaTeX{}, figuras e tabelas são incluídas como
\emph{floats} (localização flexível) usando \textsf{\textbackslash{}begin\{figure\}}
e \textsf{\textbackslash{}begin\{table\}} e o programa procura o ``melhor''
lugar para colocá-las. Dentro do \emph{float} é inserido um
\textsf{\textbackslash{}label} para que se possa fazer referência à figura/tabela
no texto (com o comando \textsf{\textbackslash{}ref}). A figura/tabela em
si é definida com \textsf{\textbackslash{}includegraphics} ou
\textsf{\textbackslash{}begin\{tabular\}}, e em geral é uma boa ideia acrescentar
uma descrição com \textsf{\textbackslash{}caption}\index{Legendas}.

\section{Fórmulas Matemáticas}

A diagramação de fórmulas matemáticas tem regras específicas; assim, para
criar fórmulas em \LaTeX{}, é preciso usar um comando para iniciar o modo
matemático. Isso pode ser feito de duas formas:

\begin{itemize}
  \item Pequenas fórmulas no meio do texto ($E=mc^2$) são inseridas com
  \textsf{\$\textit{fórmula}\$} (e, portanto, para inserir um caractere \$
  normal no texto, é preciso usar \textsf{\textbackslash{}\$}).

  \item Fórmulas mais longas ou que devem aparecer em um parágrafo
  separado são inseridas com
  \textsf{\textbackslash{}[\textit{fórmula}\textbackslash{}]} (ou
  \textsf{\textbackslash{}begin\{displaymath\}}).
\end{itemize}

No modo matemático, letras são interpretadas como variáveis e espaços
em branco são ignorados (\LaTeX{} usa o contexto da fórmula para
definir o espaçamento). Para inserir um espaço explicitamente, use
\textsf{\textbackslash{}quad} ou \textsf{\textbackslash{}enspace}.
Para inserir texto ``normal'' em uma fórmula matemática, use
\textsf{\textbackslash{}text\{texto\}} (para texto de fato) ou
\textsf{\textbackslash{}mathit\{texto\}} (para nomes de variáveis
ou funções com mais de uma letra). Pode ser necessário deixar um
espaço no início do texto para evitar que ele fique colado com o
caractere matemático que o antecede.

Usando \textsf{\textbackslash{}begin\{equation\}}, a fórmula recebe um
número (que aparece à direita) ao qual você pode se referir no texto
usando o comando ``\textsf{\textbackslash{}ref}'' (``\textsf{conforme
vimos na equação \textbackslash{}ref\{eq:bhaskara\}\ldots}'').
\textsf{\textbackslash{}begin\{equation*\}} (incluindo o *) elimina
o número e é, portanto, equivalente a
\textsf{\textbackslash{}begin\{displaymath\}}. Há outros comandos
similares, como \textsf{align}, \textsf{multline} e \textsf{gather},
definidos e documentados na package \textsf{amsmath}, e todos têm
a variante com ``*''.

\section{Referências Bibliográficas e Bibliografia}

A geração de bibliografias no \LaTeX{} é feita através da package
biblatex\index{biblatex} e do programa auxiliar
biber\index{biber}\footnote{Antigamente, usava-se a package
natbib\index{natbib} e o comando auxiliar bibtex\index{bibtex}. O
funcionamento geral dos dois mecanismos é similar e o formato do banco de
dados de ambos é o mesmo.} e envolve três passos:

\begin{enumerate}
\item A criação de um banco de dados, no formato ``.bib'', das obras de
interesse. Esse banco de dados pode incluir obras que não vão ser de fato
referenciadas no documento final. Isso significa que você pode criar um
único banco de dados e utilizá-lo em todos seus documentos\footnote{É
comum criar bancos de dados desse tipo separados por assunto, mas isso
não é necessário.}.

\item A inserção de referências às obras ao longo do texto.

\item A escolha do estilo bibliográfico (usando as opções da package
biblatex) e uso do comando \textsf{\textbackslash{}printbibliography}.
\end{enumerate}

A fonte mais detalhada de informações sobre o banco de dados é a
documentação da package biblatex, mas o material ali é um tanto denso.
Há muito material introdutório ao formato ``.bib'' e ao bibtex disponível
\emph{online}, e você pode se inspirar em exemplos para criar seu banco de
dados bibliográfico. Além disso, ferramentas como Zotero\index{Zotero} ou
Mendeley\index{Mendeley} (o uso de uma delas é altamente recomendado!)
podem exportar para o formato .bib.

Observe que existem dois formatos comumente usados para escrever títulos
de artigos, livros etc:

\begin{description}
  \item[Title case:] Substantivos, adjetivos e verbos (além de nomes
  próprios e siglas) são escritos com a primeira letra maiúscula (``Um
  Exemplo de Título no Estilo Title Case''). Em geral, a regra não se
  aplica ao título de artigos ou capítulos de livro, apenas aos livros
  dos quais eles fazem parte;

  \item[Sentence case:] O título é escrito como qualquer outra frase
  (``Um título só tem maiúsculas em abreviaturas, como ABNT, ou nomes
  próprios'').
\end{description}

Cada estilo de bibliografia utiliza um desses formatos e, portanto, é
desejável que o banco de dados funcione corretamente com ambos. No
entanto, nem sempre é claro quais palavras devem ser iniciadas com letra
maiúscula ao usar \textit{title case} e, por conta disso, não há um sistema
automático em \LaTeX{} para adaptar títulos a ele. Sendo assim, como fazer
um banco de dados bibliográfico capaz de funcionar com os dois formatos?

A solução é sempre inserir os títulos dos itens no banco de dados seguindo
o formato \textit{title case}. Se o estilo utiliza esse formato, o título
é reproduzido na bibliografia como digitado no banco de dados. Se o estilo
usa \textit{sentence case}, o texto (exceto a primeira letra) é convertido
para letras minúsculas. Para evitar que isso afete siglas e nomes próprios,
basta colocá-los entre chaves (``Automated Application-Level Checkpointing
of \{MPI\} Programs''). Isso também é útil quando o autor é, na verdade,
uma instituição (``\{Universidade de São Paulo --- Sistema Integrado de
Bibliotecas\}'').

Com o banco de dados pronto, as referências podem ser inseridas no texto
usando diferentes comandos dependendo do caso:
\textsf{\textbackslash{}cite},
\textsf{\textbackslash{}citet},
\textsf{\textbackslash{}citep} etc. Esses comandos estão descritos tanto
na documentação da package biblatex\index{biblatex} quanto na da package
natbib\index{natbib}.

\section{Formatação Manual}

Às vezes é preciso inserir formatação de forma manual; os comandos mais
importantes são \textsf{\textbackslash{}emph} (texto \emph{enfatizado}, em geral
itálico), \textsf{\textbackslash{}texttt} (texto \texttt{teletype}, imitando um
terminal de texto ou uma impressora), \textsf{\textbackslash{}textit}
(\textit{itálico}), \textsf{\textbackslash{}textbf} (\textbf{negrito}),
\textsf{\textbackslash{}textsf} (fonte \textsf{sem serifa}),
\textsf{\textbackslash{}textsc} (texto \textsc{Small Caps} --- nem todas
as fontes oferecem essa possibilidade),
\textsf{\textbackslash{}normalsize} (tamanho normal),
\textsf{\textbackslash{}footnotesize} (tamanho reduzido),
\textsf{\textbackslash{}large} (tamanho aumentado),
\textsf{\textbackslash{}Large} (tamanho ainda maior),
\textsf{\textbackslash{}LARGE} (ainda maior) e
\textsf{\textbackslash{}Huge} (ainda maior).

\section{Detalhes}

Há quatro estilos típicos de comandos \LaTeX{}:

\begin{itemize}
\item Comandos que se referem a um parâmetro; por exemplo,
\textsf{\textbackslash{}emph\{um texto\}} significa ``escreva a frase
`um texto' com ênfase'' (em geral, itálico). As chaves delimitam o início
e o final do escopo sobre o qual o comando tem efeito. Aqui entram também
comandos como \textsf{\textbackslash{}title} e \textsf{\textbackslash{}author},
que não escrevem nada diretamente mas definem o título e autoria do documento
(essa informação é usada, por exemplo, por \textsf{\textbackslash{}maketitle}).

\item Comandos que se referem a um parâmetro que é um bloco grande de
texto, possivelmente vários parágrafos; por exemplo,
\textsf{\textbackslash{}begin\{center\} um texto
\textbackslash{}end\{center\}} faz ``um texto'' (que podem ser
vários parágrafos) ser centralizado.

\item Comandos que ativam alguma opção; por exemplo,
\textsf{\textbackslash{}itshape} significa ``ative o modo itálico''.
Nesse caso, o texto vai ser impresso em itálico até outro comando selecionar
outro estilo de fonte. Se o comando for inserido dentro de um bloco
delimitado por chaves, ele ``perde o efeito'' após o caractere de fecha-chaves
(exemplo: ``\textsf{\{\textbackslash{}itshape Fulano de Tal\} é meu nome}''
será impresso como ``\textit{Fulano de Tal} é meu nome''). Você normalmente
não vai utilizar esse estilo de comando, mas ele é útil em alguns casos.

\item Comandos que fazem o programa escrever algo específico; por exemplo,
em vários modelos padrão o comando \textsf{\textbackslash{}maketitle} gera
uma página de título com o nome do trabalho, autor etc.
\end{itemize}

Nos dois últimos, não é preciso usar chaves após o comando. Ainda assim, as
chaves podem ser colocadas e muitas vezes isso é bom: sem elas, \LaTeX{}
entende que o caractere espaço que se segue a esses comandos serve apenas
como separador em relação ao que vem a seguir. Por conta disso, ele ignora
esse espaço. Quando isso não é o que se deseja, a solução é usar as chaves:
\textsf{\textbackslash{}itshape\{\}}.

Alguns comandos aceitam mais de um parâmetro, às vezes entre chaves, às
vezes entre colchetes. Você pode descobrir a sintaxe correta para cada caso
lendo a documentação de cada comando.

\section{Versões do \LaTeX{}}

Assim como há packages para o \LaTeX{}, o próprio \LaTeX{} é, na verdade, um
conjunto de extensões para o programa \TeX{}. Assim, se você encontrar
referências a ``\TeX{}'' ou a ``plain \TeX{}'', basta saber que esse é o
sistema que funciona ``por baixo'' do \LaTeX{}.

\LaTeX{} é um sistema em evolução (desde os anos 80!). Uma das consequências
disso é que há, na verdade, quatro versões diferentes dele:

\begin{enumerate}
\item \LaTeX{} ``tradicional'', que gera arquivos em formato DVI que, por
sua vez, precisam ser convertidos para o formato PDF. Essa versão não é
capaz de usar as fontes instaladas no sistema; ela só pode usar fontes
adaptadas para uso com o \LaTeX{}. Hoje em dia não há boas razões para
usar essa versão.

\item pdf\LaTeX{}, que gera arquivos PDF e dá suporte a alguns recursos
avançados de tipografia adicionais. É a versão mais usada hoje em dia,
embora também só possa usar as fontes adaptadas para uso com o \LaTeX{}.

\item \XeLaTeX{} que, além dos recursos do pdf\LaTeX{}, opera internamente
em UTF-8 (ou seja, funciona melhor com múltiplas línguas) e pode funcionar
não só com as fontes adaptadas para o \LaTeX{} como também com as fontes
instaladas no sistema. A desvantagem desta versão é que ela é um pouco
mais lenta que pdf\LaTeX{}.

\item \LuaLaTeX{}, que oferece os mesmos recursos que o \XeLaTeX{} e
também pode ser estendido internamente com mais facilidade (através da
linguagem de programação Lua). Como \XeLaTeX{}, esta versão é um pouco
mais lenta que pdf\LaTeX{}.
\end{enumerate}

Todas essas versões são instaladas quando você instala o \LaTeX{} no seu
sistema, então trocar de uma para outra é muito fácil (basta escolher o
comando a executar: pdflatex, xelatex ou lualatex). \XeLaTeX{} e
\LuaLaTeX{} são as duas propostas da comunidade para o futuro novo padrão
do sistema, mas você não tem nada a perder se escolher a ``errada'', pois
para todos os efeitos práticos elas são equivalentes.

Se você pretende escrever apenas com línguas no alfabeto latino e não
pretende usar fontes diferentes das disponíveis por padrão no \LaTeX{},
então qualquer uma das três versões modernas (pdf\LaTeX{}, \XeLaTeX{}
e \LuaLaTeX{}) é adequada. Se você pretende usar línguas com outros
alfabetos ou se gostaria de escolher fontes diferentes, use \XeLaTeX{}
ou \LuaLaTeX{}.
