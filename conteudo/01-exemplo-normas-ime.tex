%!TeX root=../tese.tex
%("dica" para o editor de texto: este arquivo é parte de um documento maior)
% para saber mais: https://tex.stackexchange.com/q/78101

\chapter{O que o IME espera (normas)}

Fica a critério do aluno definir os aspectos relacionados à aparência da
tese, como o tamanho de fonte, margens, espaçamento, estilo de referências,
cabeçalho, etc., considerando sempre o bom senso.

A CPG, em reunião realizada em junho de 2007, aprovou que as
teses/dissertações deverão seguir o formato padrão por ela definido.
Esse padrão refere-se aos itens que devem estar presentes nas teses/dissertações
(e.g. capa, formato de rosto, sumário, etc.), e não à formatação do documento.
Ele define itens obrigatórios e opcionais, conforme segue:\index{Formatação}
\index{Tese/Dissertação!itens obrigatórios}
\index{Tese/Dissertação!itens opcionais}

\begin{itemize}
  \item \textsc{Capa} (obrigatória)
  \begin{itemize}
    \item O IME usa uma capa padrão de cartolina para todas as
    teses/dissertações. Essa capa tem uma janela recortada por onde se
    vê o título e o autor do trabalho e, portanto, a capa impressa do
    trabalho deve incluir o título e o autor na posição correspondente da
    página. Ela fica centralizada na página, tem 100mm de largura, 60mm de
    altura e começa 47mm abaixo do topo da página.

    \item O título da tese/dissertação deverá começar com letra maiúscula
    e o resto deverá ser em minúsculas, salvo nomes próprios.

    \item O nome do aluno(a) deverá ser completo e sem abreviaturas.

    \item É preciso explicitar se é uma tese ou dissertação (para
    obtenção do título de doutor, tese; para obtenção do título de
    mestre, dissertação).

    \item O nome do programa deve constar da capa (Matemática,
    Matemática Aplicada, Estatística, Ciência da Computação ou
    Mestrado Profissional em Ensino de Matemática).

    \item Também devem constar o nome completo do orientador e do
    co-orientador, se houver.

    \item Se o aluno recebeu bolsa, deve-se indicar a(s) agência(s).

    \item É preciso informar o mês e ano do depósito ou da entrega da
    versão corrigida.
  \end{itemize}

  \newpage % Às vezes, o uso da força é inevitável ;-)

  \item \textsc{Folha de rosto} (obrigatória, tanto para a versão
  depositada quanto para a versão corrigida)
  \begin{itemize}
    \item O título da tese/dissertação deverá seguir o padrão da capa.

    \item Deve informar se se trata da versão original ou da versão
    corrigida (veja mais sobre isso abaixo); no segundo caso, deve
    também incluir os nomes dos membros da banca.
  \end{itemize}

  \item \textsc{Agradecimentos} (opcional)

  \item \textsc{Resumo}, em português (obrigatório)

  \item \textsc{Abstract}, em inglês (obrigatório)

  \item \textsc{Sumário} (obrigatório)

  \item \textsc{Listas} (opcionais)
  \begin{itemize}
    \item Lista de Abreviaturas
    \item Lista de Símbolos
    \item Lista de Figuras
    \item Lista de Tabelas
  \end{itemize}

  \item \textsc{Referências} (obrigatório)

  \item \textsc{Índice Remissivo} (opcional\footnote{O índice remissivo
   pode ser muito útil para a banca; assim, embora seja um item opcional,
   recomendamos que você o crie.})
\end{itemize}

Ao terminar sua tese/dissertação, você deve entregar uma cópia (digital) dela
para a CPG. Após a defesa, você tem 30 dias para revisar o texto e incorporar
as sugestões da banca. Assim, há duas versões oficiais do documento: a versão
original e a versão corrigida, o que deve ser indicado na folha de rosto.
\index{Tese/Dissertação!versões}

