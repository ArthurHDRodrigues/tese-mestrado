%!TeX root=../tese.tex
%("dica" para o editor de texto: este arquivo é parte de um documento maior)
% para saber mais: https://tex.stackexchange.com/q/78101

% As palavras-chave são obrigatórias, em português e em inglês, e devem ser
% definidas antes do resumo/abstract. Acrescente quantas forem necessárias.
\palavrachave{Grafos dinâmicos}
\palavrachave{Conexidade}
\palavrachave{Floresta maximal de peso mínimo}
\palavrachave{Euler tour trees}

\keyword{Dynamic graphs}
\keyword{Connectivity}
\keyword{Minimum spanning forest}
\keyword{Euler tour trees}

% O resumo é obrigatório, em português e inglês. Estes comandos também
% geram automaticamente a referência para o próprio documento, conforme
% as normas sugeridas da USP.
\resumo{
Essa dissertação aborda os problemas de conexidade em grafos dinâmicos e floresta maximal de peso mínimo em grafos planos ponderados dinâmicos.
Para o primeiro problema é apresentado a solução proposta por Holm, de Lichtenberg e Thorup~\cite{poly_log} junto às estruturas de dados utilizadas e
a elaboração de um algoritmo que resolve esse problema restrito a floresta dinâmicas.
Para o segundo problema é apresentada de forma didática a solução de Eppstein et al.~\cite{EPPSTEIN-planar}.  
Por fim é apresentado o limitante inferior de~$\Omega(\lg n)$ por operação para os problemas estudados.
}

\abstract{
This dissertation addresses the problems of connectivity in dynamic graphs and the minimum weight spanning forest in dynamic weighted planar graphs.
For the first problem, the solution proposed by Holm, de Lichtenberg, and Thorup~\cite{poly_log} is presented, along with the data structures used, as well as the development of an algorithm that solves this problem restricted to dynamic forests.
For the second problem, Eppstein et al.~\cite{EPPSTEIN-planar} solution is presented in a didactic manner.
Finally, the lower bound of~$\Omega(\lg n)$ per operation for the studied problems is presented.
}
