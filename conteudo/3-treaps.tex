%%%%%%%%%%%%%%%%%%%%%%%%%%%%%%%%%%%%%%%%%%%%%%%%
%        Treaps com chave implícita            %
%%%%%%%%%%%%%%%%%%%%%%%%%%%%%%%%%%%%%%%%%%%%%%%%
\chapter{Árvore binária de busca com chave implícita}
\label{sec:TreapDeChaveImplicita}

Com uma implementação tradicional de ABB, em que o campo $chave$ dos nós de uma ABB~$T$ com $t$ nós armazena explicitamente os números de $1$ a $t$, é necessário, na operação \treapJoin($T$, $R$), atualizar todas as chaves de~$R$ para que possuam os valores $1+t,2+t, \ldots, r+t$, onde $r$ é o número de nós em~$R$.

Analogamente, com essa implementação tradicional, ao realizar \treapSplit($T$, $\node$) será necessário atualizar todos os nós com chave maior do que a chave de $\node$. No pior dos casos, essas atualizações de chaves podem custar tempo $\O{n}$, onde $n$ é o tamanho das árvores envolvidas.

Para reduzir o número de atualizações desse campo na biblioteca de Euler tour trees, as ABBs utilizadas possuirão \defi{chave implícita}, ou seja, substituímos o campo $chave$ de cada nó pelo campo \defi{tam} que armazena o tamanho da subárvore enraizada naquele nó, isto é, o número de nós nessa subárvore. Assim a chave de cada nó é obtida a partir de sua posição relativa na ABB.

\begin{figure}[htb]
\centering
\begin{tikzpicture}[line cap=round,line join=round,>=triangle 45,x=1cm,y=1cm]
\clip(-.6,-1) rectangle (14.1,4.6);
\draw [line width=1pt] (1,1) circle (0.5cm);
\draw [line width=1pt] (2,2) circle (0.5cm);
\draw [line width=1pt] (3,1) circle (0.5cm);
\draw [line width=1pt] (5.5,2) circle (0.5cm);
\draw [line width=1pt] (6.5,1) circle (0.5cm);
\draw [line width=1pt] (3.5,3) circle (0.5cm);
\draw [line width=1pt] (7,4) circle (0.5cm);
\draw [line width=1pt] (10.5,3) circle (0.5cm);
\draw [line width=1pt] (9.5,2) circle (0.5cm);
\draw [line width=1pt] (8.5,1) circle (0.5cm);
\draw [line width=1pt] (11.5,2) circle (0.5cm);
\draw [line width=1pt] (10.5,1) circle (0.5cm);
\draw [line width=1pt] (12.5,1) circle (0.5cm);
\draw [line width=1pt] (4.5,1) circle (0.5cm);
\draw [line width=1pt] (7.5,0) circle (0.5cm);
\draw [line width=1pt] (9.5,0) circle (0.5cm);
\draw [line width=1pt] (1.3535533905932737,1.3535533905932737)-- (1.646446609406726,1.646446609406726);
\draw [line width=1pt] (2.353553390593274,1.646446609406726)-- (2.646446609406726,1.353553390593274);
i%\draw [line width=1pt] (2.646446609406726,0.646446609406726)-- (2.353553390593274,0.35355339059327395);
\draw [line width=1pt] (2.416025147168923,2.2773500981126156)-- (3.0839748528310773,2.722649901887385);
\draw [line width=1pt] (3.947213595499958,2.776393202250021)-- (5.052786404500042,2.223606797749979);
\draw [line width=1pt] (5.146446609406726,1.646446609406726)-- (4.853553390593274,1.353553390593274);
\draw [line width=1pt] (5.853553390593274,1.646446609406726)-- (6.146446609406726,1.353553390593274);
\draw [line width=1pt] (10.01923802617959,3.137360563948689)-- (7.480761973820412,3.8626394360513108);
\draw [line width=1pt] (6.519238026179588,3.8626394360513108)-- (3.9807619738204116,3.137360563948689);
\draw [line width=1pt] (10.146446609406727,2.646446609406727)-- (9.853553390593273,2.353553390593273);
\draw [line width=1pt] (9.146446609406727,1.646446609406727)-- (8.853553390593273,1.353553390593273);
\draw [line width=1pt] (8.146446609406727,0.6464466094067269)-- (7.853553390593274,0.35355339059327395);
\draw [line width=1pt] (8.853553390593273,0.6464466094067269)-- (9.146446609406727,0.35355339059327306);
\draw [line width=1pt] (10.853553390593273,1.353553390593273)-- (11.146446609406727,1.646446609406727);
\draw [line width=1pt] (11.146446609406727,2.353553390593273)-- (10.853553390593273,2.646446609406727);
\draw [line width=1pt] (11.853553390593273,1.646446609406727)-- (12.146446609406727,1.353553390593273);


\draw[color=black] (1,1) node {$30$};
\draw[color=black] (1,.2) node {1};
\draw[color=black] (2,2) node {$00$};
\draw[color=black] (2,1.2) node {3};
\draw[color=black] (3,1) node {$04$};
\draw[color=black] (3,.2) node {1};
\draw[color=black] (3.5,3) node {$41$};
\draw[color=black] (3.5,2.2) node {7};
\draw[color=black] (4.5,1) node {$12$};
\draw[color=black] (4.5,.2) node {1};
\draw[color=black] (5.5,2) node {$22$};
\draw[color=black] (5.5,1.2) node {3};
\draw[color=black] (6.5,1) node {$21$};
\draw[color=black] (6.5,.2) node {1};
\draw[color=black] (7,4) node {$11$};
\draw[color=black] (7,3.2) node {16};
\draw[color=black] (7.5,0) node {$14$};
\draw[color=black] (7.5,-.8) node {1};
\draw[color=black] (8.5,1) node {$44$};
\draw[color=black] (8.5,.2) node {3};
\draw[color=black] (9.5,0) node {$45$};
\draw[color=black] (9.5,-.8) node {1};
\draw[color=black] (9.5,2) node {$55$};
\draw[color=black] (9.5,1.2) node {4};
\draw[color=black] (10.5,3) node {$54$};
\draw[color=black] (10.5,2.2) node {8};
\draw[color=black] (10.5,1) node {$40$};
\draw[color=black] (10.5,.2) node {1};
\draw[color=black] (11.5,2) node {$03$};
\draw[color=black] (11.5,1.2) node {3};
\draw[color=black] (12.5,1) node {$33$};
\draw[color=black] (12.5,.2) node {1};


\end{tikzpicture}

\caption{Árvore da Figura~\ref{fig:seq-treap-indices} exibindo o valor do campo $tam$ abaixo de cada nó.}
\label{fig:seq-treap-size}
\end{figure}

Com essa mudança, não é necessário atualizar as chaves dos nós das árvores em \treapJoin{} e \treapSplit{}, pois as chaves não são mais valores explícitos, mas sim valores relativos, obtidos a partir da posição do nó em sua árvore.  Precisaremos atualizar o campo~$tam$ de alguns nós, mas o custo final do~\treapJoin{} e~\treapSplit{} será ainda assim bem menor.

Como manipularemos muitos ponteiros a ABBs que podem conter $\Nil$, é conveniente a adição da rotina interna \treapGetSize($T$), descrita no Algoritmo~\ref{Algo:treapGetSize}, que retorna $0$ caso $T$ contenha~$\Nil$ e, no caso em que $T$ aponte para uma ABB não vazia, retorna seu tamanho. Essa rotina consome tempos~$\O{1}$.

\begin{algorithm}[!htb]
\caption{\treapGetSize($T$)}
\label{Algo:treapGetSize}
\begin{algorithmic}[1]
\If { $T$ = \Nil}
\State \Return $0$
\EndIf
\State \Return $T$.$tam$
\end{algorithmic}
\end{algorithm}

Com chaves implícitas, a alteração de um nó $x$ torna necessário somente a atualização do campo $tam$ dos nós contidos no caminho entre o nó $x$ e a raiz da ABB que o contém, ou seja, o custo de tempo dessas atualizações será assintoticamente proporcional à altura da ABB. Resta então o desafio de manter a ABB balanceada. Na próxima seção apresentaremos a estrutura de dados treap, que resolve esse desafio sem onerar o custo assintótico das operações ou adicionar demasiada complexidade aos algoritmos.

\section{Treaps}

\defi{Heaps} são árvores binárias quase completas constituídas por nós que possuem quatro campos: prioridade, pai e filhos esquerdo e direito~\cite{CLRS}. Os campos referentes aos pais e filhos dão a estrutura de árvore binária ao heap e a \defi{prioridade} de um nó é um inteiro não negativo. Não apresentaremos a definição de árvore binária quase completa, pois ela é desnecessária para a definição de treaps. Para que uma árvore binária quase completa seja considerada um heap é necessário que a prioridade de cada nó seja maior do que a de seus filhos \TODO{Melhorar essa frase}.

\defi{Treaps} são uma mescla entre árvores binária de busca e heaps. Seus nós possuem cinco campos: $pai$, $esq$, $dir$, $chave$ e $prio$. Os campos $esq$ e $dir$ representam os filhos esquerdo e direito de cada nó, respectivamente, e junto ao campo $pai$ descrevem a estrutura de uma árvore binária. O campo $chave$ satisfaz a propriedade de uma ABB enquanto o campo $prio$ armazena a prioridade do nó e satisfaz a propriedade de um heap. Diferentemente de heaps, treaps não precisam ser quase completas.


Tais árvores foram inicialmente nomeadas \defi[árvore!cartesiana]{árvores cartesianas} \cite{Vuillemin1980AUL}, pois podemos representar cada nó como um par ordenado em que a primeira coordenada é a chave do nó e a segunda coordenada é sua prioridade. Ao visualizar esses pares imersos no plano cartesiano, como feito na Figura \ref{fig:TREAP}, ambas as estruturas de ABB e heap são bem representadas. Isto é, nós com maior prioridade ficam ilustrados acima de nós com menor prioridade e os nós ficam ordenados de forma crescente, da esquerda para a direita, em função de suas chaves.

\begin{figure}[htb]
\centering
\begin{tikzpicture}[line cap=round,line join=round,>=triangle 45,x=1cm,y=1cm]
\begin{axis}[
x=1.5cm,y=1.5cm,
axis lines=middle,
ymajorgrids=true,
xmajorgrids=true,
xmin=-.5,
xmax=7.5,
ymin=-.5,
ymax=7.5,
xtick={-3,-2,...,16},
ytick={-1,0,...,10},]
\clip(-1,-1) rectangle (16.11970291004026,10.724908935002318);
\draw [line width=2pt] (3.383319601128083,6.426379568244459)-- (2.0132670659467835,4.457693392208796);
\draw [line width=2pt] (3.383319601128083,6.426379568244459)-- (4.630423800125734,4.5424562038011995);
\draw [line width=2pt] (2.0132670659467835,4.457693392208796)-- (1.1677195880152356,2.990209488027528);
\draw [line width=2pt] (2.0132670659467835,4.457693392208796)-- (2.9057052270438763,3.295352004803549);
\draw [line width=2pt] (4.630423800125734,4.5424562038011995)-- (3.449654930862001,2.8177376307193422);
\draw [line width=2pt] (4.630423800125734,4.5424562038011995)-- (5.399913625039178,3.189215477229281);
\draw [line width=2pt] (3.449654930862001,2.8177376307193422)-- (4.325281283349713,0.9072801343825158);
\draw [line width=2pt] (5.399913625039178,3.189215477229281)-- (5.943863328857303,1.3981615744135059);
\draw [line width=2pt] (1.1677195880152356,2.990209488027528)-- (1.6586010280462258,1.1593543873714025);
\begin{scriptsize}
\draw [fill=black] (3.383319601128083,6.426379568244459) circle (2.5pt);
\draw[color=black] (4.073207030360826,6.711621486100305) node {$(3.38, 6.43)$};
\draw [fill=black] (2.0132670659467835,4.457693392208796) circle (2.5pt);
\draw[color=black] (1.3,4.748095725976344) node {$(2.01, 4.46)$};
\draw [fill=black] (4.630423800125734,4.5424562038011995) circle (2.5pt);
\draw[color=black] (5.320311229358477,4.827698121657045) node {$(4.63, 4.54)$};
\draw [fill=black] (1.1677195880152356,2.990209488027528) circle (2.5pt);
\draw[color=black] (.65,3.275451405883374) node {$(1.17, 3)$};
\draw [fill=black] (2.9057052270438763,3.295352004803549) circle (2.5pt);
\draw[color=black] (3.3,3.5805939226593946) node {$(2.91, 3.3)$};
\draw [fill=black] (3.449654930862001,2.8177376307193422) circle (2.5pt);
\draw[color=black] (4.2,2.8) node {$(3.45, 2.82)$};
\draw [fill=black] (5.399913625039178,3.189215477229281) circle (2.5pt);
\draw[color=black] (6.036732790484788,3.4744573950851265) node {$(5.4, 3.19)$};
\draw [fill=black] (4.325281283349713,0.9072801343825158) circle (2.5pt);
\draw[color=black] (5.015168712582456,1.1925220522383615) node {$(4.33, 0.91)$};
\draw [fill=black] (5.943863328857303,1.3981615744135059) circle (2.5pt);
\draw[color=black] (6.580682494302912,1.6834034922693517) node {$(5.94, 1.4)$};
\draw [fill=black] (1.6586010280462258,1.1593543873714025) circle (2.5pt);
\draw[color=black] (2.3484884572789686,1.4445963052272481) node {$(1.66, 1.16)$};
\end{scriptsize}
\end{axis}
\end{tikzpicture}
\caption{Uma treap imersa no plano cartesiano.}
\label{fig:TREAP}
\end{figure}

\section{Implementação de árvore binária de busca de chave implícita com treaps}
\label{sec:imple-treap}
A implementação das rotinas \treapGetRoot{}, \treapSearch{}, \treapOrder{} e~\treapGetLast{} independem da técnica que usaremos para balancear a treap, pois são consultas e somente usam a estrutura da ABB de chave implícita, assim já podem ser apresentadas nos Algoritmos~\ref{Algo:treapGetRoot},~\ref{Algo:TREAPsearch},~\ref{Algo:treapGetLast} e~\ref{Algo:TREAPorder}.

\begin{algorithm}[htb]
\caption{\treapGetRoot($\node$)}
\label{Algo:treapGetRoot}
\begin{algorithmic}[1]
\State $raiz$ $\gets$ $\node$
\While {$raiz$.$pai$ $\neq \Nil$}
\State  $raiz$ $\gets$  $raiz$.$pai$
\EndWhile
\State \Return $raiz$
\end{algorithmic}
\end{algorithm}

O consumo de tempo das rotinas \treapGetRoot{}, \treapSearch{}, \treapGetLast{} e \treapOrder{} é~$\O{h}$ onde~$h$ é a altura da árvore. Para balancear uma treap e assim garantir consumo de tempo logarítmico, Seidel e Aragon~\cite{AragonSeidel1996} propuseram escolher a prioridade de cada nó de forma aleatória com distribuição de probabilidade uniforme em um universo suficientemente grande para que a probabilidade de haver nós com a mesma prioridade seja próxima de $0$. Como exemplo, os autores sugerem usar como universo os inteiros entre $0$ e~$2^{31}$.

\begin{algorithm}
\caption{\treapSearch($T$, $k$)}
\label{Algo:TREAPsearch}
\begin{algorithmic}[1]
	\If { \treapGetSize($T$.$esq$) $\geqslant$ $k$}
\State \Return \treapSearch($T$.$esq$, $k$)
\EndIf
	\If {\treapGetSize($T$.$esq$)+1 = $k$}
\State \Return $T$
\EndIf
\State \Return \treapSearch($T.dir$, $k$-\treapGetSize($T$.$esq$)-1)
\end{algorithmic}
\end{algorithm}

\begin{algorithm}
\caption{\treapGetLast($T$)}
\label{Algo:treapGetLast}
\begin{algorithmic}[1]
\State \Return \treapSearch($T$, \treapGetSize($T$))
\end{algorithmic}
\end{algorithm}

Para representar essa escolha aleatória, usaremos uma função auxiliar \random($r$) que retorna um inteiro positivo menor do que $r$ com probabilidade uniforme. Esse tipo de função está presente nativamente em diversas linguagens de programação e a elaboração de sua implementação foge do escopo desse texto. Consideraremos que seu consumo de tempo é~$\O{1}$.

\begin{algorithm}
\caption{\treapOrder($u$)}
\label{Algo:TREAPorder}
\begin{algorithmic}[1]
\State $ordem$ $\gets$ 1 + \treapGetSize($u$.$esq$)
\While { $u$.$pai$ $\neq$ \Nil{}}
\If { $u = u$.$pai$.$dir$}
  \State $ordem$ $\gets$ $ordem$ + 1 + \treapGetSize($u$.$pai$.$esq$)
\EndIf
  \State $u$ $\gets$ $u$.$pai$
\EndWhile
\State \Return $ordem$
\end{algorithmic}
\end{algorithm}

Com essa técnica de balanceamento de treaps em mãos, podemos apresentar a implementação dos demais algoritmos que compõem essa biblioteca.

\begin{algorithm}
\caption{\treapCreate($v$)}
\label{Algo:TREAPbuild}
\begin{algorithmic}[1]
\State $\node$.$tam$ $\gets 1$
\State $\node$.$prio$ $\gets$ \random($2^{31}$)
\State $\node$.$info$ $\gets$ $v$
\State $\node$.$esq$ $\gets$ $\node$.$dir$ $\gets$ $\node$.$pai$ $\gets$ \Nil
\State \Return $\node$
\end{algorithmic}
\end{algorithm}

\newpage
A rotina \treapCreate{} consome~$\O{1}$ enquanto que \treapJoin{} consome $\O{h}$ onde~$h$ é a soma das alturas das duas árvores que são unidas e \treapSplit{} consome $\O{h}$ onde~$h$ é a altura da árvore original. Como a altura esperada dessas árvores é  $\O{\lg n}$, teremos que o consumo de tempo dessas duas rotinas também será logarítmico.

\begin{algorithm}
\caption{\treapJoin($T$, $R$)}
\label{Algo:TREAPjoin}
\begin{algorithmic}[1]
\If { $T$ = \Nil} \Return $R$
\EndIf
\If { $R$ = \Nil} \Return $T$
\EndIf

\If { $T$.$prio$ $>$ $R$.$prio$}
  \State $T$.$dir$ $\gets$ \treapJoin($T$.$dir$, $R$)
  \State $T$.$dir$.$pai$ $\gets$ $T$
  \State $T$.$tam$ $\gets$ \treapGetSize($T$.$dir$)+\treapGetSize($T$.$esq$)+$1$
  \State \Return $T$
\Else 
  \State $R$.$esq$ $\gets$ \treapJoin($T$, $R$.$esq$)
  \State $R$.$esq$.$pai$ $\gets$ $R$
  \State $R$.$tam$ $\gets$ \treapGetSize($R$.$dir$)+\treapGetSize($R$.$esq$)+$1$
  \State \Return $R$
\EndIf
\end{algorithmic}
\end{algorithm}
\begin{algorithm}
\caption{\treapSplit($u$)}
\label{Algo:TREAPsplit}
\begin{algorithmic}[1]
\If { $u$ $=$ \Nil} \Return \Nil, \Nil
\EndIf
\State $R$ $\gets $ $u$
\State $L$ $\gets $ $u$.$esq$
\State $u$.$esq$ $\gets\Nil$
\If { $L$ $\neq$ \Nil} $L$.$tam$ $\gets$ \treapGetSize($L$.$dir$)+\treapGetSize($L$.$esq$)+$1$
\EndIf
\State $tmp$ $\gets$ $u$
\While { $tmp$.$pai$ $\neq \Nil$}
  \If { $tmp$.$pai$.$dir$ = $tmp$}
    \State $tmp$.$pai$.$dir$ $\gets$ $L$ \Comment se for filho à direita, então deve ficar em $L$
    \If { $L$ $\neq$ \Nil} $L$.$pai$ $\gets$ $tmp$.$pai$
    \EndIf
    \State $L$ $\gets$ $tmp$.$pai$

    \State $L$.$tam$ $\gets$ \treapGetSize($L$.$dir$)+\treapGetSize($L$.$esq$)+$1$
  \Else
    \State $tmp$.$pai$.$esq$ $\gets$ $R$
    \If { $R \neq \Nil$} $R$.$pai$ $\gets$ $tmp$.$pai$
    \EndIf
    \State $R$ $\gets$ $tmp$.$pai$
    \State $R$.$tam$ $\gets$ \treapGetSize($R$.$dir$)+\treapGetSize($R$.$esq$)+$1$
  \EndIf
  \State $tmp$ $\gets$ $tmp$.$pai$
\EndWhile
\If { $L$ $\neq \Nil$} $L$.$pai$ $\gets\Nil$\EndIf
\If { $R$ $\neq \Nil$} $R$.$pai$ $\gets\Nil$\EndIf
\State\Return $L$, $R$
\end{algorithmic}
\end{algorithm}



