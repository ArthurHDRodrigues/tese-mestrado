
\section{Tratando o problema restrito a somente mudança de peso com Link-Cut Trees}

Para um grafo dinâmico ponderado conexo~$G$, manteremos uma árvore geradora de pesos mínimos~$T$ e sua floresta dual~$T^\star$ como definida no Teorema~\ref{teo:MSFdual} usando Link-Cut trees.
Construiremos~$T$ e~$T^\star$ na rotina \MSFCreate($G_0$), cuja implementação está detalhada no Algoritmo~\ref{Algo:MSFCreate}.
Nessa rotina, primeiro usaremos a rotina auxiliar~\order{} para ordenar as arestas de~$G_0$ em ordem crescente de peso e retornar uma lista com essas arestas ordenadas.
Lembramos que~$G_0$ mantém, para cada aresta~$uv$, sua aresta dual~$uv^\star$, dessa forma, para não perder essa informação, faremos com que cada entrada da lista retornada por~\order{}
possua ponteiros para~$uv$, seu peso~$w$ e~$uv^\star$.


\TODO{Figura de $T$ e $T^\star$ como Link-Cut Trees com nós representando vértices e arestas}

Em seguida percorreremos essa lista sequencialmente e para cada aresta~$uv$ nela, faremos um teste de conexidade entre os vértices~$u$ e~$v$.
Se~$u$ e~$v$ não estiverem conectados, então inserimos~$uv$ em~$F$ com peso~$w$, caso contrário inserimos $uv^\star$ em~$F^\star$ com o mesmo peso~$w$.

Como inserimos as arestas em ordem crescente de peso, a floresta~$F$ é de pesos mínimos e logo, pelo Teorema~\ref{teo:MSFdual}, $F^\star$ é de pesos máximos.

\begin{algorithm}[htb]
\caption{\MSFCreate($n$, $G_0$)}
\label{Algo:MSFCreate}
\begin{algorithmic}[1]
\State \varname{lista} $\gets$ \order($G_0$)
\For{$uv$ em \varname{lista}}\label{Algo:MSFCreate:linhafor}
\If \linkcutQuery($F$, $u$, $v$)\label{Algo:MSFCreate:query}
\State \linkcutAddEdge($G$.$F^\star$, $e$, $f$, $w$)\label{Algo:MSFCreate:link1}
\Else
\State \linkcutAddEdge($G$.$F$, $u$, $v$, $w$)\label{Algo:MSFCreate:link2}
\EndIf
\EndFor
\State \Return $G$
\end{algorithmic}
\end{algorithm}

Existem diversos algoritmos conhecidos na literatura que podem implementar a rotina \order{}~\cite{CLRS} com consumo de tempo~$\O{m\lg m}$, onde $m$ é o tamanho da lista,
que, no nosso caso, coincide com o número de arestas do grafo~$G_0$.
O laço da linha~\ref{Algo:MSFCreate:linhafor} do Algoritmo~\ref{Algo:MSFCreate} itera~$m$ vezes e cada teste de conexidade da linha~\ref{Algo:MSFCreate:query} e cada chamada para a rotina  \linkcutAddEdge{} nas linhas~\ref{Algo:MSFCreate:link1} e~\ref{Algo:MSFCreate:link2} possuem consumo de tempo~$\O{\lg n}$, onde~$n$ é o número de vértices do grafo.
Logo o consumo de tempo de \MSFCreate{} é $\O{m\lg n}$.

Para implementar \MSFweight{} faremos com que cada nó das link-cut trees tenha um campo~$w$ que contabiliza o peso total da subárvore enraizada naquele nó.
Dessa forma, para obter o peso da floresta maximal de peso mínimo~$F$ que estamos mantendo, basta retornar o campo $w$ da raiz de~$F$.
Essa rotina consome tempo constante.

\begin{algorithm}[htb]
\caption{\MSFweight($G$)}
\label{Algo:MSFweight}
\begin{algorithmic}[1]
\State \Return $G$.$F$.$w$
\end{algorithmic}
\end{algorithm}

Quando atualizamos o peso de uma aresta~$uv$ com a rotina \MSFupdate{} para um novo peso~$w$, cuja implementação pode ser vista no Algoritmo~\ref{Algo:MSFupdate}, precisamos garantir que a floresta~$F$ resultante continuará sendo  de peso mínimo.

Notemos que ou $uv\in F$ ou $uv^\star\in F^\star$, iremos tratar cada um desses casos separadamente.
A identificação de qual caso~$uv$ se encontra é feita na linha~\ref{Algo:MSFupdate:linhauvinF} do Algoritmo~\ref{Algo:MSFupdate}, para implementar essa consulta, não podemos fazer como foi feito linha~\ref{Algo:dymGraphReplace:linhayinTv} do \dymGraphReplace(Algoritmo~\ref{Algo:dymGraphReplace}), pois link-cut trees não possuem nós que representam arestas.
Portanto precisamos recorrer a recursos mais elaborados.
Manteremos um dicionário que guardará apontadores para os nós de~$F$ e usaremos como o conjunto de chaves os vértices de $G$. O apontador associado à chave~$u$ aponta para o nó que representa o vértice $u$.
Após usar esse dicionário para obter os nós que representam os vértices~$u$ e~$v$ usamos a rotina \linkcutPath{} para fazer o caminho entre~$u$ e~$v$ ser preferido e verificamos se~$u$ é predecessor ou sucessor de~$v$.

Se~$uv$ for uma aresta de~$F$ e se existir alguma aresta no corte~$(F, uv)$ com peso menor do que~$w$, então~$F$ não será mais de peso mínimo.
Para corrigir isso, precisamos tomar~$xy$ como a aresta de menor peso no corte~$(F, uv)$,
então remover~$uv$ de~$F$ e adicionar~$xy$ a~$F$.
Dessa forma, garantindo que~$F$ seja de peso mínimo.

Para obter a aresta~$xy$, vamos investigar o corte~$(F, uv)$.
Pelo Teorema~\ref{teo:cutset}, o conjunto~$(F, uv)^\star$ forma um ciclo em~$G^\star$.
Como~$uv^\star\in(F, uv)$ e~$uv$ é a única aresta de~$F$ cujo dual está em~$(F, uv)$, então as demais arestas desse corte formam um caminho em~$F^\star$ ligando os vértices incidentes a~$uv^\star$.
Utilizando a rotina \linkcutMin{}  da biblioteca de link-cut trees podemos obter o nó de menor peso nesse percurso.

Note também que, se modificarmos~$F$, então precisaremos atualizar~$F^\star$, removendo $xy^\star$ de~$F^\star$ e adicionando $uv^\star$ com o novo peso~$w$.

Se~$uv$ não for uma aresta de~$F$, então teremos que~$uv^\star$ é uma aresta de~$F^\star$.
O tratamento desse caso é análogo ao caso anterior e é feito entre as linhas~\ref{Algo:MSFupdate:dualinicio} e~\ref{Algo:MSFupdate:dualfim} do Algoritmo~\ref{Algo:MSFupdate}.

Como cada rotina de link-cut trees utilizada consome tempo~$\O{\lg n}$ amortizado, então o consumo de tempo de \MSFupdate{} também é~$\O{\lg n}$ amortizado.



\MSFupdate($H$, $e$, $w$) recebe a hash $H$, uma aresta~$e$ e um peso~$w$ e atualiza o peso de $e$ para ser~$w$ e atualiza $T$ e~$T^\star$.

\begin{algorithm}[htb]
\caption{\MSFupdate($H$, $e$, $w$)}
\label{Algo:MSFupdate}
\begin{algorithmic}[1]
\State $\hat e$, $e_0$, $e_1$, $e_2$, $e_3$ $\gets$ $H$($e$)
\State $\hat e.w$ $\gets$ $w$
\If {$e$ $\in$ $T$}\label{Algo:MSFupdate:linhauvinF}

\State \linkcutEvert($e_1$)
\State $\hat d$ $\gets$ \linkcutMin($e_3$)

\If {$\hat d$.$w$ $=$ $\infty$}
\Comment {A aresta $e^\star$ é um laço.}
\State \Return
\EndIf
\State $d$ $\gets$ $\hat d.\varname{id}$
\State $\hat d$, $d_0$, $d_1$, $d_2$, $d_3$ $\gets$ $H$($d$)
\If {$\hat d$.$w$ > $w$}
\State \linkcutEvert($e_0$); \linkcutDelEdge($\hat e$); \linkcutDelEdge($e_2$)

\State \linkcutEvert($d_1$); \linkcutDelEdge($\hat d$); \linkcutDelEdge($d_3$)

\State \linkcutEvert($e_3$); \linkcutAddEdge($e_3$, $\hat e$); \linkcutAddEdge($\hat e$, $e_1$)

\State \linkcutEvert($d_0$); \linkcutAddEdge($d_0$, $\hat d$); \linkcutAddEdge($\hat d$, $d_2$)
\EndIf

\Else

\State \linkcutEvert($e_0$)
\State $\hat d$ $\gets$ \linkcutMax($e_2$)

\If {$\hat d$.$w$ $=$ $-\infty$}
\State \Return
\EndIf
\State $d$ $\gets$ $\hat d.\varname{id}$
\State $\hat d$, $d_0$, $d_1$, $d_2$, $d_3$ $\gets$ $H$($d$)
\If {$\hat d$.$w$ < $w$}

\State \linkcutEvert($d_0$); \linkcutDelEdge($\hat d$); \linkcutDelEdge($d_2$)
\State \linkcutEvert($e_1$); \linkcutDelEdge($\hat e$); \linkcutDelEdge($e_3$)

\State \linkcutEvert($e_0$); \linkcutAddEdge($e_0$, $\hat e$); \linkcutAddEdge($\hat e$, $e_2$)

\State \linkcutEvert($d_1$); \linkcutAddEdge($d_1$, $\hat d$); \linkcutAddEdge($\hat d$, $d_3$)

\EndIf
\EndIf
\end{algorithmic}
\end{algorithm}
