%% ------------------------------------------------------------------------- %%
\chapter{Introdução}
\label{cap:introducao}

Escrever bem é uma arte que exige muita técnica e dedicação e,
consequentemente, há vários bons livros sobre como escrever uma boa
dissertação ou tese. Um dos trabalhos pioneiros e mais conhecidos nesse
sentido é o livro de
%Umberto Eco~\cite{eco:09} % usando o estilo alpha
Umberto~\citet{eco:09} % usando o estilo plainnat
intitulado \emph{Como se faz uma tese}; é uma leitura bem interessante mas,
como foi escrito em 1977 e é voltado para trabalhos de graduação na Itália,
não se aplica tanto a nós.

Sobre a escrita acadêmica em geral, John Carlis disponibilizou um texto curto
e interessante~\citep{carlis:09} em que advoga a preparação de um único
rascunho da tese antes da versão final. Mais importante que isso, no
entanto, são os vários \textit{insights} dele sobre a escrita acadêmica.
Dois outros bons livros sobre o tema são \emph{The Craft of Research}~\citep{craftresearch}
e \emph{The Dissertation Journey}~\citep{dissertjourney}. Além disso, a USP
tem uma compilação de normas relativas à produção de documentos
acadêmicos~\citep{usp:guidelines} que pode ser utilizada como referência.

Para a escrita de textos especificamente sobre Ciência da Computação, o
livro de Justin Zobel, \emph{Writing for Computer Science}~\citep{zobel:04}
é uma leitura obrigatória. O livro \emph{Metodologia de Pesquisa para
Ciência da Computação} de
%Raul Sidnei Wazlawick~\cite{waz:09} % usando o estilo alpha
Raul Sidnei~\citet{waz:09} % usando o estilo plainnat
também merece uma boa lida. Já para a área de Matemática, dois livros
recomendados são o de Nicholas Higham, \emph{Handbook of Writing for
Mathematical Sciences}~\citep{Higham:98} e o do criador do \TeX{}, Donald
Knuth, juntamente com Tracy Larrabee e Paul Roberts, \emph{Mathematical
Writing}~\citep{Knuth:96}.

Apresentar os resultados de forma simples, clara e completa é uma tarefa que
requer inspiração. Nesse sentido, o livro de
%Edward Tufte~\cite{tufte01:visualDisplay}, % usando o estilo alpha
Edward~\citet{tufte01:visualDisplay}, % usando o estilo plainnat
\emph{The Visual Display of Quantitative Information}, serve de ajuda na
criação de figuras que permitam entender e interpretar dados/resultados de forma
eficiente.

Além desse material, também vale muito a pena a leitura do trabalho de
%Uri Alon \cite{alon09:how}, % usando o estilo alpha
Uri \citet{alon09:how}, % usando o estilo plainnat
no qual apresenta-se uma reflexão sobre a utilização da Lei de Pareto para
tentar definir/escolher problemas para as diferentes fases da vida acadêmica.
A direção dos novos passos para a continuidade da vida acadêmica deveria ser
discutida com seu orientador.

%% ------------------------------------------------------------------------- %%
\section{Considerações de Estilo}
\label{sec:consideracoes_preliminares}

Normalmente, as citações não devem fazer parte da estrutura sintática da
frase\footnote{E não se deve abusar das notas de rodapé.\index{Notas de rodapé}}.
No entanto, usando referências em algum estilo autor-data (como o estilo
plainnat do \LaTeX{}), é comum que o nome do autor faça parte da frase. Nesses
casos, pode valer a pena mudar o formato da citação para não repetir o nome do
autor (no \LaTeX{}, isso pode ser feito usando os comandos
\textsf{\textbackslash{}citet}, \textsf{\textbackslash{}citep},
\textsf{\textbackslash{}citeyear} etc. documentados no pacote
natbib)\index{natbib}. Em geral, portanto, as citações devem seguir estes
exemplos:

\small
\begin{verbatim}
Modos de citação:
indesejável: [AF83] introduziu o algoritmo ótimo.
indesejável: (Andrew e Foster, 1983) introduziram o algoritmo ótimo.
certo: Andrew e Foster introduziram o algoritmo ótimo [AF83].
certo: Andrew e Foster introduziram o algoritmo ótimo (Andrew e Foster, 1983).
certo (\citet ou \citeyear): Andrew e Foster (1983) introduziram o algoritmo ótimo.
\end{verbatim}
\normalsize

O uso desnecessário de termos em língua estrangeira deve ser evitado. No entanto,
quando isso for necessário, os termos devem aparecer \textit{em itálico}.
\index{Língua estrangeira}
% index permite acrescentar um item no indice remissivo

Uma prática recomendável na escrita de textos é descrever as
legendas\index{Legendas} das figuras e tabelas em forma auto-contida: as
legendas devem ser razoavelmente completas, de modo que o leitor possa entender
a figura sem ler o texto onde a figura ou tabela é citada.\index{Floats}

\section{Ferramentas Bibliográficas}

Embora seja possível pesquisar por material acadêmico na Internet usando sistemas
de busca ``comuns'', existem ferramentas dedicadas, como o \textsf{Google Scholar}\index{Google Scholar}
(\url{scholar.google.com}). Você também pode querer usar o \textsf{Web of Science}\index{Web of Science}
(\url{webofscience.com}) e o \textsf{Scopus}\index{Scopus} (\url{scopus.com}), que oferecem
recursos sofisticados e limitam a busca a periódicos com boa reputação acadêmica.
Essas duas plataformas não são gratuitas, mas os alunos da USP têm acesso a elas
através da instituição. Ambas são capazes de exportar os dados para o formato .bib,
usado pelo \LaTeX{}. Algumas editoras, como a ACM e a IEEE, também têm sistemas de
busca bibliográfica.

Apenas uma parte dos artigos acadêmicos de interesse está disponível livremente
na Internet; os demais são restritos a assinantes. A CAPES assina um grande
volume de publicações e disponibiliza o acesso a elas para diversas universidades
brasileiras, entre elas a USP, através do seu portal de periódicos
(\url{periodicos.capes.gov.br}). Existe uma extensão para os navegadores
Chrome e Firefox (\url{www.infis.ufu.br/capes-periodicos}) que facilita o uso
cotidiano do portal.

Para manter um banco de dados organizado sobre artigos e outras fontes bibliográficas
relevantes para sua pesquisa, é altamente recomendável que você use uma ferramenta
como Zotero~(\url{zotero.org})\index{Zotero} ou
Mendeley~(\url{mendeley.com})\index{Mendeley}. Ambas podem exportar seus dados no
formato .bib, compatível com \LaTeX{}. Também existem três plataformas
gratuitas que permitem a busca de referências acadêmicas já no formato .bib:

\begin{itemize}
  \item \emph{CiteULike}\index{CiteULike} (patrocinados por Springer): \url{www.citeulike.org}
  \item Coleção de bibliografia em Ciência da Computação: \url{liinwww.ira.uka.de/bibliography}
  \item Google acadêmico\index{Google Scholar} (habilitar bibtex nas preferências): \url{scholar.google.com}
\end{itemize}

Lamentavelmente, ainda não existe um mecanismo de verificação ou validação das
informações nessas plataformas. Portanto, é fortemente sugerido validar todas
as informações de tal forma que as entradas bib estejam corretas.

De qualquer modo, tome muito cuidado na padronização das referências
bibliográficas: ou considere TODOS os nomes dos autores por extenso, ou TODOS
os nomes dos autores abreviados.  Evite misturas inapropriadas.

\section{O Que o IME Espera}

Ao terminar sua tese/dissertação, você deve entregar uma cópia dela para a
CPG. Após a defesa, você tem 30 dias para revisar o texto e incorporar as
sugestões da banca. Assim, há duas versões oficiais do documento: a versão
original e a versão corrigida, o que deve ser indicado na folha de rosto.
\index{Tese/Dissertação!versões}

Fica a critério do aluno definir aspectos como o tamanho de fonte, margens,
espaçamento, estilo de referências, cabeçalho, etc. considerando sempre o
bom senso. A CPG, em reunião realizada em junho de 2007, aprovou que as
teses/dissertações deverão seguir o formato padrão por ela
definido\footnote{\url{www.ime.usp.br/dcc/pos/normas/tesesedissertacoes}}.
Esse padrão refere-se aos itens que devem estar presentes nas teses/dissertações
(e.g. capa, formato de rosto, sumário, etc.), e não à formatação do documento.
Ele define itens obrigatórios e opcionais, conforme segue:\index{Formatação}
\index{Tese/Dissertação!itens obrigatórios}
\index{Tese/Dissertação!itens opcionais}

\begin{itemize}
  \item \textsc{Capa} (obrigatória)
  \begin{itemize}
    \item O IME usa uma capa padrão de cartolina para todas as
    teses/dissertações.  Essa capa tem uma janela recortada por onde se
    vê o título e o autor do trabalho e, portanto, a capa impressa do
    trabalho deve incluir o título e o autor na posição correspondente da
    página. Ela fica centralizada na página, tem 100mm de largura, 60mm de
    altura e começa 47mm abaixo do topo da página.

    \item O título da tese/dissertação deverá começar com letra maiúscula
    e o resto deverá ser em minúsculas, salvo nomes próprios.

    \item O nome do aluno(a) deverá ser completo e sem abreviaturas.

    \item É preciso explicitar se é uma tese ou dissertação (para
    obtenção do título de doutor, tese; para obtenção do título de
    mestre, dissertação).

    \item O nome do programa deve constar da capa (Matemática,
    Matemática Aplicada, Estatística ou Ciência da Computação).

    \item Também devem constar o nome completo do orientador e do
    co-orientador, se houver.

    \item Se o aluno recebeu bolsa, deve-se indicar a(s) agência(s).

    \item É preciso informar o mês e ano do depósito ou da entrega da
    versão corrigida.
  \end{itemize}

  \item \textsc{Folha de Rosto} (obrigatória, tanto para a versão
  depositada quanto para a versão corrigida)
  \begin{itemize}
    \item o título da tese/dissertação deverá seguir o padrão da capa

    \item deve informar se se trata da versão original ou da versão
    corrigida; no segundo caso, deve também incluir os nomes
    dos membros da banca.
  \end{itemize}

  \item \textsc{Agradecimentos} (opcional)

  \item \textsc{Resumo}, em português (obrigatório)

  \item \textsc{Abstract}, em inglês (obrigatório)

  \item \textsc{Sumário} (obrigatório)

  \item \textsc{Listas} (opcionais)
  \begin{itemize}
    \item Lista de Abreviaturas
    \item Lista de Símbolos
    \item Lista de Figuras
    \item Lista de Tabelas
  \end{itemize}

  \item \textsc{Referências Bibliográficas} (obrigatório)

  \item \textsc{Índice Remissivo} (opcional\footnote{O índice remissivo
   pode ser muito útil para a banca; assim, embora seja um item opcional,
   recomendamos que você o crie.})
\end{itemize}
