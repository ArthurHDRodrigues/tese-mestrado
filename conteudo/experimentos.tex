\chapter{Avaliação empírica}

Nesse capítulo faremos experimentos comparativos para avaliar a eficiência do algoritmo de conexidade dinâmica elaborado no Capítulo~\ref{sec:connDG}. Como esse algoritmo foi desenvolvido por por Holm, de Lichtenberg e Thorup, ele é conhecido na literatura pelo acrônimo \HDT.

Desde de sua invenção, houveram três estudos experimentais desse algoritmo~\cite{EmpiricalStudy1997, EmpiricalStudy2002, Zaroliagis2002}, o principal comparador é um algoritmo desenvolvido por Henzinger e King e portanto denotado por \HK, pois ele empata o consumo amortizado esperado de $\O{\lg^2 n}$ de \HDT. Ambos os algoritmos visam manter uma MSF do grafo dinâmico para realizar o teste de conexidade. A principal diferença entre eles é que \HK{} sorteia arestas reservas e as testa para verificar se substituem uma aresta removida da MSF.


Já o de Chen \textit{et al.} será chamado por \CLHB.

Em $2022$, Chen et al. \cite{QC22} propõem uma nova estrutura simplesmente chamada \defi{árvore dinâmica} (\textit{D-tree}). Essa estrutura consiste em uma árvore enraizada $k$-ária, com $k$ grande o suficiente. Os autores não apresentam análises de complexidade das operações, mas comparam o desempenho da nova estrutura de dados com a de Henzinger e King em bancos de dados imensos bem conhecidos. No entanto, não comparam sua implementação com outros algoritmos, como o de Holm et al.



O primeiro algoritmo poli-logarítmico para o problema de conectividade dinâmica foi apresentado por Henzinger e King~\cite{HenzingerKing} e possui consumo amortizado esperado $\O{\lg^2 n}$ para cada operação. 



O algoritmo de Holm et al. para MSF possui consumo amortizado $\O{\lg^4 n}$ para cada operação, sendo assim a primeira estrutura de dados determinística com consumo de tempo poli-logarítmico para o problema MSF. Uma implementação eficiente deste algoritmo foi apresentada por Cattaneo et al. \cite{xpstudy2002} junto a um outro algoritmo mais simples e assintoticamente pior, mas que teve bom desempenho nos experimentos práticos realizados.

Em $2000$, Thorup \cite{Thorup2000} apresenta uma estrutura de dados aleatorizada que resolve o problema de conectividade dinâmica com consumo de tempo amortizado $\O{\lg n(\lg\lg n)^3}$ por operação.




\TODO{Escrever sobre \cite{Wulff-Nilsen2016}}

\TODO{adicionar \cite{kejlbergrasmussen_et_al}}
