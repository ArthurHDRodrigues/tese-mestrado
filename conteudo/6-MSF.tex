\chapter{Árvores geradoras mínimas em grafos planos}
\label{sec:MSF}


Retomemos o problema da floresta maximal de peso mínimo em grafos dinâmicos inicialmente apresentado na Seção~\ref{sec:Motivação} e que é o segundo problema que iremos apresentar.
Ele consiste na busca por uma implementação tão eficiente quanto possível para a seguinte biblioteca:

\begin{itemize}
\item \MSFCreate($n$, $G_0$): recebe um grafo plano ponderado com~$n$ vértices dado por listas de adjacências e devolve um grafo dinâmico plano ponderado isomorfo a~$G_0$; Para cada aresta~$uv$ em~$G_0$ também é informado a aresta dual~$uv^\star$, assim fixando uma imersão no plano.
\item \MSFupdate($G$, $u$, $v$, $w$): atribui o peso~$w$ à aresta~$uv$ do grafo dinâmico~$G$.
%\item \MSFaddEdge($G$, $u$, $v$, $w$): adiciona a aresta $uv$ com peso $w$ ao grafo dinâmico~$G$;
%\item \MSFdelEdge($G$, $u$, $v$): remove a aresta $uv$ de $G$; e
\item \MSFweight($G$): devolve o peso de uma MSF de $G$.
\end{itemize}


\section{Subdivisão planar e suas representações}

Definir:
\begin{itemize}
\item grafo planar
\item imersão no plano
\item grafo plano
\item grafo dual
\end{itemize}

\begin{theorem}
\label{teo:MSFdual}
Dado um árvore maximal~$T$ de um grafo plano $G$, então o conjunto
$$
T^\star = \{e^\star|e\notin T\}
$$
é uma árvore maximal de $G^\star$.
Além disso, se~$T$ for de peso mínimo, então~$T^\star$ é de peso máximo e vice versa.
\end{theorem}

\section{Link/Cut Tree}
\label{sec:linkcuttree}

Para uma introdução sobre link-cut trees, recomendamos o trabalho de~\cite{linkcuttree}.

Biblioteca de link/cut tree:
\begin{itemize}
\item link
\item cut
\item max($T$, $u$, $v$): retorna o peso máximo no caminho entre $u$ e $v$ na árvore~$T$.
\item min($T$, $u$, $v$): retorna o peso mínimo no caminho entre $u$ e $v$ na árvore~$T$.
\end{itemize}

\section{Ideia do algoritmo}
Manteremos $T$ e sua dual~$T^\star$ como definida no Teorema~\ref{teo:MSFdual} usando link-cut trees.

Para implementar \MSFCreate($n$, $G_0$), primeiro ordenamos as arestas de~$G_0$ em ordem crescente de peso e em seguida as inserimos sequencialmente em~$G$.
Antes de inserir uma aresta~$uv$, faremos um teste de conexidade entre~$u$ e~$v$, se $u$ e $v$ não estiverem conectados, então inserimos $uv$ em~$T$, caso contrário inserimos $uv^\star$ em~$T^\star$.






