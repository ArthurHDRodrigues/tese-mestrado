\chapter{Árvores geradoras mínimas}
\label{sec:MSF}


Retomemos o problema da floresta maximal de peso mínimo em grafos dinâmicos inicialmente apresentado na Seção~\ref{sec:Motivação} e que é o segundo problema que iremos apresentar. Ele consiste na busca por uma implementação tão eficiente quanto possível para a seguinte biblioteca:

\begin{itemize}
\item \MSFCreate($n$): cria e devolve um grafo dinâmico com $n$ vértices isolados;
\item \MSFaddEdge($G$, $u$, $v$, $w$): adiciona a aresta $uv$ com peso $w$ ao grafo dinâmico~$G$;
\item \MSFdelEdge($G$, $u$, $v$): remove a aresta $uv$ de $G$; e
\item \MSFweight($G$): devolve o peso de uma MSF de $G$.
\end{itemize}


A estratégia predominante na literatura para solucionar esse problema é manter uma floresta dinâmica~$F$ provida pela biblioteca
elaborada no Capítulo~\ref{sec:connDF} e garantir que essa floresta seja de pesos mínimo ao longo da
sequência de modificações. As arestas que não estiverem em~$F$ continuarão sendo chamadas de arestas
reservas.

\section{MSF em grafos planares}

Nessa seção mostraremos uma solução ótima para o problema MSF para o caso em que o grafo é planar.
Começaremos com o caso em que a estrutura do grafo não é modificado, a única operação permitida é a
mudança de peso nas arestas e, em seguida, elaboraremos uma solução para o caso em que inserção e
remoção de arestas são permitidas.

\subsection{Subdivisão planar e suas representações}

\subsection{Suporte somente à mudança de peso em arestas}


Para tal, em \MSFaddEdge, faremos um teste de conexidade entre os vértices~$u$ e~$v$.
Se esses vértices não estiverem na mesma componente conexa, então inserimos a aresta~$uv$ em~$F$.
Caso contrário, existe um caminho entre~$u$ e~$v$ em~$F$ e assim precisamos verificar se o peso~$w$ de $uv$ é menor do todos os
pesos das arestas nesse caminho.

Podemos adaptar a solução para o problema de conexidade em grafos dinâmicos apresentado na Seção~\ref{sec:connDG} para obter uma solução do problema de MSF dinâmico para o caso decremental, isto é, o caso em que a única atualização aplicada ao grafo é a de remoção de arestas do grafo~\cite{poly_log}. Primeiro, evidentemente, a floresta maximal $F_{\lceil \lg n \rceil}$ mantida nessa solução deve ser de peso mínimo. Para tal, ao procurar uma substituta para alguma aresta da árvore na rotina~\dymGraphReplace{}, basta percorrer as arestas de nível $i$ em ordem crescente de peso. A prova de como essa simples mudança funciona será apresentada na dissertação.

A operação de adição de aresta exigirá um pouco mais da nossa estrutura de dados. Dada uma MSF $F$ de um grafo dinâmico~$G$, ao adicionar uma aresta $uv$ com peso $w$ a $G$, se os vértices~$u$ e~$v$ não estiverem na mesma componente conexa de~$G$, então basta adicionar a aresta a~$F$. No caso em que os vértices estão conectados, então precisamos verificar se $uv$ substitui alguma aresta de $F$. Para tal, é necessário considerar o caminho em~$F$ que liga $u$ a~$v$, verificando se~$w$ é menor do que o peso de alguma aresta nesse caminho. Se existir tal aresta com peso maior do que~$w$, então a removemos de~$F$ e adicionamos $uv$ a~$F$. Caso contrário, somente adicionamos $uv$ a $G$.

Notemos que é necessário ter informação sobre a conexidade do grafo dinâmico, o que já é resolvido pela biblioteca de grafos dinâmicos e Euler tour trees como visto nas Seções anteriores. No entanto, Euler tour trees não permitem percorrer caminhos da árvore de forma eficiente, pois muitas vezes o caminho entre dois vértices arbitrários estará fragmentado ao longo da sequência Euleriana que representa a árvore. Para obter caminhos de forma eficiente, usaremos uma estrutura de dados chamada \defi[arvore@\'arvore!topológica]{árvores topológicas} (top tree)~\cite{AHLTMinDiameter}, que será introduzida adequadamente no texto completo de dissertação.


\section{Link/Cut Tree}
\label{sec:linkcuttree}

Para uma introdução sobre link-cut trees, recomendamos o trabalho de~\cite{linkcuttree}.
