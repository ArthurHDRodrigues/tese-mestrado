\chapter{Floresta maximal de peso mínimo em grafos planos}
\label{sec:MSF}

O problema da floresta maximal de peso mínimo em grafos planos dinâmicos foi inicialmente apresentado na Seção~\ref{sec:Motivação} e é o segundo problema que iremos discutir.
O objetivo desse capítulo é apresentar o algoritmo de David Eppstein, Giuseppe Italiano, Roberto Tamassia, Robert Tarjan, Jeffery Westbrook e Moti Yung~\cite{EPPSTEIN-planar} que resolve esse problema para grafos planos que podem ter laços e arestas paralelas.

O estudo desse problema se inicia com uma revisão de conceitos de planaridade e dualidade apresentados nas Seções~\ref{sec:planaridade} e~\ref{sec:dualidade} respectivamente.
Esses conceitos são bem estabelecidos na literatura e essas seções são baseadas na Seção~4.2 do livro \textit{Graph Theory} de Reinhard Diestel~\cite{Diestel}.

Os conceitos apresentando nessas duas seções iniciais serão usados para definir formalmente o problema, o que é feito na Seção~\ref{sec:definition-MSF}, e para fundamentar a estrutura de dados que resolve esse problema, que será apresentada na Seção~\ref{sec:EODT}.

\section{Planaridade}
\label{sec:planaridade}

Intuitivamente, um grafo é dito plano se ele está desenhado em uma folha de papel de forma que suas arestas não se cruzem.
Formalmente, um \defi[grafo!plano]{grafo plano}~\cite{Diestel} é um par de conjuntos finitos $G = (V, E)$ com as seguintes propriedades:
\begin{enumerate}
\item $V\subset \R^2$;
\item Toda aresta é um arco entre dois vértices;
\item O interior de uma aresta não contém vértices nem intersecta outras arestas.
\end{enumerate}
Quando conveniente, $G$ será usado tanto para o par~$(V ,E)$ quanto para o conjunto~${V\cup\bigcup E}$.

Para cada grafo plano~$G$, ao remover~$G$ do plano~$\R^2$, é obtido um conjunto finito~$F(G)$ de regiões conexas que é chamado de \defi{conjunto de faces} de~$G$.
Naturalmente, cada uma dessas regiões é chamada de uma \defi{face} de $G$.
Uma dessas faces é ilimitada.
Essa face é chamada de \defi{face exterior}.
A Figura~\ref{fig:MSF-basico-0} mostra um grafo plano~$G$ e suas as faces.

\begin{figure}[htb]
\scalebox{1.5}{
\centering
\begin{tikzpicture}[line cap=round,line join=round,>=triangle 45,x=1cm,y=1cm]
\clip(-3.5,-.5) rectangle (6,3);


\draw [line width=1pt] (0,0) to[in=-135,out=135,looseness=1]  (0,2); % v -- u
\draw [line width=1pt] (0,0) to[in=-45,out=45,looseness=1]  (0,2);   % v -- u
\draw [line width=1pt] (2,1) to  (0,0); % 4 -- 6
\draw [line width=1pt] (2,1) to  (0,2); % 4 -- 6
\draw [line width=1pt] (2,1.05) to[out=45,in=-45,looseness=50] (2,.95); % 4 -- 4


%\draw [line width=1pt] (1,2.5) to[out=-45,in=45,looseness=1] (1,1); % F0 -- F1
%\draw [line width=1pt] (1,2.5) to[out=20,in=-70,looseness=10] (1,1); % F0 -- F1
%\draw [line width=1pt] (1,2.5) to[out=170,in=180,looseness=2.5] (0,1); % F0 -- F2
%\draw [line width=1pt] (1,2.5) to[out=0,in=90,looseness=1] (2.5,1); % F0 -- F3
%\draw [line width=1pt] (1,1) to (0,1); % F1 -- F2




\begin{scriptsize}
\draw [fill=black] (0,2) circle (1.5pt);
\draw (0,2) node[anchor=south east] {$u$};


\draw [fill=black] (0,0) circle (1.5pt);
\draw (0,0) node[anchor=north east] {$v$};
\draw [fill=black] (2,1) circle (1.5pt);
\draw (1.95,1) node[anchor=south] {$z$};


%\draw [fill=black] (1,2.5) circle (1.5pt);
\draw (1,2.5) node {$F_0$};
%\draw [fill=black] (1,1) circle (1.5pt);
\draw (1,1) node {$F_1$};
%\draw [fill=black] (0,1) circle (1.5pt);
\draw (0,1) node {$F_2$};
%\draw [fill=black] (2.5,1) circle (1.5pt);
\draw (2.5,1) node {$F_3$};

\draw (-0.3,1) node {$2$};
\draw (-0.53,1) node {$a$};

\draw (0.3,1) node {$b$};
\draw (0.53,1) node {$7$};

\draw (1.15,1.7) node[anchor=north] {$c$}; % u -- w
\draw (1.05,1.5) node[anchor=north] {$3$}; % u -- w

\draw (2.9,1) node {$d$}; % w -- w
\draw (3.15,1) node {$2$}; % w -- w

\draw (1.2,.9) node[anchor=north] {$e$}; % v -- w
\draw (1.35,.65) node[anchor=north] {$1$}; % v -- w
\end{scriptsize}
\end{tikzpicture}

}
\caption{Um grafo ponderado plano e suas faces.}
\label{fig:MSF-basico-0}
\end{figure}



\begin{lemma}[Lemma 4.2.2~\cite{Diestel}]
\label{lemma:diestel}
Seja $G$ um grafo plano e $e$ uma de suas arestas.
\begin{enumerate}
\item Se $X$ é a fronteira de uma face de~$G$, então ou $e\subseteq X$ ou a intersecção de $X$ com o interior de $e$ é vazia.
\item Se $e$ pertence a um ciclo de $G$, então $e$ pertence à fronteira de exatamente duas faces distintas de $G$;
\item Se não existe ciclo que contém $e$, então $e$ pertence à fronteira de uma única face de~$G$.
\end{enumerate}
\end{lemma}

\begin{theorem}[Fórmula de Euler, Teorema 4.2.9~\cite{Diestel}]
\label{teo:MSFEuler}
Seja $G=(V,E)$ um grafo planar conexo com~$n$ vértices e~$m$ arestas. Toda imersão de~$G$ no plano possui o mesmo número $f$ de faces e
$$
n-m+f = 2.
$$
\end{theorem}

Para cada vértice $v$ de um grafo plano, é possível construir uma ordenação cíclica~$D(v)$ das aresta incidentes a~$v$~\cite{noma2003}.
Para obter essa ordenação, percorremos as arestas incidentes a~$v$ em sentido anti-horário até retornar à aresta inicial do percurso.
Uma aresta é dita \defi{sucessora} de outra aresta se a primeira é sucessora da segunda nesse percurso.
Uma lista de todas essas ordens cíclicas é chamada de~\defi{descrição combinatória plana} da imersão.
Abaixo temos uma descrição combinatória plana do grafo da Figura~\ref{fig:MSF-basico-0}.
\begin{align*}
D(u)&=\langle (a,u), (b,u), (c,u))\rangle\\
D(v)&=\langle (a,v), (e,v), (b,v)  \rangle\\
D(z)&=\langle (e,z), (d,z), (d,z), (c,z)\rangle
\end{align*}

A descrição combinatória de um grafo plano é essencialmente uma lista de adjacências em que a ordem das células é relevante.
Como estamos admitindo arestas paralelas, o número de ocorrências de um vértice~$u$ na lista de um vértice~$v$ será exatamente o número de arestas paralelas entre~$u$ e~$v$.
Um laço num vértice~$u$ corresponderá a duas ocorrências de~$u$ na sua lista de adjacências.

Para o algoritmo que vamos descrever nesse capítulo, cada aresta possui um identificador.
Observe que cada aresta corresponde a duas células nas listas de adjacência do grafo:
Se os extremos de uma aresta são~$u$ e~$v$, há uma célula com~$u$ na lista de~$v$ e uma célula com~$v$ na lista de~$u$, ambos representando essa aresta.

Pensamos nessas células como representações das duas possíveis orientações da aresta: de~$u$ para~$v$ (algumas vezes denotada por~$uv$) e de~$v$ para~$u$ (algumas vezes denotada por $vu$).
Essas duas células armazenarão o identificador da aresta e terão um apontador $\varname{sym}$, referenciando a outra e vice-versa.
Ademais, cada uma dessas células possuirá também um campo $\varname{orig}$, que identificará o vértice a cuja lista pertence, ou seja, o vértice que é a origem da aresta orientada representada pela célula.

\section{Dualidade}
\label{sec:dualidade}


Dado um grafo plano, o grafo \defi{dual} de~$G$~\cite{Diestel} é o grafo $G^\star = (F,E^\star)$ cujo conjunto de vértices é o conjunto~$F = F(G)$ de faces de~$G$ e o conjunto de arestas $E^\star$ é construído a partir das arestas de~$G$.
Para cada aresta~$e\in E$, existe exatamente uma aresta~$e^\star$ no conjunto $E^\star$.
A aresta~$e^\star$ é chamada de \defi[aresta!dual]{aresta dual} de~$e$ e os vértices incidentes a essa aresta são as duas faces (não necessariamente distintas) cujas fronteiras contém~$e$.
Nesse contexto, chamamos~$G$ de grafo \defi{primal}.
Para ilustrar a relação entre cada aresta e seu dual, geralmente desenhamos cada aresta do grafo dual cruzando a correspondente aresta do grafo primal, como mostra a Figura~\ref{fig:MSF-basico-1}.

\begin{figure}[htb]
\scalebox{1.5}{
\centering
\begin{tikzpicture}[line cap=round,line join=round,>=triangle 45,x=1cm,y=1cm]
\clip(-3.5,-1) rectangle (6,3);


\draw [line width=1pt] (0,0) to[in=-135,out=135,looseness=1]  (0,2); % v -- u
\draw [line width=1pt] (0,0) to[in=-45,out=45,looseness=1]  (0,2);   % v -- u
\draw [line width=1pt] (2,1) to  (0,0); % 4 -- 6
\draw [line width=1pt] (2,1) to  (0,2); % 4 -- 6
\draw [line width=1pt] (2,1.05) to[out=45,in=-45,looseness=50] (2,.95); % 4 -- 4


\draw [line width=1pt] (1,2.5) to[out=-45,in=45,looseness=1] (1,1); % F0 -- F1
\draw [line width=1pt] (1,2.5) to[out=20,in=-70,looseness=10] (1,1); % F0 -- F1
\draw [line width=1pt] (1,2.5) to[out=170,in=180,looseness=2.5] (0,1); % F0 -- F2
\draw [line width=1pt] (1,2.5) to[out=0,in=90,looseness=1] (2.5,1); % F0 -- F3
\draw [line width=1pt] (1,1) to (0,1); % F1 -- F2




\begin{scriptsize}
\draw [fill=black] (0,2) circle (1.5pt);
\draw (0,2) node[anchor=south east] {$u$};


\draw [fill=black] (0,0) circle (1.5pt);
\draw (0,0) node[anchor=north east] {$v$};
\draw [fill=black] (2,1) circle (1.5pt);
\draw (1.95,1) node[anchor=south] {$w$};


\draw [fill=black] (1,2.5) circle (1.5pt);
\draw (1,2.5) node[anchor=south east] {$F_0$};
\draw [fill=black] (1,1) circle (1.5pt);
\draw (1,1) node[anchor=west] {$F_1$};
\draw [fill=black] (0,1) circle (1.5pt);
\draw (0,1) node[anchor=south] {$F_2$};
\draw [fill=black] (2.5,1) circle (1.5pt);
\draw (2.5,1) node[anchor=west] {$F_3$};

\draw (-0.3,1.05) node[anchor=north] {$a$};
\draw (-0.5,1.1) node[anchor=north] {$2$};

\draw (0.5,1.35) node[anchor=north] {$b$};
\draw (0.5,1.05) node[anchor=north] {$7$};

\draw (1.15,1.7) node[anchor=north] {$c$}; % u -- w
\draw (1.4,1.7) node[anchor=north] {$3$}; % u -- w

\draw (2.52,1.75) node[anchor=north] {$d$}; % w -- w
\draw (2.3,1.65) node[anchor=north] {$2$}; % w -- w

\draw (1.2,.9) node[anchor=north] {$e$}; % v -- w
\draw (1.35,.65) node[anchor=north] {$1$}; % v -- w
\end{scriptsize}
\end{tikzpicture}

}
\caption{Um grafo ponderado plano e seu dual.}
\label{fig:MSF-basico-1}
\end{figure}


Se~$G$ tem $n$ vértices e~$m$ arestas, o Teorema~\ref{teo:MSFEuler} mostra que o número de vértices do grafo dual é $\OTheta{n+m}$.

\begin{theorem}[\cite{EPPSTEIN-planar}]
\label{teo:MSFdual}
Seja~$T$ uma árvore geradora de um grafo plano conexo~$G$. O conjunto
$$
T^\star = \{e^\star:e\notin T\}
$$
é uma árvore geradora de~$G^\star$.
Além disso, se~$G$ for ponderado e adotarmos $w(e^\star) = w(e)$, então~$T$ será de peso mínimo se e somente se~$T^\star$ for de peso máximo e vice versa.
\end{theorem}

A Figura~\ref{fig:MSF-figura-2} mostra~$T$ e~$T^\star$ do grafo da Figura~\ref{fig:MSF-basico-1}.

\begin{figure}[htb]
\scalebox{1.5}{
\centering
\begin{tikzpicture}[line cap=round,line join=round,>=triangle 45,x=1cm,y=1cm]
\clip(-3.5,-.5) rectangle (6,3);


\draw [line width=2pt,color=cyan] (0,0) to[in=-135,out=135,looseness=1]  (0,2); % v -- u
\draw [line width=.5pt] (0,0) to[in=-135,out=135,looseness=1]  (0,2); % v -- u
%\draw [line width=1pt] (0,0) to[in=-45,out=45,looseness=1]  (0,2);   % v -- u
\draw [line width=2pt,color=cyan] (2,1) to  (0,0); % 4 -- 6
\draw [line width=.5pt] (2,1) to  (0,0); % 4 -- 6
\draw [line width=2pt,color=cyan] (2,1) to  (3,1); % 4 -- 6
\draw [line width=.5pt] (2,1) to  (3,1); % 4 -- 6
%\draw [line width=1pt] (2,1) to  (0,2); % 4 -- 6
%\draw [line width=1pt] (2,1.05) to[out=45,in=-45,looseness=50] (2,.95); % 4 -- 4

\draw [line width=2pt,color=red] (1,2.5) to[out=-45,in=45,looseness=1] (1,1); % F0 -- F1
\draw [line width=.5pt] (1,2.5) to[out=-45,in=45,looseness=1] (1,1); % F0 -- F1
%\draw [line width=1pt] (1,2.5) to[out=20,in=-70,looseness=10] (1,1); % F0 -- F1
%\draw [line width=1pt] (1,2.5) to[out=170,in=180,looseness=2.5] (0,1); % F0 -- F2
\draw [line width=2pt,color=red] (1,2.5) to[out=0,in=90,looseness=1] (3.5,1); % F0 -- F3
\draw [line width=.5pt] (1,2.5) to[out=0,in=90,looseness=1] (3.5,1); % F0 -- F3
\draw [line width=2pt,color=red] (1,1) to (0,1); % F1 -- F2
\draw [line width=.5pt] (1,1) to (0,1); % F1 -- F2

\begin{scriptsize}
\draw (-.5,1) node[anchor=south east,color=cyan] {$T$};
\draw (4,2.3) node[anchor=south east,color=red] {$T^\star$};
\draw [fill=black] (0,2) circle (1.5pt);
\draw (0,2) node[anchor=south east] {$u$};


\draw [fill=black] (0,0) circle (1.5pt);
\draw (0,0) node[anchor=north east] {$v$};
\draw [fill=black] (2,1) circle (1.5pt);
\draw (2,1) node[anchor=south] {$z$};
\draw [fill=black] (3,1) circle (1.5pt);
\draw (3,1) node[anchor=south] {$y$};


\draw [fill=black] (1,2.5) circle (1.5pt);
\draw (1,2.5) node[anchor=south east] {$F_0$};
\draw [fill=black] (1,1) circle (1.5pt);
\draw (.9,1) node[anchor=south] {$F_1$};
\draw [fill=black] (0,1) circle (1.5pt);
\draw (0,1) node[anchor=south] {$F_2$};
\draw [fill=black] (3.5,1) circle (1.5pt);
\draw (3.5,1) node[anchor=west] {$F_3$};


\draw (-0.3,1.05) node[anchor=north] {$a$};
\draw (-0.5,1.1) node[anchor=north] {$2$};

\draw (0.55,1.35) node[anchor=north] {$b^\star$};
\draw (0.5,1.05) node[anchor=north] {$7$};

\draw (1.15,1.9) node[anchor=north] {$c^\star$}; % u -- w
\draw (1.45,1.85) node[anchor=north] {$3$}; % u -- w

\draw (1.2,1) node[anchor=north] {$d$}; % v -- w
\draw (1.35,.65) node[anchor=north] {$1$}; % v -- w

\draw (2.5,1.4) node[anchor=north] {$g$}; % v -- w
\draw (2.5,1) node[anchor=north] {$4$}; % v -- w

\draw (2.55,2.7) node[anchor=north] {$f^\star$}; % w -- w
\draw (2.4,2.3) node[anchor=north] {$2$}; % w -- w

\end{scriptsize}
\end{tikzpicture}

}
\caption{Árvore geradora~$T$ de peso mínimo do grafo da Figura~\ref{fig:MSF-basico-1} e árvore correspondente~$T^\star$ do grafo.}
\label{fig:MSF-figura-2}
\end{figure}



Dado um grafo~$G$, um \defi{corte} é um conjunto de arestas cuja remoção aumenta o conjunto de componentes conexas de~$G$.
Dado $F$ uma floresta maximal de~$G$ e~$uv$ uma aresta de~$F$, então existe um corte associado ao par $(F, uv)$ dado pela união de $uv$ com as arestas que reconectam as duas árvores geradas pela remoção de $uv$.


\begin{theorem}
\label{teo:cutset}
Seja~$G$ um grafo planar, $F$ uma floresta maximal de~$G$ e~$uv$ uma aresta de~$F$, então o conjunto
$$
(F, uv)^\star := \{e^\star:e\in (F, uv)\}
$$
forma um ciclo em $G^\star$.
\end{theorem}


\section{Definição do problema}
\label{sec:definition-MSF}
O problema da floresta maximal de peso mínimo (MSF) em grafos dinâmicos planos ponderados consiste na busca por uma implementação tão eficiente quanto possível para a seguinte biblioteca:

\begin{itemize}
\item \MSFCreate($n$): Cria e devolve um grafo plano ponderado~$G$ com $n$ vértices isolados.
\item \MSFaddEdge($G$, $e$, $u$, $e_u$, $v$, $e_v$, $w$): Insere em~$G$ uma nova aresta~$e$ com peso~$w$ ligando os vértices~$u$ e~$v$. A nova aresta~$e$ é sucessora das arestas~$e_u$ e~$e_v$ nas ordenações cíclicas de~$u$ e~$v$, respectivamente.
\item \MSFdelEdge($G$, $e$): Remove a aresta~$e$ de~$G$.
\item \MSFupdate($G$, $e$, $w$): Altera o peso da aresta~$e$ do grafo dinâmico~$G$ para o valor~$w$.
\item \MSFweight($G$): Devolve o peso de uma MSF de $G$.
\end{itemize}


\section{Árvores dinâmicas planas}
\label{sec:EODT}

Para resolver o problema da floresta geradora de peso mínimo em grafos dinâmico planos ponderados, Eppstein et al.~\cite{EPPSTEIN-planar} propõem o uso de uma estrutura de dados que chamaremos de \defi[arvore@\'arvore!dinâmica plana]{Árvores dinâmicas planas} (ADP), originalmente os autores nomearam essa estrutura de \textit{edge ordered dynamic tree}.

Seguindo a descrição de Eppstein et al.~\cite[Seção 3]{EPPSTEIN-planar}, essa estrutura consiste de árvores em que cada nó possui um peso, juntamente com uma descrição combinatória plana das árvores.
Como árvores não possuem arestas paralelas, consideraremos que a ordenação cíclica para cada nó~$u$ é uma sequência dos nós vizinhos de~$u$ no ordem anti-horária em vez de arestas.

Árvores dinâmicas planas dão suporte à seguinte biblioteca:
\begin{itemize}
\item \LCOMakeNode(): Cria e retorna um novo nó.
\item \LCOLink($v$, $u$): Adiciona uma aresta ligando os nós~$v$ a~$u$, tornando~$v$ um filho de~$u$.
O nó $w$ é inserido no final de~$D(v)$ e~$v$ no inicio de~$D(w)$.
Essa operação assume que~$v$ é a raiz de sua árvore e que $v$ e $w$ estão em árvores distintas.
\item \LCOMerge($u$, $v$): Recebe dois nós~$u$ e~$v$ de mesmo peso e retorna um único nó~$z$ que é a união desses nós.
Se~$\alpha$ e~$\beta$ são respectivamente as ordenações cíclicas de arestas incidentes a~$u$ e a~$v$ seguindo a descrição combinatória plana, então $\alpha\beta$ é a lista de arestas incidentes a~$z$.
\item \LCOSplit($u$, $e$): Recebe nós adjacentes~$u$ e~$e$.
Substitui o nó~$u$ por dois nós~$v$ e~$z$.
Se a ordem cíclica $D(u)$ tem forma $\alpha e\beta$, então $\alpha e$ e~$\beta$ são as ordens cíclicas de~$v$ e~$w$, respectivamente.
\item \LCOCycle($v$, $e$): Permuta ciclicamente $D(v)$ de forma que o nó $e$ é o último na ordem. Se a ordem inicial tem forma $\alpha e \beta$, então a ordem resultante é $\beta\alpha e$.
\item \LCORoot($v$): Retorna a raiz da árvore do nó~$v$.
\item \LCOAddCost($v$, $w$): Atribui o peso~$w$ ao nó~$v$.
\item \LCOMax($u$, $v$): Retorna o nó de peso máximo no percurso entre os nós~$u$ e~$v$.
\item \LCOMin($u$, $v$): Retorna o nó de peso mínimo no percurso entre os nós~$u$ e~$v$.
\item \LCOConnected($u$, $v$): Retorna verdadeiro se os nós~$u$ e~$v$ estiverem na mesma componente conexa e falso caso contrário.
\end{itemize}

A estrutura também se aplica a grafos planos desconexos, funcionando como a união independente das estruturas correspondentes para cada componente do grafo.

\section{Resolvendo MSF com árvores dinâmicas planas}

Representaremos cada componente conexa~$C$ do grafo dinâmico plano ponderado~$G$ em um plano distinto.
Dessa forma, cada componente possui uma face exterior própria.

Cada componente conexa~$C$ de $G$ é representada por duas ADPs.
Uma delas representa uma árvore geradora de pesos mínimos~$T$ de $C$.
Enquanto que a outra representa sua correspondente dual~$T^\star$.

A primeira árvore, que é denotada por~$\hat T$, é essencialmente a árvore~$T$ submetida a duas transformações: cada uma de suas aresta é subdivida; e é adicionado uma folha a cada vértice incidente a uma aresta de~$C$ que não esteja em~$T$.

Mais precisamente, $\hat T$ possui um nó~$\hat v$ para cada vértice~$v$ de~$C$, um nó~$\hat e$ para cada aresta~$e$ de~$T$ e dois nós~$\hat d_1$ e~$\hat d_3$ para cada aresta~$d^\star$ de~$T^\star$.
O peso do nó~$\hat v$ é $-\infty$, enquanto que os pesos de~$\hat e$, $\hat d_1$ e~$\hat d_3$ são os pesos das arestas representadas por esses nós. 
Se uma aresta $e$ de $T$ é incidente aos vértices~$v$ e~$u$ de~$C$, então $\hat e$ é ligado aos nós~$\hat v$ e~$\hat u$.
Se uma aresta~$d$ de~$C$ não está em $T$ e é incidente aos vértices~$v$ e~$u$ de~$C$, então $\hat d_0$ e $\hat d_2$ são ligado a~$\hat v$ e~$\hat u$, respectivamente.
Note que $\hat d_0$ e $\hat d_2$ não estão ligados, logo são folhas da ADP.

A segunda ADP, denotada por~$\hat T^\star$  possui construção análoga à primeira, mas é baseada em~$T^\star$ ao invés de~$T$.
A ADP possui um nó $\hat F$ para cada vértice~$F$ de~$C^\star$ (isto é, cada face de~$C$), um nó~$\hat e^\star$ para cada aresta~$e^\star$ de $T^\star$ e dois nós $\hat d_1$ e~$\hat d_3$ para cada aresta~$d$ de $T$.
O peso do nó~$\hat F$ é $\infty$, enquanto que os pesos de $\hat e^\star$, $\hat d_1$ e~$\hat d_3$ são os pesos das arestas representadas por esses nós.
Se uma aresta $e^\star$ de $T^\star$ é incidente às faces~$F$ e~$G$ de~$C$, então $\hat e^\star$ é ligado ao nós~$\hat F$ e~$\hat G$.
Se uma aresta $d^\star$ de~$C^\star$ não está em $T^\star$ e é incidente às faces~$F$ e~$G$ de~$C$, então $\hat d_1$ e $\hat d_3$ são ligado a~$\hat F$ e~$\hat G$ respectivamente.
A Figura~\ref{fig:MSF-figura-3} ilustra $\hat T$ e $\hat T^\star$.

\begin{figure}[htb]
%\scalebox{2}{
\centering
\begin{tikzpicture}[line cap=round,line join=round,>=triangle 45,x=1cm,y=1cm]
\clip(-2,-1) rectangle (9,6);

\draw [line width=2pt,color=cyan] (0,0) to[in=-135,out=135,looseness=1]  (0,4); % v -- u
\draw [line width=.5pt] (0,0) to[in=-135,out=135,looseness=1]  (0,4); % v -- u
%\draw [line width=1pt] (0,0) to[in=-45,out=45,looseness=1]  (0,4);   % v -- u

\draw [line width=2pt,color=cyan] (4,2) to  (0,0); % 4 -- 6
\draw [line width=.5pt] (4,2) to  (0,0); % 4 -- 6
%\draw [line width=1pt] (4,2) to  (0,4); % 4 -- 6
%\draw [line width=1pt] (4,2.05) to[out=45,in=-45,looseness=100] (4,1.95); % w -- w

\draw [line width=2pt,color=red] (2,5) to[out=-45,in=45,looseness=1] (2,2); % F0 -- F1
\draw [line width=.5pt] (2,5) to[out=-45,in=45,looseness=1] (2,2); % F0 -- F1
%\draw [line width=1pt] (2,5) to[out=20,in=-70,looseness=7.5] (2,2); % F0 -- F1
%\draw [line width=1pt] (2,5) to[out=170,in=180,looseness=2] (0,2); % F0 -- F2
\draw [line width=2pt,color=red] (2,5) to[out=0,in=90,looseness=1] (7,2); % F0 -- F3
\draw [line width=.5pt] (2,5) to[out=0,in=90,looseness=1] (7,2); % F0 -- F3

\draw [line width=2pt,color=red] (2,2) to (0,2); % F1 -- F2
\draw [line width=.5pt] (2,2) to (0,2); % F1 -- F2

\draw [line width=2pt,color=red] (2,5) to[out=0,in=90,looseness=1] (5,2.5); % F0 -- F0
\draw [line width=.5pt] (2,5) to[out=0,in=90,looseness=1] (5,2.5); % F0 -- F0


%F0 -- e1
\draw [line width=2pt,color=red] (2,5) to[out=0,in=90,looseness=1] (8.3,2); % F0 -- F0
\draw [line width=2pt,color=red] (8.3,2) to[out=-90,in=-90,looseness=2] (5, 1.5); % F0 -- F0
\draw [line width=.5pt] (2,5) to[out=0,in=90,looseness=1] (8.3,2); % F0 -- F0
\draw [line width=.5pt] (8.3,2) to[out=-90,in=-90,looseness=2] (5, 1.5); % F0 -- F0

\begin{scriptsize}

% Aresta a
\draw [line width=2pt,color=red] (0,2) to  (-.6,2); 
\draw [line width=.5pt] (0,2) to  (-.6,2); 
\draw [line width=2pt,color=red] (-1.1,2) to[out=135,in=180,looseness=1.5]  (2,5);   % v -- u
\draw [line width=.5pt] (-1.1,2) to[out=135,in=180,looseness=1.5]  (2,5);   % v -- u
\draw [fill=black] (-.6,2) circle (1.5pt);
\draw [fill=black] (-.83,2) circle (1.5pt);
\draw [fill=black] (-1.1,2) circle (1.5pt);


% Aresta b
\draw [line width=2pt,color=cyan] (0,0) to[in=-90,out=45,looseness=1]  (.8,1.7);   % v -- u
\draw [line width=.5pt] (0,0) to[in=-90,out=45,looseness=1]  (.8,1.7);   % v -- u

\draw [line width=2pt,color=cyan] (0,4) to[out=-45,in=90,looseness=1]  (.8,2.3);   % v -- u
\draw [line width=.5pt] (0,4) to[out=-45,in=90,looseness=1]  (.8,2.3);   % v -- u
\draw [fill=black] (1,2) circle (1.5pt);
\draw [fill=black] (.8,2.3) circle (1.5pt);
\draw [fill=black] (.8,1.7) circle (1.5pt);


% Aresta c

\draw [line width=2pt,color=cyan] (0,4) to  (2.2,2.9);   
\draw [line width=.5pt] (0,4) to  (2.2,2.9);   
\draw [line width=2pt,color=cyan] (2.7,2.65) to  (4,2);   
\draw [line width=.5pt] (2.7,2.65) to  (4,2);   
\draw [fill=black] (2.2,2.9) circle (1.5pt);
\draw [fill=black] (2.54,2.9) circle (1.5pt);
\draw [fill=black] (2.7,2.65) circle (1.5pt);

% Aresta d
\draw [line width=2pt,color=red] (2,2) to  (2.2,1.5);   
\draw [line width=.5pt] (2,2) to  (2.2,1.5);   

%\draw [line width=2pt,color=red] (2.5,.9) to[out=-45,in=0,looseness=4.1]  (2,5);   
%\draw [line width=.5pt] (2.5,.9) to[out=-45,in=0,looseness=4.1]  (2,5);   

\draw [line width=2pt,color=red] (2.5,.9) to[out=-45,in=-90,looseness=1.5]  (8.5,2);   
\draw [line width=.5pt] (2.5,.9) to[out=-45,in=-90,looseness=1.5]  (8.5,2);   
\draw [line width=2pt,color=red] (2,5) to[out=10,in=90,looseness=1]  (8.5,2);   
\draw [line width=.5pt] (2,5) to[out=10,in=90,looseness=1]  (8.5,2);   



\draw [fill=black] (2.35,1.175) circle (1.5pt);
\draw [fill=black] (2.2,1.5) circle (1.5pt);
\draw [fill=black] (2.5,.9) circle (1.5pt);

% Aresta e
\draw [line width=2pt,color=cyan] (4,2) to  (6,2);   
\draw [line width=.5pt] (4,2) to  (6,2);   
\draw [fill=black] (5,2.5) circle (1.5pt);
\draw [fill=black] (5,2) circle (1.5pt);
\draw [fill=black] (5,1.5) circle (1.5pt);

% Aresta f

\draw [line width=2pt,color=cyan] (6,2) to  (6.5,2.5);   
\draw [line width=.5pt] (6,2) to  (6.5,2.5);   
\draw [line width=2pt,color=cyan] (6,2) to[out=-45,in=-45,looseness=3]  (7.4,2.8);   
\draw [line width=.5pt] (6,2) to[out=-45,in=-45,looseness=3]  (7.4,2.8);   
\draw [fill=black] (6.9,2.75) circle (1.5pt);
\draw [fill=black] (6.5,2.5) circle (1.5pt);
\draw [fill=black] (7.4,2.8) circle (1.5pt);


\draw [fill=black] (0,4) circle (1.5pt);
\draw (0,4) node[anchor=south east] {$\hat u$};

\draw [fill=black] (0,0) circle (1.5pt);
\draw (0,0) node[anchor=north east] {$\hat v$};
\draw [fill=black] (4,2) circle (1.5pt);
\draw (3.95,2) node[anchor=south] {$\hat z$};
\draw [fill=black] (6,2) circle (1.5pt);
\draw (6,2) node[anchor=south] {$\hat y$};

\draw [fill=black] (2,5) circle (1.5pt);
\draw (2,5) node[anchor=south east] {$\hat F_0$};
\draw [fill=black] (2,2) circle (1.5pt);
\draw (2,2) node[anchor=west] {$\hat F_1$};
\draw [fill=black] (0,2) circle (1.5pt);
\draw (0,2) node[anchor=south] {$\hat F_2$};
\draw [fill=black] (7,2) circle (1.5pt);
\draw (7,2) node[anchor=west] {$\hat F_3$};


\draw (-.7,2.4) node[anchor=north] {$\hat a$};
\draw (-.5,2) node[anchor=north] {$\hat a_1$};
\draw (-1.2,2.05) node[anchor=north] {$\hat a_3$};
%\draw (-1,2.4) node[anchor=north] {$2$};

\draw (1.1,2.5) node[anchor=north] {$\hat b$};
\draw (.6,2.7) node[anchor=north] {$\hat b_0$};
\draw (.6,2) node[anchor=north] {$\hat b_2$};
%\draw (1,2) node[anchor=north] {$7$};

\draw (2.45,3.3) node[anchor=north] {$\hat c$}; % u -- w
\draw (2.8,3.05) node[anchor=north] {$\hat c_0$}; % u -- w
\draw (2.1,3) node[anchor=north] {$\hat c_2$}; % u -- w
%\draw (2.55,2.7) node[anchor=north] {$3$}; % u -- w

\draw (2.45,1.6) node[anchor=north] {$\hat d$}; % v -- w
\draw (2,1.7) node[anchor=north] {$\hat d_3$}; % v -- w
\draw (2.4,1) node[anchor=north] {$\hat d_1$}; % v -- w
%\draw (2.5,1.3) node[anchor=north] {$1$}; % v -- w

\draw (5.1,2.4) node[anchor=north] {$\hat e$}; % v -- w
\draw (5.2,2.8) node[anchor=north] {$\hat e_3$}; % v -- w
\draw (5.2,1.8) node[anchor=north] {$\hat e_1$}; % v -- w
%\draw (2.5,1.3) node[anchor=north] {$1$}; % v -- w

\draw (7.04,3.2) node[anchor=north] {$\hat f$}; % w -- w
\draw (7.4,3.3) node[anchor=north] {$\hat f_2$}; % w -- w
\draw (6.5,3) node[anchor=north] {$\hat f_0$}; % w -- w
%\draw (4.7,3) node[anchor=north] {$\2$}; % w -- w

\end{scriptsize}
\end{tikzpicture}

%}
\caption{As árvores geradoras modificadas~$\hat T$ e~$\hat T^\star$ construídas a partir das árvores da Figura~\ref{fig:MSF-figura-2}.}
\label{fig:MSF-figura-3}
\end{figure}
Manteremos essas duas ADPs e duas tabelas de símbolos, $H_v$ e $H_e$. $H_v$ guarda o conjunto de nós que representam vértices e usa como chave os identificadores dos vértices.
$H_e$ guarda informações sobre as arestas de $G$, para cada aresta~$e$ retorna a quadrupla $(\varname{primal}$, $\hat e$, $\hat e_1$, $\hat e_3$), onde $\varname{primal}$ é um booleado que é igual a verdadeiro se~$e$ é uma aresta de $T$ e falso caso contrário. $\hat e$ é o nó que representa $e$ ou $e^\star$ e $\hat e_1$, $\hat e_3$ são as duas folhas associadas a~$e$. 

A primeira rotina de MSF que vamos apresentar é \MSFupdate{}, cuja implementação pode ser vista no Algoritmo~\ref{Algo:MSFupdate}.

Quando atualizamos o peso de uma aresta~$e$ para um novo peso~$w$, precisamos atualizar a árvore geradora de pesos mínimos~$T$ que está sendo mantida de forma que ela continue sendo de peso mínimo.

Se~$e$ for uma aresta de~$T$, é necessário verificar se não há alguma aresta no corte~$(T, e)$ com peso menor do que~$w$, pois se tal aresta existir, então~$T$ e~$T^\star$ não serão mais de peso mínimo e máximo, respectivamente.
Para corrigir isso, seja~$d$ a aresta de menor peso no corte~$(T, e)$,
então $e$ é removida de~$T$ e $e^\star$ adicionada a~$T^\star$ e~$d^\star$ é removida de~$T^\star$ e $d$ é adicionada a~$T$.

Pelo Teorema~\ref{teo:cutset}, o conjunto~$(T, e)^\star$ forma um ciclo em~$C^\star$.
Como~$e^\star \in (T, e)^\star$ e~$e$ é a única aresta de~$T$ cujo dual está em~$(T, e)^\star$, então as demais arestas desse corte formam um caminho em~$T^\star$ ligando os vértices incidentes a~$e^\star$.

Dessa forma, cada aresta $d\in (T, e)$ de menor peso é representada por exatamente um nó nesse percurso que possui peso mínimo nesse percurso em~$\hat T^\star$.
Para obter o nó $\hat d$ de menor peso percorrer esse caminho, usaremos os nós $\hat e_1$ e $\hat e_3$ e a rotina \LCOMin{} da biblioteca de ADPs, isso é feito na linha~\ref{Algo:MSFupdate:linhamin} do Algoritmo~\ref{Algo:MSFupdate}.

As remoções de~$e$ de~$T$ e $d^\star$ de~$T^\star$ são feitas nas linhas~\ref{Algo:MSFupdate:linhasplitd} e~\ref{Algo:MSFupdate:linhasplite} utilizando a rotina~\LCOSplit{}, que converte os nós~$\hat e$ e~$\hat d$ nos nós $\hat d_1$, $\hat d_3$, $\hat e_0$ e~$\hat e_2$, que se tornarão as novas folhas associadas às arestas~$e$ e~$d$.
Para adicionar as arestas~$e^\star$ a~$T^\star$ e $d$ a~$T$ são usados as folhas~$\hat d_0$, $\hat d_2$, $\hat e_1$ e~$\hat e_3$ junto a rotina \LCOMerge{} nas linhas \ref{Algo:MSFupdate:linhas:Merged} e~\ref{Algo:MSFupdate:linhas:Mergee} do Algoritmo~\ref{Algo:MSFupdate}.

Se~$e$ não for uma aresta de~$T$, então teremos que~$e^\star$ é uma aresta de~$T^\star$.
O tratamento desse caso é análogo ao caso anterior e é feito entre as linhas~\ref{Algo:MSFupdate:dualinicio} e~\ref{Algo:MSFupdate:dualfim} do Algoritmo~\ref{Algo:MSFupdate}.

\begin{algorithm}[htb]
\caption{\MSFupdate($G$, $e$, $w$)}
\label{Algo:MSFupdate}
\begin{algorithmic}[1]
\State primal, $\hat e$, $\hat e_1$, $\hat e_3$, $\hat v$, $\hat u$ $\gets$ $H_e$($e$)
\State \LCOAddCost($\hat e$, $w$); \LCOAddCost($\hat e_1$, $w$); \LCOAddCost($\hat e_2$, $w$)
\If {primal}
\State $d$ $\gets$ \LCOMin($\hat e_1$, $\hat e_3$)\label{Algo:MSFupdate:linhamin}

\State primal, $\hat d$, $\hat d_0$, $\hat d_2$, $\hat x$, $\hat y$ $\gets$ $H_e$($d$)
\If {$\hat d$.$w$ < $w$}
\State $\hat d_1$, $\hat d_3$ $\gets$ \LCOSplit($\hat d$, $\hat x$)\label{Algo:MSFupdate:linhasplitd}
\State $\hat e_0$, $\hat e_2$ $\gets$ \LCOSplit($\hat e$, $\hat v$)\label{Algo:MSFupdate:linhasplite}


\State $\hat d$ $\gets$ \LCOMerge( $\hat d_0$, $\hat d_2$)\label{Algo:MSFupdate:linhas:Merged}
\State $\hat e$ $\gets$ \LCOMerge( $\hat e_1$, $\hat e_3$)\label{Algo:MSFupdate:linhas:Mergee}
\State $H(e)$ $\gets$ falso, $\hat e$, $\hat e_0$, $\hat e_2$, $\hat v$, $\hat u$
\State $H(d)$ $\gets$ verdadeiro, $\hat d$, $\hat d_1$, $\hat d_3$, $\hat x$, $\hat y$

\EndIf

\Else
\State $d$ $\gets$ \LCOMax($\hat e_1$, $\hat e_3$)\label{Algo:MSFupdate:dualinicio}

\State primal, $\hat d$, $\hat d_0$, $\hat d_2$, $\hat x$, $\hat y$ $\gets$ $H$($d$)
\If {$\hat d$.$w$ > $w$}
\State $\hat d_1$, $\hat d_3$ $\gets$ \LCOSplit($\hat d$, $\hat x$)
\State $\hat e_0$, $\hat e_2$ $\gets$ \LCOSplit($\hat e$, $\hat v$))

\State $\hat d$ $\gets$ \LCOMerge( $\hat d_0$, $\hat d_2$)
\State $\hat e$ $\gets$ \LCOMerge( $\hat e_1$, $\hat e_3$)

\State $H(e)$ $\gets$ verdadeiro, $\hat e$, $\hat e_0$, $\hat e_2$, $\hat v$, $\hat u$
\State $H(d)$ $\gets$ falso, $\hat d$, $\hat d_1$, $\hat d_3$, $\hat x$, $\hat y$

\EndIf
\EndIf\label{Algo:MSFupdate:dualfim}
\end{algorithmic}
\end{algorithm}


A próxima rotina elaborada é \MSFdelEdge{}, cuja implementação pode ser vista no Algoritmo~\ref{Algo:MSFdelEdge}.

Pelo Lema~\ref{lemma:diestel}, ou uma aresta~$e$ não está em ciclos e as duas faces incidentes a ela são iguais, ou essas faces são distintas e a aresta está em pelo menos um ciclo.

No primeiro caso, a remoção de~$e$ desconecta $C$.
Logo não há trabalho de manutenção em $\hat T$ além de remover~$e$ de~$\hat T$.
A face incidente a aresta~$e$ é a face exterior~$F$.
Como cada componente conexa está imersa em um plano distinto, será necessário dividir $\hat F$ em dois nós, que representarão as duas faces exteriores das duas componentes conexas resultantes da remoção de~$e$.


No segundo caso, a remoção de~$e$ não desconecta $C$.
Dessa forma, é necessário buscar uma substituta para $e$ caso essa aresta seja uma aresta de $T$.
Para fazer essa busca eficientemente, na linha~\ref{Algo:MSFdelEdge:linha:mudaPeso} do Algoritmo~\ref{Algo:MSFdelEdge}, o peso de~$e$ é mudado para~$\infty$, fazendo com que essa aresta se torne uma aresta de~$T^\star$. 
A remoção de $e$ implica na junção das duas faces distintas $F_0$ e $F_2$ incidentes a essa aresta.
Essa mesclagem é feita de forma que a descrição combinatória resultante se mantenha plana, isto é, se $\alpha e^\star \beta$ e $\gamma e^\star \delta$ forem as ordenações de $F_0$ e $F_2$, respectivamente, então $\alpha \delta\gamma\beta$ é a ordenação ciclica da face resultante.

Para implementar essa rotina será necessário incrementar a biblioteca de árvores dinâmicas planas com uma rotina que relacione $\hat T$ com $\hat T^\star$.
A rotina adicional se chama \LCOLeftFace{} e associa os nós que são folhas com nós que representam vértices e faces.

Dada uma orientação de uma aresta~$e$ de~$C$, é possível definir a \defi[face!esquerda]{face esquerda} de uma aresta.\TODO{consultar definição de face esquerda/direita na literatura}

Cada folha pode ser visto como a orientação de uma aresta de $C$ ou $C^\star$.
Assim possuindo uma face a esquerda bem definida.

\begin{itemize}
\item \LCOLeftFace($e$): Recebe um nó folha e retorna o nó que representa a face a esquerda dessa folha.
\end{itemize}

A primeira coisa feita é mudar o peso de $e$ para $\infty$, assim tornando-a uma aresta de~$\hat T^\star$.
Dessa forma, as duas folhas~$\hat e_0$ e~$\hat e_2$ são nós de~$\hat T$.

Em seguida, a nova rotina \LCOLeftFace{} é usada para obter os dois nós que representam as faces incidentes à aresta $e$.


\begin{algorithm}[htb]
\caption{\MSFdelEdge($G$, $e$)}
\label{Algo:MSFdelEdge}
\begin{algorithmic}[1]
\State \MSFupdate($G$, $e$, $\infty$)\label{Algo:MSFdelEdge:linha:mudaPeso}
\State \varname{primal}, $\hat e$, $\hat e_0$, $\hat e_2$, $\hat u$, $\hat v$ $\gets$ $H_e(e)$
\State $\hat F_0$ $\gets$ \LCOLeftFace($\hat e_0$)
\State $\hat F_2$ $\gets$ \LCOLeftFace($\hat e_2$)
\State
\State \LCOCycle($\hat u$, $\hat e_0$)
\State \LCOSplit($\hat u$, \LCOPredecessor($\hat u$, $\hat e_0$))
\State \LCOCycle($\hat v$, $\hat e_2$)
\State \LCOSplit($\hat v$, \LCOPredecessor($\hat v$, $\hat e_2$))
\State
\State \LCOCycle($\hat F_0$, $\hat e$)
\State \LCOSplit($\hat F_0$, \LCOPredecessor($\hat F_0$, $\hat e$))
\State \LCOCycle($\hat F_2$, $\hat e$)
\State \LCOSplit($\hat F_2$, \LCOPredecessor($\hat F_2$, $\hat e$))
\State
\State \LCOMerge($\hat F_0$, $\hat F_2$)
\end{algorithmic}
\end{algorithm}

\begin{algorithm}[htb]
\caption{\MSFaddEdge($G$, $e$, $u$, $e_u$, $v$, $e_v$, $w$)}
\label{Algo:MSFaddEdge}
\begin{algorithmic}[1]

\State $\hat u$ $\gets$ $H_v(u)$; $\hat v$ $\gets$ $H_v(v)$
\State $\varname{primal}_u$, $\hat e_u$, $\hat e_{u0}$, $\hat e_{u2}$, $\hat u$, $\hat x$ $\gets$ $H_e(e_u)$
\State $\varname{primal}_v$, $\hat e_v$, $\hat e_{v0}$, $\hat e_{v2}$, $\hat v$, $\hat y$ $\gets$ $H_e(e_v)$
\For {$i \in \{0,1,2,3\}$}
\State $\hat s_i$ $\gets$ \LCOMakeNode(); $\hat e_i$ $\gets$ \LCOMakeNode()
\State  \LCOAddCost($\hat e_i$, $w$);
\State \LCOLink($\hat e_i$, $\hat s_i$)
\EndFor
\State \LCOAddCost($\hat s_0$, $-\infty$); \LCOAddCost($\hat s_2$, $-\infty$);
\State \LCOAddCost($\hat s_1$, $\infty$); \LCOAddCost($\hat s_3$, $\infty$);

\If{\LCOConnected($\hat u$, $\hat v$)}



\Else
\If{$\varname{primal}_u$}
\State \LCOCycle($\hat u$, $\hat e_u$)
\Else
\State \LCOCycle($\hat u$, $\hat e_{u0}$)
\EndIf
\State \LCOMerge($\hat u$, $\hat s_0$)
\State \LCOMerge($\hat v$, $\hat s_v$)
\State \LCOCycle($\hat u$, $\hat e_u$)
\State \MSFupdate($G$, $e$, $w$)
\EndIf
\end{algorithmic}
\end{algorithm}



\section{Estruturas auxiliares}
\subsection{Árvores binárias de busca com chave implícita}
\subsection{Link-Cut Trees}
\label{sec:linkcuttree}
Link-Cut Trees são uma estrutura de dados usada para representar e manipular florestas dinâmicas enraizadas.
Elas foram originalmente introduzidas por Sleator e Tarjan~\cite{SleatroTarjanLinkCutTree1983,} em~1983.
Elas permitem as seguintes operações
\begin{itemize}
\item \linkcutCreate(): Cria e retorna um novo nó de link-cut tree.
\item \linkcutAddEdge($v$, $w$): Adiciona uma aresta de~$v$ a~$w$, tornando~$v$ um filho de~$w$. Essa operação assume que~$v$ é a raiz de uma árvore~$T$ e que $w$ não é um nó de~$T$.
\item \linkcutDelEdge($v$): Remove a aresta de~$v$ para seu pai.
\item \linkcutEvert($v$): Torna~$v$ a raiz de sua árvore revertendo o caminho de~$v$ para a raiz original.
\item \linkcutMax($v$): Retorna o nó de peso máximo no caminho entre~$v$ e a raiz de sua árvore.
\item \linkcutMin($v$): Retorna o nó de peso mínimo no caminho entre~$v$ e a raiz de sua árvore.
\end{itemize}



Cada nó em uma Link-Cut Tree é associado a um nó em uma árvore que está sendo representada. Esses nós mantêm informações sobre seus pais e filhos, além de possíveis conexões reversas devido à natureza dinâmica da estrutura. A estrutura interna das Link-Cut Trees é composta por árvores de busca binária, especificamente utilizando uma variante conhecida como Splay Tree. Essa organização permite que operações como acesso, corte, e união de árvores sejam realizadas de forma eficiente, aproveitando as propriedades de balanceamento e acesso rápido das Splay Trees.

Link-Cut Trees são especialmente úteis em algoritmos que necessitam de operações dinâmicas sobre árvores, como em problemas de análise de conectividade em grafos dinâmicos, gerenciamento de redes, e em várias otimizações de algoritmos relacionados a árvores. A capacidade de realizar operações complexas de forma eficiente torna as Link-Cut Trees uma ferramenta poderosa em computação.

Para uma introdução sobre link-cut trees, recomendamos o trabalho de~\cite{linkcuttree}.



\section{Implementação de árvores dinâmicas planas}


Seja $\hat v$ um nó de ADP com ordem cíclica $\langle e_1, e_2, \ldots, e_k\rangle$.
O nó $\hat v$ é implementado por um conjunto de subnós de link cut trees.
Nesse conjunto há um subnó~$v_{e_i}$ para cada aresta~$e_i$ incidente a~$\hat v$.
Esse subnó representa a correspondente aresta na ordem cíclica de~$\hat v$.

Cada subnó $v_{e_i}$ é ligado aos subnós que representam a aresta sucessora e predecessora de~$e_i$ na ordem cíclica de~$\hat v$.
Com exceção de~$e_1$ e~$e_k$, que são ligados somente ao seu sucessor e predecessor, respectivamente.
Além disso, se a aresta~$e_i$ liga os nós~$\hat v$ e~$\hat u$, então~$v_{e_i}$ também é ligado ao subnó~$u_{e_i}$ que representa~$e_i$ na ordem cíclica de~$\hat u$.

Cada nó $\hat v$ possuirá dois ponteiros: $v_{first}$ e $v_{last}$.
$v_{first}$ apontará para o primeiro subnó da ordem cíclica, isto é, $s_{e_1}$ e $v_{last}$ para o último, $s_{e_k}$.
Essa estrutura tem a caracteristica de que se for feito \linkcutEvert($\hat u$.$last$), então o caminho de $\hat v$.$first$ até a raiz corresponde à sequência $\langle s_{e_1}, s_{e_2}, \ldots, s_{e_k}\rangle$.

Além disso, a sequência de subnós $\langle s_{e_1}, s_{e_2}, \ldots, s_{e_k}\rangle$ será mantida em uma árvore binária de busca com chaves ímplicitas e a raiz dessa ABB possui um campo $\node$ que aponta para o nó~$\hat v$.

A Figura~\TODO{bla} mostra.

Para implementar árvores dinâmicas planas utilizando link-cut trees e árvores de busca binária de chave implicita, são necessárias duas rotinas auxiliares, chamadas \LCOFindNode{} e \LCOFindSubNode{}, que estabelecem a relação entre os nós da ADP e os subnós das link cut trees.
Essas rotinas são fundamentais para manter a coerência entre as operações nos subnós da link cut tree e a estrutura dos nós da ADP.

\begin{itemize}
\item \LCOFindNode($s$): Recebe um sub-nó $s$ e retorna o nó que contêm $s$ em sua ordem cíclica.
\item \LCOFindSubNode($\hat v$, $\hat e$): Recebe dois nós e retorna o subnó $s$ da ordem cíclica de $\hat v$ que representa a aresta que liga~$\hat v$ a~$\hat e$.
\end{itemize}


A implementação de \LCOFindNode{} é simples e usa a estrutura de árvores binárias de busca com chave ímplicita.
Dado o subnós $s$, sabemos que ele está em alguma ABB e que a raiz dessa árvore aponta para o nó que contém~$s$ em sua ordem cíclica, logo usamos \treapGetRoot{} da biblioteca de ABBs para obter a raiz~$r$ da árvore e em seguida é retornado~$r$.$\node$.

\begin{algorithm}[htb]
\caption{\LCOFindNode($s$)}
\label{Algo:LCOFindNode}
\begin{algorithmic}[1]
\State $r$ $\gets$ \treapGetRoot($s$)
\State \Return $r$.$\node$
\end{algorithmic}
\end{algorithm}




\begin{algorithm}[htb]
\caption{\LCOFindSubNode($\hat v$, $\hat e$)}
\label{Algo:LCOFindSubNode}
\begin{algorithmic}[1]
\If{ $e$ $\in$ $T$}
\State $v_{first}$, $v_{last}$ $\gets$ $\hat v$
\State $e_{first}$, $e_{last}$ $\gets$ $\hat e$
\State \linkcutEvert($v_{first}$)
\If{$e_{first}$.$parent$ $\neq$ $e_{last}$}
\State \Return $e_{first}$.$parent$
\EndIf
\State \Return $e_{last}$.$parent$
\Else
\If { \LCOFindNode($e_{first}.rot.parent$) $=$ $v$  }\Comment{$e_{first}$ e $e_{last}$ apontam pra $T^\star$}
\State \Return $e_{first}.rot.parent$
\Else
\State \Return $e_{last}.rot.parent$
\EndIf
\EndIf
\end{algorithmic}
\end{algorithm}


A implementação da rotina \LCOMerge pode ser vista no Algoritmo~\ref{Algo:LCOMerge}.
O processo de \LCOMerge{} começa com a criação de um novo nó $\hat w$ utilizando a função \LCOMakeNode, que representará o nó resultante da operação.

A função \linkcutEvert{} é então chamada invertendo a árvore de modo que o nó~$\hat u$.$last$ se torne a nova raiz.
Isso é necessário para que a operação \linkcutAddEdge de adição de aresta subsequente assume que $\hat u$.$last$ é a raiz de sua árvore.
A função \linkcutAddEdge é usada para adicionar uma aresta entre~$\hat u$.$last$ e~$\hat v$.$first$, conectando a última posição de $\hat u$ à primeira de $\hat v$.

Os campos \varname{first} e \varname{last} do nó $\hat w$ são atualizados para refletir a nova estrutura: $\hat w$.$first$ recebe o valor de $\hat u$.$first$, e $\hat w$.$last$ recebe o valor de $\hat v$.$last$. Para garantir que a ordem cíclica das arestas seja mantida corretamente, a função \treapJoin{} é chamada para unir as raízes das ABBs associadas aos nós~$\hat u$ e~$\hat v$. A raiz da nova ABB resultante é associada ao novo nó~$\hat w$.
Por fim, o algoritmo retorna o nó resultante~$\hat w$.

\begin{algorithm}[htb]
\caption{\LCOMerge($\hat u$, $\hat v$)}
\label{Algo:LCOMerge}
\begin{algorithmic}[1]
\State $\hat w$ $\gets$ \LCOMakeNode()
\State \linkcutEvert($\hat u$.$last$)
\State \linkcutAddEdge($\hat u$.$last$, $\hat v$.$first$)
\State $\hat w$.$first$  $\gets$ $\hat u$.$first$
\State $\hat w$.$last$  $\gets$ $\hat v$.$last$
\State $T$ $\gets$ \treapJoin(\treapGetRoot($\hat u$.$first$), \treapGetRoot($\hat v$.$last$))
\State $T$.$\node$ $\gets$ $\hat w$
\State \Return $\hat w$
\end{algorithmic}
\end{algorithm}

A implementação da rotina \LCOSplit($\hat u$, $\hat e$) pode ser visto no Algoritmo~\ref{Algo:LCOSplit}.
Essa rotina realiza a divisão de um nó $\hat u$ em dois novos nós, $\hat v$ e $\hat w$, de acordo com a ordem cíclica das arestas incidentes ao nó~$\hat u$.
Primeiro, a rotina encontra o subnó correspondente à aresta~$\hat e$ em relação a~$\hat u$ usando a função \LCOFindSubNode.

Em seguida, dois novos nós, $\hat v$ e $\hat w$, são criados com a rotina \LCOMakeNode.
A função \linkcutEvert{} é aplicada para fazer uma inversão em $\hat u$.$last$, assim o caminho entre $\hat u$.$first$ e a raiz corresponde a ordem cíclica de $\hat u$.
Dessa forma, ao aplicar \linkcutDelEdge($s_e$) a ordem cíclica de~$\hat u$ é dividida em duas sequências, a primeira de $\hat u$.\varname{first} até $s_e$ e a segunda de \linkcutParent($s_e$) até~$\hat u$.\varname{last}.
Essa sequências correspondem às ordens cíclicas dos nós resultantes~$\hat v$ e~$\hat w$, respectivamente.

A árvore binária de busca é dividida em duas partes: uma que mantém a ordem das arestas até $s_e$, e outra que contém o restante.
Depois, essas duas partes são associadas aos novos nós~$\hat v$ e~$\hat w$, ajustando os ponteiros \varname{first} e \varname{last} de cada nó para refletir a nova estrutura.
Finalmente, os novos nós $\hat v$ e $\hat w$ são retornados, representando a divisão de $\hat u$.

\begin{algorithm}[htb]
\caption{\LCOSplit($\hat u$, $\hat e$)}
\label{Algo:LCOSplit}
\begin{algorithmic}[1]
\State $s_e$ $\gets$ \LCOFindSubNode($\hat u$, $\hat e$)
\State $\hat v$ $\gets$ \LCOMakeNode()
\State $\hat w$ $\gets$ \LCOMakeNode()
\State \linkcutEvert($\hat u$.$last$)
\State $y$ $\gets$ \linkcutParent($s_e$)
\State \linkcutDelEdge($s_e$)
\State $T$, $u_{last}$, $T'$ $\gets$ \treapSplit($s_e$)
\State $T$ $\gets$ \treapJoin($T$, $s_e$  )
\State $T$.$\node$ $\gets$ $\hat w$
\State $T'$.$\node$ $\gets$ $\hat v$
\State $\hat v$.$first$ $\gets$ $\hat u$.$first$; $\hat v$.$last$ $\gets$ $s_e$
\State $\hat w$.$first$ $\gets$ $y$; $\hat w$.$last$ $\gets$ $u_{last}$
\State \Return $\hat v$, $\hat w$
\end{algorithmic}
\end{algorithm}


A implementação da rotina \LCOCycle($\hat u$, $\hat e$) pode ser visto no Algoritmo~\ref{Algo:LCOCycle}.
O processo começa identificando o subnó correspondente a $\hat e$ na ordem cíclica de~$\hat v$ através da função \LCOFindSubNode.

Se $\hat e$ já for a última aresta, a função termina imediatamente.
Caso contrário, a função \linkcutEvert{} é aplicada a~$\hat v$.$last$ para reverter a ordem do caminho que vai até a raiz.
Essa operação faz com que o pai de $s_e$ seja o sucessor desse subnó na ordem cíclica de $\hat v$.
Então é usado a rotina \linkcutParent{} para obter esse sucessor~$x$ e remove a aresta que liga~$s_e$ a~$x$.
Em seguida, a função \linkcutAddEdge adiciona uma nova aresta que conecta o último subnó de $\hat v$ ao primeiro subnó, efetivamente ajustando a ordem cíclica.

Depois, os ponteiros~\varname{first} e~\varname{last} de~$\hat v$ são atualizados para refletir essa nova ordem, com \varname{first} apontando para o subnó~$x$ e \varname{last} apontando para~$s_e$.
Finalmente, a árvore binária de busca associada é dividida em partes, reorganizadas, e depois unidas para representar a nova ordem cíclica, garantindo que~$\hat e$ seja, de fato, a última aresta na sequência cíclica.




\begin{algorithm}[htb]
\caption{\LCOCycle($\hat v$, $\hat e$)}
\label{Algo:LCOCycle}
\begin{algorithmic}[1]
\State $s_e$ $\gets$ \LCOFindSubNode($\hat v$, $\hat e$)
\If{$\hat v$.$last$ $=$ $s_e$}
\State\Return
\EndIf
\State \linkcutEvert($\hat v$.$last$)
\State $x$ $\gets$ \linkcutParent($s_e$)
\State \linkcutDelEdge($s_e$)
\State \linkcutAddEdge($\hat v$.$last$, $\hat v$.$first$)
\State $\hat v$.$first$ $\gets$ $x$; $\hat v$.$last$ $\gets$ $s_e$
\State $T$, $u_{last}$, $T'$ $\gets$ \treapSplit($\hat u$.$last$)
\State $T$ $\gets$ \treapJoin($T'$, $T$, $u_{last}$)
\State $T$.$\node$ $\gets$ $\hat v$
\end{algorithmic}
\end{algorithm}


\begin{algorithm}[htb]
\caption{\LCOLink($\hat v$, $\hat w$)}
\label{Algo:LCOLink}
\begin{algorithmic}[1]
\State $x$ $\gets$ \linkcutCreate()
\State $y$ $\gets$ \linkcutCreate()
\If {$\hat v$.$last$ $=$ $\Nil$}
  \State $T$ $\gets$ \treapCreate()
  \State $T$.$\node$ $\gets$ $\hat v$
  \State $\hat v$.$first$ $\gets$ $x$; $\hat v$.$last$ $\gets$ $x$ 
\Else
  \State \linkcutEvert($\hat v$.$last$)
  \State \linkcutAddEdge($\hat v$.$last$, $x$)
  \State $T$ $\gets$ \treapGetRoot($\hat v$.$last$)
  \State $T$ $\gets$ \treapJoin($T$, $x$)
  \State $T$.$\node$ $\gets$ $\hat v$
  \State $\hat v$.$last$ $\gets$ $x$
\EndIf
\If {$\hat w$.$last$ $=$ $\Nil$}
  \State $T$ $\gets$ \treapCreate()
  \State $T$.$\node$ $\gets$ $\hat w$
  \State $\hat w$.$first$ $\gets$ $y$; $\hat w$.$last$ $\gets$ $y$ 
\Else
  \State \linkcutAddEdge($y$, $\hat w$.$first$)
  \State $T$ $\gets$ \treapGetRoot($\hat w$.$last$)
  \State $T$ $\gets$ \treapJoin($y$, $T$)
  \State $T$.$\node$ $\gets$ $\hat w$
  \State $\hat w$.$first$ $\gets$ $y$
\EndIf
\State \linkcutAddEdge($x$, $y$)
\end{algorithmic}
\end{algorithm}


\begin{figure}[htb]
\scalebox{1.5}{
\centering
\begin{tikzpicture}[line cap=round,line join=round,>=triangle 45,x=1cm,y=1cm]
\clip(-3,-.5) rectangle (8,6);

%\draw [line width=1pt] (0,0) to[in=-135,out=135,looseness=1]  (0,2); % v -- u
\begin{scriptsize}

%
% Node v
%
\draw (0,0) node {$\hat v$};
\draw [line width=1pt,color=cyan] (0,.5) to  (.35,.35); 
\draw [line width=1pt,color=cyan] (.35,.35) to  (.5,0); 

\draw [line width=1pt,color=cyan] (0,.5) to  (-1,1.8); 
\draw [line width=1pt,color=cyan] (.35,.35) to  (1.05,1.75); 
\draw [line width=1pt,color=cyan] (.5,0) to  (2.5,1); 


\draw (0,.5)[anchor=east] node {\tiny $a_v$};
\draw (.35,.35)[anchor=west] node {\tiny $b_v$};
\draw (.5,0)[anchor=north west] node {\tiny $e_v$};
\draw [fill=black] (0,.5) circle (1.5pt);
\draw [fill=black] (.35,.35) circle (1.5pt);
\draw [fill=black] (.5,0) circle (1.5pt);

\draw [line width=1pt,color=cyan] (-1,1.8) to  (-1,2.2); 
\draw (-1,1.8) node[anchor=north east] {\tiny $a_0$};
\draw [fill=white] (-1,1.8) circle (1.5pt);

% ^b
\draw [line width=1pt,color=cyan] (2.5,1) to (2.9,1.15); 
\draw (1.05,1.75)[anchor=west] node {\tiny $b_2$};
\draw [fill=white] (1.05,1.75) circle (1.5pt);
\draw (2.5,1)[anchor=north] node {\tiny $e_0$};
\draw [fill=white] (2.5,1) circle (1.5pt);

%
% Node u
%
\draw (0,4) node {$\hat u$};

\draw [line width=1pt,color=cyan] (0,3.5) to  (.3,3.7); 
\draw [line width=1pt,color=cyan] (.5,4) to  (.3,3.7); 

\draw [line width=1pt,color=cyan] (0,3.5) to  (-1,2.2); 
\draw [line width=1pt,color=cyan] (.3,3.7) to  (1.05,2.25); 
\draw [line width=1pt,color=cyan] (.5,4) to  (2.5,3); 

\draw (-1.2,2.2) node {\tiny $a_2$};
\draw [fill=white] (-1,2.2) circle (1.5pt);
\draw (1.3,2.3) node {\tiny $b_0$};
\draw [fill=white] (1.05,2.25) circle (1.5pt);

\draw (2.4,2.9) node {\tiny $c_2$};
\draw [fill=white] (2.5,3) circle (1.5pt);

% u node path
\draw [fill=black] (0,3.5) circle (1.5pt);
\draw [fill=black] (.5,4) circle (1.5pt);
\draw [fill=black] (.3,3.7) circle (1.5pt);

%
% Node z
%
\draw (4,2) node {$\hat z$};
\draw [line width=1pt,color=cyan] (4.3,2.3) to (5,2.7); 
\draw [line width=1pt,color=cyan] (3.7,2.3) to (3,2.7); 
\draw [line width=1pt,color=cyan] (3.7,1.7) to (2.9,1.15); 
\draw [line width=1pt,color=cyan] (4.3,1.7) to[in=0,out=-45,looseness=3] (5.5,2.8); 

% node path
\draw [line width=1pt,color=cyan] (4.3,2.3) to  (3.7,2.3); 
\draw [line width=1pt,color=cyan] (3.7,2.3) to  (3.7,1.7); 
\draw [line width=1pt,color=cyan] (3.7,1.7) to  (4.3,1.7); 


\draw [fill=black] (3.7,1.7)circle (1.5pt);
\draw [fill=black] (3.7,2.3)circle (1.5pt);
\draw [fill=black] (4.3,2.3)circle (1.5pt);
\draw [fill=black] (4.3,1.7)circle (1.5pt);


\draw (3,2.7)[anchor=north] node {\tiny $c_0$};
\draw [fill=white] (3,2.7)circle (1.5pt);
\draw (2.9,1.15)[anchor=north] node {\tiny $e_2$};
\draw [fill=white] (2.9,1.15)circle (1.5pt);
\draw (5,2.7)[anchor=north] node {\tiny $d_0$};
\draw [fill=white] (5,2.7)circle (1.5pt);
\draw (5.5,2.8)[anchor=north] node {\tiny $d_2$};
\draw [fill=white] (5.5,2.8)circle (1.5pt);

%
% Node F_0
%
\draw (3,5.2) node {$F_0$};
\draw [line width=1pt,color=red] (2.7,4.7) to  (3.3,4.7); 
\draw [line width=1pt,color=red] (3.5,5) to  (3.3,4.7); 
\draw [line width=1pt,color=red] (3.5,5) to  (2.5,5); 

\draw [line width=1pt,color=red] (2.5,5) to[in=180,out=180,looseness=2]  (-1.2,2); 
\draw [line width=1pt,color=red] (3.5,5) to[in=-15,out=0,looseness=3]   (2.8,.85); 
\draw [line width=1pt,color=red] (2.7,4.7) to  (2.9,3.1); 
\draw [line width=1pt,color=red] (3.3,4.7) to  (5.15,3.05); 

%^c
\draw [line width=1pt,color=red] (2.6,2.6) to  (2.9,3.1); 
%^d
\draw [line width=1pt,color=red] (5.15,3.05) to  (5.3,2.6); 

\draw [fill=white] (-1.2,2)circle (1.5pt);
\draw [fill=white] (2.9,3.1)circle (1.5pt);
\draw [fill=white] (2.8,.85)circle (1.5pt);
\draw [fill=white] (5.15,3.05)circle (1.5pt);


\draw [fill=black] (2.5,5)circle (1.5pt);
\draw [fill=black] (2.7,4.7)circle (1.5pt);
\draw [fill=black] (3.5,5)circle (1.5pt);
\draw [fill=black] (3.3,4.7)circle (1.5pt);

%
% Node F_1
%
\draw (2.1,2) node {$F_1$};
%node path
\draw [line width=1pt,color=red] (2.3,2.3) to  (2.3,1.7); 
\draw [line width=1pt,color=red] (1.7,2) to  (2.3,1.7); 

\draw [line width=1pt,color=red] (2.6,2.6) to  (2.3,2.3); 
\draw [line width=1pt,color=red] (2.6,1.3) to  (2.3,1.7); 
\draw [line width=1pt,color=red] (1.7,2) to  (1.3,2); 

% F_1 to F_2
\draw [line width=1pt,color=red] (.8,2) to  (1.3,2); 

\draw [fill=black] (1.7,2)circle (1.5pt);
\draw [fill=black] (2.3,1.7)circle (1.5pt);
\draw [fill=black] (2.3,2.3)circle (1.5pt);
\draw [fill=white] (1.3,2)circle (1.5pt);
\draw [fill=white] (2.6,1.3)circle (1.5pt);
\draw [fill=white] (2.6,2.6)circle (1.5pt);

%
% Node F_2
%
\draw (0,2.2) node {$F_2$};

\draw [line width=1pt,color=red] (-.8,2) to  (-.3,2); 
\draw [line width=1pt,color=red] (.8,2) to  (.3,2);   
\draw [line width=1pt,color=red] (-.3,2) to (.3,2);   

\draw [fill=black] (-.3,2) circle (1.5pt);
\draw [fill=black] (.3,2) circle (1.5pt);
\draw [fill=white] (-.8,2) circle (1.5pt);
\draw [fill=white] (.8,2) circle (1.5pt);

%
% Node F_3
%
\draw (6,2) node {$F_3$};
\draw [line width=1pt,color=red] (5.7,2) to  (5.3,2.6); 
\draw [fill=black] (5.7,2)circle (1.5pt);
\draw [fill=white] (5.3,2.6)circle (1.5pt);

\end{scriptsize}
\end{tikzpicture}

}
\caption{As árvores geradoras modificadas que representam o grafo da Figura~\ref{fig:MSF-basico-1}.}
\label{fig:MSF-figura-4}
\end{figure}
