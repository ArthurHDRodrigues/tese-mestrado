\chapter{Floresta maximal de peso mínimo em grafos planos}
\label{sec:MSF}

O problema da floresta maximal de peso mínimo em grafos dinâmicos planos foi inicialmente apresentado na Seção~\ref{sec:Motivação} e é o segundo problema que iremos discutir.
O objetivo desse capítulo é apresentar o algoritmo de David Eppstein, Giuseppe Italiano, Roberto Tamassia, Robert Tarjan, Jeffery Westbrook e Moti Yung~\cite{EPPSTEIN-planar} que resolve esse problema para grafos planos que podem ter laços e arestas paralelas.

O estudo desse problema se inicia com uma revisão de conceitos de planaridade e dualidade apresentados nas Seções~\ref{sec:planaridade} e~\ref{sec:dualidade} respectivamente.
Esses conceitos são bem estabelecidos na literatura e essas seções são baseadas na Seção~4.2 do livro \textit{Graph Theory} de Reinhard Diestel~\cite{Diestel}.

Os conceitos apresentando nessas seções serão usados nas Seções~\ref{sec:definition-MSF} para definir formalmente o problema de floresta maximal de peso mínimo em grafos ponderados planos dinâmicos e na Seção~\ref{sec:ADP} para fundamentar a estrutura de dados chamada árvores dinâmicas planas proposta por Eppstein et al. para solucionar esse problema.
Em seguida, a biblioteca das árvores dinâmicas planas é definida na Seção~\ref{sec:EODT} e na Seção~\ref{sec:MSFcomADPs} é mostrado como usar essa estrutura de dados para solucionar o problema MSF.

Por fim, na Seção~\ref{sec:struct-aux} são apresentadas estruturas de dados auxiliares que serão usadas para implementar as árvores dinâmicas planas. Essa implementação é feita na Seção~\ref{sec:implementacao-ADP}, que concluí o presente capítulo.

\section{Planaridade}
\label{sec:planaridade}

Intuitivamente, um grafo é dito plano se ele está desenhado em uma folha de papel de forma que suas arestas não se cruzem.
Formalmente, um \defi[grafo!plano]{grafo plano}~\cite{Diestel} é um par de conjuntos finitos $G = (V, E)$ com as seguintes propriedades:
\begin{enumerate}
\item $V\subset \R^2$;
\item Toda aresta é um arco entre dois vértices;
\item O interior de uma aresta não contém vértices nem intersecta outras arestas.
\end{enumerate}
Quando conveniente, $G$ será usado tanto para o par~$(V ,E)$ quanto para o conjunto~${V\cup\bigcup E}$.

Para cada grafo plano~$G$, ao remover~$G$ do plano~$\R^2$, é obtido um conjunto finito~$F(G)$ de regiões conexas que é chamado de \defi{conjunto de faces} de~$G$.
Naturalmente, cada uma dessas regiões é chamada de uma \defi{face} de $G$.
Uma dessas faces é ilimitada.
Essa face é chamada de \defi{face exterior}.
A Figura~\ref{fig:MSF-basico-0} mostra um grafo plano~$G$ e suas as faces.

\begin{minipage}[b]{0.4\textwidth}
\centering
\begin{tabular}{| c  c |} 
 \hline
 aresta & peso\\
 \hline
 $a$ & 2 \\ 
 \hline
$b$ & 7 \\
 \hline
$c$ & 3 \\
 \hline
$d$ & 1 \\
 \hline
$f$ & 2 \\
 \hline
$g$ & 4 \\
 \hline
\end{tabular}
\captionof{table}{Tabela de pesos}
\end{minipage}
\hfill
\begin{minipage}[b]{0.6\textwidth}
\scalebox{1.5}{
  \centering
  \begin{tikzpicture}[line cap=round,line join=round,>=triangle 45,x=1cm,y=1cm]
\clip(-3.5,-.5) rectangle (6,3);


\draw [line width=1pt] (0,0) to[in=-135,out=135,looseness=1]  (0,2); % v -- u
\draw [line width=1pt] (0,0) to[in=-45,out=45,looseness=1]  (0,2);   % v -- u
\draw [line width=1pt] (2,1) to  (0,0); % 4 -- 6
\draw [line width=1pt] (2,1) to  (0,2); % 4 -- 6
\draw [line width=1pt] (2,1.05) to[out=45,in=-45,looseness=50] (2,.95); % 4 -- 4


%\draw [line width=1pt] (1,2.5) to[out=-45,in=45,looseness=1] (1,1); % F0 -- F1
%\draw [line width=1pt] (1,2.5) to[out=20,in=-70,looseness=10] (1,1); % F0 -- F1
%\draw [line width=1pt] (1,2.5) to[out=170,in=180,looseness=2.5] (0,1); % F0 -- F2
%\draw [line width=1pt] (1,2.5) to[out=0,in=90,looseness=1] (2.5,1); % F0 -- F3
%\draw [line width=1pt] (1,1) to (0,1); % F1 -- F2




\begin{scriptsize}
\draw [fill=black] (0,2) circle (1.5pt);
\draw (0,2) node[anchor=south east] {$u$};


\draw [fill=black] (0,0) circle (1.5pt);
\draw (0,0) node[anchor=north east] {$v$};
\draw [fill=black] (2,1) circle (1.5pt);
\draw (1.95,1) node[anchor=south] {$z$};


%\draw [fill=black] (1,2.5) circle (1.5pt);
\draw (1,2.5) node {$F_0$};
%\draw [fill=black] (1,1) circle (1.5pt);
\draw (1,1) node {$F_1$};
%\draw [fill=black] (0,1) circle (1.5pt);
\draw (0,1) node {$F_2$};
%\draw [fill=black] (2.5,1) circle (1.5pt);
\draw (2.5,1) node {$F_3$};

\draw (-0.3,1) node {$2$};
\draw (-0.53,1) node {$a$};

\draw (0.3,1) node {$b$};
\draw (0.53,1) node {$7$};

\draw (1.15,1.7) node[anchor=north] {$c$}; % u -- w
\draw (1.05,1.5) node[anchor=north] {$3$}; % u -- w

\draw (2.9,1) node {$d$}; % w -- w
\draw (3.15,1) node {$2$}; % w -- w

\draw (1.2,.9) node[anchor=north] {$e$}; % v -- w
\draw (1.35,.65) node[anchor=north] {$1$}; % v -- w
\end{scriptsize}
\end{tikzpicture}

  }
  \captionof{figure}{Um grafo ponderado plano e suas faces.}
 \label{fig:MSF-basico-0}
\end{minipage}

\begin{lemma}[Lemma 4.2.2~\cite{Diestel}]
\label{lemma:diestel}
Seja $G$ um grafo plano e $e$ uma de suas arestas.
\begin{enumerate}
\item Se $X$ é a fronteira de uma face de~$G$, então ou $e\subseteq X$ ou a intersecção de $X$ com o interior de $e$ é vazia.
\item Se $e$ pertence a um ciclo de $G$, então $e$ pertence à fronteira de exatamente duas faces distintas de $G$;
\item Se não existe ciclo que contém $e$, então $e$ pertence à fronteira de uma única face de~$G$.
\end{enumerate}
\end{lemma}

Para cada vértice $v$ de um grafo plano~$G$, é possível construir uma ordenação cíclica~$D(v)$ das aresta incidentes a~$v$~\cite{noma2003}.
Para obter essa ordenação, percorremos as arestas incidentes a~$v$ em sentido anti-horário até retornar à aresta inicial do percurso.
Uma aresta é dita \defi{sucessora} de outra aresta se a primeira é sucessora da segunda nesse percurso.
Uma lista de todas essas ordens cíclicas é chamada de~\defi{descrição combinatória plana} de~$G$.

A descrição combinatória plana é essencialmente uma lista de adjacências do grafo plano em que a ordem das células é relevante.
Para o algoritmo que vamos descrever nesse capítulo, cada aresta possui um identificador e armazenamos também nas células das listas o identificador das arestas.
Observe que cada aresta corresponde a duas células nas listas de adjacência do grafo:
Se os extremos de uma aresta são~$u$ e~$v$, há uma célula com~$u$ na lista de~$v$ e uma célula com~$v$ na lista de~$u$, ambos representando essa aresta.
Como estamos admitindo arestas paralelas, o número de ocorrências de um vértice~$u$ na lista de um vértice~$v$ será exatamente o número de arestas paralelas entre~$u$ e~$v$.
Um laço num vértice~$u$ corresponderá a duas ocorrências de~$u$ na sua lista de adjacências.
Abaixo temos uma descrição combinatória plana do grafo da Figura~\ref{fig:MSF-basico-0}.
\begin{align*}
D(u)&=\langle (a,v), (b,v), (c,z))\rangle\\
D(v)&=\langle (a,u), (d,z), (b,u)  \rangle\\
D(y)&=\langle (e,z), (f,y), (f,y) \rangle
\end{align*}

Pensamos nessas células como representações das duas possíveis orientações da aresta: de~$u$ para~$v$ (algumas vezes denotada por~$uv$) e de~$v$ para~$u$ (algumas vezes denotada por $vu$).
Essas duas células terão também um apontador~$\varname{sym}$, referenciando à outra e vice-versa.

\section{Dualidade}
\label{sec:dualidade}


Dado um grafo plano~$G$, o grafo \defi{dual} de~$G$~\cite{Diestel} é o grafo $G^\star = (F,E^\star)$ cujo conjunto de vértices é o conjunto~$F = F(G)$ de faces de~$G$ e o conjunto de arestas $E^\star$ é construído a partir das arestas de~$G$.
Para cada aresta~$e\in E$, existe exatamente uma aresta~$e^\star$ no conjunto $E^\star$.
A aresta~$e^\star$ é chamada de \defi[aresta!dual]{aresta dual} de~$e$ e os vértices incidentes a essa aresta são as duas faces (não necessariamente distintas) cujas fronteiras contém~$e$.
Nesse contexto, chamamos~$G$ de grafo \defi{primal}.
O grafo dual também é um grafo plano.
Para ilustrar a relação entre cada aresta e seu dual, geralmente desenhamos cada aresta do grafo dual cruzando a correspondente aresta do grafo primal, como mostra a Figura~\ref{fig:MSF-basico-1}.

\begin{figure}[htb]
\scalebox{1.5}{
\centering
\begin{tikzpicture}[line cap=round,line join=round,>=triangle 45,x=1cm,y=1cm]
\clip(-3.5,-1) rectangle (6,3);


\draw [line width=1pt] (0,0) to[in=-135,out=135,looseness=1]  (0,2); % v -- u
\draw [line width=1pt] (0,0) to[in=-45,out=45,looseness=1]  (0,2);   % v -- u
\draw [line width=1pt] (2,1) to  (0,0); % 4 -- 6
\draw [line width=1pt] (2,1) to  (0,2); % 4 -- 6
\draw [line width=1pt] (2,1.05) to[out=45,in=-45,looseness=50] (2,.95); % 4 -- 4


\draw [line width=1pt] (1,2.5) to[out=-45,in=45,looseness=1] (1,1); % F0 -- F1
\draw [line width=1pt] (1,2.5) to[out=20,in=-70,looseness=10] (1,1); % F0 -- F1
\draw [line width=1pt] (1,2.5) to[out=170,in=180,looseness=2.5] (0,1); % F0 -- F2
\draw [line width=1pt] (1,2.5) to[out=0,in=90,looseness=1] (2.5,1); % F0 -- F3
\draw [line width=1pt] (1,1) to (0,1); % F1 -- F2




\begin{scriptsize}
\draw [fill=black] (0,2) circle (1.5pt);
\draw (0,2) node[anchor=south east] {$u$};


\draw [fill=black] (0,0) circle (1.5pt);
\draw (0,0) node[anchor=north east] {$v$};
\draw [fill=black] (2,1) circle (1.5pt);
\draw (1.95,1) node[anchor=south] {$w$};


\draw [fill=black] (1,2.5) circle (1.5pt);
\draw (1,2.5) node[anchor=south east] {$F_0$};
\draw [fill=black] (1,1) circle (1.5pt);
\draw (1,1) node[anchor=west] {$F_1$};
\draw [fill=black] (0,1) circle (1.5pt);
\draw (0,1) node[anchor=south] {$F_2$};
\draw [fill=black] (2.5,1) circle (1.5pt);
\draw (2.5,1) node[anchor=west] {$F_3$};

\draw (-0.3,1.05) node[anchor=north] {$a$};
\draw (-0.5,1.1) node[anchor=north] {$2$};

\draw (0.5,1.35) node[anchor=north] {$b$};
\draw (0.5,1.05) node[anchor=north] {$7$};

\draw (1.15,1.7) node[anchor=north] {$c$}; % u -- w
\draw (1.4,1.7) node[anchor=north] {$3$}; % u -- w

\draw (2.52,1.75) node[anchor=north] {$d$}; % w -- w
\draw (2.3,1.65) node[anchor=north] {$2$}; % w -- w

\draw (1.2,.9) node[anchor=north] {$e$}; % v -- w
\draw (1.35,.65) node[anchor=north] {$1$}; % v -- w
\end{scriptsize}
\end{tikzpicture}

}
\caption{Um grafo ponderado plano e seu dual.}
\label{fig:MSF-basico-1}
\end{figure}

\begin{theorem}[Fórmula de Euler, Teorema 4.2.9~\cite{Diestel}]
\label{teo:MSFEuler}
Seja $G=(V,E)$ um grafo planar conexo com~$n$ vértices e~$m$ arestas. Toda imersão de~$G$ no plano possui o mesmo número $f$ de faces e
$$
n-m+f = 2.
$$
\end{theorem}

Se~$G$ é um grafo plano com~$n$ vértices e~$m$ arestas, o Teorema~\ref{teo:MSFEuler} mostra que o número de vértices do seu grafo dual é $\OTheta{n+m}$.

\begin{theorem}[\cite{EPPSTEIN-planar}]
\label{teo:MSFdual}
Seja~$T$ uma árvore geradora de um grafo plano conexo~$G$. O conjunto
$$
T^\star = \{e^\star:e\notin T\}
$$
é uma árvore geradora de~$G^\star$.
Além disso, se~$G$ for ponderado e adotarmos $w(e^\star) = w(e)$, então~$T$ será de peso mínimo em~$G$ se e somente se~$T^\star$ for de peso máximo em~$G^\star$ e vice versa.
\end{theorem}

A Figura~\ref{fig:MSF-figura-2} mostra~$T$ e~$T^\star$ dos grafos da Figura~\ref{fig:MSF-basico-1}.

\begin{figure}[htb]
\scalebox{1.5}{
\centering
\begin{tikzpicture}[line cap=round,line join=round,>=triangle 45,x=1cm,y=1cm]
\clip(-3.5,-.5) rectangle (6,3);


\draw [line width=2pt,color=cyan] (0,0) to[in=-135,out=135,looseness=1]  (0,2); % v -- u
\draw [line width=.5pt] (0,0) to[in=-135,out=135,looseness=1]  (0,2); % v -- u
%\draw [line width=1pt] (0,0) to[in=-45,out=45,looseness=1]  (0,2);   % v -- u
\draw [line width=2pt,color=cyan] (2,1) to  (0,0); % 4 -- 6
\draw [line width=.5pt] (2,1) to  (0,0); % 4 -- 6
\draw [line width=2pt,color=cyan] (2,1) to  (3,1); % 4 -- 6
\draw [line width=.5pt] (2,1) to  (3,1); % 4 -- 6
%\draw [line width=1pt] (2,1) to  (0,2); % 4 -- 6
%\draw [line width=1pt] (2,1.05) to[out=45,in=-45,looseness=50] (2,.95); % 4 -- 4

\draw [line width=2pt,color=red] (1,2.5) to[out=-45,in=45,looseness=1] (1,1); % F0 -- F1
\draw [line width=.5pt] (1,2.5) to[out=-45,in=45,looseness=1] (1,1); % F0 -- F1
%\draw [line width=1pt] (1,2.5) to[out=20,in=-70,looseness=10] (1,1); % F0 -- F1
%\draw [line width=1pt] (1,2.5) to[out=170,in=180,looseness=2.5] (0,1); % F0 -- F2
\draw [line width=2pt,color=red] (1,2.5) to[out=0,in=90,looseness=1] (3.5,1); % F0 -- F3
\draw [line width=.5pt] (1,2.5) to[out=0,in=90,looseness=1] (3.5,1); % F0 -- F3
\draw [line width=2pt,color=red] (1,1) to (0,1); % F1 -- F2
\draw [line width=.5pt] (1,1) to (0,1); % F1 -- F2

\begin{scriptsize}
\draw (-.5,1) node[anchor=south east,color=cyan] {$T$};
\draw (4,2.3) node[anchor=south east,color=red] {$T^\star$};
\draw [fill=black] (0,2) circle (1.5pt);
\draw (0,2) node[anchor=south east] {$u$};


\draw [fill=black] (0,0) circle (1.5pt);
\draw (0,0) node[anchor=north east] {$v$};
\draw [fill=black] (2,1) circle (1.5pt);
\draw (2,1) node[anchor=south] {$z$};
\draw [fill=black] (3,1) circle (1.5pt);
\draw (3,1) node[anchor=south] {$y$};


\draw [fill=black] (1,2.5) circle (1.5pt);
\draw (1,2.5) node[anchor=south east] {$F_0$};
\draw [fill=black] (1,1) circle (1.5pt);
\draw (.9,1) node[anchor=south] {$F_1$};
\draw [fill=black] (0,1) circle (1.5pt);
\draw (0,1) node[anchor=south] {$F_2$};
\draw [fill=black] (3.5,1) circle (1.5pt);
\draw (3.5,1) node[anchor=west] {$F_3$};


\draw (-0.3,1.05) node[anchor=north] {$a$};
\draw (-0.5,1.1) node[anchor=north] {$2$};

\draw (0.55,1.35) node[anchor=north] {$b^\star$};
\draw (0.5,1.05) node[anchor=north] {$7$};

\draw (1.15,1.9) node[anchor=north] {$c^\star$}; % u -- w
\draw (1.45,1.85) node[anchor=north] {$3$}; % u -- w

\draw (1.2,1) node[anchor=north] {$d$}; % v -- w
\draw (1.35,.65) node[anchor=north] {$1$}; % v -- w

\draw (2.5,1.4) node[anchor=north] {$g$}; % v -- w
\draw (2.5,1) node[anchor=north] {$4$}; % v -- w

\draw (2.55,2.7) node[anchor=north] {$f^\star$}; % w -- w
\draw (2.4,2.3) node[anchor=north] {$2$}; % w -- w

\end{scriptsize}
\end{tikzpicture}

}
\caption{Árvore geradora~$T$, em azul, de peso mínimo do grafo ponderado da Figura~\ref{fig:MSF-basico-1} e a árvore correspondente~$T^\star$ do seu grafo dual, em vermelho.}
\label{fig:MSF-figura-2}
\end{figure}





\section{Definição do problema}
\label{sec:definition-MSF}
O problema da floresta maximal de peso mínimo (MSF) em grafos dinâmicos planos ponderados consiste na busca por uma implementação tão eficiente quanto possível para a seguinte biblioteca:

\begin{itemize}
\item \MSFCreate($n$): Cria e devolve um grafo plano ponderado~$G$ com $n$ vértices isolados.
\item \MSFaddEdge($G$, $e$, $u$, $e_u$, $v$, $e_v$, $w$): Insere em~$G$ uma nova aresta~$e$ com peso~$w$ ligando os vértices~$u$ e~$v$. A nova aresta~$e$ é sucessora das arestas~$e_u$ e~$e_v$ nas ordenações cíclicas de~$u$ e~$v$, respectivamente.
\item \MSFdelEdge($G$, $e$): Remove a aresta~$e$ de~$G$.
\item \MSFupdate($G$, $e$, $w$): Altera o peso da aresta~$e$ de~$G$ para o valor~$w$.
\item \MSFweight($G$): Devolve o peso de uma MSF de $G$.
\end{itemize}


\section{Árvores dinâmicas planas}
\label{sec:ADP}

Para resolver o problema da floresta geradora de peso mínimo em grafos planos ponderados dinâmicos, Eppstein et al.~\cite{EPPSTEIN-planar} propuseram o uso de uma estrutura de dados que chamaremos de \defi[arvore@\'arvore!dinâmica plana]{árvores dinâmicas planas} (ADP). Originalmente os autores denominaram essa estrutura de \textit{edge ordered dynamic tree}.
Com essa estrutura, eles obtém implementações de \MSFCreate{} e \MSFweight{} que consomem tempo constante e \MSFaddEdge{}, \MSFdelEdge{} e \MSFupdate{} que consomem tempo~$\O{\lg m}$ esperado amortizado, onde~$m$ é o número de arestas no grafo corrente. 



Cada componente conexa do grafo plano dinâmico ponderado~$G$ é representada em um plano distinto.
Consequentemente, a face exterior de cada componente conexa é considerada distinta das faces exteriores das outras componentes.
Cada componente~$C$ de~$G$ é representada por duas árvores com pesos em seus \textit{vértices}, sendo a primeira construída a partir de uma árvore geradora de peso mínimo~$T$ de~$C$ e a segunda a partir da correspondente dual~$T^\star$.
A Figura~\ref{fig:MSF-figura-3} ilustra essas duas árvores cuja construção será detalhada no próximo parágrafo.

A primeira dessas árvores, denotada por~$\hat T$, é essencialmente a árvore~$T$ submetida a duas transformações:
cada uma de suas arestas é subdivida; e é adicionada uma folha a cada vértice incidente a uma aresta de~$C$ que não esteja em~$T$.
Mais precisamente, $\hat T$ possui um vértice~$\hat v$, de peso~$-\infty$, para cada vértice~$v$ de~$C$, um vértice~$\hat e$, com o peso de $e$, para cada aresta~$e$ de~$T$ e dois vértices~$\hat d_0$ e~$\hat d_2$, ambos com o peso de~$d$, para cada aresta~$d$ de~$C$ que não é uma aresta de~$T$.
Se uma aresta~$e$ de~$T$ é incidente aos vértices~$v$ e~$u$ de~$C$, então o vértice~$\hat e$ é ligado aos vértices~$\hat v$ e~$\hat u$.
Se uma aresta~$d$ de~$C$ não está em~$T$ e é incidente aos vértices~$v$ e~$u$ de~$C$, então~$\hat d_0$ e~$\hat d_2$ são ligados a~$\hat v$ e~$\hat u$, respectivamente.
Note que~$\hat d_0$ e~$\hat d_2$ são folhas da árvore~$\hat T$.
Na Figura~\ref{fig:MSF-figura-3}, a árvore $\hat T$ está colorida em azul.

A segunda árvore, denotada por~$\hat T^\star$, possui construção análoga à primeira, mas é baseada em~$T^\star$ em vez de~$T$.
Ela possui um vértice~$\hat F$ para cada vértice~$F$ de~$C^\star$ (isto é, para cada face de~$C$), um vértice~$\hat e^\star$ para cada aresta~$e^\star$ de $T^\star$ e dois vértices $\hat d_1$ e~$\hat d_3$ para cada aresta~$d$ de $C$ que não está em $T^\star$.
O peso do vértice~$\hat F$ é $\infty$, enquanto que o peso de~$\hat e^\star$ é o peso de~$e$ e o peso de~$\hat d_1$ e~$\hat d_3$ é o peso de~$d$.
Se uma aresta $e^\star$ de $T^\star$ é incidente às faces~$F_1$ e~$F_2$ de~$C$, então o vértice~$\hat e^\star$ é ligado aos vértices~$\hat F_1$ e~$\hat F_2$.
Se uma aresta $d^\star$ de~$C^\star$ não está em $T^\star$ e é incidente às faces~$F_1$ e~$F_2$ de~$C$, então $\hat d_1$ e $\hat d_3$ são ligados a~$\hat F_1$ e~$\hat F_2$ respectivamente.
Na Figura~\ref{fig:MSF-figura-3}, a árvore $\hat T^\star$ está colorida em vermelho.

\begin{minipage}[b]{0.4\textwidth}
\centering
\begin{tabular}{| c  c |} 
 \hline
 vértices & pesos\\
 \hline
 $\hat a$, $\hat a_1$, $\hat a_3$ & 2 \\ 
 \hline
 $\hat b$, $\hat b_0$, $\hat b_2$ & 7 \\
 \hline
 $\hat c$, $\hat c_0$, $\hat c_2$ & 3 \\
 \hline
 $\hat d$, $\hat d_0$, $\hat d_2$ & 2 \\
 \hline
 $\hat f$, $\hat f_1$, $\hat f_3$ & 1 \\
 \hline
 $\hat g$, $\hat g_1$, $\hat g_3$ & 4 \\
 \hline
 $\hat u$, $\hat v$, $\hat y$, $\hat z$ & $-\infty$ \\
 \hline
 $\hat F_0$, $\hat F_1$, $\hat F_2$, $\hat F_3$ & $\infty$ \\
 \hline
\end{tabular}
\captionof{table}{Tabela de pesos}
\end{minipage}
\hfill
\begin{minipage}[b]{0.5\textwidth}
    \centering
\begin{tikzpicture}[line cap=round,line join=round,>=triangle 45,x=1cm,y=1cm]
\clip(-2,-1) rectangle (9,6);

\draw [line width=2pt,color=cyan] (0,0) to[in=-135,out=135,looseness=1]  (0,4); % v -- u
\draw [line width=.5pt] (0,0) to[in=-135,out=135,looseness=1]  (0,4); % v -- u
%\draw [line width=1pt] (0,0) to[in=-45,out=45,looseness=1]  (0,4);   % v -- u

\draw [line width=2pt,color=cyan] (4,2) to  (0,0); % 4 -- 6
\draw [line width=.5pt] (4,2) to  (0,0); % 4 -- 6
%\draw [line width=1pt] (4,2) to  (0,4); % 4 -- 6
%\draw [line width=1pt] (4,2.05) to[out=45,in=-45,looseness=100] (4,1.95); % w -- w

\draw [line width=2pt,color=red] (2,5) to[out=-45,in=45,looseness=1] (2,2); % F0 -- F1
\draw [line width=.5pt] (2,5) to[out=-45,in=45,looseness=1] (2,2); % F0 -- F1
%\draw [line width=1pt] (2,5) to[out=20,in=-70,looseness=7.5] (2,2); % F0 -- F1
%\draw [line width=1pt] (2,5) to[out=170,in=180,looseness=2] (0,2); % F0 -- F2
\draw [line width=2pt,color=red] (2,5) to[out=0,in=90,looseness=1] (7,2); % F0 -- F3
\draw [line width=.5pt] (2,5) to[out=0,in=90,looseness=1] (7,2); % F0 -- F3

\draw [line width=2pt,color=red] (2,2) to (0,2); % F1 -- F2
\draw [line width=.5pt] (2,2) to (0,2); % F1 -- F2

\draw [line width=2pt,color=red] (2,5) to[out=0,in=90,looseness=1] (5,2.5); % F0 -- F0
\draw [line width=.5pt] (2,5) to[out=0,in=90,looseness=1] (5,2.5); % F0 -- F0


%F0 -- e1
\draw [line width=2pt,color=red] (2,5) to[out=0,in=90,looseness=1] (8.3,2); % F0 -- F0
\draw [line width=2pt,color=red] (8.3,2) to[out=-90,in=-90,looseness=2] (5, 1.5); % F0 -- F0
\draw [line width=.5pt] (2,5) to[out=0,in=90,looseness=1] (8.3,2); % F0 -- F0
\draw [line width=.5pt] (8.3,2) to[out=-90,in=-90,looseness=2] (5, 1.5); % F0 -- F0

\begin{scriptsize}

% Aresta a
\draw [line width=2pt,color=red] (0,2) to  (-.6,2); 
\draw [line width=.5pt] (0,2) to  (-.6,2); 
\draw [line width=2pt,color=red] (-1.1,2) to[out=135,in=180,looseness=1.5]  (2,5);   % v -- u
\draw [line width=.5pt] (-1.1,2) to[out=135,in=180,looseness=1.5]  (2,5);   % v -- u
\draw [fill=black] (-.6,2) circle (1.5pt);
\draw [fill=black] (-.83,2) circle (1.5pt);
\draw [fill=black] (-1.1,2) circle (1.5pt);


% Aresta b
\draw [line width=2pt,color=cyan] (0,0) to[in=-90,out=45,looseness=1]  (.8,1.7);   % v -- u
\draw [line width=.5pt] (0,0) to[in=-90,out=45,looseness=1]  (.8,1.7);   % v -- u

\draw [line width=2pt,color=cyan] (0,4) to[out=-45,in=90,looseness=1]  (.8,2.3);   % v -- u
\draw [line width=.5pt] (0,4) to[out=-45,in=90,looseness=1]  (.8,2.3);   % v -- u
\draw [fill=black] (1,2) circle (1.5pt);
\draw [fill=black] (.8,2.3) circle (1.5pt);
\draw [fill=black] (.8,1.7) circle (1.5pt);


% Aresta c

\draw [line width=2pt,color=cyan] (0,4) to  (2.2,2.9);   
\draw [line width=.5pt] (0,4) to  (2.2,2.9);   
\draw [line width=2pt,color=cyan] (2.7,2.65) to  (4,2);   
\draw [line width=.5pt] (2.7,2.65) to  (4,2);   
\draw [fill=black] (2.2,2.9) circle (1.5pt);
\draw [fill=black] (2.54,2.9) circle (1.5pt);
\draw [fill=black] (2.7,2.65) circle (1.5pt);

% Aresta d
\draw [line width=2pt,color=red] (2,2) to  (2.2,1.5);   
\draw [line width=.5pt] (2,2) to  (2.2,1.5);   

%\draw [line width=2pt,color=red] (2.5,.9) to[out=-45,in=0,looseness=4.1]  (2,5);   
%\draw [line width=.5pt] (2.5,.9) to[out=-45,in=0,looseness=4.1]  (2,5);   

\draw [line width=2pt,color=red] (2.5,.9) to[out=-45,in=-90,looseness=1.5]  (8.5,2);   
\draw [line width=.5pt] (2.5,.9) to[out=-45,in=-90,looseness=1.5]  (8.5,2);   
\draw [line width=2pt,color=red] (2,5) to[out=10,in=90,looseness=1]  (8.5,2);   
\draw [line width=.5pt] (2,5) to[out=10,in=90,looseness=1]  (8.5,2);   



\draw [fill=black] (2.35,1.175) circle (1.5pt);
\draw [fill=black] (2.2,1.5) circle (1.5pt);
\draw [fill=black] (2.5,.9) circle (1.5pt);

% Aresta e
\draw [line width=2pt,color=cyan] (4,2) to  (6,2);   
\draw [line width=.5pt] (4,2) to  (6,2);   
\draw [fill=black] (5,2.5) circle (1.5pt);
\draw [fill=black] (5,2) circle (1.5pt);
\draw [fill=black] (5,1.5) circle (1.5pt);

% Aresta f

\draw [line width=2pt,color=cyan] (6,2) to  (6.5,2.5);   
\draw [line width=.5pt] (6,2) to  (6.5,2.5);   
\draw [line width=2pt,color=cyan] (6,2) to[out=-45,in=-45,looseness=3]  (7.4,2.8);   
\draw [line width=.5pt] (6,2) to[out=-45,in=-45,looseness=3]  (7.4,2.8);   
\draw [fill=black] (6.9,2.75) circle (1.5pt);
\draw [fill=black] (6.5,2.5) circle (1.5pt);
\draw [fill=black] (7.4,2.8) circle (1.5pt);


\draw [fill=black] (0,4) circle (1.5pt);
\draw (0,4) node[anchor=south east] {$\hat u$};

\draw [fill=black] (0,0) circle (1.5pt);
\draw (0,0) node[anchor=north east] {$\hat v$};
\draw [fill=black] (4,2) circle (1.5pt);
\draw (3.95,2) node[anchor=south] {$\hat z$};
\draw [fill=black] (6,2) circle (1.5pt);
\draw (6,2) node[anchor=south] {$\hat y$};

\draw [fill=black] (2,5) circle (1.5pt);
\draw (2,5) node[anchor=south east] {$\hat F_0$};
\draw [fill=black] (2,2) circle (1.5pt);
\draw (2,2) node[anchor=west] {$\hat F_1$};
\draw [fill=black] (0,2) circle (1.5pt);
\draw (0,2) node[anchor=south] {$\hat F_2$};
\draw [fill=black] (7,2) circle (1.5pt);
\draw (7,2) node[anchor=west] {$\hat F_3$};


\draw (-.7,2.4) node[anchor=north] {$\hat a$};
\draw (-.5,2) node[anchor=north] {$\hat a_1$};
\draw (-1.2,2.05) node[anchor=north] {$\hat a_3$};
%\draw (-1,2.4) node[anchor=north] {$2$};

\draw (1.1,2.5) node[anchor=north] {$\hat b$};
\draw (.6,2.7) node[anchor=north] {$\hat b_0$};
\draw (.6,2) node[anchor=north] {$\hat b_2$};
%\draw (1,2) node[anchor=north] {$7$};

\draw (2.45,3.3) node[anchor=north] {$\hat c$}; % u -- w
\draw (2.8,3.05) node[anchor=north] {$\hat c_0$}; % u -- w
\draw (2.1,3) node[anchor=north] {$\hat c_2$}; % u -- w
%\draw (2.55,2.7) node[anchor=north] {$3$}; % u -- w

\draw (2.45,1.6) node[anchor=north] {$\hat d$}; % v -- w
\draw (2,1.7) node[anchor=north] {$\hat d_3$}; % v -- w
\draw (2.4,1) node[anchor=north] {$\hat d_1$}; % v -- w
%\draw (2.5,1.3) node[anchor=north] {$1$}; % v -- w

\draw (5.1,2.4) node[anchor=north] {$\hat e$}; % v -- w
\draw (5.2,2.8) node[anchor=north] {$\hat e_3$}; % v -- w
\draw (5.2,1.8) node[anchor=north] {$\hat e_1$}; % v -- w
%\draw (2.5,1.3) node[anchor=north] {$1$}; % v -- w

\draw (7.04,3.2) node[anchor=north] {$\hat f$}; % w -- w
\draw (7.4,3.3) node[anchor=north] {$\hat f_2$}; % w -- w
\draw (6.5,3) node[anchor=north] {$\hat f_0$}; % w -- w
%\draw (4.7,3) node[anchor=north] {$\2$}; % w -- w

\end{scriptsize}
\end{tikzpicture}

\label{fig:MSF-figura-3}
	\captionof{figure}{As árvores~$\hat T$ e~$\hat T^\star$ construídas a partir das árvores da Figura~\ref{fig:MSF-figura-2}.}
    \end{minipage}

Note que há uma correspondência entre as arestas incidentes a um vértice~$v$ de~$G$ e os vértices vizinhos de~$\hat v$ em~$\hat T$.
Cada aresta~$e$ na vizinhança de um vértice~$v$ de~$G$ corresponde ou a $\hat e$, $\hat e_0$ ou~$\hat e_2$. 
Logo é possível construir, para cada vértice $v$ de~$G$, uma ordem cíclica dos vizinhos de~$\hat v$ a partir da descrição combinatória plana de~$G$.
Por exemplo, se temos \nolbreaks{$D(u) =\langle (a,v), (b,v), (c,z))\rangle$}, então a respectiva ordem cíclica de~$\hat u$ é~$\langle \hat a, \hat b_0, \hat c_2 \rangle$.



A implementação de árvores dinâmicas planas utiliza link cut trees e árvores binárias de busca com chave implícita.
Árvores binárias de busca com chave implícita foram tratadas em detalhes no Capitulo~\ref{sec:TreapDeChaveImplicita} e sua biblioteca será expandida na Seção~\ref{sec:extrastructABB} para poder ser usada para implementar árvores dinâmicas planas.
Já link cut trees serão brevemente abordadas na Seção~\ref{sec:linkcuttree}.

Cada vértice~$\hat v$ de~$\hat T$ e~$\hat T^\star$ é implementado por um conjunto de nós de link cut trees.
Esse conjunto possui um nó~$e_v$ para cada aresta~$e$ na ordem cíclica de~$\hat v$.
Cada nó $e_v$ é ligado aos nós que representam a aresta sucessora e predecessora de~$e$ na ordem cíclica de~$\hat v$,
com exceção do primeiro e do último na ordem, que são ligados somente ao seu sucessor e predecessor, respectivamente.
Dessa forma, a ordem cíclica é representada por um caminho na link cut tree.
Além disso, se a aresta~$e$ liga os nós~$\hat v$ e~$\hat u$, então~$e_v$ é ligado ao nó~$e_u$ que representa~$e$ na ordem cíclica de~$\hat u$.
A Figura~\ref{fig:MSF-figura-4} ilustra as duas link cut trees que representam as árvores~$\hat T$ e~$\hat T^\star$ ilustradas na Figura~\ref{fig:MSF-figura-3}.

\begin{figure}[htb]
\scalebox{1.3}{
\centering
\begin{tikzpicture}[line cap=round,line join=round,>=triangle 45,x=1cm,y=1cm]
\clip(-3,-.5) rectangle (8,6);

%\draw [line width=1pt] (0,0) to[in=-135,out=135,looseness=1]  (0,2); % v -- u
\begin{scriptsize}

%
% Node v
%
\draw (0,0) node {$\hat v$};
\draw [line width=1pt,color=cyan] (0,.5) to  (.35,.35); 
\draw [line width=1pt,color=cyan] (.35,.35) to  (.5,0); 

\draw [line width=1pt,color=cyan] (0,.5) to  (-1,1.8); 
\draw [line width=1pt,color=cyan] (.35,.35) to  (1.05,1.75); 
\draw [line width=1pt,color=cyan] (.5,0) to  (2.5,1); 


\draw (0,.5)[anchor=east] node {\tiny $a_v$};
\draw (.35,.35)[anchor=west] node {\tiny $b_v$};
\draw (.5,0)[anchor=north west] node {\tiny $e_v$};
\draw [fill=black] (0,.5) circle (1.5pt);
\draw [fill=black] (.35,.35) circle (1.5pt);
\draw [fill=black] (.5,0) circle (1.5pt);

\draw [line width=1pt,color=cyan] (-1,1.8) to  (-1,2.2); 
\draw (-1,1.8) node[anchor=north east] {\tiny $a_0$};
\draw [fill=white] (-1,1.8) circle (1.5pt);

% ^b
\draw [line width=1pt,color=cyan] (2.5,1) to (2.9,1.15); 
\draw (1.05,1.75)[anchor=west] node {\tiny $b_2$};
\draw [fill=white] (1.05,1.75) circle (1.5pt);
\draw (2.5,1)[anchor=north] node {\tiny $e_0$};
\draw [fill=white] (2.5,1) circle (1.5pt);

%
% Node u
%
\draw (0,4) node {$\hat u$};

\draw [line width=1pt,color=cyan] (0,3.5) to  (.3,3.7); 
\draw [line width=1pt,color=cyan] (.5,4) to  (.3,3.7); 

\draw [line width=1pt,color=cyan] (0,3.5) to  (-1,2.2); 
\draw [line width=1pt,color=cyan] (.3,3.7) to  (1.05,2.25); 
\draw [line width=1pt,color=cyan] (.5,4) to  (2.5,3); 

\draw (-1.2,2.2) node {\tiny $a_2$};
\draw [fill=white] (-1,2.2) circle (1.5pt);
\draw (1.3,2.3) node {\tiny $b_0$};
\draw [fill=white] (1.05,2.25) circle (1.5pt);

\draw (2.4,2.9) node {\tiny $c_2$};
\draw [fill=white] (2.5,3) circle (1.5pt);

% u node path
\draw [fill=black] (0,3.5) circle (1.5pt);
\draw [fill=black] (.5,4) circle (1.5pt);
\draw [fill=black] (.3,3.7) circle (1.5pt);

%
% Node z
%
\draw (4,2) node {$\hat z$};
\draw [line width=1pt,color=cyan] (4.3,2.3) to (5,2.7); 
\draw [line width=1pt,color=cyan] (3.7,2.3) to (3,2.7); 
\draw [line width=1pt,color=cyan] (3.7,1.7) to (2.9,1.15); 
\draw [line width=1pt,color=cyan] (4.3,1.7) to[in=0,out=-45,looseness=3] (5.5,2.8); 

% node path
\draw [line width=1pt,color=cyan] (4.3,2.3) to  (3.7,2.3); 
\draw [line width=1pt,color=cyan] (3.7,2.3) to  (3.7,1.7); 
\draw [line width=1pt,color=cyan] (3.7,1.7) to  (4.3,1.7); 


\draw [fill=black] (3.7,1.7)circle (1.5pt);
\draw [fill=black] (3.7,2.3)circle (1.5pt);
\draw [fill=black] (4.3,2.3)circle (1.5pt);
\draw [fill=black] (4.3,1.7)circle (1.5pt);


\draw (3,2.7)[anchor=north] node {\tiny $c_0$};
\draw [fill=white] (3,2.7)circle (1.5pt);
\draw (2.9,1.15)[anchor=north] node {\tiny $e_2$};
\draw [fill=white] (2.9,1.15)circle (1.5pt);
\draw (5,2.7)[anchor=north] node {\tiny $d_0$};
\draw [fill=white] (5,2.7)circle (1.5pt);
\draw (5.5,2.8)[anchor=north] node {\tiny $d_2$};
\draw [fill=white] (5.5,2.8)circle (1.5pt);

%
% Node F_0
%
\draw (3,5.2) node {$F_0$};
\draw [line width=1pt,color=red] (2.7,4.7) to  (3.3,4.7); 
\draw [line width=1pt,color=red] (3.5,5) to  (3.3,4.7); 
\draw [line width=1pt,color=red] (3.5,5) to  (2.5,5); 

\draw [line width=1pt,color=red] (2.5,5) to[in=180,out=180,looseness=2]  (-1.2,2); 
\draw [line width=1pt,color=red] (3.5,5) to[in=-15,out=0,looseness=3]   (2.8,.85); 
\draw [line width=1pt,color=red] (2.7,4.7) to  (2.9,3.1); 
\draw [line width=1pt,color=red] (3.3,4.7) to  (5.15,3.05); 

%^c
\draw [line width=1pt,color=red] (2.6,2.6) to  (2.9,3.1); 
%^d
\draw [line width=1pt,color=red] (5.15,3.05) to  (5.3,2.6); 

\draw [fill=white] (-1.2,2)circle (1.5pt);
\draw [fill=white] (2.9,3.1)circle (1.5pt);
\draw [fill=white] (2.8,.85)circle (1.5pt);
\draw [fill=white] (5.15,3.05)circle (1.5pt);


\draw [fill=black] (2.5,5)circle (1.5pt);
\draw [fill=black] (2.7,4.7)circle (1.5pt);
\draw [fill=black] (3.5,5)circle (1.5pt);
\draw [fill=black] (3.3,4.7)circle (1.5pt);

%
% Node F_1
%
\draw (2.1,2) node {$F_1$};
%node path
\draw [line width=1pt,color=red] (2.3,2.3) to  (2.3,1.7); 
\draw [line width=1pt,color=red] (1.7,2) to  (2.3,1.7); 

\draw [line width=1pt,color=red] (2.6,2.6) to  (2.3,2.3); 
\draw [line width=1pt,color=red] (2.6,1.3) to  (2.3,1.7); 
\draw [line width=1pt,color=red] (1.7,2) to  (1.3,2); 

% F_1 to F_2
\draw [line width=1pt,color=red] (.8,2) to  (1.3,2); 

\draw [fill=black] (1.7,2)circle (1.5pt);
\draw [fill=black] (2.3,1.7)circle (1.5pt);
\draw [fill=black] (2.3,2.3)circle (1.5pt);
\draw [fill=white] (1.3,2)circle (1.5pt);
\draw [fill=white] (2.6,1.3)circle (1.5pt);
\draw [fill=white] (2.6,2.6)circle (1.5pt);

%
% Node F_2
%
\draw (0,2.2) node {$F_2$};

\draw [line width=1pt,color=red] (-.8,2) to  (-.3,2); 
\draw [line width=1pt,color=red] (.8,2) to  (.3,2);   
\draw [line width=1pt,color=red] (-.3,2) to (.3,2);   

\draw [fill=black] (-.3,2) circle (1.5pt);
\draw [fill=black] (.3,2) circle (1.5pt);
\draw [fill=white] (-.8,2) circle (1.5pt);
\draw [fill=white] (.8,2) circle (1.5pt);

%
% Node F_3
%
\draw (6,2) node {$F_3$};
\draw [line width=1pt,color=red] (5.7,2) to  (5.3,2.6); 
\draw [fill=black] (5.7,2)circle (1.5pt);
\draw [fill=white] (5.3,2.6)circle (1.5pt);

\end{scriptsize}
\end{tikzpicture}

}
\caption{Link cut trees que implementam $\hat T$ e $\hat T^\star$ ilustradas na Figura~\ref{fig:MSF-figura-3}.}
\label{fig:MSF-figura-4}
\end{figure}

Cada aresta~$e$ de~$G$, que liga vértices~$u$ e~$v$ e cuja dual $e^\star$ liga vértices~$F_1$ e $F_2$ de~$G^\star$, é implementada por oito nós de link cut tree.
Os quatro primeiros, denotados por~$e_0$, $e_1$, $e_2$ e~$e_3$, representam~$e$.
Os nós $e_0$, $e_2$ são nós de~$\hat T$, enquanto que $e_1$, $e_3$ são nós de~$\hat T^\star$.
Os nós $e_0$ e $e_2$ representam as orientações~$uv$ e~$vu$ de~$e$.
Nesse caso se $F_1$ é a face à esquerda de~$e$ quando vamos de~$u$ para~$v$, o nó $e_1$ representa a orientação $F_1F_2$ de~$e^\star$ enquanto que $e_3$ representa a orientação $F_2F_1$ de~$e^\star$.
Isso pode ser visto na Figura~\ref{fig:MSF-figura-4}.

Caso~$e$ seja uma aresta de~$T$, então o vértice~$\hat e$ de $\hat T$ é implementado pelos nós~$e_0$,~$e_2$ e a aresta entre eles.
Já os nós $e_1$, $e_3$ individualmente formam as duas folhas $\hat e_1$, $\hat e_3$ de $\hat T^\star$.
Caso~$d$ seja uma aresta de~$T^\star$, então o vértice~$\hat d$ de $\hat T^\star$ é implementado pelos nós~$d_1$,~$d_3$ e a aresta entre eles.
Já os nós $d_0$, $d_2$ individualmente formam as duas folhas $\hat d_0$, $\hat d_2$ de $\hat T$.

Os quatro últimos nós que representam~$e$ na link cut tree são~$e_u$, $e_v$, $e^\star_{F_1}$ e~$e^\star_{F_2}$, que mantém a posição de~$e$ e~$e^\star$ nas ordens cíclicas de~$u$, $v$, $F_1$ e $F_2$, respectivamente.
Por exemplo, a ordem cíclica $D(u)$ é representada pela sequência de nós~$\langle a_u, b_u, c_u\rangle$.
No restante desse capítulo, será usado~$D(u)$ para referenciar a sequência de nós de link cut tree que representam essa ordem cíclica.
À frente, nos referiremos à óctupla $( e_0, e_1, e_2, e_3, e_u, e_v, {e^\star}_{F_1}, {e^\star}_{F_2})$ como a \defi[octupla]{óctupla} de~$e$.
A Figura~\ref{fig:MSF-octo} ilustra a óctupla das arestas~$b$ e $g$.


\begin{figure}[htb]
\centering
\begin{tikzpicture}[line cap=round,line join=round,>=triangle 45,x=1cm,y=1cm]
\clip(-3,-.3) rectangle (10,5.6);

\begin{scriptsize}

%
% Node v
%
\draw [line width=1pt,color=cyan] (0,.5) to  (.35,.35); 
\draw [line width=1pt,color=cyan] (.35,.35) to  (.5,0); 

\draw [line width=1pt,color=cyan] (0,.5) to  (-1,1.8); 
\draw [line width=1pt,color=cyan] (.35,.35) to  (1.05,1.75); 
\draw [line width=1pt,color=cyan] (.5,0) to  (2.5,1); 



\draw [line width=1pt,color=cyan] (-1,1.8) to  (-1,2.2); 
\draw (-.6,1.9) node[anchor=north east] {\tiny $a_0$};
\draw [fill=white] (-1,1.8) circle (1.5pt);

% ^b
\draw [line width=1pt,color=cyan] (2.5,1) to (2.9,1.15); 
\draw (2.5,1)[anchor=north] node {\tiny $d_0$};
\draw [fill=white] (2.5,1) circle (1.5pt);

%
% Node u
%

\draw [line width=1pt,color=cyan] (0,3.5) to  (.3,3.7); 
\draw [line width=1pt,color=cyan] (.5,4) to  (.3,3.7); 

\draw [line width=1pt,color=cyan] (0,3.5) to  (-1,2.2); 
\draw [line width=1pt,color=cyan] (.3,3.7) to  (1.05,2.25); 
\draw [line width=1pt,color=cyan] (.5,4) to  (2.5,3); 

\draw (-1.1,2.35) node {\tiny $a_2$};
\draw [fill=white] (-1,2.2) circle (1.5pt);

\draw (2.4,2.9) node {\tiny $c_2$};
\draw [fill=white] (2.5,3) circle (1.5pt);

% u node path
\draw (-.2,3.5) node {\tiny $a_u$};
\draw [fill=black] (0,3.5) circle (1.5pt);
\draw (.5,4.15) node {\tiny $c_u$};
\draw [fill=black] (.5,4) circle (1.5pt);

%
% Node z
%
\draw [line width=1pt,color=cyan] (4.3,2) to (6.5,2); 
\draw [line width=1pt,color=cyan] (3.7,2.3) to (3,2.7); 
\draw [line width=1pt,color=cyan] (3.7,1.7) to (2.9,1.15); 

% node path
\draw [line width=1pt,color=cyan] (3.7,2.3) to  (3.7,1.7); 
\draw [line width=1pt,color=cyan] (3.7,1.7) to  (4.3,2); 


\draw (3.75,1.55) node {\tiny $d_z$};
\draw [fill=black] (3.7,1.7)circle (1.5pt);
\draw (3.7,2.43) node {\tiny $c_z$};
\draw [fill=black] (3.7,2.3)circle (1.5pt);

\draw (3.2,2.7) node {\tiny $c_0$};
\draw [fill=white] (3,2.7)circle (1.5pt);
\draw (3.05,1.25)[anchor=north] node {\tiny $d_2$};
\draw [fill=white] (2.9,1.15)circle (1.5pt);

%eF0 -- e3
\draw [line width=1pt,color=red] (3.3,4.7) to  (5.25,2.25);


%
% Node F_0
%
\draw [line width=1pt,color=red] (2.7,4.7) to  (3.8,4.7); 
\draw [line width=1pt,color=red] (3.8,4.7) to  (4,5); 
\draw [line width=1pt,color=red] (4,5) to  (2.5,5); 

\draw [line width=1pt,color=red] (2.5,5) to[in=180,out=180,looseness=2]  (-1.2,2); 
\draw [line width=1pt,color=red] (4,5) to[in=-20,out=10,looseness=3]  (5.25,1.75); 
\draw [line width=1pt,color=red] (3.5,5) to[in=-30,out=20,looseness=4.5]  (2.8,.85); 

%^c
\draw [line width=1pt,color=red] (2.6,2.6) to  (2.9,3.1);
\draw [line width=1pt,color=red] (2.7,4.7) to  (2.9,3.1);

\draw (-1.3,1.9) node {\tiny $a_3$};
\draw [fill=white] (-1.2,2)circle (1.5pt);
\draw (3.07,3.1) node {\tiny $c_1$};
\draw [fill=white] (2.9,3.1)circle (1.5pt);
\draw (2.9,.7) node {\tiny $d_1$};
\draw [fill=white] (2.8,.85)circle (1.5pt);


%fF3 -- 
\draw [line width=1pt,color=red] (6.5,3.9) to  (6.5,3);

% ^f
\draw [line width=1pt,color=cyan] (6.8,2.5) to[in=-60,out=60,looseness=1.5] (6.65,3.8);
\draw [line width=1pt,color=cyan] (6.2,2.5) to[in=-120,out=120,looseness=1.5] (6.35,3.8);

\draw [line width=1pt,color=red] (3.8,4.7) to  (6.5,3.95);
\draw (6.65,4.05) node {\tiny $f_0$};
\draw [fill=white] (6.5,3.95)circle (1.5pt);
\draw (6.65,3.55) node {\tiny $f_2$};
\draw [fill=white] (6.5,3.65)circle (1.5pt);
\draw (6.8,3.8) node {\tiny $f_3$};
\draw [fill=white] (6.65,3.8)circle (1.5pt);
\draw (6.2,3.8) node {\tiny $f_1$};
\draw [fill=white] (6.35,3.8)circle (1.5pt);


\draw (2.5,5)[anchor=south] node {\tiny $a_{F_0}$};
\draw [fill=black] (2.5,5)circle (1.5pt);
\draw (2.5,4.6) node {\tiny $c_{F_0}$};
\draw [fill=black] (2.7,4.7)circle (1.5pt);
\draw (3.5,5)[anchor=south] node {\tiny $d_{F_0}$};
\draw [fill=black] (3.5,5)circle (1.5pt);

%
% Node y
%

%node path
\draw [line width=1pt,color=cyan] (6.5,2) to  (6.2,2.5); 
\draw [line width=1pt,color=cyan] (6.5,2) to  (6.8,2.5); 


\draw (6,2.5) node {\tiny $f_y$};
\draw (7,2.5) node {\tiny $f_y$};
\draw [fill=black] (6.2,2.5)circle (1.5pt);
\draw [fill=black] (6.8,2.5)circle (1.5pt);


%
% Node F_1
%
%node path
\draw [line width=1pt,color=red] (2.3,2.3) to  (2.3,1.7); 
\draw [line width=1pt,color=red] (1.7,2) to  (2.3,1.7); 

\draw [line width=1pt,color=red] (2.6,2.6) to  (2.3,2.3); 
\draw [line width=1pt,color=red] (2.6,1.3) to  (2.3,1.7); 
\draw [line width=1pt,color=red] (1.7,2) to  (1.3,2); 

% F_1 to F_2
\draw [line width=1pt,color=red] (.8,2) to  (1.3,2); 

\draw (2.3,1.7)[anchor=north] node {\tiny $d_{F_1}$};
\draw [fill=black] (2.3,1.7)circle (1.5pt);
\draw (2.25,2.2)[anchor=west] node {\tiny $c_{F_1}$};
\draw [fill=black] (2.3,2.3)circle (1.5pt);

\draw (2.8,1.4) node {\tiny $d_3$};
\draw [fill=white] (2.6,1.3)circle (1.5pt);
\draw (2.75,2.55) node {\tiny $c_3$};
\draw [fill=white] (2.6,2.6)circle (1.5pt);

%
% Node F_2
%

\draw [line width=1pt,color=red] (-.8,2) to  (-.3,2); 
\draw [line width=1pt,color=red] (.8,2) to  (.3,2);   
\draw [line width=1pt,color=red] (-.3,2) to (.3,2);   

\draw [fill=black] (-.2,2) circle (1.5pt);
\draw (-.2,1.8) node {\tiny $a_{F_2}$};
\draw (-.75,2.1) node {\tiny $a_1$};
\draw [fill=white] (-.8,2) circle (1.5pt);

%
% Node F_3
%
\draw (6.3,3) node {\tiny $f_{F_3}$};
\draw [fill=black] (6.5,3)circle (1.5pt);

\draw (5,2)[anchor=south] node {\tiny $e_0$};
\draw [fill=white] (5,2)circle (1.5pt);
\draw (5.5,2)[anchor=south] node {\tiny $e_2$};
\draw [fill=white] (5.5,2)circle (1.5pt);
\draw (5.25,2.25)[anchor=south] node {\tiny $e_3$};
\draw [fill=white] (5.25,2.25)circle (1.5pt);
\draw (5.25,1.75)[anchor=west] node {\tiny $e_1$};
\draw [fill=white] (5.25,1.75)circle (1.5pt);
\draw (4.3,2.2) node {\tiny $e_z$};
\draw [fill=black] (4.3,2)circle (1.5pt);
\draw (4,5)[anchor=north west] node {\tiny $e_{F_0}$};
\draw [fill=black] (4,5)circle (1.5pt);
\draw (3.3,4.7)[anchor=north] node {\tiny $e_{F_0}$};
\draw [fill=black] (3.3,4.7)circle (1.5pt);
\draw (6.5,1.8) node {\tiny $e_y$};
\draw [fill=black] (6.5,2)circle (1.5pt);

\draw (1.2,2.33) node {\tiny $b_0$};
\draw [fill=white] (1.05,2.25) circle (1.5pt);
\draw (.8,2.16) node {\tiny $b_1$};
\draw [fill=white] (.8,2) circle (1.5pt);
\draw (1,1.65)[anchor=west] node {\tiny $b_2$};
\draw [fill=white] (1.05,1.75) circle (1.5pt);
\draw (1.4,1.85) node {\tiny $b_3$};
\draw [fill=white] (1.3,2)circle (1.5pt);
\draw (.35,.35)[anchor=west] node {\tiny $b_v$};
\draw [fill=black] (.35,.35) circle (1.5pt);
\draw (.19,3.82) node {\tiny $b_u$};
\draw [fill=black] (.3,3.7) circle (1.5pt);
\draw (.3,1.8) node {\tiny $b_{F_2}$};
\draw [fill=black] (.2,2) circle (1.5pt);
\draw (1.8,2.13) node {\tiny $b_{F_1}$};
\draw [fill=black] (1.7,2)circle (1.5pt);
\end{scriptsize}
\end{tikzpicture}

\caption{Óctuplas das arestas~$b$ e~$g$.}
\label{fig:MSF-octo}
\end{figure}


Ademais, a sequência dos nós de link cut tree que representam as arestas de~$G$ incidentes a~$v$ é armazenada em uma árvore binária de busca com chave implícita.
A ABB é utilizada para obter o primeiro e o último nó da ordem cíclica, o predecessor de um dado nó e identificar se dois nós estão na mesma ordem cíclica.
A raiz dessa ABB possui um campo $\node$ que armazena o identificador $v$.

Uma árvore dinâmica plana é uma link cut tree (como as ilustradas na Figura~\ref{fig:MSF-figura-4}) que representa uma dada árvore em junção às ABBs que armazenam as sequências de nós que representam as ordens cíclicas de cada vértice dessa árvore.
Para solucionar o problema da floresta maximal de peso mínimo em grafos dinâmicos planos, serão mantidas as duas ADPs~$\hat T$ e~$\hat T^\star$ e uma tabela de símbolos~$H$, que guarda as óctuplas de cada aresta de~$G$ e usa como chave o identificador dessas arestas.


Dessa forma, $G$ e uma floresta geradora maximal~$F$ de~$G$ de peso mínimo é representada por dois conjuntos de ADPs: o conjunto~$\hat F$ das ADPs das componentes de~$F$ e o conjunto~$\hat F^\star$ das ADPs do dual de cada componente de~$F$.

Agora que a estrutura de ADPs foi apresentada, podemos introduzir na próxima seção a biblioteca dessa estrutura de dados.

\section{Biblioteca de árvores dinâmicas planas}
\label{sec:EODT}

Árvores dinâmicas planas dão suporte à biblioteca listada a seguir.
Para as descrições que se seguem, considere~$u$ e~$v$ vértices da árvore representada,
$p$ um nó da ordem cíclica~$D(u)$ e~$q$ um nó de~$D(v)$.

\begin{itemize}
\item \LCOMakeOcto($e$, $w$): Recebe um identificador~$e$ e um peso~$w$ e cria e retorna uma óctupla de nós de ADP associados a~$e$ com peso~$w$.
\item \LCODestroyOcto($H$, $e$): Recebe uma tabela de símbolos~$H$ e um identificador~$e$ e desaloca a óctupla associada a~$e$ da memória.

\item \LCOConnected($p$, $q$): Retorna verdadeiro se os nós~$p$ e~$q$ estiverem na mesma ADP e falso caso contrário.
\item \LCOFindNode($p$): Recebe um nó $p$ e retorna o vértice da~$\hat F$ que contêm $p$ em sua ordem cíclica.

Por exemplo: Na Figura~\ref{fig:MSF-figura-4}, \LCOFindNode($a_u$) retorna $\hat u$, \LCOFindNode($b_{F_1}$) retorna $\hat F_1$, \LCOFindNode($a_0$) e \LCOFindNode($a_2$) retornam~$\hat a$, enquanto que \LCOFindNode($a_1$) e \LCOFindNode($a_3$) retornam $\hat a_1$ e~$\hat a_3$, respectivamente.
\item \LCOAddCost($p$, $w$): Atribui o peso~$w$ ao vértice que contém $p$ em sua ordem cíclica.
\item \LCOMax($p$, $q$): Retorna o nó de peso máximo no percurso entre os nós~$p$ e~$q$.
Essa operação assume que $p$ e~$q$ são nós da mesma ADP.
\item \LCOMin($p$, $q$): Retorna o nó de peso mínimo no percurso entre os nós~$p$ e~$q$.
Essa operação assume que~$p$ e~$q$ são nós da mesma ADP.
\item \LCOCycle($p$): Permuta ciclicamente $D(u)$ de forma que o nó~$p$ seja o último na ordem.
Se a ordem inicial tem forma $\alpha p \beta$, então a ordem resultante é $\beta\alpha p$.
\item \treapPredecessor($p$): Retorna o predecessor do nó~$p$ na ordem cíclica.
\end{itemize}

É possível unir e separar nós das ADPs com as rotinas \LCOMerge{} e \LCOSplit{}.

\begin{itemize}
\item \LCOMerge($p$, $q$, $z$): Recebe dois nós~$p$ e~$q$ de mesmo peso e une os vértices $u$ e $v$ que possuem $p$ e $q$ em suas ordens cíclicas.
Se~$\alpha$ e~$\beta$ são respectivamente as ordens cíclicas de~$u$ e~$v$, então $\alpha\beta$ será a ordem cíclica do vértice resultante~$z$. 
Essa operação assume que~$u$ e~$v$ estão em ADPs distintas.

\item \LCOSplit($p$, $v$, $z$): Substitui o vértice~$u$ que contém~$p$ em sua ordem cíclica por dois vértices~$v$ e~$z$.
Se a ordem cíclica $D(u)$ tem forma $\alpha p\beta$, então $\alpha$ e~$p\beta$ serão as ordens cíclicas de~$v$ e~$z$, respectivamente.

\end{itemize}
\begin{figure}[htb]
\scalebox{1}{
\centering
\begin{tikzpicture}[line cap=round,line join=round,>=triangle 45,x=1cm,y=1cm]
\clip(-1.2,-.8) rectangle (7,3);


% u

\draw [line width=1pt] (0,2) to[in=-90,out=180,looseness=1]  (-1,3); 
\draw [line width=1pt] (0,2) to[in=-90,out=180,looseness=1]  (-.5,3); 
\draw [line width=1pt] (0,2) to (0,3); 
\draw [line width=1pt] (0,2) to[in=-90,out=0,looseness=1]  (.5,3); 
\draw [line width=1pt] (0,2) to[in=-90,out=0,looseness=1]  (1,3); 

%v
\draw [line width=1pt] (0,0.3) to[in=90,out=180,looseness=1]  (-1,-1); 
\draw [line width=1pt] (0,0.3) to[in=90,out=180,looseness=1]  (-.5,-1); 
\draw [line width=1pt] (0,0.3) to (0,-1); 
\draw [line width=1pt] (0,0.3) to[in=90,out=0,looseness=1]  (.5,-1); 
\draw [line width=1pt] (0,0.3) to[in=90,out=0,looseness=1]  (1,-1); 


%z

\draw [line width=1pt] (5,1) to[in=-90,out=180,looseness=1]  (4,3); 
\draw [line width=1pt] (5,1) to[in=-90,out=135,looseness=1]  (4.5,3); 
\draw [line width=1pt] (5,1) to (5,3); 
\draw [line width=1pt] (5,1) to[in=-90,out=45,looseness=1]  (5.5,3); 
\draw [line width=1pt] (5,1) to[in=-90,out=0,looseness=1]  (6,3); 

\draw [line width=1pt] (5,1) to[in=90,out=180,looseness=1]  (4,-1); 
\draw [line width=1pt] (5,1) to[in=90,out=225,looseness=1]  (4.5,-1); 
\draw [line width=1pt] (5,1) to (5,-1); 
\draw [line width=1pt] (5,1) to[in=90,out=-45,looseness=1]  (5.5,-1); 
\draw [line width=1pt] (5,1) to[in=90,out=0,looseness=1]  (6,-1); 



\draw (0.25,2.7) node[color=red] {$\alpha$};
\draw [line width=1pt,color=red] (.5,2) to[in=90,out=90,looseness=2]  (-.5,2); % 4 -- 6
\draw [line width=.1pt,color=red,->] (-.5,2.1) to  (-.5,1.9); % 4 -- 6

\draw (0.25,-.5) node[color=red] {$\beta$};
\draw [line width=1pt,color=red] (.5,.3) to[in=-90,out=-90,looseness=2]  (-.5,.3); % 4 -- 6
\draw [line width=.1pt,color=red,->] (.5,.2) to  (.5,.4);


\draw [line width=1pt,->] (1.5,1) to  (3.5,1); % 4 -- 6
\draw [line width=1pt,<-] (1.5,.8) to  (3.5,.8); % 4 -- 6

\begin{scriptsize}
\draw [fill=black] (0,0.3) circle (2.5pt);
\draw (0,0.4) node[anchor=south] {$v$};
\draw (-.6,.3) node[anchor=north east] {$q$};
\draw [fill=black] (0,2) circle (2.5pt);
\draw (0,1.9) node[anchor=north] {$u$};
\draw (-.6,2.1) node[anchor=north east] {$p$};


\draw [fill=black] (5,1) circle (2.5pt);
\draw (5.1,1) node[anchor=north west] {$z$};
\draw (4,.3) node[anchor=north east] {$q$};
\draw (4,2.1) node[anchor=north east] {$p$};


\draw (2.1,1) node[anchor=south] {\LCOMerge($p$, $q$, $z$)};
\draw (2.55,.8) node[anchor=north] {\LCOSplit($q$, $u$, $v$)};




\end{scriptsize}
\end{tikzpicture}

}
\caption{Efeito de \LCOMerge{} e \LCOSplit{}.}
\label{fig:MSF-Merge-Split}
\end{figure}



A implementação dessas rotinas será apresentada na Seção~\ref{sec:implementacao-ADP}.
O consumo de tempo de $\LCOMakeOcto$ e $\LCODestroyOcto$ é~$\O{1}$ e o consumo das restantes rotinas é $\O{\lg m}$ esperado amortizado, onde $m$ é o número de arestas no grafo corrente.

\section{Resolvendo MSF com ADPs}
\label{sec:MSFcomADPs}

\subsection{Criação de grafo plano ponderado dinâmico}

Um grafo plano ponderado dinâmico~$G$ será implementado como a tabela de símbolos~$H$ descrita no final da Seção~\ref{sec:ADP} e um inteiro $p$ que representa o peso da floresta geradora de pesos mínimos~$F$.
Como~$G$ é inicializado vazio, então $p$ é igual a~$0$.
A implementação de \MSFCreate{} é descrita no Algoritmo~\ref{Algo:MSFCreate} e seu consumo de tempo é~$\O{1}$.

\begin{algorithm}[htb]
\caption{\MSFCreate($n$)}
\label{Algo:MSFCreate}
\begin{algorithmic}[1]
\State $G.H$ $\gets$ \hashCreate($n$)
\State $G.p$ $\gets$ $0$
\State \Return $G$
\end{algorithmic}
\end{algorithm}

\subsection{Obtenção de peso}

Para obter o peso da floresta geradora de pesos mínimos~$F$, simplesmente retornamos o valor~$p$, como pode ser visto na implementação de $\MSFweight$ no Algoritmo~\ref{Algo:MSFweight}.
Seu consumo é~$\O{1}$.
\begin{algorithm}[htb]
\caption{\MSFweight($G$)}
\label{Algo:MSFweight}
\begin{algorithmic}[1]
\State \Return $G$.$p$
\end{algorithmic}
\end{algorithm}

\subsection{Mudança de peso}
A primeira rotina de MSF que vamos apresentar é \MSFupdate{}, cuja implementação pode ser vista no Algoritmo~\ref{Algo:MSFupdate}.
Essa é a primeira rotina apresentada, pois não modifica a estrutura do grafo.
Encorajamos a leitora ou o leitor a acompanhar a explicação do funcionamento dessa rotina antes de ler o pseudocódigo linha-a-linha.
Essa explicação é feita nos próximos parágrafos e detalha o embasamento teórico utilizado pelo algoritmo, seu funcionamento e ilustra a execução de \MSFupdate($G$, $a$, $5$) como exemplo, onde $G$ é o grafo da Figura~\ref{fig:MSF-figura-4}.


\begin{algorithm}[htb]
\caption{\MSFupdate($G$, $e$, $w$)}
\label{Algo:MSFupdate}
\begin{algorithmic}[1]
	\State $e_0$, $e_1$, $e_2$, $e_3$, $v_0$, $v_1$, $v_2$, $v_3$ $\gets$ $G.H(e)$ \label{Algo:MSFupdate:linha:const1}
\For {$i \in \{0,1,2,3\}$} \label{Algo:MSFupdate:linha:laçosetpeso}
\State  \LCOAddCost($e_i$, $w$) \label{Algo:MSFupdate:linha:LCOAddCost}
\EndFor

\If {\LCOFindNode($e_0$) $=$ \LCOFindNode($e_2$)}\label{Algo:MSFupdate:linha:if} \Comment{$e\in F$}
\State $d$ $\gets$ \LCOMin($e_1$, $e_3$)\label{Algo:MSFupdate:linhamin}

\State $d_0$, $d_1$, $d_2$, $d_3$, $y_0$, $y_1$, $y_2$, $y_3$ $\gets$ $G.H(d)$ \label{Algo:MSFupdate:linha:const2}
\If {$d_0$.$w$ < $w$}\label{Algo:MSFupdate:linha:compara}
\State $G$.$p$ $\gets$ $G.p-e_0.w+d_0$.$w$ \label{Algo:MSFupdate:linha:atualizaPeso}
\State \LCOCycle($d_3$); \LCOSplit($d_3$, $\hat d_1$, $\hat d_3$)
\State \LCOCycle($e_2$); \LCOSplit($e_2$, $\hat e_0$, $\hat e_2$ )\label{Algo:MSFupdate:linhasplit}

\State \LCOMerge($d_0$, $d_2$, $\hat d$); \LCOMerge($e_1$, $e_3$, $\hat e$)\label{Algo:MSFupdate:linhas:Merge}

\EndIf

\Else\Comment{$e^\star\in F^\star$}
\State $d$ $\gets$ \LCOMax($e_0$, $e_2$)\label{Algo:MSFupdate:dualinicio}

\State $d_0$, $d_1$, $d_2$, $d_3$, $y_0$, $y_1$, $y_2$, $y_3$ $\gets$ $G.H(d)$\label{Algo:MSFupdate:linha:const3}
\If {$d_0$.$w$ > $w$}
\State $G$.$p$ $\gets$ $G$.$p-e_0.w+d_0$.$w$\label{Algo:MSFupdate:linha:const4}
\State \LCOCycle($d_0$); \LCOSplit($d_0$, $\hat d_2$, $\hat d_0$)
\State \LCOCycle($e_1$); \LCOSplit($e_1$, $\hat e_3$, $\hat e_1$)

\State \LCOMerge($d_1$, $d_3$, $\hat d$); \LCOMerge($e_0$, $e_2$, $\hat e$)

\EndIf
\EndIf\label{Algo:MSFupdate:dualfim}
\end{algorithmic}
\end{algorithm}

Dado um grafo~$G$, um \defi{corte} é um conjunto de arestas cuja remoção aumenta o número de componentes conexas de~$G$.
Seja~$F$ uma floresta maximal de~$G$ e~$uv$ uma aresta de~$F$.
Então existe um corte associado ao par $(F, uv)$ dado pela união de $uv$ com as arestas de~$G$ que reconectam as duas árvores de~$F$ geradas pela remoção de $uv$.
Por exemplo, o corte~$(F, a)$ é o conjunto de arestas~$\{a,~b,~c\}$.

\begin{theorem}[Proposição 4.6.1~\cite{Diestel}]
\label{teo:cutset}
Seja~$G$ um grafo plano, $F$ uma floresta maximal de~$G$ e~$uv$ uma aresta de~$F$.
Então o conjunto
$$
(F, uv)^\star := \{e^\star:e\in (F, uv)\}
$$
forma um ciclo em $G^\star$.
\end{theorem}

Quando atualizamos o peso de uma aresta~$e$ de~$G$ para um novo peso~$w$, precisamos eventualmente atualizar a floresta~$F$ de peso mínimo que está sendo mantida de forma que ela continue sendo maximal e de peso mínimo.
Existem dois casos a serem tratados: quando $e$ é uma aresta de $F$ e quando $e^\star$ é uma aresta de~$F^\star$.

No Algoritmo~\ref{Algo:MSFupdate}, essa distinção de casos é feita na linha~\ref{Algo:MSFupdate:linha:if}.
Se \LCOFindNode($e_0$) e \LCOFindNode($e_2$) retornarem o mesmo identificador~$\hat e$, então deduzimos que $e$ é uma aresta de~$F$.
Caso contrário, temos que $e^\star$ é uma aresta de $F^\star$.
Por exemplo, observando a Figura~\ref{fig:MSF-figura-3}, notamos que~\LCOFindNode($a_0$) e \LCOFindNode($a_2$) retornam o mesmo identificador~$\hat a$,
enquanto que \LCOFindNode($c_0$) e \LCOFindNode($c_2$) retornam os identificadores distintos~$\hat c_0$ e~$\hat c_2$.

O primeiro caso é tratado entre as linhas~\ref{Algo:MSFupdate:linhamin} e~\ref{Algo:MSFupdate:linhas:Merge} do Algoritmo~\ref{Algo:MSFupdate}.
Nele~$e$ é uma aresta de~$F$ e é necessário verificar se não há alguma aresta no corte~$(F, e)$ com peso menor do que novo peso~$w$, pois se tal aresta existir, então~$F$ e~$F^\star$ não serão mais de peso mínimo em~$G$ e máximo em~$G^\star$, respectivamente.
Nesse caso, para corrigir isso, é necessário remover a aresta $e$ de~$F$ e adicionar em~$F$ uma aresta~$d$ de menor peso do corte~$(F, e)$.
Simetricamente, também é necessário remover $d^\star$ de~$F^\star$ e adicionar~$e^\star$ nessa floresta.

Pelo Teorema~\ref{teo:cutset}, o conjunto~$(F, e)^\star$ forma um ciclo em~$G^\star$.
Como~$e$ é a única aresta de~$F$ cuja aresta dual~$e^\star$ está em~$(F, e)^\star$, então as demais arestas desse corte formam um caminho~$P$ em~$F^\star$ ligando as faces incidentes a~$e^\star$.
Dessa forma, para obter~$d$, basta obter o mínimo no caminho~$P$.

As faces incidentes a~$e^\star$ são acessíveis nas ADPs pelos nós~$e_1$ e~$e_3$.
Assim, para obter~$d$, é executado \LCOMin($e_1$, $e_3$), que é feito na linha~\ref{Algo:MSFupdate:linhamin} do Algoritmo~\ref{Algo:MSFupdate}.
A Figura~\ref{fig:MSF-figura-5} ilustra o caminho~$P$ para o corte $(F,~a)$.
No exemplo, \LCOMin($a_1$, $a_3$) retorna o identificado $\hat c$.


\begin{figure}[!h]
\begin{subfigure}{.3\textwidth}
\scalebox{.8}{
\begin{tikzpicture}[line cap=round,line join=round,>=triangle 45,x=1cm,y=1cm]
\clip(-3,1.5) rectangle (4.5,5.4);

\begin{scriptsize}




%
% Node F_0
%
\draw (3,5.2) node {$\hat F_0$};
\draw [line width=1pt,color=red] (2.7,4.7) to  (3.8,4.7); 
\draw [line width=1pt,color=red] (3.8,4.7) to  (4,5); 
\draw [line width=1pt,color=red] (4,5) to  (2.5,5); 

\draw [line width=1pt,color=red] (2.5,5) to[in=180,out=180,looseness=2]  (-1.2,2); 

%^c
\draw [line width=1pt,color=red] (2.6,2.6) to  (2.9,3.1);
\draw [line width=1pt,color=red] (2.7,4.7) to  (2.9,3.1);

\draw (-1.2,2.15) node {\tiny $a_3$};
\draw (-1.2,1.85) node {\tiny $5$};
\draw [fill=white] (-1.2,2)circle (1.5pt);
\draw (2.7,3.1) node {\tiny $c_1$};
\draw (3.07,3.1) node {\tiny $3$};
\draw [fill=white] (2.9,3.1)circle (1.5pt);



\draw (2.5,5)[anchor=south] node {\tiny $\infty$};
\draw [fill=black] (2.5,5)circle (1.5pt);
\draw (2.5,4.6) node {\tiny $\infty$};
\draw [fill=black] (2.7,4.7)circle (1.5pt);
\draw (3.5,5)[anchor=south] node {\tiny $\infty$};
\draw [fill=black] (3.5,5)circle (1.5pt);
\draw (4,5)[anchor=north west] node {\tiny $\infty$};
\draw [fill=black] (4,5)circle (1.5pt);
\draw (3.8,4.7)[anchor=north] node {\tiny $\infty$};
\draw [fill=black] (3.8,4.7)circle (1.5pt);
\draw (3.3,4.7)[anchor=north] node {\tiny $\infty$};
\draw [fill=black] (3.3,4.7)circle (1.5pt);




%
% Node F_1
%
\draw (2.1,2) node {$\hat F_1$};
%node path
\draw [line width=1pt,color=red] (2.3,2.3) to  (2.3,1.7); 
\draw [line width=1pt,color=red] (1.7,2) to  (2.3,1.7); 

\draw [line width=1pt,color=red] (2.6,2.6) to  (2.3,2.3); 
\draw [line width=1pt,color=red] (1.7,2) to  (1.3,2); 

% F_1 to F_2
\draw [line width=1pt,color=red] (.8,2) to  (1.3,2); 

\draw (1.8,2.13) node {\tiny $\infty$};
\draw [fill=black] (1.7,2)circle (1.5pt);
\draw (2.3,1.7)[anchor=north] node {\tiny $\infty$};
\draw [fill=black] (2.3,1.7)circle (1.5pt);
\draw (2.25,2.2)[anchor=west] node {\tiny $\infty$};
\draw [fill=black] (2.3,2.3)circle (1.5pt);
\draw (1.3,2.16) node {\tiny $7$}; %b_3
\draw [fill=white] (1.3,2)circle (1.5pt);

\draw (2.45,2.6) node {\tiny $c_3$};
\draw (2.75,2.55) node {\tiny $3$};
\draw [fill=white] (2.6,2.6)circle (1.5pt);

%
% Node F_2
%
\draw (0,2.2) node {$\hat F_2$};

\draw [line width=1pt,color=red] (-.8,2) to  (-.3,2); 
\draw [line width=1pt,color=red] (.8,2) to  (.3,2);   
\draw [line width=1pt,color=red] (-.3,2) to (.3,2);   

\draw [fill=black] (-.2,2) circle (1.5pt);
\draw (-.2,1.8) node {\tiny $\infty$};
\draw [fill=black] (.2,2) circle (1.5pt);
\draw (.3,1.8) node {\tiny $\infty$};
\draw (-.75,2.15) node {\tiny $a_1$};
\draw [fill=white] (-.8,2) circle (1.5pt);
\draw (-.75,1.85) node {\tiny $5$};
\draw (.8,2.16) node {\tiny $7$}; %b_1
\draw [fill=white] (.8,2) circle (1.5pt);

\end{scriptsize}
\end{tikzpicture}

}
\caption{Caminho entre os nós $a_1$ e~$a_3$ na árvore~$\hat T^\star$ da Figura~\ref{fig:MSF-figura-4}.}
\label{fig:MSF-figura-5}
\end{subfigure}
\hspace{1cm}
\begin{subfigure}{.6\textwidth}
\scalebox{.8}{
\begin{tikzpicture}[line cap=round,line join=round,>=triangle 45,x=1cm,y=1cm]
\clip(-3,-.3) rectangle (10,5.6);

\begin{scriptsize}

%
% Node v
%
\draw (0,0) node {$\hat v$};
\draw [line width=1pt,color=cyan] (0,.5) to  (.35,.35); 
\draw [line width=1pt,color=cyan] (.35,.35) to  (.5,0); 

\draw [line width=1pt,color=cyan] (0,.5) to  (-1,1.8); 
\draw [line width=1pt,color=cyan] (.35,.35) to  (1.05,1.75); 
\draw [line width=1pt,color=cyan] (.5,0) to  (2.5,1); 


%\draw (0,.5)[anchor=east] node {\tiny $a_v$};
%%\draw (.35,.35)[anchor=west] node {\tiny $b_v$};
%%\draw (.5,0)[anchor=north west] node {\tiny $d_v$};
\draw [fill=black] (0,.5) circle (1.5pt);
\draw [fill=black] (.35,.35) circle (1.5pt);
\draw [fill=black] (.5,0) circle (1.5pt);

%\draw [line width=1pt,color=cyan] (-1,1.8) to  (-1,2.2); 
\draw (-.6,1.9) node[anchor=north east] {\tiny $a_0$};
\draw [fill=white] (-1,1.8) circle (1.5pt);

% ^b
\draw [line width=1pt,color=cyan] (2.5,1) to (2.9,1.15); 
%\draw (1,1.65)[anchor=west] node {\tiny $b_2$};
\draw [fill=white] (1.05,1.75) circle (1.5pt);
%\draw (2.5,1)[anchor=north] node {\tiny $d_0$};
\draw [fill=white] (2.5,1) circle (1.5pt);

%
% Node u
%
\draw (0,4) node {$\hat u$};

\draw [line width=1pt,color=cyan] (0,3.5) to  (.3,3.7); 
\draw [line width=1pt,color=cyan] (.5,4) to  (.3,3.7); 

\draw [line width=1pt,color=cyan] (0,3.5) to  (-1,2.2); 
\draw [line width=1pt,color=cyan] (.3,3.7) to  (1.05,2.25); 
\draw [line width=1pt,color=cyan] (.5,4) to  (2.5,3); 

\draw (-1.1,2.35) node {\tiny $a_2$};
\draw [fill=white] (-1,2.2) circle (1.5pt);
%\draw (1.2,2.33) node {\tiny $b_0$};
\draw [fill=white] (1.05,2.25) circle (1.5pt);

\draw (2.4,2.9) node {\tiny $c_2$};
\draw [fill=white] (2.5,3) circle (1.5pt);

% u node path
%\draw (-.2,3.5) node {\tiny $a_u$};
\draw [fill=black] (0,3.5) circle (1.5pt);
%\draw (.5,4.15) node {\tiny $c_u$};
\draw [fill=black] (.5,4) circle (1.5pt);
%%\draw (.19,3.82) node {\tiny $b_u$};
\draw [fill=black] (.3,3.7) circle (1.5pt);

%
% Node z
%
\draw (3.95,2.1) node {$\hat z$};
\draw [line width=1pt,color=cyan] (4.3,2) to (6.5,2); 
\draw [line width=1pt,color=cyan] (3.7,2.3) to (3,2.7); 
\draw [line width=1pt,color=cyan] (3.7,1.7) to (2.9,1.15); 

% node path
\draw [line width=1pt,color=cyan] (3.7,2.3) to  (3.7,1.7); 
\draw [line width=1pt,color=cyan] (3.7,1.7) to  (4.3,2); 


%\draw (3.75,1.55) node {\tiny $d_z$};
\draw [fill=black] (3.7,1.7)circle (1.5pt);
%\draw (3.7,2.43) node {\tiny $c_z$};
\draw [fill=black] (3.7,2.3)circle (1.5pt);
%\draw (4.3,2.2) node {\tiny $e_z$};
\draw [fill=black] (4.3,2)circle (1.5pt);

\draw (3.2,2.7) node {\tiny $c_0$};
\draw [fill=white] (3,2.7)circle (1.5pt);
%\draw (3.05,1.25)[anchor=north] node {\tiny $d_2$};
\draw [fill=white] (2.9,1.15)circle (1.5pt);

%eF0 -- e3
\draw [line width=1pt,color=red] (3.3,4.7) to  (5.25,2.25);


%
% Node F_0
%
\draw (3,5.2) node {$\hat F_0$};
\draw [line width=1pt,color=red] (2.7,4.7) to  (3.8,4.7); 
\draw [line width=1pt,color=red] (3.8,4.7) to  (4,5); 
\draw [line width=1pt,color=red] (4,5) to  (2.5,5); 

\draw [line width=1pt,color=red] (2.5,5) to[in=180,out=180,looseness=2]  (-1.2,2); 
\draw [line width=1pt,color=red] (4,5) to[in=-20,out=10,looseness=3]  (5.25,1.75); 
\draw [line width=1pt,color=red] (3.5,5) to[in=-30,out=20,looseness=4.5]  (2.8,.85); 

%^c
%\draw [line width=1pt,color=red] (2.6,2.6) to  (2.9,3.1);
\draw [line width=1pt,color=red] (2.7,4.7) to  (2.9,3.1);

\draw (-1.3,1.9) node {\tiny $a_3$};
\draw [fill=white] (-1.2,2)circle (1.5pt);
\draw (3.07,3.1) node {\tiny $c_1$};
\draw [fill=white] (2.9,3.1)circle (1.5pt);
%\draw (2.9,.7) node {\tiny $d_1$};
\draw [fill=white] (2.8,.85)circle (1.5pt);


%fF3 -- 
\draw [line width=1pt,color=red] (6.5,3.9) to  (6.5,3);

% ^f
\draw [line width=1pt,color=cyan] (6.8,2.5) to[in=-60,out=60,looseness=1.5] (6.65,3.8);
\draw [line width=1pt,color=cyan] (6.2,2.5) to[in=-120,out=120,looseness=1.5] (6.35,3.8);

\draw [line width=1pt,color=red] (3.8,4.7) to  (6.5,3.95);
%\draw (6.65,4.05) node {\tiny $f_0$};
\draw [fill=white] (6.5,3.95)circle (1.5pt);
%\draw (6.65,3.55) node {\tiny $f_2$};
\draw [fill=white] (6.5,3.65)circle (1.5pt);
%\draw (6.8,3.8) node {\tiny $f_3$};
\draw [fill=white] (6.65,3.8)circle (1.5pt);
%\draw (6.2,3.8) node {\tiny $f_1$};
\draw [fill=white] (6.35,3.8)circle (1.5pt);


%\draw (2.5,5)[anchor=south] node {\tiny $a_{F_0}$};
\draw [fill=black] (2.5,5)circle (1.5pt);
%\draw (2.5,4.6) node {\tiny $c_{F_0}$};
\draw [fill=black] (2.7,4.7)circle (1.5pt);
%\draw (3.5,5)[anchor=south] node {\tiny $d_{F_0}$};
\draw [fill=black] (3.5,5)circle (1.5pt);
%\draw (4,5)[anchor=north west] node {\tiny $e_{F_0}$};
\draw [fill=black] (4,5)circle (1.5pt);
%\draw (3.8,4.7)[anchor=north] node {\tiny $f_{F_0}$};
\draw [fill=black] (3.8,4.7)circle (1.5pt);
%\draw (3.3,4.7)[anchor=north] node {\tiny $e_{F_0}$};
\draw [fill=black] (3.3,4.7)circle (1.5pt);

%
% Node y
%
\draw (6.5,2.35) node {$\hat y$};

%node path
\draw [line width=1pt,color=cyan] (6.5,2) to  (6.2,2.5); 
\draw [line width=1pt,color=cyan] (6.5,2) to  (6.8,2.5); 


%\draw (6.5,1.8) node {\tiny $e_y$};
%\draw (6,2.5) node {\tiny $f_y$};
%\draw (7,2.5) node {\tiny $f_y$};
\draw [fill=black] (6.5,2)circle (1.5pt);
\draw [fill=black] (6.2,2.5)circle (1.5pt);
\draw [fill=black] (6.8,2.5)circle (1.5pt);


%
% Node F_1
%
\draw (2.1,2) node {$\hat F_1$};
%node path
\draw [line width=1pt,color=red] (2.3,2.3) to  (2.3,1.7); 
\draw [line width=1pt,color=red] (1.7,2) to  (2.3,1.7); 

\draw [line width=1pt,color=red] (2.6,2.6) to  (2.3,2.3); 
\draw [line width=1pt,color=red] (2.6,1.3) to  (2.3,1.7); 
\draw [line width=1pt,color=red] (1.7,2) to  (1.3,2); 

% F_1 to F_2
\draw [line width=1pt,color=red] (.8,2) to  (1.3,2); 

%\draw (1.8,2.13) node {\tiny $b_{F_1}$};
\draw [fill=black] (1.7,2)circle (1.5pt);
%\draw (2.3,1.7)[anchor=north] node {\tiny $d_{F_1}$};
\draw [fill=black] (2.3,1.7)circle (1.5pt);
%\draw (2.25,2.2)[anchor=west] node {\tiny $c_{F_1}$};
\draw [fill=black] (2.3,2.3)circle (1.5pt);
%\draw (1.4,1.85) node {\tiny $b_3$};
\draw [fill=white] (1.3,2)circle (1.5pt);

%\draw (2.8,1.4) node {\tiny $d_3$};
\draw [fill=white] (2.6,1.3)circle (1.5pt);
\draw (2.75,2.55) node {\tiny $c_3$};
\draw [fill=white] (2.6,2.6)circle (1.5pt);

%
% Node F_2
%
\draw (0,2.2) node {$\hat F_2$};

\draw [line width=1pt,color=red] (-.8,2) to  (-.3,2); 
\draw [line width=1pt,color=red] (.8,2) to  (.3,2);   
\draw [line width=1pt,color=red] (-.3,2) to (.3,2);   

\draw [fill=black] (-.2,2) circle (1.5pt);
%\draw (-.2,1.8) node {\tiny $a_{F_2}$};
\draw [fill=black] (.2,2) circle (1.5pt);
%\draw (.3,1.8) node {\tiny $b_{F_2}$};
\draw (-.75,2.1) node {\tiny $a_1$};
\draw [fill=white] (-.8,2) circle (1.5pt);
%\draw (.8,2.16) node {\tiny $b_1$};
\draw [fill=white] (.8,2) circle (1.5pt);

%
% Node F_3
%
\draw (6.7,3.2) node {$\hat F_3$};
%\draw (6.3,3) node {\tiny $f_{F_3}$};
\draw [fill=black] (6.5,3)circle (1.5pt);

%\draw (5,2)[anchor=south] node {\tiny $e_0$};
\draw [fill=white] (5,2)circle (1.5pt);
%\draw (5.5,2)[anchor=south] node {\tiny $e_2$};
\draw [fill=white] (5.5,2)circle (1.5pt);
%\draw (5.25,2.25)[anchor=south] node {\tiny $e_3$};
\draw [fill=white] (5.25,2.25)circle (1.5pt);
%\draw (5.25,1.75)[anchor=west] node {\tiny $e_1$};
\draw [fill=white] (5.25,1.75)circle (1.5pt);

\end{scriptsize}
\end{tikzpicture}

}
\caption{ADPs com as arestas $a$ e $c^\star$ cortadas de $\hat T$ e~$\hat T^\star$, respectivamente.}
\label{fig:MSF-figura-6}
\end{subfigure}
\caption{Etapas da mudança de peso \MSFupdate($G$, $a$, $5$).}
\end{figure}

Em seguida, na linha~\ref{Algo:MSFupdate:linha:compara} do Algoritmo~\ref{Algo:MSFupdate}, o novo peso~$w$ é comparado com o peso do vértice~$d$.
Caso $w$ seja menor do que o peso de~$d$, nada precisa ser feito.
Caso contrário, na linha~\ref{Algo:MSFupdate:linha:atualizaPeso} do Algoritmo~\ref{Algo:MSFupdate}, o peso da floresta geradora de pesos mínimos é atualizada.
Em seguida, na linha~\ref{Algo:MSFupdate:linhasplit}, as arestas~$e$ e~$d^\star$ são removidas de~$F$ e de~$F^\star$ utilizando a rotina~\LCOSplit{}, que converte os nós~$\hat e$ e~$\hat d$ nos nós $\hat e_0$,~$\hat e_2$, $\hat d_1$ e~$\hat d_3$, que são as novas folhas associadas às arestas~$e$ e~$d$.
Para adicionar as arestas~$e^\star$ a~$F^\star$ e $d$ a~$F$ são usadas as folhas~$\hat d_0$, $\hat d_2$, $\hat e_1$ e~$\hat e_3$ junto à rotina \LCOMerge{} na linha~\ref{Algo:MSFupdate:linhas:Merge} do Algoritmo~\ref{Algo:MSFupdate}.

Concluindo o exemplo, o peso de $\hat c$ é $3$ que é menor do que o novo peso~$5$ de~$\hat a$.
A Figura~\ref{fig:MSF-figura-6} ilustra esse exemplo logo após a linha~\ref{Algo:MSFupdate:linhasplit} e antes da linha~\ref{Algo:MSFupdate:linhas:Merge} do Algoritmo~\ref{Algo:MSFupdate}.
Note que há duas árvores azuis e duas vermelhas.
A Figura~\ref{fig:MSF-figura-7} ilustra esse exemplo após a conclusão da chamada \MSFupdate($G$, $a$, $5$).

\begin{figure}[htb]
\scalebox{1}{
\centering
\begin{tikzpicture}[line cap=round,line join=round,>=triangle 45,x=1cm,y=1cm]
\clip(-3,-.3) rectangle (10,5.6);

\begin{scriptsize}

%
% Node v
%
\draw (0,0) node {$\hat v$};
\draw [line width=1pt,color=cyan] (0,.5) to  (.35,.35); 
\draw [line width=1pt,color=cyan] (.35,.35) to  (.5,0); 

\draw [line width=1pt,color=cyan] (0,.5) to  (-1,1.8); 
\draw [line width=1pt,color=cyan] (.35,.35) to  (1.05,1.75); 
\draw [line width=1pt,color=cyan] (.5,0) to  (2.5,1); 


\draw (0,.5)[anchor=east] node {\tiny $a_v$};
\draw (.35,.35)[anchor=west] node {\tiny $b_v$};
\draw (.5,0)[anchor=north west] node {\tiny $d_v$};
\draw [fill=black] (0,.5) circle (1.5pt);
\draw [fill=black] (.35,.35) circle (1.5pt);
\draw [fill=black] (.5,0) circle (1.5pt);

\draw (-.6,1.9) node[anchor=north east] {\tiny $a_0$};
\draw [fill=white] (-1,1.8) circle (1.5pt);

% ^b
\draw [line width=1pt,color=cyan] (2.5,1) to (2.9,1.15); 
\draw (1,1.65)[anchor=west] node {\tiny $b_2$};
\draw [fill=white] (1.05,1.75) circle (1.5pt);
\draw (2.5,1)[anchor=north] node {\tiny $d_0$};
\draw [fill=white] (2.5,1) circle (1.5pt);

%
% Node u
%
\draw (0,4) node {$\hat u$};

\draw [line width=1pt,color=cyan] (0,3.5) to  (.3,3.7); 
\draw [line width=1pt,color=cyan] (.5,4) to  (.3,3.7); 

\draw [line width=1pt,color=cyan] (0,3.5) to  (-1,2.2); 
\draw [line width=1pt,color=cyan] (.3,3.7) to  (1.05,2.25); 
\draw [line width=1pt,color=cyan] (.5,4) to  (2.5,3); 

\draw (-1.1,2.35) node {\tiny $a_2$};
\draw [fill=white] (-1,2.2) circle (1.5pt);
\draw (1.2,2.33) node {\tiny $b_0$};
\draw [fill=white] (1.05,2.25) circle (1.5pt);

% u node path
\draw (-.2,3.5) node {\tiny $a_u$};
\draw [fill=black] (0,3.5) circle (1.5pt);
\draw (.5,4.15) node {\tiny $c_u$};
\draw [fill=black] (.5,4) circle (1.5pt);
\draw (.19,3.82) node {\tiny $b_u$};
\draw [fill=black] (.3,3.7) circle (1.5pt);

%
% Node z
%
\draw (3.95,2.1) node {$\hat z$};
\draw [line width=1pt,color=cyan] (4.3,2) to (6.5,2); 
\draw [line width=1pt,color=cyan] (3.7,2.3) to (2.5,3); 
\draw [line width=1pt,color=cyan] (3.7,1.7) to (2.9,1.15); 
\draw (2.4,2.9) node {\tiny $c_2$};
\draw [fill=white] (2.5,3) circle (1.5pt);


% node path
\draw [line width=1pt,color=cyan] (3.7,2.3) to  (3.7,1.7); 
\draw [line width=1pt,color=cyan] (3.7,1.7) to  (4.3,2); 


\draw (3.75,1.55) node {\tiny $d_z$};
\draw [fill=black] (3.7,1.7)circle (1.5pt);
\draw (3.7,2.43) node {\tiny $c_z$};
\draw [fill=black] (3.7,2.3)circle (1.5pt);
\draw (4.3,2.2) node {\tiny $g_z$};
\draw [fill=black] (4.3,2)circle (1.5pt);

\draw (3.2,2.7) node {\tiny $c_0$};
\draw [fill=white] (3,2.7)circle (1.5pt);
\draw (3.05,1.25)[anchor=north] node {\tiny $d_2$};
\draw [fill=white] (2.9,1.15)circle (1.5pt);

%eF0 -- e3
\draw [line width=1pt,color=red] (3.3,4.7) to  (5.25,2.25);


%
% Node F_0
%
\draw (3,5.2) node {$\hat F_0$};
\draw [line width=1pt,color=red] (2.7,4.7) to  (3.8,4.7); 
\draw [line width=1pt,color=red] (3.8,4.7) to  (4,5); 
\draw [line width=1pt,color=red] (4,5) to  (2.5,5); 

\draw [line width=1pt,color=red] (2.5,5) to[in=180,out=180,looseness=2]  (-1.2,2); 
\draw [line width=1pt,color=red] (4,5) to[in=-20,out=10,looseness=3]  (5.25,1.75); 
\draw [line width=1pt,color=red] (3.5,5) to[in=-30,out=20,looseness=4.5]  (2.8,.85); 

%^c
\draw [line width=1pt,color=red] (2.7,4.7) to  (2.9,3.1);

\draw (3.07,3.1) node {\tiny $c_1$};
\draw [fill=white] (2.9,3.1)circle (1.5pt);
\draw (2.9,.7) node {\tiny $d_1$};
\draw [fill=white] (2.8,.85)circle (1.5pt);


%fF3 -- 
\draw [line width=1pt,color=red] (6.5,3.9) to  (6.5,3);

% ^f
\draw [line width=1pt,color=cyan] (6.8,2.5) to[in=-60,out=60,looseness=1.5] (6.65,3.8);
\draw [line width=1pt,color=cyan] (6.2,2.5) to[in=-120,out=120,looseness=1.5] (6.35,3.8);

\draw [line width=1pt,color=red] (3.8,4.7) to  (6.5,3.95);
\draw (6.65,4.05) node {\tiny $f_0$};
\draw [fill=white] (6.5,3.95)circle (1.5pt);
\draw (6.65,3.55) node {\tiny $f_2$};
\draw [fill=white] (6.5,3.65)circle (1.5pt);
\draw (6.8,3.8) node {\tiny $f_3$};
\draw [fill=white] (6.65,3.8)circle (1.5pt);
\draw (6.2,3.8) node {\tiny $f_1$};
\draw [fill=white] (6.35,3.8)circle (1.5pt);


\draw (2.5,5)[anchor=south] node {\tiny $a_{F_0}$};
\draw [fill=black] (2.5,5)circle (1.5pt);
\draw (2.5,4.6) node {\tiny $c_{F_0}$};
\draw [fill=black] (2.7,4.7)circle (1.5pt);
\draw (3.5,5)[anchor=south] node {\tiny $d_{F_0}$};
\draw [fill=black] (3.5,5)circle (1.5pt);
\draw (4,5)[anchor=north west] node {\tiny $g_{F_0}$};
\draw [fill=black] (4,5)circle (1.5pt);
\draw (3.8,4.7)[anchor=north] node {\tiny $f_{F_0}$};
\draw [fill=black] (3.8,4.7)circle (1.5pt);
\draw (3.3,4.7)[anchor=north] node {\tiny $g_{F_0}$};
\draw [fill=black] (3.3,4.7)circle (1.5pt);

%
% Node y
%
\draw (6.5,2.35) node {$\hat y$};

%node path
\draw [line width=1pt,color=cyan] (6.5,2) to  (6.2,2.5); 
\draw [line width=1pt,color=cyan] (6.5,2) to  (6.8,2.5); 


\draw (6.5,1.8) node {\tiny $g_y$};
\draw (6,2.5) node {\tiny $f_y$};
\draw (7,2.5) node {\tiny $f_y$};
\draw [fill=black] (6.5,2)circle (1.5pt);
\draw [fill=black] (6.2,2.5)circle (1.5pt);
\draw [fill=black] (6.8,2.5)circle (1.5pt);


%
% Node F_1
%
\draw (2.1,2) node {$\hat F_1$};
%node path
\draw [line width=1pt,color=red] (2.3,2.3) to  (2.3,1.7); 
\draw [line width=1pt,color=red] (1.7,2) to  (2.3,1.7); 

\draw [line width=1pt,color=red] (2.6,2.6) to  (2.3,2.3); 
\draw [line width=1pt,color=red] (2.6,1.3) to  (2.3,1.7); 
\draw [line width=1pt,color=red] (1.7,2) to  (1.3,2); 

% F_1 to F_2
\draw [line width=1pt,color=red] (.8,2) to  (1.3,2); 

\draw (1.8,2.13) node {\tiny $b_{F_1}$};
\draw [fill=black] (1.7,2)circle (1.5pt);
\draw (2.3,1.7)[anchor=north] node {\tiny $d_{F_1}$};
\draw [fill=black] (2.3,1.7)circle (1.5pt);
\draw (2.25,2.2)[anchor=west] node {\tiny $c_{F_1}$};
\draw [fill=black] (2.3,2.3)circle (1.5pt);
\draw (1.4,1.85) node {\tiny $b_3$};
\draw [fill=white] (1.3,2)circle (1.5pt);

\draw (2.8,1.4) node {\tiny $d_3$};
\draw [fill=white] (2.6,1.3)circle (1.5pt);
\draw (2.75,2.55) node {\tiny $c_3$};
\draw [fill=white] (2.6,2.6)circle (1.5pt);

%
% Node F_2
%
\draw (0,2.2) node {$\hat F_2$};

\draw [line width=1pt,color=red] (-1.2,2) to  (-.3,2); 
\draw [line width=1pt,color=red] (.8,2) to  (.3,2);   
\draw [line width=1pt,color=red] (-.3,2) to (.3,2);   

\draw [fill=black] (-.2,2) circle (1.5pt);
\draw (-.2,1.8) node {\tiny $a_{F_2}$};
\draw [fill=black] (.2,2) circle (1.5pt);
\draw (.3,1.8) node {\tiny $b_{F_2}$};
\draw (-.75,2.1) node {\tiny $a_1$};
\draw [fill=white] (-.8,2) circle (1.5pt);
\draw (.8,2.16) node {\tiny $b_1$};
\draw [fill=white] (.8,2) circle (1.5pt);

\draw (-1.3,1.9) node {\tiny $a_3$};
\draw [fill=white] (-1.2,2)circle (1.5pt);

%
% Node F_3
%
\draw (6.7,3.2) node {$\hat F_3$};
\draw (6.3,3) node {\tiny $f_{F_3}$};
\draw [fill=black] (6.5,3)circle (1.5pt);

\draw (5,2)[anchor=south] node {\tiny $g_0$};
\draw [fill=white] (5,2)circle (1.5pt);
\draw (5.5,2)[anchor=south] node {\tiny $g_2$};
\draw [fill=white] (5.5,2)circle (1.5pt);
\draw (5.3,2.25)[anchor=south] node {\tiny $g_3$};
\draw [fill=white] (5.25,2.25)circle (1.5pt);
\draw (5.25,1.75)[anchor=north] node {\tiny $g_1$};
\draw [fill=white] (5.25,1.75)circle (1.5pt);

\end{scriptsize}
\end{tikzpicture}

}
\caption{ADPs após a execução de~\MSFupdate($G$, $a$, $5$).}
\label{fig:MSF-figura-7}
\end{figure}

O segundo caso, quando~$e$ não for uma aresta de~$F$, então~$e^\star$ é uma aresta de~$F^\star$.
O tratamento desse caso é simétrico ao primeiro caso e é feito entre as linhas~\ref{Algo:MSFupdate:dualinicio} e~\ref{Algo:MSFupdate:dualfim} do Algoritmo~\ref{Algo:MSFupdate}.

O custo de tempo assintótico de $\MSFupdate$ é $\O{\lg m}$ esperado amortizado, onde $m$ é o número de arestas do grafo $G$ corrente.
As linhas~\ref{Algo:MSFupdate:linha:const1}, \ref{Algo:MSFupdate:linha:const2}, \ref{Algo:MSFupdate:linha:atualizaPeso}, \ref{Algo:MSFupdate:linha:const3} e~\ref{Algo:MSFupdate:linha:const4} do Algoritmo~\ref{Algo:MSFupdate} possuem consumo de tempo constante amortizado, pois são consultas à tabela de símbolos ou operações aritméticas básicas.
As linhas remanescentes são formadas por um número constante de chamadas de \LCOAddCost, \LCOFindNode, \LCOMerge{} e~\LCOSplit{} e cada uma dessas chamadas possui consumo de tempo assintótico~$\O{\lg m}$ esperado amortizado, concluindo que $\MSFupdate$ possui consumo de tempo assintótico~$\O{\lg m}$ esperado amortizado.

\subsection{Remoção de aresta}

Ao remover uma aresta~$e$ de um grafo plano ponderado dinâmico~$G$, além de remover $e$ de $\hat F$ ou $e^\star$ de~$\hat F^\star$ é necessário também atualizar a estrutura de~$\hat F^\star$, pois a remoção de $e$ modifica a estrutura de faces do grafo plano.
Pelo Lema~\ref{lemma:diestel}, existem dois tipos de arestas em~$G$: ou a aresta não está em ciclos em~$G$ e as duas faces incidentes a ela são iguais, ou essas faces são distintas e a aresta está em pelo menos um ciclo de~$G$.

No primeiro caso, a aresta~$e$ é uma ponte e sua remoção aumenta o número de componentes conexas de~$G$.
Como consideramos que cada uma dessas componentes possui uma face exterior própria, a remoção de~$e$ implica na divisão da face exterior em duas, que serão as faces exteriores das duas componentes resultantes.
Como exemplo, ao remover a aresta~$e$ do grafo~$G$ ilustrado na Figura~\ref{fig:MSF-basico-reprodicao} a face exterior~$F_0$ é separada nas faces $F_4$ e~$F_5$ da Figura~\ref{fig:MSF-remove-ponte}.


Já no segundo caso, a remoção de uma aresta~$e$ implica na junção das duas faces incidentes a essa aresta.
No exemplo, a remoção da aresta~$b$ do grafo ilustrado na Figura~\ref{fig:MSF-basico-reprodicao} resulta na junção das faces~$F_1$ e~$F_2$ resultando na face~$F_6$ ilustrada na Figura~\ref{fig:MSF-remove-nao-ponte}.

\begin{figure}[htb]
\begin{subfigure}{0.3\textwidth}
\scalebox{0.7}{
\begin{tikzpicture}[line cap=round,line join=round,>=triangle 45,x=1cm,y=1cm]
\clip(-3.5,-1) rectangle (6,3);


\draw [line width=1pt] (0,0) to[in=-135,out=135,looseness=1]  (0,2); % v -- u
\draw [line width=1pt] (0,0) to[in=-45,out=45,looseness=1]  (0,2);   % v -- u
\draw [line width=1pt] (2,1) to  (0,0); % 4 -- 6
\draw [line width=1pt] (2,1) to  (0,2); % 4 -- 6
\draw [line width=1pt] (2,1.05) to[out=45,in=-45,looseness=50] (2,.95); % 4 -- 4


\draw [line width=1pt] (1,2.5) to[out=-45,in=45,looseness=1] (1,1); % F0 -- F1
\draw [line width=1pt] (1,2.5) to[out=20,in=-70,looseness=10] (1,1); % F0 -- F1
\draw [line width=1pt] (1,2.5) to[out=170,in=180,looseness=2.5] (0,1); % F0 -- F2
\draw [line width=1pt] (1,2.5) to[out=0,in=90,looseness=1] (2.5,1); % F0 -- F3
\draw [line width=1pt] (1,1) to (0,1); % F1 -- F2




\begin{scriptsize}
\draw [fill=black] (0,2) circle (1.5pt);
\draw (0,2) node[anchor=south east] {$u$};


\draw [fill=black] (0,0) circle (1.5pt);
\draw (0,0) node[anchor=north east] {$v$};
\draw [fill=black] (2,1) circle (1.5pt);
\draw (1.95,1) node[anchor=south] {$w$};


\draw [fill=black] (1,2.5) circle (1.5pt);
\draw (1,2.5) node[anchor=south east] {$F_0$};
\draw [fill=black] (1,1) circle (1.5pt);
\draw (1,1) node[anchor=west] {$F_1$};
\draw [fill=black] (0,1) circle (1.5pt);
\draw (0,1) node[anchor=south] {$F_2$};
\draw [fill=black] (2.5,1) circle (1.5pt);
\draw (2.5,1) node[anchor=west] {$F_3$};

\draw (-0.3,1.05) node[anchor=north] {$a$};
\draw (-0.5,1.1) node[anchor=north] {$2$};

\draw (0.5,1.35) node[anchor=north] {$b$};
\draw (0.5,1.05) node[anchor=north] {$7$};

\draw (1.15,1.7) node[anchor=north] {$c$}; % u -- w
\draw (1.4,1.7) node[anchor=north] {$3$}; % u -- w

\draw (2.52,1.75) node[anchor=north] {$d$}; % w -- w
\draw (2.3,1.65) node[anchor=north] {$2$}; % w -- w

\draw (1.2,.9) node[anchor=north] {$e$}; % v -- w
\draw (1.35,.65) node[anchor=north] {$1$}; % v -- w
\end{scriptsize}
\end{tikzpicture}

}
\caption{Grafo da Figura~\ref{fig:MSF-basico-1}.}
\label{fig:MSF-basico-reprodicao}
\end{subfigure}
\begin{subfigure}{0.3\textwidth}
\scalebox{0.7}{
\begin{tikzpicture}[line cap=round,line join=round,>=triangle 45,x=1cm,y=1cm]
\clip(-1,-1) rectangle (6,3);


\draw [line width=1pt] (0,0) to[in=-135,out=135,looseness=1]  (0,2); % v -- u
\draw [line width=1pt] (0,0) to[in=-45,out=45,looseness=1]  (0,2);   % v -- u
\draw [line width=1pt] (2,1) to  (0,0); % 4 -- 6
\draw [line width=1pt] (2,1) to  (0,2); % 4 -- 6
%\draw [line width=1pt] (2,1) to  (3,1); % 4 -- 6
\draw [line width=1pt] (3,1.05) to[out=45,in=-45,looseness=50] (3,.95); % 4 -- 4


\draw [line width=1pt] (1,2.5) to[out=-45,in=45,looseness=1] (1,1); % c: F0 -- F1
\draw [line width=1pt] (1,2.5) to[out=0,in=90,looseness=1] (2.5,1); % d: F0 -- F1
\draw [line width=1pt] (1,1) to[out=-55,in=-90,looseness=2] (2.5,1); % d: F0 -- F1


% laço F0 para F0
%\draw [line width=1pt] (1,2.5) to[out=-30,in=90,looseness=1] (2.5,1); % F0 -- F1
%\draw [line width=1pt] (2.5,1) to[out=-90,in=-90,looseness=2] (4.5,1); % F0 -- F1
%\draw [line width=1pt] (1,2.5) to[out=0,in=90,looseness=1] (4.5,1); % F0 -- F1


\draw [line width=1pt] (1,2.5) to[out=170,in=180,looseness=2.5] (0,1); % F0 -- F2
\draw [line width=1pt] (3.5,2) to (3.5,1); % F0 -- F3
\draw [line width=1pt] (1,1) to (0,1); % F1 -- F2




\begin{scriptsize}
\draw [fill=black] (0,2) circle (1.5pt);
\draw (0,2) node[anchor=south east] {$u$};


\draw [fill=black] (0,0) circle (1.5pt);
\draw (0,0) node[anchor=north east] {$v$};
\draw [fill=black] (2,1) circle (1.5pt);
\draw (1.95,1) node[anchor=south] {$z$};
\draw [fill=black] (3,1) circle (1.5pt);
\draw (2.95,1) node[anchor=south] {$y$};


\draw [fill=black] (1,2.5) circle (1.5pt);
\draw (1,2.5) node[anchor=south east] {$F_4$};
\draw [fill=black] (1,1) circle (1.5pt);
\draw (1,1) node[anchor=west] {$F_1$};
\draw [fill=black] (0,1) circle (1.5pt);
\draw (0,1) node[anchor=south] {$F_2$};
\draw [fill=black] (3.5,1) circle (1.5pt);
\draw (3.5,1) node[anchor=west] {$F_3$};
\draw [fill=black] (3.5,2) circle (1.5pt);
\draw (3.5,2) node[anchor=west] {$F_5$};

\draw (-0.5,1.05) node[anchor=north] {$a$};
%\draw (-0.5,1.1) node[anchor=north] {$2$};

\draw (0.5,1.35) node[anchor=north] {$b$};
%\draw (0.5,1.05) node[anchor=north] {$7$};

\draw (1.15,1.7) node[anchor=north] {$c$}; % u -- w
%\draw (1.4,1.7) node[anchor=north] {$3$}; % u -- w

\draw (1.1,.6) node[anchor=north] {$d$}; % v -- w
%\draw (1.2,.65) node[anchor=north] {$1$}; % v -- w

%\draw (2.3,1.3) node[anchor=north] {$e$}; % v -- w
%\draw (2.3,.95) node[anchor=north] {$4$}; % v -- w

\draw (4.15,1) node {$f$}; % w -- w
%\draw (4.15,1) node {$2$}; % w -- w

\end{scriptsize}
\end{tikzpicture}

}
\caption{Após a remoção de~$g$.}
\label{fig:MSF-remove-ponte}
\end{subfigure}
\begin{subfigure}{0.3\textwidth}
\scalebox{0.7}{
\begin{tikzpicture}[line cap=round,line join=round,>=triangle 45,x=1cm,y=1cm]
\clip(-1,-1) rectangle (6,3);


\draw [line width=1pt] (0,0) to[in=-135,out=135,looseness=1]  (0,2); % v -- u
%\draw [line width=1pt] (0,0) to[in=-45,out=45,looseness=1]  (0,2);   % v -- u
\draw [line width=1pt] (2,1) to  (0,0); % 4 -- 6
\draw [line width=1pt] (2,1) to  (0,2); % 4 -- 6
\draw [line width=1pt] (2,1) to  (3,1); % 4 -- 6
\draw [line width=1pt] (3,1.05) to[out=45,in=-45,looseness=50] (3,.95); % 4 -- 4


\draw [line width=1pt] (1,2.5) to[out=-45,in=45,looseness=1] (1,1); % c: F0 -- F1
\draw [line width=1pt] (1,2.5) to[out=20,in=160,looseness=1] (4.2,2.5); % d: F0 -- F1
\draw [line width=1pt] (4.2,2.5) to[out=-20,in=20,looseness=1] (4.2,-.5); % d: F0 -- F1
\draw [line width=1pt] (1,1) to[out=-70,in=200,looseness=1] (4.2,-.5); % d: F0 -- F1


% laço F0 para F0
\draw [line width=1pt] (1,2.5) to[out=-30,in=90,looseness=1] (2.5,1); % F0 -- F1
\draw [line width=1pt] (2.5,1) to[out=-90,in=-90,looseness=2] (4.5,1); % F0 -- F1
\draw [line width=1pt] (1,2.5) to[out=0,in=90,looseness=1] (4.5,1); % F0 -- F1


\draw [line width=1pt] (1,2.5) to[out=170,in=180,looseness=2.5] (0,1); % F0 -- F2
\draw [line width=1pt] (1,2.5) to[out=0,in=90,looseness=1] (3.5,1); % F0 -- F3
\draw [line width=1pt] (1,1) to (0,1); % F1 -- F2




\begin{scriptsize}
\draw [fill=black] (0,2) circle (1.5pt);
\draw (0,2) node[anchor=south east] {$u$};


\draw [fill=black] (0,0) circle (1.5pt);
\draw (0,0) node[anchor=north east] {$v$};
\draw [fill=black] (2,1) circle (1.5pt);
\draw (1.95,1) node[anchor=south] {$z$};
\draw [fill=black] (3,1) circle (1.5pt);
\draw (2.95,1) node[anchor=south] {$y$};


\draw [fill=black] (1,2.5) circle (1.5pt);
\draw (1,2.5) node[anchor=south east] {$F_0$};
\draw [fill=black] (1,1) circle (1.5pt);
\draw (1,1) node[anchor=west] {$F_6$};
%\draw [fill=black] (0,1) circle (1.5pt);
%\draw (0,1) node[anchor=south] {$F_2$};
\draw [fill=black] (3.5,1) circle (1.5pt);
\draw (3.5,1) node[anchor=west] {$F_3$};

\draw (-0.5,1.05) node[anchor=north] {$a$};
%\draw (-0.5,1.1) node[anchor=north] {$2$};

%\draw (0.5,1.35) node[anchor=north] {$b$};
%\draw (0.5,1.05) node[anchor=north] {$7$};

\draw (1.15,1.7) node[anchor=north] {$c$}; % u -- w
%\draw (1.4,1.7) node[anchor=north] {$3$}; % u -- w

\draw (1.1,.6) node[anchor=north] {$d$}; % v -- w
%\draw (1.2,.65) node[anchor=north] {$1$}; % v -- w

\draw (2.3,1.35) node[anchor=north] {$g$}; % v -- w
%\draw (2.3,.95) node[anchor=north] {$4$}; % v -- w

\draw (4.15,1) node {$f$}; % w -- w
%\draw (4.15,1) node {$2$}; % w -- w

\end{scriptsize}
\end{tikzpicture}

}
\caption{Após remoção de~$b$.}
\label{fig:MSF-remove-nao-ponte}
\end{subfigure}
\caption{Exemplos de remoção de arestas.}
\label{fig:antes-depois-remocao}
\end{figure}


A implementação de \MSFdelEdge{} pode ser vista no Algoritmo~\ref{Algo:MSFdelEdge}.
Junto com a explicação do pseudocódigo, detalharemos o processamento das chamadas \MSFdelEdge($G$, $g$) e \MSFdelEdge($G$, $b$) para exemplificar os dois casos de remoção de aresta.

Antes de identificar em qual caso a aresta~$e$ está, sua óctupla é obtida da tabela de símbolos e o identificado de cada nó dessa óctupla é obtido usando a rotina \LCOFindNode{} no laço da linha~\ref{Algo:MSFdelEdge:linha:laco} do Algoritmo~\ref{Algo:MSFdelEdge}.

A identificação de qual caso uma aresta $e$ se encontra é feita na linha~\ref{Algo:MSFdelEdge:linha2} do Algoritmos~\ref{Algo:MSFdelEdge}.
Nessa linha são comparados os valores de \LCOFindNode($v_1$) e \LCOFindNode($v_3$).
Esses valores são os identificadores das faces adjacentes a~$e$.
Caso esses identificadores sejam iguais, então as faces incidentes a~$e$ são iguais e logo~$e$ não está em ciclos de~$G$.
Caso contrário, então~$e$ está em pelo menos um ciclo.
A aresta~$g$ se encontra no primeiro caso, pois ambas as chamadas de~\LCOFindNode($g_{F_0}$) retornam o identificador $\hat F_0$.
Enquanto que~$b$ se encontra no segundo caso, já que~\LCOFindNode($b_u$) retorna $\hat u$, enquanto que~\LCOFindNode($b_v$) retorna~$\hat v$.

O primeiro caso é tratado entre as linhas~\ref{Algo:MSFdelEdge:linha3} e~\ref{Algo:MSFdelEdge:linha12} do Algoritmo~\ref{Algo:MSFdelEdge}.
Remover uma aresta~$e$ das ADPs consiste em desconectar a óctupla de~$e$ do restante da ADP. 
Para desconectar um nó~$v_i$ de sua ordem cíclica são chamadas as rotinas \LCOCycle($v_i$) e \LCOSplit($v_i$, $\hat v_1$, \Nil) em sucessão.
A rotina \LCOCycle($v_i$) garante que a ordem cíclica que contém~$v_i$ tenha forma $\alpha v_i$, enquanto que a rotina \LCOSplit($v_i$, $\hat v_i$, \Nil) separa essa ordem em duas: a primeira contendo todos os nós anteriores a~$v_i$ e a segunda com~$v_i$ e todos os nós sucessores, mas como a ordem tem forma $\alpha v_i$, então as duas ordens resultantes são~$\alpha$ e~$v_i$.
Ou seja, o nó~$v_i$ foi desconectado do restante da ordem cíclica.
Essa rotina também atribui o identificador $\hat v_i$ à primeira ordem, isto é, o antigo identificador do vértice que continha~$v_i$ em sua ordem cíclica.
O terceiro parâmetro da rotina \LCOSplit{} é o identificador da segunda ordem cíclica obtida por essa rotina, nesse caso a ordem é composta somente pelo nó~$v_i$.
Como essa ordem será apagada da memória, esse novo identificador atribuído por \LCOSplit{} não é relevante para a operação da rotina \MSFdelEdge{} e para simbolizar isso no pseudocódigo, optamos por passar $\Nil$ como parâmetro nesse caso.
Esse processo é feito para $v_0$, $v_1$ e~$v_2$ nas linhas~\ref{Algo:MSFdelEdge:linha3} a~\ref{Algo:MSFdelEdge:c1:v2}.
Relembramos que a óctupla de~$g$ e~$b$ estão ilustradas na Figura~\ref{fig:MSF-octo}.

\begin{algorithm}[hbt]
\caption{\MSFdelEdge($G$, $e$)}
\label{Algo:MSFdelEdge}
\begin{algorithmic}[1]
\State $e_0$, $e_1$, $e_2$, $e_3$, $v_0$, $v_1$, $v_2$, $v_3$ $\gets$ $G.H(e)$ \label{Algo:MSFdelEdge:linha:1}
\For {$i\in \{0, 1, 2, 3\}$}\label{Algo:MSFdelEdge:linha:laco}
\State $\hat e_i$ $\gets$ \LCOFindNode($e_i$)
\State $\hat v_i$ $\gets$ \LCOFindNode($v_i$)
\EndFor

\If {$\hat v_1$ $=$ $\hat v_3$ }\label{Algo:MSFdelEdge:linha2}\Comment{$e^\star$ é um laço de $G^\star$}

\State \LCOCycle($v_0$); \LCOSplit($v_0$, $\hat v_0$, \Nil)\label{Algo:MSFdelEdge:linha3}
\State \LCOCycle($v_1$); \LCOSplit($v_1$, $\hat v_1$, \Nil)\label{Algo:MSFdelEdge:c1:v1}
\State \LCOCycle($v_2$); \LCOSplit($v_2$, $\hat v_2$, \Nil)\label{Algo:MSFdelEdge:c1:v2}
\State \LCOSplit($v_3$, $F_0$, $F_1$) \label{Algo:MSFdelEdge:c1:split}
\State \LCOCycle($v_3$); \LCOSplit($v_3$, $\hat v_3$, \Nil)\label{Algo:MSFdelEdge:linha12}
\Else\Comment{$e$ está em um ciclo em $G$}
\State \MSFupdate($G$, $e$, $\infty$)\label{Algo:MSFdelEdge:linha:mudaPeso}
\State \LCOCycle($e_1$); \LCOSplit($e_1$, $\hat e_3$, $\hat e_1$) \label{Algo:MSFdelEdge:linha:split}
\State \LCOCycle($v_1$); \LCOCycle($v_3$); \LCOMerge($v_1$, $v_3$, $F_0$) \label{Algo:MSFdelEdge:linha:merge}
\State \LCOCycle($v_0$); \LCOSplit($v_0$, $\hat v_0$, \Nil) \label{Algo:MSFdelEdge:linha:1cs}
\State \LCOCycle($v_1$); \LCOSplit($v_1$, $F_0$, \Nil )
\State \LCOCycle($v_2$); \LCOSplit($v_2$, $\hat v_2$, \Nil)
\State \LCOCycle($v_3$); \LCOSplit($v_3$, $F_0$, \Nil)
	\label{Algo:MSFdelEdge:linha:ucs}
	\label{Algo:MSFdelEdge:linha:fim}
\EndIf
\State \LCODestroyOcto($G.H$, $e$)\label{Algo:MSFdelEdge:linha:desaloca}
\end{algorithmic}
\end{algorithm}

\newpage
Antes de desconectar $v_3$ de sua ordem cíclica, a face exterior~$F_e$ que contém~$v_1$ e~$v_3$ em sua ordem cíclica é dividida nas duas faces exteriores das duas componentes conexas~$G_1$ e~$G_2$ obtidas pela remoção de~$e$ de~$G$.
As duas pontas do laço~$e^\star$, representadas por~$v_1$ e~$v_3$ nas ADPs, dividem a ordem cíclica de~$F_e$ em duas partes.
A primeira contendo todos os nós entre~$v_1$ e~$v_3$ e a segunda possuindo todos os nós de~$v_3$ a~$v_1$ seguindo a ordem. 
Isto é, se a ordem de~$F_e$ tem forma $\alpha v_1\beta v_3\gamma$, então essas duas partes têm forma~$\gamma\alpha$ e~$\beta$.
Essas duas partes são as ordens cíclicas das futuras faces exteriores de~$G_1$ e~$G_2$, como ilustra a Figura~\ref{fig:MSF-ordem-ciclica-ponte}.

Após a desconexão de $v_1$ com as chamadas \LCOCycle($v_1$) e \LCOSplit($v_1$, $\hat v_1$, \Nil), feitas na linha~\ref{Algo:MSFdelEdge:c1:v1}, a ordem cíclica da face~$\hat F_e$ resultante dessas operações tem forma $\beta v_3\gamma\alpha$.
Então, para separar $\beta$ de~$\gamma\alpha$, é executado um $\LCOSplit$ em $v_3$ na linha~\ref{Algo:MSFdelEdge:c1:split}, obtendo dois vértices $F_0$ e~$F_1$ com ordens cíclicas~$\beta$ e~$v_3\gamma\alpha$.
É importante que $F_0$ e~$F_1$ sejam identificadores únicos para as novas faces criadas, pois esses identificadores são usados na linha~\ref{Algo:MSFdelEdge:linha2} do Algoritmos~\ref{Algo:MSFdelEdge}.
Em seguida, $v_3$ é desconectado da segunda ordem cíclica na linha~\ref{Algo:MSFdelEdge:linha12}.
Por fim, na linha~\ref{Algo:MSFdelEdge:linha:desaloca} a óctupla de~$e$ é desalocada da memória.

Na Figura~\ref{fig:MSF-ordem-ciclica-ponte-e}, podemos ver as ADPs logo antes da linha~\ref{Algo:MSFdelEdge:c1:split} do Algoritmo~\ref{Algo:MSFdelEdge}.
Após desconectar o nó~$g_{F_0}$ que está ligado a $g_1$, a ordem cíclica da face exterior~$\hat F_0$ é $\langle d_{F_0}, a_{F_0}, c_{F_0}, g_{F_0}, f_{F_0}\rangle$.
Logo, nesse caso, temos que $\gamma\alpha$ e $\beta$ são respectivamente $\langle f_{F_0} \rangle$ e~$\langle d_{F_0}, a_{F_0}, c_{F_0}\rangle$.
Essas são as duas ordens cíclicas das faces exteriores~$F_5$ e $F_4$ ilustradas na Figura~\ref{fig:MSF-remove-ponte}.

\begin{figure}[htb]
\begin{subfigure}{.4\textwidth}
\begin{tikzpicture}[line cap=round,line join=round,>=triangle 45,x=1cm,y=1cm]
\clip(-1,-2) rectangle (6,3);

\draw (0,0) circle [radius=1cm];
\draw (4,0) circle [radius=1cm];
\draw (0,0) node[anchor=north] {$G_1$};
\draw (4,0) node[anchor=north] {$G_2$};


\draw [line width=1pt] (2,2.5) to[out=170,in=90,looseness=1] (-.5,.5); % F0 -- F2
\draw [line width=1pt] (2,2.5) to[out=175,in=90,looseness=1] (-.3,.3); % F0 -- F2
\draw [line width=1pt] (2,2.5) to[out=185,in=90,looseness=1] (0,0); % F0 -- F2
\draw [line width=1pt] (2,2.5) to[out=195,in=90,looseness=1] (.2,-.3); % F0 -- F2



%e*
\draw [line width=1pt] (2,2.5) to (2,0); % F0 -- F1
\draw [line width=1pt] (2,0) to[out=-90,in=-90,looseness=1.5] (5.5,0); % F0 -- F1
\draw [line width=1pt] (2,2.5) to[out=20,in=90,looseness=1] (5.5,0); % F0 -- F1


\draw [line width=1pt] (2,2.5) to[out=0,in=90,looseness=1] (4.3,.7); % F0 -- F3
\draw [line width=1pt] (2,2.5) to[out=0,in=90,looseness=1] (4.2,.5); % F0 -- F3
\draw [line width=1pt] (2,2.5) to[out=0,in=90,looseness=1] (3.9,.4); % F0 -- F3

\draw [line width=1pt] (2,2.5) to[out=-40,in=180,looseness=1] (3.5,.3); % F0 -- F3
\draw [line width=1pt] (2,2.5) to[out=-50,in=180,looseness=1] (3.4,.1); % F0 -- F3
\draw [line width=1pt] (2,2.5) to[out=-60,in=180,looseness=1] (3.3,.0); % F0 -- F3

% e
\draw [line width=1pt] (1,0) to (3,0); 
\draw (2,0) node[anchor=south east] {$e$};


\draw (2,1.5) node[anchor=south east] {$v_3$};
\draw (3,2.8) node[anchor=west] {$v_1$};


\draw (2,2.5) node[anchor=south] {$F_e$};
\draw [fill=black] (2,2.5) circle (1.5pt);


% Arcos
%\draw (2,2.5) circle [radius=1cm];
\draw[red,line width=1pt] (2.15,1.5) arc (-80:-50:1);
\draw[line width=.1pt,->,red] (2.7,1.8) to (2.75,1.85);
\draw (2.75,1.5) node[red] {$\gamma$};

\draw[red,line width=1pt] (2.15,1.5) arc (-80:-50:1);
\draw[line width=.1pt,->,red] (2.7,1.8) to (2.75,1.85);
\draw (2.75,1.5) node[red] {$\gamma$};

\draw[red,line width=1pt] (2.85,2) arc (-35:-5:1);
\draw[line width=.1pt,->,red] (3.025,2.5) to (3.025,2.6);
\draw (3.3,2.4) node[red] {$\alpha$};

\draw[red,line width=1pt] (1,2.5) arc (180:220:1);
\draw[line width=.1pt,->,red] (1.2,1.9) to (1.3,1.8);
\draw (1.2,1.55) node[red] {$\beta$};

\end{tikzpicture}

\caption{Ordens cíclicas da face exterior~$F_e$.}
\label{fig:MSF-ordem-ciclica-ponte}
\end{subfigure}
\hfill
\begin{subfigure}{.4\textwidth}
\begin{tikzpicture}[line cap=round,line join=round,>=triangle 45,x=1cm,y=1cm]
\clip(2,-.3) rectangle (10,5.6);

\begin{scriptsize}

%
% Node v
%
\draw (0,0) node {$\hat v$};
\draw [line width=1pt,color=cyan] (0,.5) to  (.35,.35); 
\draw [line width=1pt,color=cyan] (.35,.35) to  (.5,0); 

\draw [line width=1pt,color=cyan] (0,.5) to  (-1,1.8); 
\draw [line width=1pt,color=cyan] (.35,.35) to  (1.05,1.75); 
\draw [line width=1pt,color=cyan] (.5,0) to  (2.5,1); 


\draw (0,.5)[anchor=east] node {\tiny $a_v$};
\draw (.35,.35)[anchor=west] node {\tiny $b_v$};
%\draw (.5,0)[anchor=north west] node {\tiny $d_v$};
\draw [fill=black] (0,.5) circle (1.5pt);
\draw [fill=black] (.35,.35) circle (1.5pt);
\draw [fill=black] (.5,0) circle (1.5pt);

\draw (-.6,1.9) node[anchor=north east] {\tiny $a_0$};
\draw [fill=white] (-1,1.8) circle (1.5pt);

% ^b
\draw [line width=1pt,color=cyan] (2.5,1) to (2.9,1.15); 
\draw (1,1.65)[anchor=west] node {\tiny $b_2$};
\draw [fill=white] (1.05,1.75) circle (1.5pt);
%\draw (2.5,1)[anchor=north] node {\tiny $d_0$};
\draw [fill=white] (2.5,1) circle (1.5pt);

%
% Node u
%
\draw (0,4) node {$\hat u$};

\draw [line width=1pt,color=cyan] (0,3.5) to  (.3,3.7); 
\draw [line width=1pt,color=cyan] (.5,4) to  (.3,3.7); 

\draw [line width=1pt,color=cyan] (0,3.5) to  (-1,2.2); 
\draw [line width=1pt,color=cyan] (.3,3.7) to  (1.05,2.25); 
\draw [line width=1pt,color=cyan] (.5,4) to  (2.5,3); 

\draw (-1.1,2.35) node {\tiny $a_2$};
\draw [fill=white] (-1,2.2) circle (1.5pt);
\draw (1.2,2.33) node {\tiny $b_0$};
\draw [fill=white] (1.05,2.25) circle (1.5pt);

% u node path
\draw (-.2,3.5) node {\tiny $a_u$};
\draw [fill=black] (0,3.5) circle (1.5pt);
%\draw (.5,4.15) node {\tiny $c_u$};
\draw [fill=black] (.5,4) circle (1.5pt);
\draw (.19,3.82) node {\tiny $b_u$};
\draw [fill=black] (.3,3.7) circle (1.5pt);

%
% Node z
%
\draw (3.8,2) node {$\hat z$};
\draw [line width=1pt,color=cyan] (4.3,2) to (6.5,2); 
\draw [line width=1pt,color=cyan] (3.7,2.3) to (2.5,3); 
\draw [line width=1pt,color=cyan] (3.7,1.7) to (2.9,1.15); 
%\draw (2.4,2.9) node {\tiny $c_2$};
\draw [fill=white] (2.5,3) circle (1.5pt);


% node path
\draw [line width=1pt,color=cyan] (3.7,2.3) to  (3.7,1.7); 
%\draw [line width=1pt,color=cyan] (3.7,1.7) to  (4.3,2); 


%\draw (3.75,1.55) node {\tiny $d_z$};
\draw [fill=black] (3.7,1.7)circle (1.5pt);
%\draw (3.7,2.43) node {\tiny $c_z$};
\draw [fill=black] (3.7,2.3)circle (1.5pt);
\draw (4.3,2.2) node {\tiny $g_z$};
\draw [fill=black] (4.3,2)circle (1.5pt);

%\draw (3.2,2.7) node {\tiny $c_0$};
\draw [fill=white] (3,2.7)circle (1.5pt);
%\draw (3.05,1.25)[anchor=north] node {\tiny $d_2$};
\draw [fill=white] (2.9,1.15)circle (1.5pt);

%eF0 -- e3
\draw [line width=1pt,color=red] (3.3,4.7) to  (5.25,2.25);


%
% Node F_0
%
\draw (3,5.2) node {$\hat F_0$};
\draw [line width=1pt,color=red] (2.7,4.7) to  (3.8,4.7); 
\draw [line width=1pt,color=red] (2.5,5) to  (2.7,4.7); 
\draw [line width=1pt,color=red] (3.5,5) to  (2.5,5); 

\draw [line width=1pt,color=red] (2.5,5) to[in=180,out=180,looseness=2]  (-1.2,2); 
\draw [line width=1pt,color=red] (4,5) to[in=-20,out=10,looseness=3]  (5.25,1.75); 
\draw [line width=1pt,color=red] (3.5,5) to[in=-30,out=20,looseness=4.5]  (2.8,.85); 

%^c
\draw [line width=1pt,color=red] (2.7,4.7) to  (2.9,3.1);

%\draw (3.07,3.1) node {\tiny $c_1$};
\draw [fill=white] (2.9,3.1)circle (1.5pt);
%\draw (2.9,.7) node {\tiny $d_1$};
\draw [fill=white] (2.8,.85)circle (1.5pt);


%fF3 -- 
\draw [line width=1pt,color=red] (6.5,3.9) to  (6.5,3);

% ^f
\draw [line width=1pt,color=cyan] (6.8,2.5) to[in=-60,out=60,looseness=1.5] (6.65,3.8);
\draw [line width=1pt,color=cyan] (6.2,2.5) to[in=-120,out=120,looseness=1.5] (6.35,3.8);

\draw [line width=1pt,color=red] (3.8,4.7) to  (6.5,3.95);
%\draw (6.65,4.05) node {\tiny $f_0$};
\draw [fill=white] (6.5,3.95)circle (1.5pt);
%\draw (6.65,3.55) node {\tiny $f_2$};
\draw [fill=white] (6.5,3.65)circle (1.5pt);
%\draw (6.8,3.8) node {\tiny $f_3$};
\draw [fill=white] (6.65,3.8)circle (1.5pt);
%\draw (6.2,3.8) node {\tiny $f_1$};
\draw [fill=white] (6.35,3.8)circle (1.5pt);


\draw (2.5,5)[anchor=south] node {\tiny $a_{F_0}$};
\draw [fill=black] (2.5,5)circle (1.5pt);
\draw (2.5,4.6) node {\tiny $c_{F_0}$};
\draw [fill=black] (2.7,4.7)circle (1.5pt);
\draw (3.5,5)[anchor=south] node {\tiny $d_{F_0}$};
\draw [fill=black] (3.5,5)circle (1.5pt);
\draw (4,5)[anchor=north west] node {\tiny $g_{F_0}$};
\draw [fill=black] (4,5)circle (1.5pt);
\draw (3.8,4.7)[anchor=north] node {\tiny $f_{F_0}$};
\draw [fill=black] (3.8,4.7)circle (1.5pt);
\draw (3.3,4.7)[anchor=north] node {\tiny $g_{F_0}$};
\draw [fill=black] (3.3,4.7)circle (1.5pt);

%
% Node y
%
\draw (7,2.35) node {$\hat y$};

%node path
\draw [line width=1pt,color=cyan] (6.8,2.5) to  (6.2,2.5); 
%\draw [line width=1pt,color=cyan] (6.5,2) to  (6.8,2.5); 


\draw (6.5,1.8) node {\tiny $g_y$};
%\draw (6,2.5) node {\tiny $f_y$};
%\draw (7,2.5) node {\tiny $f_y$};
\draw [fill=black] (6.5,2)circle (1.5pt);
\draw [fill=black] (6.2,2.5)circle (1.5pt);
\draw [fill=black] (6.8,2.5)circle (1.5pt);


%
% Node F_1
%
%\draw (2.1,2) node {$\hat F_1$};
%node path
\draw [line width=1pt,color=red] (2.3,2.3) to  (2.3,1.7); 
\draw [line width=1pt,color=red] (1.7,2) to  (2.3,1.7); 

\draw [line width=1pt,color=red] (2.6,2.6) to  (2.3,2.3); 
\draw [line width=1pt,color=red] (2.6,1.3) to  (2.3,1.7); 
\draw [line width=1pt,color=red] (1.7,2) to  (1.3,2); 

% F_1 to F_2
\draw [line width=1pt,color=red] (.8,2) to  (1.3,2); 

\draw (1.8,2.13) node {\tiny $b_{F_1}$};
\draw [fill=black] (1.7,2)circle (1.5pt);
%\draw (2.3,1.7)[anchor=north] node {\tiny $d_{F_1}$};
\draw [fill=black] (2.3,1.7)circle (1.5pt);
%\draw (2.25,2.2)[anchor=west] node {\tiny $c_{F_1}$};
\draw [fill=black] (2.3,2.3)circle (1.5pt);
\draw (1.4,1.85) node {\tiny $b_3$};
\draw [fill=white] (1.3,2)circle (1.5pt);

%\draw (2.8,1.4) node {\tiny $d_3$};
\draw [fill=white] (2.6,1.3)circle (1.5pt);
%\draw (2.75,2.55) node {\tiny $c_3$};
\draw [fill=white] (2.6,2.6)circle (1.5pt);

%
% Node F_2
%
\draw (0,2.2) node {$\hat F_2$};

\draw [line width=1pt,color=red] (-1.2,2) to  (-.3,2); 
\draw [line width=1pt,color=red] (.8,2) to  (.3,2);   
\draw [line width=1pt,color=red] (-.3,2) to (.3,2);   

\draw [fill=black] (-.2,2) circle (1.5pt);
\draw (-.2,1.8) node {\tiny $a_{F_2}$};
\draw [fill=black] (.2,2) circle (1.5pt);
\draw (.3,1.8) node {\tiny $b_{F_2}$};
\draw (-.75,2.1) node {\tiny $a_1$};
\draw [fill=white] (-.8,2) circle (1.5pt);
\draw (.8,2.16) node {\tiny $b_1$};
\draw [fill=white] (.8,2) circle (1.5pt);

\draw (-1.3,1.9) node {\tiny $a_3$};
\draw [fill=white] (-1.2,2)circle (1.5pt);

%
% Node F_3
%
%\draw (6.7,3.2) node {$\hat F_3$};
%\draw (6.3,3) node {\tiny $f_{F_3}$};
\draw [fill=black] (6.5,3)circle (1.5pt);

\draw (5,2)[anchor=south] node {\tiny $g_0$};
\draw [fill=white] (5,2)circle (1.5pt);
\draw (5.5,2)[anchor=south] node {\tiny $g_2$};
\draw [fill=white] (5.5,2)circle (1.5pt);
\draw (5.25,2.25)[anchor=south] node {\tiny $g_3$};
\draw [fill=white] (5.25,2.25)circle (1.5pt);
\draw (5.25,1.75)[anchor=north] node {\tiny $g_1$};
\draw [fill=white] (5.25,1.75)circle (1.5pt);

\end{scriptsize}
\end{tikzpicture}

\caption{ADPs logo antes da linha~\ref{Algo:MSFdelEdge:c1:split} do Algoritmo~\ref{Algo:MSFdelEdge}.}
\label{fig:MSF-ordem-ciclica-ponte-e}
\end{subfigure}
	\caption{Ordens cíclicas de faces exteriores. Figura~\ref{fig:MSF-ordem-ciclica-ponte} representa o caso genérico. Figura~\ref{fig:MSF-ordem-ciclica-ponte-e} ilustra as ADPS da Figura~\ref{fig:MSF-figura-4} ao remover a aresta~$e$.}
\end{figure}




O segundo caso é tratado entre as linhas~\ref{Algo:MSFdelEdge:linha:mudaPeso} e~\ref{Algo:MSFdelEdge:linha:fim} do Algoritmo~\ref{Algo:MSFdelEdge}.
Nesse caso, a remoção de~$e$ não desconecta sua componente conexa.
Dessa forma, caso~$e$ seja uma aresta de~$F$, é necessário buscar uma aresta que a substituirá em~$F$, assim mantendo~$F$ como maximal.
Para fazer essa busca eficientemente, na linha~\ref{Algo:MSFdelEdge:linha:mudaPeso} do Algoritmo~\ref{Algo:MSFdelEdge}, o peso de~$e$ é mudado para~$\infty$, fazendo com que $e^\star$ se torne uma aresta de~$F^\star$. 
Em nosso exemplo, note que \MSFupdate($G$, $b$, $\infty$) não altera as estruturas das ADPs, já que $b^\star$ já é uma aresta de~$\hat T^\star$.

Em seguida, na linha~\ref{Algo:MSFdelEdge:linha:split} do Algoritmo~\ref{Algo:MSFdelEdge}, o vértice~$\hat e^\star$ é dividido em dois vértices~$\hat e_1$ e~$\hat e_3$. 
Essa operação quebra a ADP~$\hat T^\star$ em duas, assim tornando possível juntar as faces~$F_e$ e~$F'_e$ incidentes a~$e$ sem criar ciclos em~$\hat T^\star$.
A junção dessas faces ocorre na linha~\ref{Algo:MSFdelEdge:linha:merge} do Algoritmo~\ref{Algo:MSFdelEdge} e é feita de forma que a descrição combinatória resultante se mantenha plana, isto é, se $\alpha v_1 \beta$ e $\gamma v_3 \delta$ são as ordens cíclicas de~$\hat F_e$ e~$\hat F'_e$, respectivamente, então $\alpha \delta\gamma\beta$ é a ordem cíclica da face resultante.
Como ilustra a Figura~\ref{fig:MSF-ordem-ciclica-nao-ponte}.
Em mais detalhes, as chamadas \LCOCycle($v_1$) e \LCOCycle($v_3$) garantem que as ordens cíclicas de~$F'_e$ e~$F_e$ tenham forma~$\delta\gamma v_1$ e~$\beta\alpha v_3$.
Em seguida, a chamada \LCOMerge($v_1$, $v_3$, $F_0$) junta as faces~$F'_e$ e~$F_e$ criando uma nova face~$F_0$ com ordem cíclica $\delta\gamma v_1\beta\alpha v_3$.
Entre as linhas~\ref{Algo:MSFdelEdge:linha:1cs} e~\ref{Algo:MSFdelEdge:linha:ucs}, cada nó~$v_i$ é desconectado de sua ordem cíclica com um par de chamadas \LCOCycle($v_i$) e \LCOSplit($v_i$, $\hat v_i$, $\Nil$) como feito no caso anterior.
Em particular, os nós~$v_1$ e $v_3$ são desconectado da ordem cíclica $\delta\gamma v_1\beta\alpha v_3$ obtendo a ordem cíclica $\delta\gamma\beta\alpha$ como desejado.
Por fim, a óctupla é desalocada da memória.


\begin{figure}[h!]
\begin{subfigure}{.4\textwidth}
\begin{tikzpicture}[line cap=round,line join=round,>=triangle 45,x=1cm,y=1cm]
\clip(-1,-2.3) rectangle (6,3);

\draw (0,0) circle [radius=1cm];
\draw (4,0) circle [radius=1cm];
\draw (0,0) node[anchor=north] {$G$};
\draw (4,0) node[anchor=north] {$G$};


\draw [line width=1pt] (2,1.5) to[out=170,in=90,looseness=1] (-.5,.5); % F0 -- F2
\draw [line width=1pt] (2,1.5) to[out=175,in=90,looseness=1] (-.3,.3); % F0 -- F2
\draw [line width=1pt] (2,1.5) to[out=185,in=90,looseness=1] (0,0); % F0 -- F2
\draw [line width=1pt] (2,1.5) to[out=195,in=90,looseness=1] (.4,0); % F0 -- F2


\draw [line width=1pt] (2,1.5) to[out=0,in=90,looseness=1] (4.3,.7); % F0 -- F3
\draw [line width=1pt] (2,1.5) to[out=0,in=90,looseness=1] (4.2,.5); % F0 -- F3
\draw [line width=1pt] (2,1.5) to[out=0,in=90,looseness=1] (3.9,.4); % F0 -- F3


\draw [line width=1pt] (2,-1.5) to[out=-170,in=-90,looseness=1] (-.7,-.3); % F0 -- F2
\draw [line width=1pt] (2,-1.5) to[out=-175,in=-90,looseness=1] (-.3,-.5); % F0 -- F2
\draw [line width=1pt] (2,-1.5) to[out=-185,in=-90,looseness=1] (.3,-.7); % F0 -- F2
\draw [line width=1pt] (2,-1.5) to[out=-195,in=-90,looseness=1] (.7,-.6); % F0 -- F2

\draw [line width=1pt] (2,-1.5) to[out=0,in=-90,looseness=1] (4.3,-.7); % F0 -- F3
\draw [line width=1pt] (2,-1.5) to[out=0,in=-90,looseness=1] (4.2,-.5); % F0 -- F3
\draw [line width=1pt] (2,-1.5) to[out=0,in=-90,looseness=1] (3.9,-.4); % F0 -- F3
% e
\draw [line width=1pt] (1,0) to (3,0); 
\draw (2,0) node[anchor=south east] {$e$};


%e*
\draw [line width=1pt] (2,1.5) to (2,-1.5); 
%\draw (2.6,.5) node[anchor=south east] {$e^\star$};


\draw (2.1,.7) node[anchor=south east] {$v_3$};
\draw (2.1,-.9) node[anchor=east] {$v_1$};


\draw (2,1.5) node[anchor=south] {$F_e$};
\draw [fill=black] (2,1.5) circle (1.5pt);

\draw (2,-1.5) node[anchor=north] {$F'_e$};
\draw [fill=black] (2,-1.5) circle (1.5pt);

% Arcos
%\draw (2,-1.5) circle [radius=1cm];

\draw[red,line width=1pt] (2.85,1.2) arc (-35:-5:1);
\draw[line width=.1pt,->,red] (3.025,1.7) to (3.025,1.8);
\draw (3.1,2) node[red] {$\beta$};

\draw[red,line width=1pt] (1,1.7) arc (180:220:1);
\draw[line width=.1pt,->,red] (1.2,1.1) to (1.3,1);
\draw (1,1.9) node[red] {$\alpha$};

\draw[red,line width=1pt] (1.5,-.65) arc (120:200:1);
\draw[line width=.1pt,->,red] (1.05,-1.8) to (1.1,-1.9);
\draw (1.2,-.6) node[red] {$\delta$};

\draw[red,line width=1pt] (2.5,-.65) arc (60:-20:1);
\draw[line width=.1pt,->,red] (2.5,-.62) to (2.45,-.57);
\draw (2.9,-.6) node[red] {$\gamma$};
\end{tikzpicture}

\caption{Antes de remover $e$.}
\label{fig:MSF-ordem-ciclica-nao-ponte}
\end{subfigure}
\hfill
\begin{subfigure}{.4\textwidth}
\begin{tikzpicture}[line cap=round,line join=round,>=triangle 45,x=1cm,y=1cm]
\clip(-1,-2.3) rectangle (6,3);

\draw (0,0) circle [radius=1cm];
\draw (4,0) circle [radius=1cm];
\draw (0,0) node[anchor=north] {$G$};
\draw (4,.3) node[anchor=north] {$G$};


\draw [line width=1pt] (2,0) to[out=90,in=90,looseness=1] (-.5,.5); % F0 -- F2
\draw [line width=1pt] (2,0) to[out=100,in=90,looseness=1] (-.3,.3); % F0 -- F2
\draw [line width=1pt] (2,0) to[out=110,in=90,looseness=1] (0,0); % F0 -- F2
\draw [line width=1pt] (2,0) to[out=120,in=90,looseness=1] (.4,0); % F0 -- F2


\draw [line width=1pt] (2,0) to[out=90,in=90,looseness=1] (4.3,.7); % F0 -- F3
\draw [line width=1pt] (2,0) to[out=80,in=90,looseness=1] (4.2,.5); % F0 -- F3
\draw [line width=1pt] (2,0) to[out=70,in=90,looseness=1] (3.9,.4); % F0 -- F3


\draw [line width=1pt] (2,0) to[out=-90,in=-90,looseness=1] (-.7,-.5); % F0 -- F2
\draw [line width=1pt] (2,0) to[out=-100,in=-90,looseness=1] (-.3,-.5); % F0 -- F2
\draw [line width=1pt] (2,0) to[out=-110,in=-90,looseness=1] (.3,-.5); % F0 -- F2
\draw [line width=1pt] (2,0) to[out=-120,in=-90,looseness=1] (.7,-.3); % F0 -- F2

\draw [line width=1pt] (2,0) to[out=-90,in=-90,looseness=1] (4.3,-.7); % F0 -- F3
\draw [line width=1pt] (2,0) to[out=-90,in=-90,looseness=1] (4.2,-.5); % F0 -- F3
\draw [line width=1pt] (2,0) to[out=-90,in=-90,looseness=1] (3.9,-.4); % F0 -- F3

\draw (2,0) node[anchor=east] {$F_e$};
\draw [fill=black] (2,0) circle (1.5pt);

% Arcos
%\draw (2,0) circle [radius=1cm];

\draw[red,line width=1pt] (2.33,.96) arc (70:30:1);
\draw[line width=.1pt,->,red] (2.3,0.97) to (2.25,1);
\draw (2.7,1.2) node[red] {$\beta$};

\draw[red,line width=1pt] (1.5,.85) arc (120:160:1);
\draw[line width=.1pt,->,red] (1.05,.25) to (1.04,.2);
\draw (1.4,1) node[red] {$\alpha$};

\draw[red,line width=1pt] (1.08,-.34) arc (200:250:1);
\draw[line width=.1pt,->,red] (1.8,-.98) to (1.85,-1);
\draw (1.5,-1.2) node[red] {$\delta$};

\draw[red,line width=1pt] (2.5,-.86) arc (-60:-30:1);
\draw[line width=.1pt,->,red] (2.88,-.46) to (2.93,-.35);
\draw (2.7,-1.1) node[red] {$\gamma$};
\end{tikzpicture}

\caption{Depois de remover~$e$.}
\label{fig:MSF-ordem-ciclica-nao-ponte-depois}
\end{subfigure}
\caption{Ordens cíclicas das faces incidentes à aresta~$e$ antes e depois de sua remoção.}
\end{figure}

Na simulação de \MSFdelEdge($G$, $b$), após cortar o vértice~$\hat b$ em dois, obtém-se duas ADPs, uma contendo somente o vértice~$\hat F_2$ e as duas folhas $\hat b_1$ e~$\hat a_1$, como pode ser visto na Figura~\ref{fig:MSF-remove-aresta-ex-1}.
Na linha~\ref{Algo:MSFdelEdge:linha:merge} do Algoritmo~\ref{Algo:MSFdelEdge} as ordens cíclicas de~$\hat F_1$ e~$\hat F_2$ são unidas.
Primeiro, \LCOCycle($b_{F_2}$) e \LCOCycle($b_{F_1}$) rearranjam as ordens de~$\hat F_2$ e~$\hat F_1$ para terem forma~$\langle  a_{F_2}, b_{F_2} \rangle$ e~$\langle d_{F_1}, c_{F_1}, b_{F_1}\rangle$, respectivamente.
Em seguida, esses vértices são juntados com \LCOMerge($b_{F_2}$, $b_{F_1}$), obtendo o vértice com ordem~$\langle a_{F_2}, b_{F_2}, d_{F_1}, c_{F_1}, b_{F_1}\rangle$, como ilustrado na Figura~\ref{fig:MSF-remove-aresta-ex-2}.
Por fim, a óctupla de~$b$ é desconectada das ordens cíclicas e desalocada da memória.
A ordem da face resultante é $\langle a_{F_2}, d_{F_1}, c_{F_1}\rangle$, como ilustrado na Figura~\ref{Algo:MSFdelEdge:linha:desaloca}.
Note que esse vértice de ADP representa a face~$F_6$ da Figura~\ref{fig:MSF-remove-nao-ponte}. 

\begin{figure}[h!]
\begin{subfigure}{.2\textwidth}
\begin{tikzpicture}[line cap=round,line join=round,>=triangle 45,x=1cm,y=1cm]
\clip(-1.2,-.3) rectangle (3,5.6);

\begin{scriptsize}

%
% Node v
%
%\draw (0,0) node {$\hat v$};
\draw [line width=1pt,color=cyan] (0,.5) to  (.35,.35); 
\draw [line width=1pt,color=cyan] (.35,.35) to  (.5,0); 

\draw [line width=1pt,color=cyan] (0,.5) to  (-1,1.8); 
\draw [line width=1pt,color=cyan] (.35,.35) to  (1.05,1.75); 
\draw [line width=1pt,color=cyan] (.5,0) to  (2.5,1); 


%\draw (0,.5)[anchor=east] node {\tiny $a_v$};
\draw (.35,.35)[anchor=west] node {\tiny $b_v$};
%\draw (.5,0)[anchor=north west] node {\tiny $d_v$};
\draw [fill=black] (0,.5) circle (1.5pt);
\draw [fill=black] (.35,.35) circle (1.5pt);
\draw [fill=black] (.5,0) circle (1.5pt);

\draw [line width=1pt,color=cyan] (-1,1.8) to  (-1,2.2); 
%\draw (-.6,1.9) node[anchor=north east] {\tiny $a_0$};
\draw [fill=white] (-1,1.8) circle (1.5pt);

% ^b
\draw [line width=1pt,color=cyan] (2.5,1) to (2.9,1.15); 
\draw (1,1.65)[anchor=west] node {\tiny $b_2$};
\draw [fill=white] (1.05,1.75) circle (1.5pt);
%\draw (2.5,1)[anchor=north] node {\tiny $d_0$};
\draw [fill=white] (2.5,1) circle (1.5pt);

%
% Node u
%
%\draw (0,4) node {$\hat u$};

\draw [line width=1pt,color=cyan] (0,3.5) to  (.3,3.7); 
\draw [line width=1pt,color=cyan] (.5,4) to  (.3,3.7); 

\draw [line width=1pt,color=cyan] (0,3.5) to  (-1,2.2); 
\draw [line width=1pt,color=cyan] (.3,3.7) to  (1.05,2.25); 
\draw [line width=1pt,color=cyan] (.5,4) to  (2.5,3); 

%\draw (-1.1,2.35) node {\tiny $a_2$};
\draw [fill=white] (-1,2.2) circle (1.5pt);
\draw (1.2,2.33) node {\tiny $b_0$};
\draw [fill=white] (1.05,2.25) circle (1.5pt);

%\draw (2.4,2.9) node {\tiny $c_2$};
\draw [fill=white] (2.5,3) circle (1.5pt);

% u node path
%\draw (-.2,3.5) node {\tiny $a_u$};
\draw [fill=black] (0,3.5) circle (1.5pt);
%\draw (.5,4.15) node {\tiny $c_u$};
\draw [fill=black] (.5,4) circle (1.5pt);
\draw (.19,3.82) node {\tiny $b_u$};
\draw [fill=black] (.3,3.7) circle (1.5pt);

%
% Node z
%
\draw (3.95,2.1) node {$\hat z$};
\draw [line width=1pt,color=cyan] (4.3,2) to (6.5,2); 
%\draw [line width=1pt,color=cyan] (3.7,2.3) to (3,2.7); 
\draw [line width=1pt,color=cyan] (3.7,1.7) to (2.9,1.15); 

% node path
\draw [line width=1pt,color=cyan] (3.7,2.3) to  (3.7,1.7); 
\draw [line width=1pt,color=cyan] (3.7,1.7) to  (4.3,2); 


%\draw (3.75,1.55) node {\tiny $d_z$};
\draw [fill=black] (3.7,1.7)circle (1.5pt);
%\draw (3.7,2.43) node {\tiny $c_z$};
\draw [fill=black] (3.7,2.3)circle (1.5pt);
\draw (4.3,2.2) node {\tiny $g_z$};
\draw [fill=black] (4.3,2)circle (1.5pt);

%\draw (3.2,2.7) node {\tiny $c_0$};
%\draw [fill=white] (3,2.7)circle (1.5pt);
%\draw (3.05,1.25)[anchor=north] node {\tiny $d_2$};
\draw [fill=white] (2.9,1.15)circle (1.5pt);

%eF0 -- e3
\draw [line width=1pt,color=red] (3.3,4.7) to  (5.25,2.25);


%
% Node F_0
%
%\draw (3,5.2) node {$\hat F_0$};
%\draw [line width=1pt,color=red] (2.7,4.7) to  (3.8,4.7); 
%\draw [line width=1pt,color=red] (3.8,4.7) to  (4,5); 
%\draw [line width=1pt,color=red] (4,5) to  (2.5,5); 

%\draw [line width=1pt,color=red] (2.5,5) to[in=180,out=180,looseness=2]  (-1.2,2); 
%\draw [line width=1pt,color=red] (4,5) to[in=-20,out=10,looseness=3]  (5.25,1.75); 
%\draw [line width=1pt,color=red] (3.5,5) to[in=-30,out=20,looseness=4.5]  (2.8,.85); 

%^c
%\draw [line width=1pt,color=red] (2.6,2.6) to  (2.9,3.1);
%\draw [line width=1pt,color=red] (2.7,4.7) to  (2.9,3.1);

%\draw (-1.3,1.9) node {\tiny $a_3$};
%\draw [fill=white] (-1.2,2)circle (1.5pt);
%\draw (3.07,3.1) node {\tiny $c_1$};
%\draw [fill=white] (2.9,3.1)circle (1.5pt);
%\draw (2.9,.7) node {\tiny $d_1$};
%\draw [fill=white] (2.8,.85)circle (1.5pt);


%fF3 -- 
%\draw [line width=1pt,color=red] (6.5,3.9) to  (6.5,3);

% ^f
%\draw [line width=1pt,color=cyan] (6.8,2.5) to[in=-60,out=60,looseness=1.5] (6.65,3.8);
%\draw [line width=1pt,color=cyan] (6.2,2.5) to[in=-120,out=120,looseness=1.5] (6.35,3.8);

\draw [line width=1pt,color=red] (3.8,4.7) to  (6.5,3.95);
%\draw (6.65,4.05) node {\tiny $f_0$};
%\draw [fill=white] (6.5,3.95)circle (1.5pt);
\draw (6.65,3.55) node {\tiny $f_2$};
%\draw [fill=white] (6.5,3.65)circle (1.5pt);
\draw (6.8,3.8) node {\tiny $f_3$};
%\draw [fill=white] (6.65,3.8)circle (1.5pt);
\draw (6.2,3.8) node {\tiny $f_1$};
%\draw [fill=white] (6.35,3.8)circle (1.5pt);


%\draw (2.5,5)[anchor=south] node {\tiny $a_{F_0}$};
%\draw [fill=black] (2.5,5)circle (1.5pt);
%\draw (2.5,4.6) node {\tiny $c_{F_0}$};
%\draw [fill=black] (2.7,4.7)circle (1.5pt);
%\draw (3.5,5)[anchor=south] node {\tiny $d_{F_0}$};
\draw [fill=black] (3.5,5)circle (1.5pt);
\draw (4,5)[anchor=north west] node {\tiny $g_{F_0}$};
\draw [fill=black] (4,5)circle (1.5pt);
\draw (3.8,4.7)[anchor=north] node {\tiny $f_{F_0}$};
\draw [fill=black] (3.8,4.7)circle (1.5pt);
\draw (3.3,4.7)[anchor=north] node {\tiny $g_{F_0}$};
\draw [fill=black] (3.3,4.7)circle (1.5pt);

%
% Node y
%
\draw (6.5,2.35) node {$\hat y$};

%node path
\draw [line width=1pt,color=cyan] (6.5,2) to  (6.2,2.5); 
\draw [line width=1pt,color=cyan] (6.5,2) to  (6.8,2.5); 


\draw (6.5,1.8) node {\tiny $g_y$};
\draw (6,2.5) node {\tiny $f_y$};
\draw (7,2.5) node {\tiny $f_y$};
\draw [fill=black] (6.5,2)circle (1.5pt);
\draw [fill=black] (6.2,2.5)circle (1.5pt);
\draw [fill=black] (6.8,2.5)circle (1.5pt);


%
% Node F_1
%
\draw (2.1,2) node {$\hat F_1$};
%node path
\draw [line width=1pt,color=red] (2.3,2.3) to  (2.3,1.7); 
\draw [line width=1pt,color=red] (1.7,2) to  (2.3,1.7); 

\draw [line width=1pt,color=red] (2.6,2.6) to  (2.3,2.3); 
\draw [line width=1pt,color=red] (2.6,1.3) to  (2.3,1.7); 
\draw [line width=1pt,color=red] (1.7,2) to  (1.3,2); 

% F_1 to F_2
%\draw [line width=1pt,color=red] (.8,2) to  (1.3,2); 

\draw (1.8,2.13) node {\tiny $b_{F_1}$};
\draw [fill=black] (1.7,2)circle (1.5pt);
\draw (2.3,1.7)[anchor=north] node {\tiny $d_{F_1}$};
\draw [fill=black] (2.3,1.7)circle (1.5pt);
\draw (2.25,2.2)[anchor=west] node {\tiny $c_{F_1}$};
\draw [fill=black] (2.3,2.3)circle (1.5pt);
\draw (1.4,1.85) node {\tiny $b_3$};
\draw [fill=white] (1.3,2)circle (1.5pt);

%\draw (2.8,1.4) node {\tiny $d_3$};
\draw [fill=white] (2.6,1.3)circle (1.5pt);
%\draw (2.75,2.55) node {\tiny $c_3$};
\draw [fill=white] (2.6,2.6)circle (1.5pt);

%
% Node F_2
%
\draw (0,2.2) node {$\hat F_2$};

\draw [line width=1pt,color=red] (-.8,2) to  (-.3,2); 
\draw [line width=1pt,color=red] (.8,2) to  (.3,2);   
\draw [line width=1pt,color=red] (-.3,2) to (.3,2);   

\draw [fill=black] (-.2,2) circle (1.5pt);
\draw (-.2,1.8) node {\tiny $a_{F_2}$};
\draw [fill=black] (.2,2) circle (1.5pt);
\draw (.3,1.8) node {\tiny $b_{F_2}$};
%\draw (-.75,2.1) node {\tiny $a_1$};
\draw [fill=white] (-.8,2) circle (1.5pt);
\draw (.8,2.16) node {\tiny $b_1$};
\draw [fill=white] (.8,2) circle (1.5pt);

%
% Node F_3
%
\draw (6.7,3.2) node {$\hat F_3$};
\draw (6.3,3) node {\tiny $f_{F_3}$};
\draw [fill=black] (6.5,3)circle (1.5pt);

\draw (5,2)[anchor=south] node {\tiny $g_0$};
\draw [fill=white] (5,2)circle (1.5pt);
\draw (5.5,2)[anchor=south] node {\tiny $g_2$};
\draw [fill=white] (5.5,2)circle (1.5pt);
\draw (5.3,2.25)[anchor=south] node {\tiny $g_3$};
\draw [fill=white] (5.25,2.25)circle (1.5pt);
\draw (5.25,1.75)[anchor=north] node {\tiny $g_1$};
\draw [fill=white] (5.25,1.75)circle (1.5pt);

\end{scriptsize}
\end{tikzpicture}

\caption{Após linha~\ref{Algo:MSFdelEdge:linha:split}.}
\label{fig:MSF-remove-aresta-ex-1}
\end{subfigure}
\hfill
\begin{subfigure}{.2\textwidth}
\begin{tikzpicture}[line cap=round,line join=round,>=triangle 45,x=1cm,y=1cm]
\clip(-1.2,-.3) rectangle (3,5.6);

\begin{scriptsize}

%
% Node v
%
%\draw (0,0) node {$\hat v$};
\draw [line width=1pt,color=cyan] (0,.5) to  (.35,.35); 
\draw [line width=1pt,color=cyan] (.35,.35) to  (.5,0); 

\draw [line width=1pt,color=cyan] (0,.5) to  (-1,1.8); 
\draw [line width=1pt,color=cyan] (.35,.35) to  (1.05,.75); 
\draw [line width=1pt,color=cyan] (.5,0) to  (2.5,1); 


%\draw (0,.5)[anchor=east] node {\tiny $a_v$};
\draw (.35,.35)[anchor=west] node {\tiny $b_v$};
%\draw (.5,0)[anchor=north west] node {\tiny $d_v$};
\draw [fill=black] (0,.5) circle (1.5pt);
\draw [fill=black] (.35,.35) circle (1.5pt);
\draw [fill=black] (.5,0) circle (1.5pt);

\draw [line width=1pt,color=cyan] (-1,1.8) to  (-1,2.2); 
%\draw (-.6,1.9) node[anchor=north east] {\tiny $a_0$};
\draw [fill=white] (-1,1.8) circle (1.5pt);

% ^b
\draw [line width=1pt,color=cyan] (2.5,1) to (2.9,1.15); 
\draw (1,.65)[anchor=west] node {\tiny $b_2$};
\draw [fill=white] (1.05,.75) circle (1.5pt);
%\draw (2.5,1)[anchor=north] node {\tiny $d_0$};
\draw [fill=white] (2.5,1) circle (1.5pt);

%
% Node u
%
%\draw (0,4) node {$\hat u$};

\draw [line width=1pt,color=cyan] (0,3.5) to  (.3,3.7); 
\draw [line width=1pt,color=cyan] (.5,4) to  (.3,3.7); 

\draw [line width=1pt,color=cyan] (0,3.5) to  (-1,2.2); 
\draw [line width=1pt,color=cyan] (.3,3.7) to  (1.05,2.25); 
\draw [line width=1pt,color=cyan] (.5,4) to  (2.5,3); 

%\draw (-1.1,2.35) node {\tiny $a_2$};
\draw [fill=white] (-1,2.2) circle (1.5pt);
\draw (1.2,2.33) node {\tiny $b_0$};
\draw [fill=white] (1.05,2.25) circle (1.5pt);

%\draw (2.4,2.9) node {\tiny $c_2$};
\draw [fill=white] (2.5,3) circle (1.5pt);

% u node path
%\draw (-.2,3.5) node {\tiny $a_u$};
\draw [fill=black] (0,3.5) circle (1.5pt);
%\draw (.5,4.15) node {\tiny $c_u$};
\draw [fill=black] (.5,4) circle (1.5pt);
\draw (.19,3.82) node {\tiny $b_u$};
\draw [fill=black] (.3,3.7) circle (1.5pt);

%
% Node z
%
\draw (3.95,2.1) node {$\hat z$};
\draw [line width=1pt,color=cyan] (4.3,2) to (6.5,2); 
%\draw [line width=1pt,color=cyan] (3.7,2.3) to (3,2.7); 
\draw [line width=1pt,color=cyan] (3.7,1.7) to (2.9,1.15); 

% node path
\draw [line width=1pt,color=cyan] (3.7,2.3) to  (3.7,1.7); 
\draw [line width=1pt,color=cyan] (3.7,1.7) to  (4.3,2); 


%\draw (3.75,1.55) node {\tiny $d_z$};
\draw [fill=black] (3.7,1.7)circle (1.5pt);
%\draw (3.7,2.43) node {\tiny $c_z$};
\draw [fill=black] (3.7,2.3)circle (1.5pt);
\draw (4.3,2.2) node {\tiny $g_z$};
\draw [fill=black] (4.3,2)circle (1.5pt);

%\draw (3.2,2.7) node {\tiny $c_0$};
%\draw [fill=white] (3,2.7)circle (1.5pt);
%\draw (3.05,1.25)[anchor=north] node {\tiny $d_2$};
\draw [fill=white] (2.9,1.15)circle (1.5pt);

%eF0 -- e3
\draw [line width=1pt,color=red] (3.3,4.7) to  (5.25,2.25);


%
% Node F_0
%
%\draw (3,5.2) node {$\hat F_0$};
%\draw [line width=1pt,color=red] (2.7,4.7) to  (3.8,4.7); 
%\draw [line width=1pt,color=red] (3.8,4.7) to  (4,5); 
%\draw [line width=1pt,color=red] (4,5) to  (2.5,5); 

%\draw [line width=1pt,color=red] (2.5,5) to[in=180,out=180,looseness=2]  (-1.2,2); 
%\draw [line width=1pt,color=red] (4,5) to[in=-20,out=10,looseness=3]  (5.25,1.75); 
%\draw [line width=1pt,color=red] (3.5,5) to[in=-30,out=20,looseness=4.5]  (2.8,.85); 

%^c
%\draw [line width=1pt,color=red] (2.6,2.6) to  (2.9,3.1);
%\draw [line width=1pt,color=red] (2.7,4.7) to  (2.9,3.1);

%\draw (-1.3,1.9) node {\tiny $a_3$};
%\draw [fill=white] (-1.2,2)circle (1.5pt);
%\draw (3.07,3.1) node {\tiny $c_1$};
%\draw [fill=white] (2.9,3.1)circle (1.5pt);
%\draw (2.9,.7) node {\tiny $d_1$};
%\draw [fill=white] (2.8,.85)circle (1.5pt);


%fF3 -- 
%\draw [line width=1pt,color=red] (6.5,3.9) to  (6.5,3);

% ^f
%\draw [line width=1pt,color=cyan] (6.8,2.5) to[in=-60,out=60,looseness=1.5] (6.65,3.8);
%\draw [line width=1pt,color=cyan] (6.2,2.5) to[in=-120,out=120,looseness=1.5] (6.35,3.8);

\draw [line width=1pt,color=red] (3.8,4.7) to  (6.5,3.95);
%\draw (6.65,4.05) node {\tiny $f_0$};
%\draw [fill=white] (6.5,3.95)circle (1.5pt);
\draw (6.65,3.55) node {\tiny $f_2$};
%\draw [fill=white] (6.5,3.65)circle (1.5pt);
\draw (6.8,3.8) node {\tiny $f_3$};
%\draw [fill=white] (6.65,3.8)circle (1.5pt);
\draw (6.2,3.8) node {\tiny $f_1$};
%\draw [fill=white] (6.35,3.8)circle (1.5pt);


%\draw (2.5,5)[anchor=south] node {\tiny $a_{F_0}$};
%\draw [fill=black] (2.5,5)circle (1.5pt);
%\draw (2.5,4.6) node {\tiny $c_{F_0}$};
%\draw [fill=black] (2.7,4.7)circle (1.5pt);
%\draw (3.5,5)[anchor=south] node {\tiny $d_{F_0}$};
\draw [fill=black] (3.5,5)circle (1.5pt);
\draw (4,5)[anchor=north west] node {\tiny $g_{F_0}$};
\draw [fill=black] (4,5)circle (1.5pt);
\draw (3.8,4.7)[anchor=north] node {\tiny $f_{F_0}$};
\draw [fill=black] (3.8,4.7)circle (1.5pt);
\draw (3.3,4.7)[anchor=north] node {\tiny $g_{F_0}$};
\draw [fill=black] (3.3,4.7)circle (1.5pt);

%
% Node y
%
\draw (6.5,2.35) node {$\hat y$};

%node path
\draw [line width=1pt,color=cyan] (6.5,2) to  (6.2,2.5); 
\draw [line width=1pt,color=cyan] (6.5,2) to  (6.8,2.5); 


\draw (6.5,1.8) node {\tiny $g_y$};
\draw (6,2.5) node {\tiny $f_y$};
\draw (7,2.5) node {\tiny $f_y$};
\draw [fill=black] (6.5,2)circle (1.5pt);
\draw [fill=black] (6.2,2.5)circle (1.5pt);
\draw [fill=black] (6.8,2.5)circle (1.5pt);


\draw [line width=1pt,color=red] (.2,2) to[in=200,out=-30,looseness=.5]  (2.3,1.7); 

%
% Node F_1
%
%\draw (2.1,2) node {$\hat F_1$};
%node path
\draw [line width=1pt,color=red] (2.3,2.3) to  (2.3,1.7); 
\draw [line width=1pt,color=red] (1.7,2) to  (2.3,2.3); 

\draw [line width=1pt,color=red] (2.6,2.6) to  (2.3,2.3); 
\draw [line width=1pt,color=red] (2.6,1.3) to  (2.3,1.7); 
\draw [line width=1pt,color=red] (1.7,2) to  (1.3,2); 

% F_1 to F_2
%\draw [line width=1pt,color=red] (.8,2) to  (1.3,2); 

\draw (1.84,1.83) node {\tiny $b_{F_1}$};
\draw [fill=black] (1.7,2)circle (1.5pt);
\draw (2.3,1.7)[anchor=north] node {\tiny $d_{F_1}$};
\draw [fill=black] (2.3,1.7)circle (1.5pt);
\draw (2.25,2.2)[anchor=west] node {\tiny $c_{F_1}$};
\draw [fill=black] (2.3,2.3)circle (1.5pt);
\draw (1.4,1.85) node {\tiny $b_3$};
\draw [fill=white] (1.3,2)circle (1.5pt);

%\draw (2.8,1.4) node {\tiny $d_3$};
\draw [fill=white] (2.6,1.3)circle (1.5pt);
%\draw (2.75,2.55) node {\tiny $c_3$};
\draw [fill=white] (2.6,2.6)circle (1.5pt);

%
% Node F_2
%
%\draw (0,2.2) node {$\hat F_2$};

\draw [line width=1pt,color=red] (-.8,2) to  (-.3,2); 
\draw [line width=1pt,color=red] (.2,1.5) to  (.2,2);   
\draw [line width=1pt,color=red] (-.3,2) to (.2,2);   

\draw [fill=black] (-.2,2) circle (1.5pt);
\draw (-.2,1.8) node {\tiny $a_{F_2}$};
\draw [fill=black] (.2,2) circle (1.5pt);
\draw (.3,2.2) node {\tiny $b_{F_2}$};
%\draw (-.75,2.1) node {\tiny $a_1$};
\draw [fill=white] (-.8,2) circle (1.5pt);
\draw (.2,1.5) node[anchor=north] {\tiny $b_1$};
\draw [fill=white] (.2,1.5) circle (1.5pt);

%
% Node F_3
%
\draw (6.7,3.2) node {$\hat F_3$};
\draw (6.3,3) node {\tiny $f_{F_3}$};
\draw [fill=black] (6.5,3)circle (1.5pt);

\draw (5,2)[anchor=south] node {\tiny $g_0$};
\draw [fill=white] (5,2)circle (1.5pt);
\draw (5.5,2)[anchor=south] node {\tiny $g_2$};
\draw [fill=white] (5.5,2)circle (1.5pt);
\draw (5.3,2.25)[anchor=south] node {\tiny $g_3$};
\draw [fill=white] (5.25,2.25)circle (1.5pt);
\draw (5.25,1.75)[anchor=north] node {\tiny $g_1$};
\draw [fill=white] (5.25,1.75)circle (1.5pt);

\end{scriptsize}
\end{tikzpicture}

\caption{Após linha~\ref{Algo:MSFdelEdge:linha:merge}.}
\label{fig:MSF-remove-aresta-ex-2}
\end{subfigure}
\hfill
\begin{subfigure}{.2\textwidth}
\begin{tikzpicture}[line cap=round,line join=round,>=triangle 45,x=1cm,y=1cm]
\clip(-1.2,-.3) rectangle (3,5.6);

\begin{scriptsize}

%
% Node v
%
%\draw (0,0) node {$\hat v$};
\draw [line width=1pt,color=cyan] (0,.5) to  (.5,0); 
%\draw [line width=1pt,color=cyan] (.35,.35) to  (.5,0); 

\draw [line width=1pt,color=cyan] (0,.5) to  (-1,1.8); 
%\draw [line width=1pt,color=cyan] (.35,.35) to  (1.05,1.75); 
\draw [line width=1pt,color=cyan] (.5,0) to  (2.5,1); 


\draw (0,.5)[anchor=east] node {\tiny $a_v$};
%\draw (.35,.35)[anchor=west] node {\tiny $b_v$};
\draw (.5,0)[anchor=north west] node {\tiny $d_v$};
\draw [fill=black] (0,.5) circle (1.5pt);
%\draw [fill=black] (.35,.35) circle (1.5pt);
\draw [fill=black] (.5,0) circle (1.5pt);

\draw [line width=1pt,color=cyan] (-1,1.8) to  (-1,2.2); 
%\draw (-.6,1.9) node[anchor=north east] {\tiny $a_0$};
\draw [fill=white] (-1,1.8) circle (1.5pt);

% ^b
\draw [line width=1pt,color=cyan] (2.5,1) to (2.9,1.15); 
%\draw (1,1.65)[anchor=west] node {\tiny $b_2$};
%\draw [fill=white] (1.05,1.75) circle (1.5pt);
%\draw (2.5,1)[anchor=north] node {\tiny $d_0$};
\draw [fill=white] (2.5,1) circle (1.5pt);

%
% Node u
%
%\draw (0,4) node {$\hat u$};

\draw [line width=1pt,color=cyan] (0,3.5) to  (.5,4); 
%\draw [line width=1pt,color=cyan] (.5,4) to  (.3,3.7); 

\draw [line width=1pt,color=cyan] (0,3.5) to  (-1,2.2); 
%\draw [line width=1pt,color=cyan] (.3,3.7) to  (1.05,2.25); 
\draw [line width=1pt,color=cyan] (.5,4) to  (2.5,3); 

%\draw (-1.1,2.35) node {\tiny $a_2$};
\draw [fill=white] (-1,2.2) circle (1.5pt);
%\draw (1.2,2.33) node {\tiny $b_0$};
%\draw [fill=white] (1.05,2.25) circle (1.5pt);

%\draw (2.4,2.9) node {\tiny $c_2$};
\draw [fill=white] (2.5,3) circle (1.5pt);

% u node path
\draw (-.2,3.5) node {\tiny $a_u$};
\draw [fill=black] (0,3.5) circle (1.5pt);
\draw (.5,4.15) node {\tiny $c_u$};
\draw [fill=black] (.5,4) circle (1.5pt);
%\draw (.19,3.82) node {\tiny $b_u$};
%\draw [fill=black] (.3,3.7) circle (1.5pt);

%
% Node z
%
\draw (3.95,2.1) node {$\hat z$};
\draw [line width=1pt,color=cyan] (4.3,2) to (6.5,2); 
%\draw [line width=1pt,color=cyan] (3.7,2.3) to (3,2.7); 
\draw [line width=1pt,color=cyan] (3.7,1.7) to (2.9,1.15); 

% node path
\draw [line width=1pt,color=cyan] (3.7,2.3) to  (3.7,1.7); 
\draw [line width=1pt,color=cyan] (3.7,1.7) to  (4.3,2); 


%\draw (3.75,1.55) node {\tiny $d_z$};
\draw [fill=black] (3.7,1.7)circle (1.5pt);
%\draw (3.7,2.43) node {\tiny $c_z$};
\draw [fill=black] (3.7,2.3)circle (1.5pt);
\draw (4.3,2.2) node {\tiny $g_z$};
\draw [fill=black] (4.3,2)circle (1.5pt);

%\draw (3.2,2.7) node {\tiny $c_0$};
%\draw [fill=white] (3,2.7)circle (1.5pt);
%\draw (3.05,1.25)[anchor=north] node {\tiny $d_2$};
\draw [fill=white] (2.9,1.15)circle (1.5pt);

%eF0 -- e3
\draw [line width=1pt,color=red] (3.3,4.7) to  (5.25,2.25);


%
% Node F_0
%
%\draw (3,5.2) node {$\hat F_0$};
%\draw [line width=1pt,color=red] (2.7,4.7) to  (3.8,4.7); 
%\draw [line width=1pt,color=red] (3.8,4.7) to  (4,5); 
%\draw [line width=1pt,color=red] (4,5) to  (2.5,5); 

%\draw [line width=1pt,color=red] (2.5,5) to[in=180,out=180,looseness=2]  (-1.2,2); 
%\draw [line width=1pt,color=red] (4,5) to[in=-20,out=10,looseness=3]  (5.25,1.75); 
%\draw [line width=1pt,color=red] (3.5,5) to[in=-30,out=20,looseness=4.5]  (2.8,.85); 

%^c
%\draw [line width=1pt,color=red] (2.6,2.6) to  (2.9,3.1);
%\draw [line width=1pt,color=red] (2.7,4.7) to  (2.9,3.1);

%\draw (-1.3,1.9) node {\tiny $a_3$};
%\draw [fill=white] (-1.2,2)circle (1.5pt);
%\draw (3.07,3.1) node {\tiny $c_1$};
%\draw [fill=white] (2.9,3.1)circle (1.5pt);
%\draw (2.9,.7) node {\tiny $d_1$};
%\draw [fill=white] (2.8,.85)circle (1.5pt);


%fF3 -- 
%\draw [line width=1pt,color=red] (6.5,3.9) to  (6.5,3);

% ^f
%\draw [line width=1pt,color=cyan] (6.8,2.5) to[in=-60,out=60,looseness=1.5] (6.65,3.8);
%\draw [line width=1pt,color=cyan] (6.2,2.5) to[in=-120,out=120,looseness=1.5] (6.35,3.8);

\draw [line width=1pt,color=red] (3.8,4.7) to  (6.5,3.95);
%\draw (6.65,4.05) node {\tiny $f_0$};
%\draw [fill=white] (6.5,3.95)circle (1.5pt);
\draw (6.65,3.55) node {\tiny $f_2$};
%\draw [fill=white] (6.5,3.65)circle (1.5pt);
\draw (6.8,3.8) node {\tiny $f_3$};
%\draw [fill=white] (6.65,3.8)circle (1.5pt);
\draw (6.2,3.8) node {\tiny $f_1$};
%\draw [fill=white] (6.35,3.8)circle (1.5pt);


%\draw (2.5,5)[anchor=south] node {\tiny $a_{F_0}$};
%\draw [fill=black] (2.5,5)circle (1.5pt);
%\draw (2.5,4.6) node {\tiny $c_{F_0}$};
%\draw [fill=black] (2.7,4.7)circle (1.5pt);
%\draw (3.5,5)[anchor=south] node {\tiny $d_{F_0}$};
\draw [fill=black] (3.5,5)circle (1.5pt);
\draw (4,5)[anchor=north west] node {\tiny $g_{F_0}$};
\draw [fill=black] (4,5)circle (1.5pt);
\draw (3.8,4.7)[anchor=north] node {\tiny $f_{F_0}$};
\draw [fill=black] (3.8,4.7)circle (1.5pt);
\draw (3.3,4.7)[anchor=north] node {\tiny $g_{F_0}$};
\draw [fill=black] (3.3,4.7)circle (1.5pt);

%
% Node y
%
\draw (6.5,2.35) node {$\hat y$};

%node path
\draw [line width=1pt,color=cyan] (6.5,2) to  (6.2,2.5); 
\draw [line width=1pt,color=cyan] (6.5,2) to  (6.8,2.5); 


\draw (6.5,1.8) node {\tiny $g_y$};
\draw (6,2.5) node {\tiny $f_y$};
\draw (7,2.5) node {\tiny $f_y$};
\draw [fill=black] (6.5,2)circle (1.5pt);
\draw [fill=black] (6.2,2.5)circle (1.5pt);
\draw [fill=black] (6.8,2.5)circle (1.5pt);


%
% Node F_1
%
%\draw (2.1,2) node {$\hat F_1$};
%node path
\draw [line width=1pt,color=red] (2.3,2.3) to  (2.3,1.7); 
%\draw [line width=1pt,color=red] (1.7,2) to  (2.3,1.7); 

\draw [line width=1pt,color=red] (2.6,2.6) to  (2.3,2.3); 
\draw [line width=1pt,color=red] (2.6,1.3) to  (2.3,1.7); 
%\draw [line width=1pt,color=red] (1.7,2) to  (1.3,2); 

% F_1 to F_2
%\draw [line width=1pt,color=red] (.8,2) to  (1.3,2); 

%\draw (1.8,2.13) node {\tiny $b_{F_1}$};
%\draw [fill=black] (1.7,2)circle (1.5pt);
\draw (2.25,2.2)[anchor=west] node {\tiny $c_{F_1}$};
\draw [fill=black] (2.3,2.3)circle (1.5pt);
%\draw (1.4,1.85) node {\tiny $b_3$};
%\draw [fill=white] (1.3,2)circle (1.5pt);

%\draw (2.8,1.4) node {\tiny $d_3$};
\draw [fill=white] (2.6,1.3)circle (1.5pt);
%\draw (2.75,2.55) node {\tiny $c_3$};
\draw [fill=white] (2.6,2.6)circle (1.5pt);

%
% Node F_2
%
%\draw (0,2.2) node {$\hat F_2$};

\draw [line width=1pt,color=red] (-.8,2) to  (-.2,2); 
\draw [line width=1pt,color=red] (-.2,2) to  (2.3,1.7);   
%\draw [line width=1pt,color=red] (-.3,2) to (.2,2);   
\draw (2.3,1.7)[anchor=north] node {\tiny $d_{F_1}$};
\draw [fill=black] (2.3,1.7)circle (1.5pt);

\draw [fill=black] (-.2,2) circle (1.5pt);
\draw (-.2,1.8) node {\tiny $a_{F_2}$};
%\draw [fill=black] (.2,2) circle (1.5pt);
%\draw (.3,1.8) node {\tiny $b_{F_2}$};
%\draw (-.75,2.1) node {\tiny $a_1$};
\draw [fill=white] (-.8,2) circle (1.5pt);
%\draw (.8,2.16) node {\tiny $b_1$};
%\draw [fill=white] (.8,2) circle (1.5pt);

%
% Node F_3
%
\draw (6.7,3.2) node {$\hat F_3$};
\draw (6.3,3) node {\tiny $f_{F_3}$};
\draw [fill=black] (6.5,3)circle (1.5pt);

\draw (5,2)[anchor=south] node {\tiny $g_0$};
\draw [fill=white] (5,2)circle (1.5pt);
\draw (5.5,2)[anchor=south] node {\tiny $g_2$};
\draw [fill=white] (5.5,2)circle (1.5pt);
\draw (5.3,2.25)[anchor=south] node {\tiny $g_3$};
\draw [fill=white] (5.25,2.25)circle (1.5pt);
\draw (5.25,1.75)[anchor=north] node {\tiny $g_1$};
\draw [fill=white] (5.25,1.75)circle (1.5pt);

\end{scriptsize}
\end{tikzpicture}

\caption{Após linha~\ref{Algo:MSFdelEdge:linha:desaloca}.}
\label{fig:MSF-remove-aresta-ex-3}
\end{subfigure}
\caption{Diferentes momentos da execução de \MSFdelEdge($G$, $b$).}
\label{fig:MSF-remocao-b}
\end{figure}


As linhas~\ref{Algo:MSFdelEdge:linha:1} e~\ref{Algo:MSFdelEdge:linha:desaloca} do Algoritmo~\ref{Algo:MSFdelEdge} possuem consumo de tempo constante, pois são uma consulta à tabela de símbolos~$G.H$ e uma chamada da rotina \LCODestroyOcto{}.
As linhas remanescentes são formadas por um número constante de chamadas de \LCOFindNode, \LCOCycle{}, \LCOMerge{}, \LCOSplit{} e \MSFupdate{} e cada uma dessas chamadas possui consumo de tempo assintótico igual a~$\O{\lg m}$ esperado, onde~$m$ é o número de arestas no grafo corrente.
Dessa forma, $\MSFdelEdge$ também possui consumo de tempo assintótico~$\O{\lg m}$ esperado.


\subsection{Adição de aresta}

A última rotina de MSF elaborada é \MSFaddEdge{}, cuja implementação é descrita no Algoritmo~\ref{Algo:MSFaddEdge}.
Como em \MSFdelEdge{}, existem dois casos a serem tratados: O primeiro é quando a nova aresta inserida no grafo plano ponderado dinâmico~$G$ liga duas componentes conexas distintas, isto é, a nova aresta é uma ponte.
Nesse caso, as faces exteriores das duas componentes conexas são unidas em uma única face exterior.
O segundo caso é quando a nova aresta liga dois vértices de uma mesma componente conexa.
Nesse caso, a rotina \MSFaddEdge{} assume que as arestas~$f$ e $g$ passadas como parâmetros compartilham uma face~$F_{fg}$, pois a inserção de uma aresta que não obedece essa hipótese não estaria de acordo com a imersão corrente do grafo~$G$.
Nesse caso a face~$F_{fg}$ é dividida em duas faces e será elaborado adiante como $F_{fg}$ é encontrada.

\begin{algorithm}[htb]
\caption{\MSFaddEdge($G$, $e$, $u$, $f$, $v$, $g$, $w$)}
\label{Algo:MSFaddEdge}
\begin{algorithmic}[1]

\State $f_0$, $f_1$, $f_2$, $f_3$, $u_0$, $u_1$, $u_2$, $u_3$ $\gets$ $G.H(f)$ \label{Algo:MSFaddEdge:linha:1}
\State $g_0$, $g_1$, $g_2$, $g_3$, $v_0$, $v_1$, $v_2$, $v_3$ $\gets$ $G.H(g)$ \label{Algo:MSFaddEdge:linha:2}

\If {\LCOFindNode($u_0$) $=$ $u$} $i$ $\gets$ $0$ \textbf{senão} $i$ $\gets$ $2$
\EndIf
\If {\LCOFindNode($v_0$) $=$ $v$} $j$ $\gets$ $0$ \textbf{senão} $j$ $\gets$ $2$
\EndIf


\State $G.H(e)$ $\gets$ $e_0$, $e_1$, $e_2$, $e_3$, $s_0$, $s_1$, $s_2$, $s_3$ $\gets$ \LCOMakeOcto($e$, $w$) \label{Algo:MSFaddEdge:linha:const3}

\State \LCOCycle($u_i$); \LCOMerge($u_i$, $s_0$, $u$)\label{Algo:MSFaddEdge:merges0}
\State \LCOCycle($v_j$); \LCOMerge($v_j$, $s_2$, $v$)\label{Algo:MSFaddEdge:merges2}


\State \LCOCycle(\treapPredecessor($u_{i-1}$)); \LCOMerge($u_{i-1}$, $s_3$, \LCOFindNode($u_{i-1}$))\label{Algo:MSFaddEdge:merges1}
\State \LCOCycle(\treapPredecessor($v_{j-1}$)); \LCOMerge($v_{j-1}$, $s_1$, \LCOFindNode($v_{j-1}$))\label{Algo:MSFaddEdge:merges3}

\If{\LCOConnected($u$, $v$)}\label{Algo:MSFaddEdge:linha:deteccao-caso}
\State \LCOSplit($u_{i-1}$, $F_u$, $F'_u$)\label{Algo:MSFaddEdge:linha:splitnaoponte}
\State \MSFupdate($G$, $e$, $w$)
\Else
\State \LCOMerge($u_{i+1}$, $v_{j+1}$, $F_{uv}$)\label{Algo:MSFaddEdge:linha:mergeponte}
\State \LCOMerge($e_0$, $e_2$, $\hat e$)
\EndIf
\end{algorithmic}
\end{algorithm}

Como exemplos, vamos inserir uma nova aresta~$h$ ligando os vértices~$v$ e~$y$ em dois grafos.
O primeiro grafo está ilustrado na Figura~\ref{fig:MSF-remove-ponte} e possui duas componentes que são ligadas por~$h$. 
A nova aresta é inserida após as arestas~$a$ e~$f$ nas ordens cíclicas de $v$ e~$y$, respectivamente, logo a chamada para essa rotina é \MSFaddEdge($G$, $h$, $v$, $a$, $y$, $f$, $7$).
O segundo grafo é o da Figura~\ref{fig:MSF-basico-1} e~$h$ é inserida após as aresta~$a$ e~$g$ nas ordens cíclicas, logo a chamada feita nesse exemplo é~\MSFaddEdge($G$, $h$, $v$, $a$, $y$, $g$, $7$).
A face comum que será dividida em duas é~$F_0$.
A Figura~\ref{fig:antes-depois-adiciona} ilustra os grafos obtidos por essas inserções.


\begin{figure}[htb]
\begin{subfigure}{0.45\textwidth}
\scalebox{1}{
\begin{tikzpicture}[line cap=round,line join=round,>=triangle 45,x=1cm,y=1cm]
\clip(-3,-1) rectangle (6,3);


\draw [line width=1pt] (0,0) to[in=-135,out=135,looseness=1]  (0,2); % v -- u
\draw [line width=1pt] (0,0) to[in=-45,out=45,looseness=1]  (0,2);   % v -- u
\draw [line width=1pt] (2,1) to  (0,0); % 4 -- 6
\draw [line width=1pt] (2,1) to  (0,2); % 4 -- 6
%\draw [line width=1pt] (2,1) to  (3,1); % 4 -- 6
\draw [line width=1pt] (3,1.05) to[out=45,in=-45,looseness=50] (3,.95); % 4 -- 4


\draw [line width=1pt] (1,2.5) to[out=-45,in=45,looseness=1] (1,1); % c: F0 -- F1
\draw [line width=1pt] (1,2.5) to[out=0,in=90,looseness=1] (2.3,1.5); % d: F0 -- F1
\draw [line width=1pt] (1,1) to[out=-55,in=-90,looseness=2] (2.3,1.5); % d: F0 -- F1


% laço F0 para F0
%\draw [line width=1pt] (1,2.5) to[out=-30,in=90,looseness=1] (2.5,1); % F0 -- F1
%\draw [line width=1pt] (2.5,1) to[out=-90,in=-90,looseness=2] (4.5,1); % F0 -- F1
%\draw [line width=1pt] (1,2.5) to[out=0,in=90,looseness=1] (4.5,1); % F0 -- F1


\draw [line width=1pt] (1,2.5) to[out=170,in=180,looseness=2.5] (0,1); % F0 -- F2
\draw [line width=1pt] (1,2.5) to[out=30,in=90,looseness=1] (3.5,1); % F0 -- F3
\draw [line width=1pt] (1,1) to (0,1); % F1 -- F2


\draw [line width=1pt] (1,2.5) to[out=70,in=90,looseness=.8] (4.7,1); % F0 -- F3
\draw [line width=1pt] (4.7,1) to[out=-90,in=-90,looseness=2] (2.5,1); % F0 -- F3
\draw [line width=1pt] (1,2.5) to[out=20,in=90,looseness=1.3] (2.5,1); % F0 -- F3

\draw [line width=1pt] (0,0) to[out=0,in=-90,looseness=1] (3,1); % F0 -- F1


\begin{scriptsize}
\draw [fill=black] (0,2) circle (1.5pt);
\draw (0,2) node[anchor=south east] {$u$};


\draw [fill=black] (0,0) circle (1.5pt);
\draw (0,0) node[anchor=north east] {$v$};
\draw [fill=black] (2,1) circle (1.5pt);
\draw (1.95,1) node[anchor=south] {$z$};
\draw [fill=black] (3,1) circle (1.5pt);
\draw (2.95,1) node[anchor=south] {$y$};


\draw [fill=black] (1,2.5) circle (1.5pt);
\draw (1,2.5) node[anchor=south east] {$F_4$};
\draw [fill=black] (1,1) circle (1.5pt);
\draw (1,1) node[anchor=west] {$F_1$};
\draw [fill=black] (0,1) circle (1.5pt);
\draw (0,1) node[anchor=south] {$F_2$};
\draw [fill=black] (3.5,1) circle (1.5pt);
\draw (3.5,1) node[anchor=west] {$F_3$};
%\draw [fill=black] (3.5,2) circle (1.5pt);
%\draw (3.5,2) node[anchor=west] {$F_5$};

\draw (-0.5,1.05) node[anchor=north] {$a$};
%\draw (-0.5,1.1) node[anchor=north] {$2$};

\draw (0.5,1.35) node[anchor=north] {$b$};
%\draw (0.5,1.05) node[anchor=north] {$7$};

\draw (1.15,1.7) node[anchor=north] {$c$}; % u -- w
%\draw (1.4,1.7) node[anchor=north] {$3$}; % u -- w

\draw (1.1,.6) node[anchor=north] {$d$}; % v -- w
%\draw (1.2,.65) node[anchor=north] {$1$}; % v -- w

%\draw (2.3,1.3) node[anchor=north] {$e$}; % v -- w
%\draw (2.3,.95) node[anchor=north] {$4$}; % v -- w

\draw (4.15,1) node {$f$}; % w -- w
%\draw (4.15,1) node {$2$}; % w -- w

\draw (3,.2) node {$h$}; % w -- w
\end{scriptsize}
\end{tikzpicture}

}
\caption{Após a adição da ponte~$h$.}
\label{fig:MSF-adiciona-ponte}
\end{subfigure}
\begin{subfigure}{0.45\textwidth}
\scalebox{1}{
\begin{tikzpicture}[line cap=round,line join=round,>=triangle 45,x=1cm,y=1cm]
\clip(-3,-1) rectangle (4.6,3);


\draw [line width=1pt] (0,0) to[in=-135,out=135,looseness=1]  (0,2); % v -- u
\draw [line width=1pt] (0,0) to[in=-45,out=45,looseness=1]  (0,2);   % v -- u
\draw [line width=1pt] (2,1) to  (0,0); % 4 -- 6
\draw [line width=1pt] (2,1) to  (0,2); % 4 -- 6
\draw [line width=1pt] (2,1) to  (3,1); % 4 -- 6
\draw [line width=1pt] (3,1.05) to[out=45,in=-45,looseness=50] (3,.95); % 4 -- 4


\draw [line width=1pt] (1,2.5) to[out=-45,in=45,looseness=1] (1,1); % c: F0 -- F1
%\draw [line width=1pt] (1,2.5) to[out=20,in=160,looseness=1] (4.2,2.5); % d: F0 -- F1
%\draw [line width=1pt] (4.2,2.5) to[out=-20,in=20,looseness=1] (4.2,-.5); % d: F0 -- F1
%\draw [line width=1pt] (1,1) to[out=-70,in=200,looseness=1] (4.2,-.5); % d: F0 -- F1


% laço F0 para F0
\draw [line width=1pt] (1,2.5) to[out=-30,in=90,looseness=1] (2.5,1); % F0 -- F1
%\draw [line width=1pt] (2.5,1) to[out=-90,in=-90,looseness=2] (4.5,1); % F0 -- F1
\draw [line width=1pt] (1,2.5) to[out=0,in=90,looseness=1] (4.5,1); % F0 -- F1
\draw [line width=1pt] (1.5,.5) to[out=-90,in=-90,looseness=1.5] (4.5,1); % F0 -- F1


% a
\draw [line width=1pt] (1,2.5) to[out=170,in=180,looseness=2.5] (0,1); % F0 -- F2
\draw [line width=1pt] (1,2.5) to[out=0,in=90,looseness=1] (3.5,1); % F0 -- F3
\draw [line width=1pt] (1,1) to (0,1); % F1 -- F2


\draw [line width=1pt] (0,0) to[out=0,in=220,looseness=1] (3,1); % F1 -- F2
\draw [line width=1pt] (2.5,1) to[out=-90,in=0,looseness=1] (1.5,.5); % F0 -- F1


\draw [line width=1pt] (1,1) to (1.5,.5); % F1 -- F2


\begin{scriptsize}
\draw [fill=black] (0,2) circle (1.5pt);
\draw (0,2) node[anchor=south east] {$u$};


\draw [fill=black] (0,0) circle (1.5pt);
\draw (0,0) node[anchor=north east] {$v$};
\draw [fill=black] (2,1) circle (1.5pt);
\draw (1.95,1) node[anchor=south] {$z$};
\draw [fill=black] (3,1) circle (1.5pt);
\draw (2.95,1) node[anchor=south] {$y$};


\draw [fill=black] (1,2.5) circle (1.5pt);
\draw (1,2.5) node[anchor=south east] {$F_0$};
\draw [fill=black] (1,1) circle (1.5pt);
\draw (1,1) node[anchor=west] {$F_1$};
\draw [fill=black] (0,1) circle (1.5pt);
\draw (0,1) node[anchor=south] {$F_2$};
\draw [fill=black] (3.5,1) circle (1.5pt);
\draw (3.5,1) node[anchor=west] {$F_3$};
\draw [fill=black] (1.5,.5) circle (1.5pt);
\draw (1.5,.65) node[anchor=west] {$F_4$};

\draw (-0.5,1.05) node[anchor=north] {$a$};
%\draw (-0.5,1.1) node[anchor=north] {$2$};

\draw (0.5,1.35) node[anchor=north] {$b$};
%\draw (0.5,1.05) node[anchor=north] {$7$};

\draw (1.15,1.7) node[anchor=north] {$c$}; % u -- w
%\draw (1.4,1.7) node[anchor=north] {$3$}; % u -- w

\draw (1.1,.6) node[anchor=north] {$d$}; % v -- w
%\draw (1.2,.65) node[anchor=north] {$1$}; % v -- w

\draw (2.35,1.35) node[anchor=north] {$g$}; % v -- w
%\draw (2.3,.95) node[anchor=north] {$4$}; % v -- w

\draw (4.15,1) node {$f$}; % w -- w
%\draw (4.15,1) node {$2$}; % w -- w

\draw (2.35,.5) node[anchor=north] {$h$}; % v -- w
\end{scriptsize}
\end{tikzpicture}

}
\caption{Após a adição de~$h$.}
\label{fig:MSF-adiciona-nao-ponte}
\end{subfigure}
\caption{Exemplos de adição de arestas.}
\label{fig:antes-depois-adiciona}
\end{figure}

Note que adicionar uma aresta~$e$ a~$G$ como sucessora de~$f$ e~$g$ implica adicionar $e^\star$ a $G^\star$ como predecessora de~$f^\star$ e $g^\star$ na ordem cíclica das faces apropriadas.  
Na Figura~\ref{fig:MSF-adiciona-nao-ponte} é possível ver que $h^\star$ é predecessora de~$a^\star$ e~$g^\star$ na ordem cíclica de~$F_0$ e~$F_4$, respectivamente.


O processamento da rotina \MSFaddEdge{} começa obtendo as óctuplas de~$f$ e~$g$ e comparando o vértice~$u$ com o identificador retornado por \LCOFindNode($u_{0}$).
Esse teste é feito para descobrir qual nó, $u_0$ ou~$u_2$, pertence à ordem cíclica de~$u$.
O índice ($0$ ou $2$) é salvo em uma variável~$i$ para ser usado futuramente.
Esse mesmo processo é feito com $v$.
Então é criado uma nova óctupla com \LCOMakeOcto{}, que é guardada na tabela de símbolos~$H$ com chave associada~$e$.



As linhas~\ref{Algo:MSFaddEdge:merges0} e~\ref{Algo:MSFaddEdge:merges2} executam \LCOCycle{} e~\LCOMerge{} para juntar os nós~$s_0$ e~$s_2$ às ordens cíclicas dos vértices~$u$ e~$v$ de~$\hat T$.  
Ao executar \LCOCycle($u_i$), a ordem cíclica de $u$ obtém forma~$\alpha u_i$, assim a operação \LCOMerge($u_i$, $s_0$, $u$) resulta na ordem cíclica $\alpha u_is_0$.
Analogamente, \LCOCycle($v_j$) seguido de \LCOMerge($v_j$, $s_2$, $v$) resulta na ordem cíclica $\beta v_j s_2$ para o vértice~$v$.
Isto é, a nova aresta é inserida após $f$ e $g$ nas ordens cíclicas.

As linhas~\ref{Algo:MSFaddEdge:merges1} e~\ref{Algo:MSFaddEdge:merges3} fazem o mesmo para os nós~$s_1$ e~$s_3$ nas ordens cíclicas dos vértices de~$\hat T^\star$.
Note que, como a nova aresta~$e$ é inserida após $f$ e $g$ em suas ordens cíclicas, então~$e^\star$ é inserida nas faces obtidas por \LCOFindNode($u_{i-1}$) e \LCOFindNode($v_{j-1}$).
A única diferença é que esses nós são inseridos como predecessores de $u_{i-1}$ e $v_{j-1}$.
Para tal, a rotina \LCOCycle{} é chamada para o predecessor de $u_{i-1}$ obtido com a rotina \treapPredecessor.

Em seguida, o algoritmo acerta a estrutura da face com base nos dois casos comentados anteriormente.
A detecção de qual caso a aresta~$e$ se encontra é feita na linha~\ref{Algo:MSFaddEdge:linha:deteccao-caso} com um teste de conexidade entre os vértices~$u$ e~$v$.
Se os dois vértices estiverem conectados, então é necessário aplicar \LCOSplit{} para dividir a face \LCOFindNode($u_{i-1}$) em duas e o peso de~$e$ é atualizado com \MSFupdate($G$, $e$, $w$), que também determina se a nova aresta~$e$ será uma aresta de $\hat T$ ou de~$\hat T^\star$.
Caso contrário, é necessário aplicar um \LCOMerge{} para juntar as faces exteriores das duas componentes conexas e como a aresta é uma ponte, então necessariamente $\hat e$ é um vértice de $\hat T$, logo é feito um \LCOMerge($e_0$, $e_2$, $\hat e$).

Simularemos primeiro a inserção da aresta~$h$ no caso em que ela é uma ponte, isto é, a chamada \MSFaddEdge($G$, $h$, $v$, $a$, $y$, $f$, $7$) comentada anteriormente.
A Figura~\ref{fig:MSF-adiciona-ponte-1} ilustra as ADPs das duas componentes conexas do grafo da Figura~\ref{fig:MSF-remove-ponte}.

\begin{figure}
\scalebox{1}{
\begin{tikzpicture}[line cap=round,line join=round,>=triangle 45,x=1cm,y=1cm]
\clip(-2.8,-.3) rectangle (7.3,5.6);

\begin{scriptsize}

%
% Node v
%
\draw (0,0) node {$\hat v$};
\draw [line width=1pt,color=cyan] (0,.5) to  (.35,.35); 
\draw [line width=1pt,color=cyan] (.35,.35) to  (.5,0); 

\draw [line width=1pt,color=cyan] (0,.5) to  (-1,1.8); 
\draw [line width=1pt,color=cyan] (.35,.35) to  (1.05,1.75); 
\draw [line width=1pt,color=cyan] (.5,0) to  (2.5,1); 


\draw (0,.5)[anchor=east] node {\tiny $a_v$};
\draw (.35,.35)[anchor=west] node {\tiny $b_v$};
\draw (.5,0)[anchor=north west] node {\tiny $d_v$};
\draw [fill=black] (0,.5) circle (1.5pt);
\draw [fill=black] (.35,.35) circle (1.5pt);
\draw [fill=black] (.5,0) circle (1.5pt);

\draw (-.6,1.9) node[anchor=north east] {\tiny $a_0$};
\draw [fill=white] (-1,1.8) circle (1.5pt);

% ^b
\draw [line width=1pt,color=cyan] (2.5,1) to (2.9,1.15); 
\draw (1,1.65)[anchor=west] node {\tiny $b_2$};
\draw [fill=white] (1.05,1.75) circle (1.5pt);
%\draw (2.5,1)[anchor=north] node {\tiny $d_0$};
\draw [fill=white] (2.5,1) circle (1.5pt);

%
% Node u
%
\draw (0,4) node {$\hat u$};

\draw [line width=1pt,color=cyan] (0,3.5) to  (.3,3.7); 
\draw [line width=1pt,color=cyan] (.5,4) to  (.3,3.7); 

\draw [line width=1pt,color=cyan] (0,3.5) to  (-1,2.2); 
\draw [line width=1pt,color=cyan] (.3,3.7) to  (1.05,2.25); 
\draw [line width=1pt,color=cyan] (.5,4) to  (2.5,3); 

\draw (-1.1,2.35) node {\tiny $a_2$};
\draw [fill=white] (-1,2.2) circle (1.5pt);
\draw (1.2,2.33) node {\tiny $b_0$};
\draw [fill=white] (1.05,2.25) circle (1.5pt);

% u node path
\draw (-.2,3.5) node {\tiny $a_u$};
\draw [fill=black] (0,3.5) circle (1.5pt);
\draw (.5,4.15) node {\tiny $c_u$};
\draw [fill=black] (.5,4) circle (1.5pt);
\draw (.19,3.82) node {\tiny $b_u$};
\draw [fill=black] (.3,3.7) circle (1.5pt);

%
% Node z
%
\draw (3.8,2) node {$\hat z$};
%\draw [line width=1pt,color=cyan] (4.3,2) to (6.5,2); 
\draw [line width=1pt,color=cyan] (3.7,2.3) to (2.5,3); 
\draw [line width=1pt,color=cyan] (3.7,1.7) to (2.9,1.15); 
%\draw (2.4,2.9) node {\tiny $c_2$};
\draw [fill=white] (2.5,3) circle (1.5pt);


% node path
\draw [line width=1pt,color=cyan] (3.7,2.3) to  (3.7,1.7); 
%\draw [line width=1pt,color=cyan] (3.7,1.7) to  (4.3,2); 


\draw (3.75,1.55) node {\tiny $d_z$};
\draw [fill=black] (3.7,1.7)circle (1.5pt);
\draw (3.7,2.43) node {\tiny $c_z$};
\draw [fill=black] (3.7,2.3)circle (1.5pt);
%\draw (4.3,2.2) node {\tiny $g_z$};
%\draw [fill=black] (4.3,2)circle (1.5pt);

%\draw (3.2,2.7) node {\tiny $c_0$};
\draw [fill=white] (3,2.7)circle (1.5pt);
%\draw (3.05,1.25)[anchor=north] node {\tiny $d_2$};
\draw [fill=white] (2.9,1.15)circle (1.5pt);

%eF0 -- e3
%\draw [line width=1pt,color=red] (3.3,4.7) to  (5.25,2.25);


%
% Node F_0
%
\draw (3,5.2) node {$\hat F_0$};
%\draw [line width=1pt,color=red] (2.7,4.7) to  (3.8,4.7); 
\draw [line width=1pt,color=red] (2.5,5) to  (2.7,4.7); 
\draw [line width=1pt,color=red] (3.5,5) to  (2.5,5); 

\draw [line width=1pt,color=red] (2.5,5) to[in=180,out=180,looseness=2]  (-1.2,2); 
%\draw [line width=1pt,color=red] (4,5) to[in=-20,out=10,looseness=3]  (5.25,1.75); 
\draw [line width=1pt,color=red] (3.5,5) to[in=-30,out=20,looseness=2]  (2.8,.85); 

%^c
\draw [line width=1pt,color=red] (2.7,4.7) to  (2.9,3.1);

%\draw (3.07,3.1) node {\tiny $c_1$};
\draw [fill=white] (2.9,3.1)circle (1.5pt);
%\draw (2.9,.7) node {\tiny $d_1$};
\draw [fill=white] (2.8,.85)circle (1.5pt);


%fF3 -- 
\draw [line width=1pt,color=red] (6.5,3.9) to  (6.5,3);

% ^f
\draw [line width=1pt,color=cyan] (6.8,2.5) to[in=-60,out=60,looseness=1.5] (6.65,3.8);
\draw [line width=1pt,color=cyan] (6.2,2.5) to[in=-120,out=120,looseness=1.5] (6.35,3.8);

\draw [line width=1pt,color=red] (6.5,4.7) to  (6.5,3.95);
\draw (6.65,4.05) node {\tiny $f_1$};
\draw [fill=white] (6.5,3.95)circle (1.5pt);
\draw (6.65,3.55) node {\tiny $f_3$};
\draw [fill=white] (6.5,3.65)circle (1.5pt);
\draw (6.8,3.8) node {\tiny $f_0$};
\draw [fill=white] (6.65,3.8)circle (1.5pt);
\draw (6.2,3.8) node {\tiny $f_2$};
\draw [fill=white] (6.35,3.8)circle (1.5pt);


\draw (2.5,5)[anchor=south] node {\tiny $a_{F_0}$};
\draw [fill=black] (2.5,5)circle (1.5pt);
\draw (2.5,4.6) node {\tiny $c_{F_0}$};
\draw [fill=black] (2.7,4.7)circle (1.5pt);
\draw (3.5,5)[anchor=south] node {\tiny $d_{F_0}$};
\draw [fill=black] (3.5,5)circle (1.5pt);
%\draw (4,5)[anchor=north west] node {\tiny $g_{F_0}$};
%\draw [fill=black] (4,5)circle (1.5pt);
\draw (6.5,4.7)[anchor=south] node {\tiny $f_{F_4}$};
\draw [fill=black] (6.5,4.7)circle (1.5pt);
%\draw (3.3,4.7)[anchor=north] node {\tiny $g_{F_0}$};
%\draw [fill=black] (3.3,4.7)circle (1.5pt);

%
% Node y
%
\draw (6.5,2.3) node {$\hat y$};

%node path
\draw [line width=1pt,color=cyan] (6.8,2.5) to  (6.2,2.5); 
%\draw [line width=1pt,color=cyan] (6.5,2) to  (6.8,2.5); 


%\draw (6.5,1.8) node {\tiny $g_y$};
\draw (6,2.5) node {\tiny $f_y$};
\draw (7,2.5) node {\tiny $f_y$};
%\draw [fill=black] (6.5,2)circle (1.5pt);
\draw [fill=black] (6.2,2.5)circle (1.5pt);
\draw [fill=black] (6.8,2.5)circle (1.5pt);


%
% Node F_1
%
%\draw (2.1,2) node {$\hat F_1$};
%node path
\draw [line width=1pt,color=red] (2.3,2.3) to  (2.3,1.7); 
\draw [line width=1pt,color=red] (1.7,2) to  (2.3,1.7); 

\draw [line width=1pt,color=red] (2.6,2.6) to  (2.3,2.3); 
\draw [line width=1pt,color=red] (2.6,1.3) to  (2.3,1.7); 
\draw [line width=1pt,color=red] (1.7,2) to  (1.3,2); 

% F_1 to F_2
\draw [line width=1pt,color=red] (.8,2) to  (1.3,2); 

\draw (1.8,2.13) node {\tiny $b_{F_1}$};
\draw [fill=black] (1.7,2)circle (1.5pt);
%\draw (2.3,1.7)[anchor=north] node {\tiny $d_{F_1}$};
\draw [fill=black] (2.3,1.7)circle (1.5pt);
%\draw (2.25,2.2)[anchor=west] node {\tiny $c_{F_1}$};
\draw [fill=black] (2.3,2.3)circle (1.5pt);
\draw (1.4,1.85) node {\tiny $b_3$};
\draw [fill=white] (1.3,2)circle (1.5pt);

%\draw (2.8,1.4) node {\tiny $d_3$};
\draw [fill=white] (2.6,1.3)circle (1.5pt);
%\draw (2.75,2.55) node {\tiny $c_3$};
\draw [fill=white] (2.6,2.6)circle (1.5pt);

%
% Node F_2
%
\draw (0,2.2) node {$\hat F_2$};

\draw [line width=1pt,color=red] (-1.2,2) to  (-.3,2); 
\draw [line width=1pt,color=red] (.8,2) to  (.3,2);   
\draw [line width=1pt,color=red] (-.3,2) to (.3,2);   

\draw [fill=black] (-.2,2) circle (1.5pt);
\draw (-.2,1.8) node {\tiny $a_{F_2}$};
\draw [fill=black] (.2,2) circle (1.5pt);
\draw (.3,1.8) node {\tiny $b_{F_2}$};
\draw (-.75,2.1) node {\tiny $a_1$};
\draw [fill=white] (-.8,2) circle (1.5pt);
\draw (.8,2.16) node {\tiny $b_1$};
\draw [fill=white] (.8,2) circle (1.5pt);

\draw (-1.3,1.9) node {\tiny $a_3$};
\draw [fill=white] (-1.2,2)circle (1.5pt);

%
% Node F_3
%
%\draw (6.7,3.2) node {$\hat F_3$};
\draw (6.3,3) node {\tiny $f_{F_3}$};
\draw [fill=black] (6.5,3)circle (1.5pt);

%\draw (5,2)[anchor=south] node {\tiny $g_0$};
%\draw [fill=white] (5,2)circle (1.5pt);
%\draw (5.5,2)[anchor=south] node {\tiny $g_2$};
%\draw [fill=white] (5.5,2)circle (1.5pt);
%\draw (5.25,2.25)[anchor=south] node {\tiny $g_3$};
%\draw [fill=white] (5.25,2.25)circle (1.5pt);
%\draw (5.25,1.75)[anchor=north] node {\tiny $g_1$};
%\draw [fill=white] (5.25,1.75)circle (1.5pt);

\end{scriptsize}
\end{tikzpicture}

}
\caption{ADPs antes da adição de~$h$.}
\label{fig:MSF-adiciona-ponte-1}
\end{figure}

Após obter as óctuplas de~$a$ e~$f$, o algoritmo compara~$v$ com o identificador retornado por \LCOFindNode($a_v$), que também é~$v$ e assim é atribuído o valor~$0$ à variável~$i$.
O algoritmo faz o mesmo processo com $y$ e atribui $0$ à~$j$.
Em seguida é criada uma nova óctupla associada a~$h$ e os nós $s_i$ são adicionados nas respectivas ordens cíclicas.

Primeiro é feito \LCOCycle($a_v$) e \LCOMerge($a_v$, $h_0$, $v$).
A rotina \LCOCycle($a_v$) não modifica a estrutura dessa ordem cíclica, já que o nó~$a_v$ já é o último nó de sua ordem cíclica.
Por sua vez, a rotina \LCOMerge($a_v$, $s_0$, $v$) resulta na ordem cíclica de nós $\langle d_v, b_v, a_v, s_0 \rangle$.

Depois é feito \LCOCycle($f_y$) e \LCOMerge($f_y$, $h_y$, $y$).
A rotina \LCOCycle($f_y$) não modifica a estrutura dessa ordem cíclica, já que o nó~$f_y$ já é o último nó de sua ordem cíclica.
A rotina \nolbreaks{\LCOMerge($f_y$, $h_y$, $y$)} por sua vez resulta na ordem cíclica de nós $\langle f_y, f_y, h_y \rangle$.

Como $i=0$, temos que~$i-1 = 3$ módulo~$4$.
Assim o nó~$u_{i-1}$ em que é calculado o predecessor na linha~\ref{Algo:MSFaddEdge:merges1} do Algoritmo~\ref{Algo:MSFaddEdge} é o nó~$a_{F_0}$, pois esse está ligado ao nó~$a_3$.
O predecessor de $a_{F_0}$ em sua ordem cíclica é~$d_{F_0}$.
Assim \LCOCycle(\treapPredecessor($a_{F_0}$)) reestrutura a ordem cíclica de~$\hat F_0$ para ser~$\langle a_{F_0}, c_{F_0}, d_{F_0}\rangle$ e \LCOMerge($a_{F_0}$, $h_{F_0}$, $F_0$) resulta na ordem $\langle a_{F_0}, c_{F_0}, d_{F_0}, h_{F_0}\rangle$.

Como $j=0$, temos que~$j-1 = 3$ módulo~$4$.
Assim o nó~$v_{i-1}$ em que é calculado o predecessor na linha~\ref{Algo:MSFaddEdge:merges3} do Algoritmo~\ref{Algo:MSFaddEdge} é o nó~$f_{F_3}$, pois esse está ligado ao nó~$f_3$.
O nó $f_{F_3}$ é o próprio predecessor.
Assim \LCOCycle(\treapPredecessor($f_{F_3}$)) não altera a estrutura da ordem cíclica de~$\hat F_3$ e \LCOMerge($f_{F_3}$, $h_{F_3}$, $F_3$) resulta na ordem $\langle  f_{F_3}, h_{F_3}\rangle$.
A Figura~\ref{fig:MSF-adiciona-ponte-2} ilustra as ADPs após esse bloco de execuções de \LCOCycle{} e \LCOMerge.

\begin{figure}[htb]
\scalebox{1}{
\begin{tikzpicture}[line cap=round,line join=round,>=triangle 45,x=1cm,y=1cm]
\clip(-2.8,-1) rectangle (8,5.6);

\begin{scriptsize}

%
% Node v
%
\draw (0,0) node {$\hat v$};
\draw [line width=1pt,color=cyan] (0,.5) to  (.35,.35); 
\draw [line width=1pt,color=cyan] (.35,.35) to  (.5,0); 
\draw [line width=1pt,color=cyan] (0,.5) to  (.5,-.5); 

\draw [line width=1pt,color=cyan] (0,.5) to  (-1,1.8); 
\draw [line width=1pt,color=cyan] (.35,.35) to  (1.05,1.75); 
\draw [line width=1pt,color=cyan] (.5,0) to  (2.5,1); 
\draw [line width=1pt,color=cyan] (.5,-.5) to  (2,-.5); 


\draw (0,.5)[anchor=east] node {\tiny $a_v$};
\draw (.35,.35)[anchor=west] node {\tiny $b_v$};
\draw (.5,0)[anchor=north west] node {\tiny $d_v$};
\draw (.5,-.5)[anchor=east] node {\tiny $h_v$};
\draw [fill=black] (0,.5) circle (1.5pt);
\draw [fill=black] (.35,.35) circle (1.5pt);
\draw [fill=black] (.5,0) circle (1.5pt);
\draw [fill=black] (.5,-.5) circle (1.5pt);

\draw (-.6,1.9) node[anchor=north east] {\tiny $a_0$};
\draw [fill=white] (-1,1.8) circle (1.5pt);

\draw (2,-.5)[anchor=south] node {\tiny $h_0$};
\draw (2,-.5)[fill=white] circle (1.5pt);
% ^b
\draw [line width=1pt,color=cyan] (2.5,1) to (2.9,1.15); 
\draw (1,1.65)[anchor=west] node {\tiny $b_2$};
\draw [fill=white] (1.05,1.75) circle (1.5pt);
%\draw (2.5,1)[anchor=north] node {\tiny $d_0$};
\draw [fill=white] (2.5,1) circle (1.5pt);

%
% Node u
%
\draw (0,4) node {$\hat u$};

\draw [line width=1pt,color=cyan] (0,3.5) to  (.3,3.7); 
\draw [line width=1pt,color=cyan] (.5,4) to  (.3,3.7); 

\draw [line width=1pt,color=cyan] (0,3.5) to  (-1,2.2); 
\draw [line width=1pt,color=cyan] (.3,3.7) to  (1.05,2.25); 
\draw [line width=1pt,color=cyan] (.5,4) to  (2.5,3); 

\draw (-1.1,2.35) node {\tiny $a_2$};
\draw [fill=white] (-1,2.2) circle (1.5pt);
\draw (1.2,2.33) node {\tiny $b_0$};
\draw [fill=white] (1.05,2.25) circle (1.5pt);

% u node path
\draw (-.2,3.5) node {\tiny $a_u$};
\draw [fill=black] (0,3.5) circle (1.5pt);
\draw (.5,4.15) node {\tiny $c_u$};
\draw [fill=black] (.5,4) circle (1.5pt);
\draw (.19,3.82) node {\tiny $b_u$};
\draw [fill=black] (.3,3.7) circle (1.5pt);

%
% Node z
%
\draw (3.8,2) node {$\hat z$};
%\draw [line width=1pt,color=cyan] (4.3,2) to (6.5,2); 
\draw [line width=1pt,color=cyan] (3.7,2.3) to (2.5,3); 
\draw [line width=1pt,color=cyan] (3.7,1.7) to (2.9,1.15); 
%\draw (2.4,2.9) node {\tiny $c_2$};
\draw [fill=white] (2.5,3) circle (1.5pt);


% node path
\draw [line width=1pt,color=cyan] (3.7,2.3) to  (3.7,1.7); 
%\draw [line width=1pt,color=cyan] (3.7,1.7) to  (4.3,2); 


\draw (3.75,1.55) node {\tiny $d_z$};
\draw [fill=black] (3.7,1.7)circle (1.5pt);
\draw (3.7,2.43) node {\tiny $c_z$};
\draw [fill=black] (3.7,2.3)circle (1.5pt);
%\draw (4.3,2.2) node {\tiny $g_z$};
%\draw [fill=black] (4.3,2)circle (1.5pt);

%\draw (3.2,2.7) node {\tiny $c_0$};
\draw [fill=white] (3,2.7)circle (1.5pt);
%\draw (3.05,1.25)[anchor=north] node {\tiny $d_2$};
\draw [fill=white] (2.9,1.15)circle (1.5pt);

%eF0 -- e3
%\draw [line width=1pt,color=red] (3.3,4.7) to  (5.25,2.25);


%
% Node F_0
%
\draw (3,5.2) node {$\hat F_0$};
%\draw [line width=1pt,color=red] (2.7,4.7) to  (3.8,4.7); 
\draw [line width=1pt,color=red] (2.5,5) to  (2.7,4.7); 
\draw [line width=1pt,color=red] (3.5,5) to  (2.7,4.7); 

\draw [line width=1pt,color=red] (2.5,5) to[in=180,out=180,looseness=2]  (-1.2,2); 
%\draw [line width=1pt,color=red] (4,5) to[in=-20,out=10,looseness=3]  (5.25,1.75); 
\draw [line width=1pt,color=red] (3.5,5) to[in=-30,out=20,looseness=2]  (2.8,.85); 

%^c
\draw [line width=1pt,color=red] (2.7,4.7) to  (2.9,3.1);

%\draw (3.07,3.1) node {\tiny $c_1$};
\draw [fill=white] (2.9,3.1)circle (1.5pt);
%\draw (2.9,.7) node {\tiny $d_1$};
\draw [fill=white] (2.8,.85)circle (1.5pt);


%fF3 -- 
\draw [line width=1pt,color=red] (6.5,4.7) to  (7,4.7);
\draw [line width=1pt,color=red] (6.5,3.9) to  (6.5,3);
\draw [line width=1pt,color=red] (3.5,5.5) to  (4.5,5.5);

% ^f
\draw [line width=1pt,color=cyan] (6.8,2.5) to[in=-60,out=60,looseness=1.5] (6.65,3.8);
\draw [line width=1pt,color=cyan] (6.2,2.5) to[in=-120,out=120,looseness=1.5] (6.35,3.8);
\draw [line width=1pt,color=red] (6.5,4.7) to  (6.5,3.95);
\draw [line width=1pt,color=red] (3.5,5) to  (3.5,5.5);
\draw [line width=1pt,color=red] (7.5,4) to  (7,4.7);

\draw (6.65,4.05) node {\tiny $f_3$};
\draw [fill=white] (6.5,3.95)circle (1.5pt);
\draw (6.65,3.55) node {\tiny $f_1$};
\draw [fill=white] (6.5,3.65)circle (1.5pt);
\draw (6.8,3.8) node {\tiny $f_2$};
\draw [fill=white] (6.65,3.8)circle (1.5pt);
\draw (6.2,3.8) node {\tiny $f_0$};
\draw [fill=white] (6.35,3.8)circle (1.5pt);
\draw (4.7,5.5) node {\tiny $h_3$};
\draw [fill=white] (4.5,5.5)circle (1.5pt);



\draw (2.5,5)[anchor=south] node {\tiny $a_{F_0}$};
\draw [fill=black] (2.5,5)circle (1.5pt);
\draw (2.5,4.5) node {\tiny $c_{F_0}$};
\draw [fill=black] (2.7,4.7)circle (1.5pt);
\draw (3.5,5)[anchor=north] node {\tiny $d_{F_0}$};
\draw [fill=black] (3.5,5)circle (1.5pt);
\draw (3.5,5.5)[anchor=east] node {\tiny $h_{F_0}$};
\draw [fill=black] (3.5,5.5)circle (1.5pt);
\draw (6.5,4.7)[anchor=south] node {\tiny $f_{F_4}$};
\draw [fill=black] (6.5,4.7)circle (1.5pt);
%\draw (3.3,4.7)[anchor=north] node {\tiny $g_{F_0}$};
%\draw [fill=black] (3.3,4.7)circle (1.5pt);
\draw (7,4.7)[anchor=south] node {\tiny $h_{F_4}$};
\draw [fill=black] (7,4.7)circle (1.5pt);
\draw (7.5,4)[anchor=north] node {\tiny $h_1$};
\draw [fill=white] (7.5,4)circle (1.5pt);

%
% Node y
%
\draw (6.5,2.3) node {$\hat y$};

%node path
\draw [line width=1pt,color=cyan] (6.8,2.5) to  (6.2,2.5); 
\draw [line width=1pt,color=cyan] (6.5,2) to  (6.2,2.5); 

\draw [line width=1pt,color=cyan] (6.5,2) to  (6.5,1); 

\draw (6.7,2) node {\tiny $h_y$};
\draw (6,2.5) node {\tiny $f_y$};
\draw (7,2.5) node {\tiny $f_y$};
\draw [fill=black] (6.5,2)circle (1.5pt);
\draw [fill=black] (6.2,2.5)circle (1.5pt);
\draw [fill=black] (6.8,2.5)circle (1.5pt);


\draw (6.5,.8) node {\tiny $h_2$};
\draw [fill=white] (6.5,1)circle (1.5pt);
%
% Node F_1
%
%\draw (2.1,2) node {$\hat F_1$};
%node path
\draw [line width=1pt,color=red] (2.3,2.3) to  (2.3,1.7); 
\draw [line width=1pt,color=red] (1.7,2) to  (2.3,1.7); 

\draw [line width=1pt,color=red] (2.6,2.6) to  (2.3,2.3); 
\draw [line width=1pt,color=red] (2.6,1.3) to  (2.3,1.7); 
\draw [line width=1pt,color=red] (1.7,2) to  (1.3,2); 

% F_1 to F_2
\draw [line width=1pt,color=red] (.8,2) to  (1.3,2); 

\draw (1.8,2.13) node {\tiny $b_{F_1}$};
\draw [fill=black] (1.7,2)circle (1.5pt);
%\draw (2.3,1.7)[anchor=north] node {\tiny $d_{F_1}$};
\draw [fill=black] (2.3,1.7)circle (1.5pt);
%\draw (2.25,2.2)[anchor=west] node {\tiny $c_{F_1}$};
\draw [fill=black] (2.3,2.3)circle (1.5pt);
\draw (1.4,1.85) node {\tiny $b_3$};
\draw [fill=white] (1.3,2)circle (1.5pt);

%\draw (2.8,1.4) node {\tiny $d_3$};
\draw [fill=white] (2.6,1.3)circle (1.5pt);
%\draw (2.75,2.55) node {\tiny $c_3$};
\draw [fill=white] (2.6,2.6)circle (1.5pt);

%
% Node F_2
%
\draw (0,2.2) node {$\hat F_2$};

\draw [line width=1pt,color=red] (-1.2,2) to  (-.3,2); 
\draw [line width=1pt,color=red] (.8,2) to  (.3,2);   
\draw [line width=1pt,color=red] (-.3,2) to (.3,2);   

\draw [fill=black] (-.2,2) circle (1.5pt);
\draw (-.2,1.8) node {\tiny $a_{F_2}$};
\draw [fill=black] (.2,2) circle (1.5pt);
\draw (.3,1.8) node {\tiny $b_{F_2}$};
\draw (-.75,2.1) node {\tiny $a_1$};
\draw [fill=white] (-.8,2) circle (1.5pt);
\draw (.8,2.16) node {\tiny $b_1$};
\draw [fill=white] (.8,2) circle (1.5pt);

\draw (-1.3,1.9) node {\tiny $a_3$};
\draw [fill=white] (-1.2,2)circle (1.5pt);

%
% Node F_3
%
%\draw (6.7,3.2) node {$\hat F_3$};
\draw (6.3,3) node {\tiny $f_{F_3}$};
\draw [fill=black] (6.5,3)circle (1.5pt);

%\draw (5,2)[anchor=south] node {\tiny $g_0$};
%\draw [fill=white] (5,2)circle (1.5pt);
%\draw (5.5,2)[anchor=south] node {\tiny $g_2$};
%\draw [fill=white] (5.5,2)circle (1.5pt);
%\draw (5.25,2.25)[anchor=south] node {\tiny $g_3$};
%\draw [fill=white] (5.25,2.25)circle (1.5pt);
%\draw (5.25,1.75)[anchor=north] node {\tiny $g_1$};
%\draw [fill=white] (5.25,1.75)circle (1.5pt);

\draw [fill=white] (-1.2,2)circle (1.5pt);

\end{scriptsize}
\end{tikzpicture}

	}
\caption{ADPs após adição da óctupla de~$h$.}
\label{fig:MSF-adiciona-ponte-2}
\end{figure}

Em seguida é feito um teste de conexidade entre $v$ e $y$ que retorna falso, assim identificando que a nova aresta é uma ponte.
Assim é feito um \nolbreaks{\LCOMerge($a_{F_0}$, $f_{F_3}$, $F_0$)}, correspondente à linha~\ref{Algo:MSFaddEdge:linha:mergeponte} do Algoritmo~\ref{Algo:MSFaddEdge},
obtendo assim a ordem cíclica $\langle a_{F_0}, c_{F_0}, d_{F_0}, h_{F_0}, f_{F_4}, h_{F_4}\rangle$ correspondente à nova face exterior do grafo.
Por fim, é feito \LCOMerge($h_0$, $h_2$, $\hat h$).
A Figura~\ref{fig:MSF-adiciona-ponte-3} ilustra o grafo resultante dessa rotina.


\begin{figure}[htb]
\scalebox{1}{
\begin{tikzpicture}[line cap=round,line join=round,>=triangle 45,x=1cm,y=1cm]
\clip(-2.8,-1) rectangle (8,5.6);

\begin{scriptsize}

%
% Node v
%
\draw (0,0) node {$\hat v$};
\draw [line width=1pt,color=cyan] (0,.5) to  (.35,.35); 
\draw [line width=1pt,color=cyan] (.35,.35) to  (.5,0); 
\draw [line width=1pt,color=cyan] (0,.5) to  (.5,-.5); 

\draw [line width=1pt,color=cyan] (0,.5) to  (-1,1.8); 
\draw [line width=1pt,color=cyan] (.35,.35) to  (1.05,1.75); 
\draw [line width=1pt,color=cyan] (.5,0) to  (2.5,1); 
\draw [line width=1pt,color=cyan] (.5,-.5) to  (5,.5); 


\draw (0,.5)[anchor=east] node {\tiny $a_v$};
\draw (.35,.35)[anchor=west] node {\tiny $b_v$};
\draw (.5,0)[anchor=north west] node {\tiny $d_v$};
\draw (.5,-.5)[anchor=east] node {\tiny $h_v$};
\draw [fill=black] (0,.5) circle (1.5pt);
\draw [fill=black] (.35,.35) circle (1.5pt);
\draw [fill=black] (.5,0) circle (1.5pt);
\draw [fill=black] (.5,-.5) circle (1.5pt);

\draw (-.6,1.9) node[anchor=north east] {\tiny $a_0$};
\draw [fill=white] (-1,1.8) circle (1.5pt);

%h 
\draw [line width=1pt,color=cyan] (5.5,1) to  (5,.5); 
\draw (5,.5)[anchor=south] node {\tiny $h_0$};
\draw (5,.5)[fill=white] circle (1.5pt);
% ^b
\draw [line width=1pt,color=cyan] (2.5,1) to (2.9,1.15); 
\draw (1,1.65)[anchor=west] node {\tiny $b_2$};
\draw [fill=white] (1.05,1.75) circle (1.5pt);
%\draw (2.5,1)[anchor=north] node {\tiny $d_0$};
\draw [fill=white] (2.5,1) circle (1.5pt);

%
% Node u
%
\draw (0,4) node {$\hat u$};

\draw [line width=1pt,color=cyan] (0,3.5) to  (.3,3.7); 
\draw [line width=1pt,color=cyan] (.5,4) to  (.3,3.7); 

\draw [line width=1pt,color=cyan] (0,3.5) to  (-1,2.2); 
\draw [line width=1pt,color=cyan] (.3,3.7) to  (1.05,2.25); 
\draw [line width=1pt,color=cyan] (.5,4) to  (2.5,3); 

\draw (-1.1,2.35) node {\tiny $a_2$};
\draw [fill=white] (-1,2.2) circle (1.5pt);
\draw (1.2,2.33) node {\tiny $b_0$};
\draw [fill=white] (1.05,2.25) circle (1.5pt);

% u node path
\draw (-.2,3.5) node {\tiny $a_u$};
\draw [fill=black] (0,3.5) circle (1.5pt);
\draw (.5,4.15) node {\tiny $c_u$};
\draw [fill=black] (.5,4) circle (1.5pt);
\draw (.19,3.82) node {\tiny $b_u$};
\draw [fill=black] (.3,3.7) circle (1.5pt);

%
% Node z
%
\draw (3.8,2) node {$\hat z$};
%\draw [line width=1pt,color=cyan] (4.3,2) to (6.5,2); 
\draw [line width=1pt,color=cyan] (3.7,2.3) to (2.5,3); 
\draw [line width=1pt,color=cyan] (3.7,1.7) to (2.9,1.15); 
%\draw (2.4,2.9) node {\tiny $c_2$};
\draw [fill=white] (2.5,3) circle (1.5pt);


% node path
\draw [line width=1pt,color=cyan] (3.7,2.3) to  (3.7,1.7); 
%\draw [line width=1pt,color=cyan] (3.7,1.7) to  (4.3,2); 


\draw (3.75,1.55) node {\tiny $d_z$};
\draw [fill=black] (3.7,1.7)circle (1.5pt);
\draw (3.7,2.43) node {\tiny $c_z$};
\draw [fill=black] (3.7,2.3)circle (1.5pt);
%\draw (4.3,2.2) node {\tiny $g_z$};
%\draw [fill=black] (4.3,2)circle (1.5pt);

%\draw (3.2,2.7) node {\tiny $c_0$};
\draw [fill=white] (3,2.7)circle (1.5pt);
%\draw (3.05,1.25)[anchor=north] node {\tiny $d_2$};
\draw [fill=white] (2.9,1.15)circle (1.5pt);

%eF0 -- e3
%\draw [line width=1pt,color=red] (3.3,4.7) to  (5.25,2.25);


%
% Node F_0
%
\draw (3,5.2) node {$\hat F_0$};
%\draw [line width=1pt,color=red] (2.7,4.7) to  (3.8,4.7); 
\draw [line width=1pt,color=red] (2.5,5) to  (2.7,4.7); 
\draw [line width=1pt,color=red] (3.5,5) to  (2.7,4.7); 

\draw [line width=1pt,color=red] (2.5,5) to[in=180,out=180,looseness=2]  (-1.2,2); 
%\draw [line width=1pt,color=red] (4,5) to[in=-20,out=10,looseness=3]  (5.25,1.75); 
\draw [line width=1pt,color=red] (3.5,5) to[in=-0,out=-60,looseness=1.4]  (2.8,.85); 

%^c
\draw [line width=1pt,color=red] (2.7,4.7) to  (2.9,3.1);

%\draw (3.07,3.1) node {\tiny $c_1$};
\draw [fill=white] (2.9,3.1)circle (1.5pt);
%\draw (2.9,.7) node {\tiny $d_1$};
\draw [fill=white] (2.8,.85)circle (1.5pt);


%fF3 -- 
\draw [line width=1pt,color=red] (6.5,4.7) to  (7,4.7);
\draw [line width=1pt,color=red] (6.5,3.9) to  (6.5,3);
\draw [line width=1pt,color=red] (4.5,5) to  (5,1);

% ^f
\draw [line width=1pt,color=cyan] (6.8,2.5) to[in=-60,out=60,looseness=1.5] (6.65,3.8);
\draw [line width=1pt,color=cyan] (6.2,2.5) to[in=-120,out=120,looseness=1.5] (6.35,3.8);
\draw [line width=1pt,color=red] (6.5,4.7) to  (6.5,3.95);
\draw [line width=1pt,color=red] (3.5,5) to  (4.5,5);
\draw [line width=1pt,color=red] (5.5,.5) to[in=0,out=-20,looseness=1]  (7,4.7);

\draw [line width=1pt,color=red] (6.5,4.7) to  (4.5,5);

\draw (6.65,4.05) node {\tiny $f_3$};
\draw [fill=white] (6.5,3.95)circle (1.5pt);
\draw (6.65,3.55) node {\tiny $f_1$};
\draw [fill=white] (6.5,3.65)circle (1.5pt);
\draw (6.8,3.8) node {\tiny $f_2$};
\draw [fill=white] (6.65,3.8)circle (1.5pt);
\draw (6.2,3.8) node {\tiny $f_0$};
\draw [fill=white] (6.35,3.8)circle (1.5pt);
\draw (4.8,1) node {\tiny $h_3$};
\draw [fill=white] (5,1)circle (1.5pt);



\draw (2.5,5)[anchor=south] node {\tiny $a_{F_0}$};
\draw [fill=black] (2.5,5)circle (1.5pt);
\draw (2.5,4.5) node {\tiny $c_{F_0}$};
\draw [fill=black] (2.7,4.7)circle (1.5pt);
\draw (3.5,5)[anchor=south] node {\tiny $d_{F_0}$};
\draw [fill=black] (3.5,5)circle (1.5pt);
\draw (4.5,5)[anchor=south] node {\tiny $h_{F_0}$};
\draw [fill=black] (4.5,5)circle (1.5pt);
\draw (6.5,4.7)[anchor=south] node {\tiny $f_{F_4}$};
\draw [fill=black] (6.5,4.7)circle (1.5pt);
%\draw (3.3,4.7)[anchor=north] node {\tiny $g_{F_0}$};
%\draw [fill=black] (3.3,4.7)circle (1.5pt);
\draw (7,4.7)[anchor=south] node {\tiny $h_{F_4}$};
\draw [fill=black] (7,4.7)circle (1.5pt);
\draw (5.5,.5)[anchor=north] node {\tiny $h_1$};
\draw [fill=white] (5.5,.5)circle (1.5pt);

%
% Node y
%
\draw (6.5,2.3) node {$\hat y$};

%node path
\draw [line width=1pt,color=cyan] (6.8,2.5) to  (6.2,2.5); 
\draw [line width=1pt,color=cyan] (6.5,2) to  (6.2,2.5); 

\draw [line width=1pt,color=cyan] (6.5,2) to  (5.5,1); 


\draw (6.7,2) node {\tiny $h_y$};
\draw (6,2.5) node {\tiny $f_y$};
\draw (7,2.5) node {\tiny $f_y$};
\draw [fill=black] (6.5,2)circle (1.5pt);
\draw [fill=black] (6.2,2.5)circle (1.5pt);
\draw [fill=black] (6.8,2.5)circle (1.5pt);


\draw (5.5,.8) node {\tiny $h_2$};
\draw [fill=white] (5.5,1)circle (1.5pt);
%
% Node F_1
%
%\draw (2.1,2) node {$\hat F_1$};
%node path
\draw [line width=1pt,color=red] (2.3,2.3) to  (2.3,1.7); 
\draw [line width=1pt,color=red] (1.7,2) to  (2.3,1.7); 

\draw [line width=1pt,color=red] (2.6,2.6) to  (2.3,2.3); 
\draw [line width=1pt,color=red] (2.6,1.3) to  (2.3,1.7); 
\draw [line width=1pt,color=red] (1.7,2) to  (1.3,2); 

% F_1 to F_2
\draw [line width=1pt,color=red] (.8,2) to  (1.3,2); 

\draw (1.8,2.13) node {\tiny $b_{F_1}$};
\draw [fill=black] (1.7,2)circle (1.5pt);
%\draw (2.3,1.7)[anchor=north] node {\tiny $d_{F_1}$};
\draw [fill=black] (2.3,1.7)circle (1.5pt);
%\draw (2.25,2.2)[anchor=west] node {\tiny $c_{F_1}$};
\draw [fill=black] (2.3,2.3)circle (1.5pt);
\draw (1.4,1.85) node {\tiny $b_3$};
\draw [fill=white] (1.3,2)circle (1.5pt);

%\draw (2.8,1.4) node {\tiny $d_3$};
\draw [fill=white] (2.6,1.3)circle (1.5pt);
%\draw (2.75,2.55) node {\tiny $c_3$};
\draw [fill=white] (2.6,2.6)circle (1.5pt);

%
% Node F_2
%
\draw (0,2.2) node {$\hat F_2$};

\draw [line width=1pt,color=red] (-1.2,2) to  (-.3,2); 
\draw [line width=1pt,color=red] (.8,2) to  (.3,2);   
\draw [line width=1pt,color=red] (-.3,2) to (.3,2);   

\draw [fill=black] (-.2,2) circle (1.5pt);
\draw (-.2,1.8) node {\tiny $a_{F_2}$};
\draw [fill=black] (.2,2) circle (1.5pt);
\draw (.3,1.8) node {\tiny $b_{F_2}$};
\draw (-.75,2.1) node {\tiny $a_1$};
\draw [fill=white] (-.8,2) circle (1.5pt);
\draw (.8,2.16) node {\tiny $b_1$};
\draw [fill=white] (.8,2) circle (1.5pt);

\draw (-1.3,1.9) node {\tiny $a_3$};
\draw [fill=white] (-1.2,2)circle (1.5pt);

%
% Node F_3
%
%\draw (6.7,3.2) node {$\hat F_3$};
\draw (6.3,3) node {\tiny $f_{F_3}$};
\draw [fill=black] (6.5,3)circle (1.5pt);

%\draw (5,2)[anchor=south] node {\tiny $g_0$};
%\draw [fill=white] (5,2)circle (1.5pt);
%\draw (5.5,2)[anchor=south] node {\tiny $g_2$};
%\draw [fill=white] (5.5,2)circle (1.5pt);
%\draw (5.25,2.25)[anchor=south] node {\tiny $g_3$};
%\draw [fill=white] (5.25,2.25)circle (1.5pt);
%\draw (5.25,1.75)[anchor=north] node {\tiny $g_1$};
%\draw [fill=white] (5.25,1.75)circle (1.5pt);

\draw [fill=white] (-1.2,2)circle (1.5pt);

\end{scriptsize}
\end{tikzpicture}

}
\caption{ADPs obtidas por \MSFaddEdge($G$, $h$, $v$, $a$, $y$, $f$, $7$).}
\label{fig:MSF-adiciona-ponte-3}
\end{figure}


Agora a chamada \MSFaddEdge($G$, $h$, $v$, $a$, $y$, $g$, $7$) será simulada.
O processo é análogo ao caso anterior até a linha~\ref{Algo:MSFaddEdge:linha:deteccao-caso} do Algoritmo~\ref{Algo:MSFaddEdge}.
Os vértices $v$ e $y$ são comparados com \LCOFindNode($a_v$) e \LCOFindNode($g_y$) e são atribuídos $0$ à variável~$i$ e $2$ a~$j$.
Uma nova óctupla é criada e associada a~$h$ e os nós $s_i$ são adicionados nas respectivas ordens cíclicas.
As chamadas \LCOCycle($a_v$) e \LCOMerge($a_v$, $h_0$, $\hat v$), correspondentes à linha~\ref{Algo:MSFaddEdge:merges0} do Algoritmo~\ref{Algo:MSFaddEdge}, reestruturam a ordem cíclica de~$v$ para ser $\langle d_v, b_v, a_v, h_v \rangle$.
Já as chamadas~\LCOCycle($g_y$) e \nolbreaks{\LCOMerge($g_y$, $h_y$, $\hat y$)}, correspondentes à linha~\ref{Algo:MSFaddEdge:merges2} da mesma rotina, reestruturam a ordem cíclica de~$y$ para ser~$\langle f_y, f_y, g_y, h_y \rangle$.


As linhas~\ref{Algo:MSFaddEdge:merges1} e~\ref{Algo:MSFaddEdge:merges3} fazem o mesmo para os nós~$s_1$ e~$s_3$ nas ordens cíclicas dos vértices de $\hat F^\star$.
Como $i=0$, temos que~$i-1 = 3$ módulo~$4$.
Assim o nó~$u_{i-1}$ em que é calculado o predecessor na linha~\ref{Algo:MSFaddEdge:merges1} do Algoritmo~\ref{Algo:MSFaddEdge} é o nó~$a_{F_0}$, pois esse está ligado ao nó~$a_3$.
O predecessor de $a_{F_0}$ em sua ordem cíclica é $d_{F_0}$.
A ordem cíclica de $F_0$ antes da execução dessa rotina é $\langle c_{F_0}, g_{F_0}, f_{F_0}, g_{F_0}, d_{F_0}, a_{F_0} \rangle$.
Assim \LCOCycle(\treapPredecessor($a_{F_0}$)) reestrutura a ordem cíclica de~$\hat F_0$ para ser~$\langle a_{F_0}, c_{F_0}, d_{F_0}\rangle$ e \LCOMerge($a_{F_0}$, $h_{F_0}$) resulta na ordem $\langle a_{F_0}, c_{F_0}, d_{F_0}, h_{F_0}\rangle$.
Em seguida, como $j=2$, temos que~$j-1 = 1$ módulo~$4$.
Assim o nó~$v_{i-1}$ em que é calculado o predecessor na linha~\ref{Algo:MSFaddEdge:merges3} do Algoritmo~\ref{Algo:MSFaddEdge} é o nó~$g_{F_0}$, pois esse está ligado ao nó~$g_1$.
O predecessor de $g_{F_0}$ é $f_{F_0}$, assim \LCOCycle(\treapPredecessor($a_{F_0}$)) reestrutura a ordem cíclica de~$\hat F_0$ para ser~$\langle g_{F_0}, d_{F_0}, h_{F_0}, a_{F_0}, c_{F_0}, g_{F_0}, f_{F_0}\rangle$ e \LCOMerge($a_{F_0}$, $h_{F_0}$) resulta na ordem~$\langle g_{F_0}, d_{F_0}, h_{F_0}, a_{F_0}, c_{F_0}, g_{F_0}, f_{F_0}, h_{F_0}\rangle$.
A Figura~\ref{fig:MSF-adiciona-nao-ponte-1} ilustra as ADPs após essas modificações.

\begin{figure}[htb]
\scalebox{.9}{
\begin{tikzpicture}[line cap=round,line join=round,>=triangle 45,x=1cm,y=1cm]
\clip(-3,-1) rectangle (8.7,6);

\begin{scriptsize}

%
% Node v
%
\draw (0,0) node {$\hat v$};
\draw [line width=1pt,color=cyan] (0,.5) to  (.35,.35); 
\draw [line width=1pt,color=cyan] (.35,.35) to  (.5,0); 
\draw [line width=1pt,color=cyan] (0,.5) to  (.5,-.5); 

\draw [line width=1pt,color=cyan] (0,.5) to  (-1,1.8); 
\draw [line width=1pt,color=cyan] (.35,.35) to  (1.05,1.75); 
\draw [line width=1pt,color=cyan] (.5,0) to  (2.5,1); 
\draw [line width=1pt,color=cyan] (.5,-.5) to  (2,-.5); 


\draw (0,.5)[anchor=east] node {\tiny $a_v$};
\draw (.35,.35)[anchor=west] node {\tiny $b_v$};
\draw (.5,0)[anchor=north west] node {\tiny $d_v$};
\draw (.5,-.5)[anchor=east] node {\tiny $h_v$};
\draw [fill=black] (0,.5) circle (1.5pt);
\draw [fill=black] (.35,.35) circle (1.5pt);
\draw [fill=black] (.5,0) circle (1.5pt);
\draw [fill=black] (.5,-.5) circle (1.5pt);

\draw [line width=1pt,color=cyan] (-1,1.8) to  (-1,2.2); 
\draw (-.6,1.9) node[anchor=north east] {\tiny $a_0$};
\draw [fill=white] (-1,1.8) circle (1.5pt);

\draw (2,-.5)[anchor=south] node {\tiny $h_0$};
\draw (2,-.5)[fill=white] circle (1.5pt);
% ^b
\draw [line width=1pt,color=cyan] (2.5,1) to (2.9,1.15); 
\draw (1,1.65)[anchor=west] node {\tiny $b_2$};
\draw [fill=white] (1.05,1.75) circle (1.5pt);
\draw (2.5,1)[anchor=north] node {\tiny $d_0$};
\draw [fill=white] (2.5,1) circle (1.5pt);

%
% Node u
%
\draw (0,4) node {$\hat u$};

\draw [line width=1pt,color=cyan] (0,3.5) to  (.3,3.7); 
\draw [line width=1pt,color=cyan] (.5,4) to  (.3,3.7); 

\draw [line width=1pt,color=cyan] (0,3.5) to  (-1,2.2); 
\draw [line width=1pt,color=cyan] (.3,3.7) to  (1.05,2.25); 
\draw [line width=1pt,color=cyan] (.5,4) to  (2.5,3); 

\draw (-1.1,2.35) node {\tiny $a_2$};
\draw [fill=white] (-1,2.2) circle (1.5pt);
\draw (1.2,2.33) node {\tiny $b_0$};
\draw [fill=white] (1.05,2.25) circle (1.5pt);

\draw (2.4,2.9) node {\tiny $c_2$};
\draw [fill=white] (2.5,3) circle (1.5pt);

% u node path
\draw (-.2,3.5) node {\tiny $a_u$};
\draw [fill=black] (0,3.5) circle (1.5pt);
\draw (.5,4.15) node {\tiny $c_u$};
\draw [fill=black] (.5,4) circle (1.5pt);
\draw (.19,3.82) node {\tiny $b_u$};
\draw [fill=black] (.3,3.7) circle (1.5pt);

%
% Node z
%
\draw (3.95,2.1) node {$\hat z$};
\draw [line width=1pt,color=cyan] (4.3,2) to (6.5,2); 
\draw [line width=1pt,color=cyan] (3.7,2.3) to (3,2.7); 
\draw [line width=1pt,color=cyan] (3.7,1.7) to (2.9,1.15); 

% node path
\draw [line width=1pt,color=cyan] (3.7,2.3) to  (3.7,1.7); 
\draw [line width=1pt,color=cyan] (3.7,1.7) to  (4.3,2); 


\draw (3.75,1.55) node {\tiny $d_z$};
\draw [fill=black] (3.7,1.7)circle (1.5pt);
\draw (3.7,2.43) node {\tiny $c_z$};
\draw [fill=black] (3.7,2.3)circle (1.5pt);
\draw (4.3,2.2) node {\tiny $g_z$};
\draw [fill=black] (4.3,2)circle (1.5pt);

\draw (3.2,2.7) node {\tiny $c_0$};
\draw [fill=white] (3,2.7)circle (1.5pt);
\draw (3.05,1.25)[anchor=north] node {\tiny $d_2$};
\draw [fill=white] (2.9,1.15)circle (1.5pt);

%eF0 -- e3
\draw [line width=1pt,color=red] (3.3,4.7) to  (5.25,2.25);


%
% Node F_0
%
\draw (3,5) node {$\hat F_0$};
\draw [line width=1pt,color=red] (2.7,4.7) to  (3.8,4.7); 
\draw [line width=1pt,color=red] (3.8,4.7) to  (4.3,4.8); 
\draw [line width=1pt,color=red] (3.5,5.4) to  (3,5.3); 
\draw [line width=1pt,color=red] (3,5.3) to  (2.5,5); 
\draw [line width=1pt,color=red] (2.5,5) to  (2.7,4.7); 
\draw [line width=1pt,color=red] (3.5,5.4) to  (4,5.2); 

\draw [line width=1pt,color=red] (2.5,5) to[in=100,out=180,looseness=1.5]  (-1.2,2); 
\draw [line width=1pt,color=red] (4,5.2) to[in=-30,out=20,looseness=3.5]  (5.25,1.75); 
\draw [line width=1pt,color=red] (5,5.8) to[in=90,out=180,looseness=1]  (3.5,5.4); 
\draw [line width=1pt,color=red] (5,5.8) to[in=-30,out=0,looseness=3]  (2.8,.85); 
\draw [line width=1pt,color=red] (4.3,4.8) to (4.8,4.8); 

%^c
\draw [line width=1pt,color=red] (2.6,2.6) to  (2.9,3.1);
\draw [line width=1pt,color=red] (2.7,4.7) to  (2.9,3.1);

\draw (-1.3,1.9) node {\tiny $a_3$};
\draw [fill=white] (-1.2,2)circle (1.5pt);
\draw (3.07,3.1) node {\tiny $c_1$};
\draw [fill=white] (2.9,3.1)circle (1.5pt);
\draw (2.9,.7) node {\tiny $d_1$};
\draw [fill=white] (2.8,.85)circle (1.5pt);
\draw (4.8,4.8)[anchor=south] node {\tiny $h_1$};
\draw [fill=white] (4.8,4.8)circle (1.5pt);


%fF3 -- 
\draw [line width=1pt,color=red] (6.5,3.9) to  (6.5,3);

% ^f
\draw [line width=1pt,color=cyan] (6.8,2.5) to[in=-60,out=60,looseness=1.5] (6.65,3.8);
\draw [line width=1pt,color=cyan] (6.2,2.5) to[in=-120,out=120,looseness=1.5] (6.35,3.8);

\draw [line width=1pt,color=red] (3.8,4.7) to  (6.5,3.95);



\draw (6.65,4.05) node {\tiny $f_0$};
\draw [fill=white] (6.5,3.95)circle (1.5pt);
\draw (6.65,3.55) node {\tiny $f_2$};
\draw [fill=white] (6.5,3.65)circle (1.5pt);
\draw (6.8,3.8) node {\tiny $f_3$};
\draw [fill=white] (6.65,3.8)circle (1.5pt);
\draw (6.2,3.8) node {\tiny $f_1$};
\draw [fill=white] (6.35,3.8)circle (1.5pt);



\draw [line width=1pt,color=red] (2.5,5.5) to  (3,5.3);

\draw (3,5.3)[anchor=south] node {\tiny $h_{F_0}$};
\draw [fill=black] (3,5.3)circle (1.5pt);
\draw (2.5,5)[anchor=south] node {\tiny $a_{F_0}$};
\draw [fill=black] (2.5,5)circle (1.5pt);
\draw (2.5,4.6) node {\tiny $c_{F_0}$};
\draw [fill=black] (2.7,4.7)circle (1.5pt);
\draw (3.4,5.4)[anchor=south] node {\tiny $d_{F_0}$};
\draw [fill=black] (3.5,5.4)circle (1.5pt);
\draw (4.3,4.8)[anchor=south] node {\tiny $h_{F_0}$};
\draw [fill=black] (4.3,4.8)circle (1.5pt);
\draw (4,5.2)[anchor=south] node {\tiny $g_{F_0}$};
\draw [fill=black] (4,5.2)circle (1.5pt);
\draw (3.8,4.7)[anchor=north] node {\tiny $f_{F_0}$};
\draw [fill=black] (3.8,4.7)circle (1.5pt);
\draw (3.3,4.7)[anchor=north] node {\tiny $g_{F_0}$};
\draw [fill=black] (3.3,4.7)circle (1.5pt);

\draw (2.5,5.5)[anchor=south] node {\tiny $h_3$};
\draw [fill=white] (2.5,5.5)circle (1.5pt);
%
% Node y
%
\draw (6.7,2.3) node {$\hat y$};

%node path
\draw [line width=1pt,color=cyan] (6.5,2) to  (6.2,2.5); 
\draw [line width=1pt,color=cyan] (6.2,2.5) to  (6.8,2.5); 
\draw [line width=1pt,color=cyan] (6.5,2) to  (7.1,2); 


\draw [line width=1pt,color=cyan] (7.1,2) to  (6.6,1.5); 

\draw (6.5,1.8) node {\tiny $g_y$};
\draw (6,2.5) node {\tiny $f_y$};
\draw (7,2.5) node {\tiny $f_y$};
\draw (7.3,2) node {\tiny $h_y$};
\draw [fill=black] (6.5,2)circle (1.5pt);
\draw [fill=black] (6.2,2.5)circle (1.5pt);
\draw [fill=black] (6.8,2.5)circle (1.5pt);
\draw [fill=black] (7.1,2)circle (1.5pt);



\draw (6.8,1.5) node {\tiny $h_2$};
\draw [fill=white] (6.6,1.5)circle (1.5pt);

%
% Node F_1
%
\draw (2.1,2) node {$\hat F_1$};
%node path
\draw [line width=1pt,color=red] (2.3,2.3) to  (2.3,1.7); 
\draw [line width=1pt,color=red] (1.7,2) to  (2.3,1.7); 

\draw [line width=1pt,color=red] (2.6,2.6) to  (2.3,2.3); 
\draw [line width=1pt,color=red] (2.6,1.3) to  (2.3,1.7); 
\draw [line width=1pt,color=red] (1.7,2) to  (1.3,2); 

% F_1 to F_2
\draw [line width=1pt,color=red] (.8,2) to  (1.3,2); 

\draw (1.8,2.13) node {\tiny $b_{F_1}$};
\draw [fill=black] (1.7,2)circle (1.5pt);
\draw (2.3,1.7)[anchor=north] node {\tiny $d_{F_1}$};
\draw [fill=black] (2.3,1.7)circle (1.5pt);
\draw (2.25,2.2)[anchor=west] node {\tiny $c_{F_1}$};
\draw [fill=black] (2.3,2.3)circle (1.5pt);
\draw (1.4,1.85) node {\tiny $b_3$};
\draw [fill=white] (1.3,2)circle (1.5pt);

\draw (2.8,1.4) node {\tiny $d_3$};
\draw [fill=white] (2.6,1.3)circle (1.5pt);
\draw (2.75,2.55) node {\tiny $c_3$};
\draw [fill=white] (2.6,2.6)circle (1.5pt);

%
% Node F_2
%
\draw (0,2.2) node {$\hat F_2$};

\draw [line width=1pt,color=red] (-.8,2) to  (-.3,2); 
\draw [line width=1pt,color=red] (.8,2) to  (.3,2);   
\draw [line width=1pt,color=red] (-.3,2) to (.3,2);   

\draw [fill=black] (-.2,2) circle (1.5pt);
\draw (-.2,1.8) node {\tiny $a_{F_2}$};
\draw [fill=black] (.2,2) circle (1.5pt);
\draw (.3,1.8) node {\tiny $b_{F_2}$};
\draw (-.75,2.1) node {\tiny $a_1$};
\draw [fill=white] (-.8,2) circle (1.5pt);
\draw (.8,2.16) node {\tiny $b_1$};
\draw [fill=white] (.8,2) circle (1.5pt);

%
% Node F_3
%
\draw (6.7,3.2) node {$\hat F_3$};
\draw (6.3,3) node {\tiny $f_{F_3}$};
\draw [fill=black] (6.5,3)circle (1.5pt);

\draw (5,2)[anchor=south] node {\tiny $g_0$};
\draw [fill=white] (5,2)circle (1.5pt);
\draw (5.5,2)[anchor=south] node {\tiny $g_2$};
\draw [fill=white] (5.5,2)circle (1.5pt);
\draw (5.3,2.25)[anchor=south] node {\tiny $g_3$};
\draw [fill=white] (5.25,2.25)circle (1.5pt);
\draw (5.25,1.75)[anchor=north] node {\tiny $g_1$};
\draw [fill=white] (5.25,1.75)circle (1.5pt);

\end{scriptsize}
\end{tikzpicture}

}
\caption{ADPs após adição da óctupla de~$h$.}
\label{fig:MSF-adiciona-nao-ponte-1}
\end{figure}

Após essas modificações, é feita a consulta de conexidade para verificar em qual caso a aresta~$h$ se encontra.
Nesse caso esse teste retorna positivo e assim constata-se que~$h$ está no segundo caso.
Assim é executado \LCOSplit($a_{F_0}$, $F_4$, $F_0$), correspondente à linha~\ref{Algo:MSFaddEdge:linha:splitnaoponte} do Algoritmo~\ref{Algo:MSFaddEdge}, que corta a ordem cíclica em duas,a primeira sendo $\langle g_{F_0}, d_{F_0}, h_{F_0}\rangle$ e a segunda $\langle a_{F_0}, c_{F_0}, g_{F_0}, f_{F_0}, h_{F_0}\rangle$.
A primeira ordem cíclica corresponde à face~$F_4$ da Figura~\ref{fig:MSF-adiciona-nao-ponte-2}, enquanto que a segunda se mantém sendo a face exterior~$F_0$.
Por fim, é chamado \MSFupdate($G$, $h$, $7$) que atesta que essa aresta pertence a $\hat F^\star$ e internamente executa $\LCOMerge(h_1, h_3, \hat h)$.

\begin{figure}
\scalebox{.95}{
\begin{tikzpicture}[line cap=round,line join=round,>=triangle 45,x=1cm,y=1cm]
\clip(-3,-1) rectangle (8.7,6);

\begin{scriptsize}

%
% Node v
%
\draw (0,0) node {$\hat v$};
\draw [line width=1pt,color=cyan] (0,.5) to  (.35,.35); 
\draw [line width=1pt,color=cyan] (.35,.35) to  (.5,0); 
\draw [line width=1pt,color=cyan] (0,.5) to  (.5,-.5); 

\draw [line width=1pt,color=cyan] (0,.5) to  (-1,1.8); 
\draw [line width=1pt,color=cyan] (.35,.35) to  (1.05,1.75); 
\draw [line width=1pt,color=cyan] (.5,0) to  (2.5,1); 
\draw [line width=1pt,color=cyan] (.5,-.5) to  (5.2,0); 


\draw (0,.5)[anchor=east] node {\tiny $a_v$};
\draw (.35,.35)[anchor=west] node {\tiny $b_v$};
\draw (.5,0)[anchor=north west] node {\tiny $d_v$};
\draw (.5,-.5)[anchor=east] node {\tiny $h_v$};
\draw [fill=black] (0,.5) circle (1.5pt);
\draw [fill=black] (.35,.35) circle (1.5pt);
\draw [fill=black] (.5,0) circle (1.5pt);
\draw [fill=black] (.5,-.5) circle (1.5pt);

\draw [line width=1pt,color=cyan] (-1,1.8) to  (-1,2.2); 
\draw (-.6,1.9) node[anchor=north east] {\tiny $a_0$};
\draw [fill=white] (-1,1.8) circle (1.5pt);

\draw (5.2,0)[anchor=south] node {\tiny $h_0$};
\draw (5.2,0)[fill=white] circle (1.5pt);
% ^b
\draw [line width=1pt,color=cyan] (2.5,1) to (2.9,1.15); 
\draw (1,1.65)[anchor=west] node {\tiny $b_2$};
\draw [fill=white] (1.05,1.75) circle (1.5pt);
\draw (2.5,1)[anchor=north] node {\tiny $d_0$};
\draw [fill=white] (2.5,1) circle (1.5pt);

%
% Node u
%
\draw (0,4) node {$\hat u$};

\draw [line width=1pt,color=cyan] (0,3.5) to  (.3,3.7); 
\draw [line width=1pt,color=cyan] (.5,4) to  (.3,3.7); 

\draw [line width=1pt,color=cyan] (0,3.5) to  (-1,2.2); 
\draw [line width=1pt,color=cyan] (.3,3.7) to  (1.05,2.25); 
\draw [line width=1pt,color=cyan] (.5,4) to  (2.5,3); 

\draw (-1.1,2.35) node {\tiny $a_2$};
\draw [fill=white] (-1,2.2) circle (1.5pt);
\draw (1.2,2.33) node {\tiny $b_0$};
\draw [fill=white] (1.05,2.25) circle (1.5pt);

\draw (2.4,2.9) node {\tiny $c_2$};
\draw [fill=white] (2.5,3) circle (1.5pt);

% u node path
\draw (-.2,3.5) node {\tiny $a_u$};
\draw [fill=black] (0,3.5) circle (1.5pt);
\draw (.5,4.15) node {\tiny $c_u$};
\draw [fill=black] (.5,4) circle (1.5pt);
\draw (.19,3.82) node {\tiny $b_u$};
\draw [fill=black] (.3,3.7) circle (1.5pt);

%
% Node z
%
\draw (3.95,2.1) node {$\hat z$};
\draw [line width=1pt,color=cyan] (4.3,2) to (6.5,2); 
\draw [line width=1pt,color=cyan] (3.7,2.3) to (3,2.7); 
\draw [line width=1pt,color=cyan] (3.7,1.7) to (2.9,1.15); 

% node path
\draw [line width=1pt,color=cyan] (3.7,2.3) to  (3.7,1.7); 
\draw [line width=1pt,color=cyan] (3.7,1.7) to  (4.3,2); 


\draw (3.75,1.55) node {\tiny $d_z$};
\draw [fill=black] (3.7,1.7)circle (1.5pt);
\draw (3.7,2.43) node {\tiny $c_z$};
\draw [fill=black] (3.7,2.3)circle (1.5pt);
\draw (4.3,2.2) node {\tiny $g_z$};
\draw [fill=black] (4.3,2)circle (1.5pt);

\draw (3.2,2.7) node {\tiny $c_0$};
\draw [fill=white] (3,2.7)circle (1.5pt);
\draw (3.05,1.25)[anchor=north] node {\tiny $d_2$};
\draw [fill=white] (2.9,1.15)circle (1.5pt);

%eF0 -- e3
\draw [line width=1pt,color=red] (3.3,4.7) to  (5.25,2.25);


%
% Node F_0
%
\draw (3,5) node {$\hat F_0$};
\draw [line width=1pt,color=red] (2.7,4.7) to  (3.8,4.7); 
\draw [line width=1pt,color=red] (3.8,4.7) to  (4.3,4.8); 
\draw [line width=1pt,color=red] (2.5,5) to  (2.7,4.7); 

\draw [line width=1pt,color=red] (2.5,5) to[in=100,out=180,looseness=1.5]  (-1.2,2); 
\draw [line width=1pt,color=red] (4.3,4.8) to[in=0,out=20,looseness=2] (5.7,-.1); 

%^c
\draw [line width=1pt,color=red] (2.6,2.6) to  (2.9,3.1);
\draw [line width=1pt,color=red] (2.7,4.7) to  (2.9,3.1);

\draw (-1.3,1.9) node {\tiny $a_3$};
\draw [fill=white] (-1.2,2)circle (1.5pt);
\draw (3.07,3.1) node {\tiny $c_1$};
\draw [fill=white] (2.9,3.1)circle (1.5pt);
\draw (5.7,-.1)[anchor=north] node {\tiny $h_1$};
\draw [fill=white] (5.7,-.1)circle (1.5pt);


%fF3 -- 
\draw [line width=1pt,color=red] (6.5,3.9) to  (6.5,3);

% ^f
\draw [line width=1pt,color=cyan] (6.8,2.5) to[in=-60,out=60,looseness=1.5] (6.65,3.8);
\draw [line width=1pt,color=cyan] (6.2,2.5) to[in=-120,out=120,looseness=1.5] (6.35,3.8);

\draw [line width=1pt,color=red] (3.8,4.7) to  (6.5,3.95);



\draw (6.65,4.05) node {\tiny $f_0$};
\draw [fill=white] (6.5,3.95)circle (1.5pt);
\draw (6.65,3.55) node {\tiny $f_2$};
\draw [fill=white] (6.5,3.65)circle (1.5pt);
\draw (6.8,3.8) node {\tiny $f_3$};
\draw [fill=white] (6.65,3.8)circle (1.5pt);
\draw (6.2,3.8) node {\tiny $f_1$};
\draw [fill=white] (6.35,3.8)circle (1.5pt);




\draw (2.5,5)[anchor=south] node {\tiny $a_{F_0}$};
\draw [fill=black] (2.5,5)circle (1.5pt);
\draw (2.5,4.6) node {\tiny $c_{F_0}$};
\draw [fill=black] (2.7,4.7)circle (1.5pt);
\draw (4.3,4.8)[anchor=south] node {\tiny $h_{F_0}$};
\draw [fill=black] (4.3,4.8)circle (1.5pt);
\draw (3.3,4.7)[anchor=south] node {\tiny $g_{F_0}$};
\draw [fill=black] (3.3,4.7)circle (1.5pt);
\draw (3.8,4.7)[anchor=north] node {\tiny $f_{F_0}$};
\draw [fill=black] (3.8,4.7)circle (1.5pt);



%
% F4
%
\draw [line width=1pt,color=red] (4.5,1) to  (3.7,.5);
\draw [line width=1pt,color=red] (3.7,.5) to  (4.5,.5);

\draw [line width=1pt,color=red] (2.8,.85) to  (3.7,.5); % d
\draw [line width=1pt,color=red] (5.5,.3) to  (4.5,.5); % h
\draw [line width=1pt,color=red] (5.25,1.75) to  (4.5,1); % g

\draw (4,.2) node {$\hat F_4$};
\draw (3.6,.5)[anchor=south] node {\tiny $d_{F_0}$};
\draw [fill=black] (3.7,.5)circle (1.5pt);
\draw (4.5,.5)[anchor=south] node {\tiny $h_{F_0}$};
\draw [fill=black] (4.5,.5)circle (1.5pt);
\draw (4.5,1)[anchor=south east] node {\tiny $g_{F_0}$};
\draw [fill=black] (4.5,1)circle (1.5pt);


\draw (2.9,.7) node {\tiny $d_1$};
\draw [fill=white] (2.8,.85)circle (1.5pt);
\draw (5.25,1.75)[anchor=north] node {\tiny $g_1$};
\draw [fill=white] (5.25,1.75)circle (1.5pt);
\draw (5.5,.3)[anchor=south] node {\tiny $h_3$};
\draw [fill=white] (5.5,.3)circle (1.5pt);

%
% Node y
%
\draw (6.7,2.3) node {$\hat y$};

%node path
\draw [line width=1pt,color=cyan] (6.5,2) to  (6.2,2.5); 
\draw [line width=1pt,color=cyan] (6.2,2.5) to  (6.8,2.5); 
\draw [line width=1pt,color=cyan] (6.5,2) to  (7.1,2); 


\draw [line width=1pt,color=cyan] (7.1,2) to  (5.9,.2); 

\draw (6.5,1.8) node {\tiny $g_y$};
\draw (6,2.5) node {\tiny $f_y$};
\draw (7,2.5) node {\tiny $f_y$};
\draw (7.3,2) node {\tiny $h_y$};
\draw [fill=black] (6.5,2)circle (1.5pt);
\draw [fill=black] (6.2,2.5)circle (1.5pt);
\draw [fill=black] (6.8,2.5)circle (1.5pt);
\draw [fill=black] (7.1,2)circle (1.5pt);

\draw (5.85,.4) node {\tiny $h_2$};
\draw [fill=white] (5.9,.2)circle (1.5pt);

%
% Node F_1
%
\draw (2.1,2) node {$\hat F_1$};
%node path
\draw [line width=1pt,color=red] (2.3,2.3) to  (2.3,1.7); 
\draw [line width=1pt,color=red] (1.7,2) to  (2.3,1.7); 

\draw [line width=1pt,color=red] (2.6,2.6) to  (2.3,2.3); 
\draw [line width=1pt,color=red] (2.6,1.3) to  (2.3,1.7); 
\draw [line width=1pt,color=red] (1.7,2) to  (1.3,2); 

% F_1 to F_2
\draw [line width=1pt,color=red] (.8,2) to  (1.3,2); 

\draw (1.8,2.13) node {\tiny $b_{F_1}$};
\draw [fill=black] (1.7,2)circle (1.5pt);
\draw (2.3,1.7)[anchor=north] node {\tiny $d_{F_1}$};
\draw [fill=black] (2.3,1.7)circle (1.5pt);
\draw (2.25,2.2)[anchor=west] node {\tiny $c_{F_1}$};
\draw [fill=black] (2.3,2.3)circle (1.5pt);
\draw (1.4,1.85) node {\tiny $b_3$};
\draw [fill=white] (1.3,2)circle (1.5pt);

\draw (2.8,1.4) node {\tiny $d_3$};
\draw [fill=white] (2.6,1.3)circle (1.5pt);
\draw (2.75,2.55) node {\tiny $c_3$};
\draw [fill=white] (2.6,2.6)circle (1.5pt);

%
% Node F_2
%
\draw (0,2.2) node {$\hat F_2$};

\draw [line width=1pt,color=red] (-.8,2) to  (-.3,2); 
\draw [line width=1pt,color=red] (.8,2) to  (.3,2);   
\draw [line width=1pt,color=red] (-.3,2) to (.3,2);   

\draw [fill=black] (-.2,2) circle (1.5pt);
\draw (-.2,1.8) node {\tiny $a_{F_2}$};
\draw [fill=black] (.2,2) circle (1.5pt);
\draw (.3,1.8) node {\tiny $b_{F_2}$};
\draw (-.75,2.1) node {\tiny $a_1$};
\draw [fill=white] (-.8,2) circle (1.5pt);
\draw (.8,2.16) node {\tiny $b_1$};
\draw [fill=white] (.8,2) circle (1.5pt);

%
% Node F_3
%
\draw (6.7,3.2) node {$\hat F_3$};
\draw (6.3,3) node {\tiny $f_{F_3}$};
\draw [fill=black] (6.5,3)circle (1.5pt);

\draw (5,2)[anchor=south] node {\tiny $g_0$};
\draw [fill=white] (5,2)circle (1.5pt);
\draw (5.5,2)[anchor=south] node {\tiny $g_2$};
\draw [fill=white] (5.5,2)circle (1.5pt);
\draw (5.3,2.25)[anchor=south] node {\tiny $g_3$};
\draw [fill=white] (5.25,2.25)circle (1.5pt);

\end{scriptsize}
\end{tikzpicture}

}
\caption{ADPs após o corte da face exterior.}
\label{fig:MSF-adiciona-nao-ponte-2}
\end{figure}



As linhas~\ref{Algo:MSFaddEdge:linha:1}, \ref{Algo:MSFaddEdge:linha:2} e~\ref{Algo:MSFaddEdge:linha:const3} do Algoritmo~\ref{Algo:MSFaddEdge} possuem consumo de tempo constante, pois são consultas a tabela de símbolos~$G.H$ e uma chamada da rotina \LCOMakeOcto{}.
As linhas remanescentes são formadas por um número constante de chamadas de \LCOFindNode, \LCOCycle{}, \LCOMerge{}, \LCOSplit{}, \MSFupdate{} e \LCOConnected{} e cada uma dessas chamadas possui consumo de tempo assintótico igual a~$\O{\lg m}$ esperado amortizado, onde~$m$ é o número de arestas no grafo corrente.
Dessa forma, $\MSFaddEdge$ também possui consumo esperado de tempo assintótico~$\O{\lg m}$ amortizado.

\section{Estruturas auxiliares}
\label{sec:struct-aux}

A implementação da biblioteca das árvores dinâmicas planas será detalhada na próxima seção.
Para construir essa estrutura de dados, são necessárias duas estruturas fundamentais: a primeira é a árvore binária de busca com chave implícita, discutida inicialmente em detalhes no Capítulo~\ref{sec:TreapDeChaveImplicita} e a segunda são as link cut trees, que serão abordadas brevemente na Seção~\ref{sec:linkcuttree}.

\subsection{Árvores binárias de busca com chave implícita}
\label{sec:extrastructABB}

Árvores binárias de busca com chave implícita são particularmente úteis para armazenar sequências de objetos.
No Capítulo~\ref{sec:TreapDeChaveImplicita} elas foram utilizadas para armazenar uma sequência de arestas que representa uma árvore.
No presente capítulo, elas serão usadas para armazenar a sequência de nós de link cut tree que representa a ordem cíclica de cada vértice.
Para implementar as rotinas presentes na biblioteca de ADPs, é necessário complementar a biblioteca de ABBs de chave implícita com as seguintes rotinas:
\begin{itemize}
\item \treapFirst($\node$): Retorna o nó de menor chave da ABB que contém $\node$.
\item \treapLast($\node$): Retorna o nó de maior chave da ABB que contém $\node$.
\item \treapSplitRight($\node$): Corta a ABB que contém um nó~$\node$ em duas ABBs. A primeira ABB contém todos os nós com chave menor ou igual a chave de~$\node$ e a segunda contém todos os nós com chave estritamente maior do que a chave de~$\node$. Essa rotina retorna as raízes dessas duas ABBs.
\end{itemize}

As implementações dessas rotinas estão descritas nos Algoritmos~\ref{Algo:TREAPfirst},~\ref{Algo:TREAPlast} e~\ref{Algo:TREAPsplitRight}.
Seu consumo de tempo será $\O{\lg t}$ esperado, onde $t$ é o número de nós da árvore.
Como descrito no final da Seção~\ref{sec:ADP}, essas ABBs são usadas para armazenar a ordem cíclica de cada vértice.
Como cada nó é incidente a no máximo todas as~$m$ de arestas no grafo dinâmico, concluímos que essas rotinas consumem $\O{\lg m}$ esperado.  

A rotina \treapFirst{}, descrita no Algoritmo~\ref{Algo:TREAPfirst}, começa seu processamento inicializando uma variável temporária, $p$, com o nó passado como argumento.
Em seguida, é usado a rotina $\treapGetRoot$, descrita no Algoritmo~\ref{Algo:treapGetRoot}, para obter a raiz da ABB que contém $\node$.
Depois de obter a raiz, o algoritmo desce pela árvore, movendo-se sempre para o filho esquerdo, até alcançar o nó mais à esquerda, que representa o nó com menor chave.
Finalmente, o nó encontrado é retornado.

\begin{algorithm}
\caption{\treapFirst($\node$)}
\label{Algo:TREAPfirst}
\begin{algorithmic}[1]
\State $p$ $\gets$ $\treapGetRoot(\node)$

\While { $p$.$esq$ $\neq \Nil$}
  \State $p$ $\gets$ $p$.$esq$
\EndWhile

\State\Return $p$
\end{algorithmic}
\end{algorithm}

A rotina \treapLast{}, descrita no Algoritmo~\ref{Algo:TREAPlast}, é análoga à rotina \treapFirst{}.
Ele também começa inicializando uma variável temporária, $u$, com o nó passado como argumento e em seguida, usando $\treapGetRoot$, obtém a raiz da ABB que contém $\node$.
Mas diferentemente de \treapFirst{}, após obter a raiz, $\treapLast$ desce pela árvore, movendo-se sempre para o filho direito, até alcançar o nó mais à direita, que representa o nó com maior chave, que é retornado.

\begin{algorithm}
\caption{\treapLast($\node$)}
\label{Algo:TREAPlast}
\begin{algorithmic}[1]
\State $u$ $\gets$ $\treapGetRoot(\node)$

\While { $u$.\varname{dir} $\neq \Nil$}
  \State $u$ $\gets$ $u$.\varname{dir}
\EndWhile

\State\Return $u$
\end{algorithmic}
\end{algorithm}

A implementação de \treapSplitRight{}, que pode ser vista no Algoritmo~\ref{Algo:TREAPsplitRight}, é quase idêntica a de \treapSplit{}, que foi descrito no Algoritmo~\ref{Algo:TREAPsplit}.
A única diferença entre essas rotinas é na linha~\ref{Algo:TREAPsplitRight:linha:atribuicao}, em que atribuímos~$\node$ a~$L$ em vez de~$\node$.\varname{esq}, como feito em \treapSplit{}.

\begin{algorithm}[H]
\caption{\treapSplitRight($\node$)}
\label{Algo:TREAPsplitRight}
\begin{algorithmic}[1]
\State $R$ $\gets $ $\node$.\varname{dir}
\State $L$ $\gets $ $\node$ \label{Algo:TREAPsplitRight:linha:atribuicao}
\State \varname{tmp} $\gets$ $\node$
\While { \varname{tmp}.\varname{pai} $\neq \Nil$}
  \If { \varname{tmp}.\varname{pai}.\varname{esq} = \varname{tmp}}
    \State \varname{tmp}.\varname{pai}.\varname{esq} $\gets$ $R$
    \State \varname{tmp}.\varname{pai}.$tam$ $\gets$ \varname{tmp}.\varname{pai}.$tam$ $\mathit{-}$ \treapGetSize($L$)
    \If { $R \neq \Nil$}
    \State $R$.\varname{pai} $\gets$ \varname{tmp}.\varname{pai}
    \EndIf
    \State $R$ $\gets$ \varname{tmp}.\varname{pai}
  \Else
    \State \varname{tmp}.\varname{pai}.\varname{dir} $\gets$ $L$ 
    \State \varname{tmp}.\varname{pai}.$tam$ $\gets$ \varname{tmp}.\varname{pai}.$tam$ $\mathit{-}$ \treapGetSize($R$)
    \If { $L$ $\neq$ \Nil}
    \State $L$.\varname{pai} $\gets$ \varname{tmp}.\varname{pai}
    \EndIf
    \State $L$ $\gets$ \varname{tmp}.\varname{pai}
  \EndIf
  \State \varname{tmp} $\gets$ \varname{tmp}.\varname{pai}
\EndWhile
\If { $L$ $\neq \Nil$} $L$.\varname{pai} $\gets\Nil$\EndIf
\If { $R$ $\neq \Nil$} $R$.\varname{pai} $\gets\Nil$\EndIf
\State $\node$.\varname{dir} $\gets $ $\node$.\varname{esq} $\gets$ $\node$.\varname{pai} $\gets$ $\Nil$
\State\Return $L$, $R$
\end{algorithmic}
\end{algorithm}

\subsection{Link cut trees}
\label{sec:linkcuttree}
Link cut trees são uma estrutura de dados usada para representar e manipular florestas dinâmicas enraizadas.
Elas foram originalmente introduzidas por Sleator e Tarjan~\cite{SleatroTarjanLinkCutTree1983,} em~1983 e já são bem conhecidas na literatura~\cite{linkcuttree,EncyclopediaAlgorithms,tarjanWerneck2010}.
Para um texto introdutório sobre link-cut trees, recomendamos o trabalho de Felipe Noronha e Cristina Gomes Fernandes~\cite{linkcuttree}.
Cada nó de link cut tree possui um campo \varname{abb} que aponta para o nó de ABB de chave implícita que representa esse nó na ordem cíclica.
Essa estrutura de dados dá suporte às seguintes operações:
\begin{itemize}
\item \linkcutCreate(): Cria e retorna um novo nó de link cut tree.
\item \linkcutDestroy($v$): Remove o nó~$v$ de link cut tree da memória.
\item \linkcutAddEdge($v$, $w$): Adiciona uma aresta de~$v$ a~$w$, tornando~$v$ um filho de~$w$. Essa operação assume que~$v$ é a raiz de sua árvore e que $w$ não é um nó da árvore de~$v$.
\item \linkcutDelEdge($v$): Remove a aresta de~$v$ para seu pai.
	Essa operação assume que~$v$ não é a raiz de sua árvore.
\item \linkcutEvert($v$): Torna~$v$ a raiz de sua árvore revertendo o caminho de~$v$ para a raiz original.
\item \linkcutMax($v$): Retorna o nó de peso máximo no caminho entre~$v$ e a raiz de sua árvore.
\item \linkcutMin($v$): Retorna o nó de peso mínimo no caminho entre~$v$ e a raiz de sua árvore.
\item \linkcutWeight($v$, $w$): Atribui o peso~$w$ a todos os nós entre~$v$ e a raiz da link cut tree.
\item \linkcutRoot($v$): Retorna a raiz da link cut tree que contém o nó~$v$.
\end{itemize}

O consumo de tempo de $\linkcutCreate$ é constante, o consumo de \linkcutWeight($v$, $w$) é $\O{t}$, onde $t$ é o número de nós entre o nó~$v$ e a raiz da sua árvore. O consumo de tempo das demais rotinas dessa biblioteca consomem tempo $\O{\lg t}$ amortizado, onde $t$ é o número de nós na link cut tree. Como $t$ é $\Theta(m)$, onde $m$ é o número de arestas no grafo corrente, então essas rotinas consumem $\O{\lg m}$ amortizado.


\section{Implementação de árvores dinâmicas planas}
\label{sec:implementacao-ADP}

Nessa seção, vamos mostrar como usar link cut trees e árvores binárias de busca de chave implícita para implementar biblioteca de árvores dinâmicas planas.

A primeira implementação elaborada é da rotina \LCOMakeOcto{}, que pode ser vista no Algoritmo~\ref{Algo:LCOMakeOcto}.
O processamento dessa rotina começa com um laço que para cada $0\leq i\leq 3$ cria novos nós de link cut tree~$s_i$ e~$e_i$ com a rotina \linkcutCreate{}.
Em seguida, o campo \varname{abb} desses nós é populado com um novo nó de ABB de chave implícita e~$e_i$ é atribuído ao campo $\node$ desse nó de ABB.
Então $s_i$ e~$e_i$ são ligados usando a rotina \linkcutAddEdge{} e o peso de~$e_i$ é inicializado com o valor~$w$.
Por fim o peso de cada $s_i$ é atribuído usando a rotina $\linkcutWeight$ e a óctupla é retornada.
O consumo assintótico de tempo dessa rotina é~$\O{1}$.

\begin{algorithm}[htb]
\caption{\LCOMakeOcto($e$, $w$)}
\label{Algo:LCOMakeOcto}
\begin{algorithmic}[1]
\For {$i \in \{0,1,2,3\}$}
  \State $s_i$ $\gets$ \linkcutCreate(); $e_i$ $\gets$ \linkcutCreate()
  \State $e_i$.\varname{abb} $\gets$ \treapCreate(); $s_i$.\varname{abb} $\gets$ \treapCreate()
  \State $e_i$.\varname{abb}.$\node$ $\gets$ $e$
  \State \linkcutAddEdge($e_i$, $s_i$)
  \State \linkcutWeight($e_i$, $w$)
\EndFor
\State \linkcutWeight($s_0$, $-\infty$); \linkcutWeight($s_2$, $-\infty$);
\State \linkcutWeight($s_1$, $\infty$); \linkcutWeight($s_3$, $\infty$);
\State \Return $e_0$, $e_1$, $e_2$, $e_3$, $s_0$, $s_1$, $s_2$, $s_3$
\end{algorithmic}
\end{algorithm}

A implementação de \LCODestroyOcto{} está descrita no Algoritmo~\ref{Algo:LCODestroyOcto}.
Primeiro a óctupla de~$e$ é obtida e apagada da tabela de símbolos, em seguida, cada nó de link cut tree é apagado da memória usando a rotina \linkcutDestroy{}. Seu consumo de tempo é constante.
\begin{algorithm}[htb]
\caption{\LCODestroyOcto($H$, $e$)}
\label{Algo:LCODestroyOcto}
\begin{algorithmic}[1]
\State $e_0$, $e_1$, $e_2$, $e_3$, $s_0$, $s_1$, $s_2$, $s_3$ $\gets$ $H(e)$
\State $H(e)$ $\gets$ $\Nil$
\For {$i \in \{0,1,2,3\}$}
  \State \linkcutDestroy($s_i$)
  \State \linkcutDestroy($e_i$)
\EndFor
\end{algorithmic}
\end{algorithm}


A implementação de \LCOConnected{} é descrita no Algoritmo~\ref{Algo:LCOConnected}. Para consultar se dois nós de link cut trees estão na mesma árvore, simplesmente comparamos a raiz das árvores que contém~$p$ e~$q$.
Como essa rotina é composta somente por duas chamadas de $\linkcutRoot$ e uma comparação, então seu consumo de tempo é~$\O{\lg m}$ amortizado, onde~$m$ é o número de arestas no grafo corrente, que é o consumo de $\linkcutRoot$.

\begin{algorithm}[htb]
\caption{\LCOConnected($p$, $q$)}
\label{Algo:LCOConnected}
\begin{algorithmic}[1]
\State \Return \linkcutRoot($p$) = \linkcutRoot($q$)
\end{algorithmic}
\end{algorithm}

A implementação de \LCOFindNode{} é simples e usa a estrutura de árvores binárias de busca com chave implícita.
Dado o nó~$s$, sabemos que ele está em alguma ABB e que a raiz dessa árvore aponta para o nó que contém~$s$ em sua ordem cíclica,
logo usamos \treapGetRoot{} da biblioteca de ABBs para obter a raiz~$r$ da árvore e em seguida é retornado~$r$.$\node$.
O consumo de tempo de $\LCOFindNode$ é o mesmo consumo de $\treapGetRoot$, isto é, $\O{\lg m}$ esperado, onde~$m$ é o número de arestas no grafo corrente.

\begin{algorithm}[htb]
\caption{\LCOFindNode($s$)}
\label{Algo:LCOFindNode}
\begin{algorithmic}[1]
\State $r$ $\gets$ \treapGetRoot($s.abb$)
\State \Return $r$.$\node$
\end{algorithmic}
\end{algorithm}

A rotina \LCOAddCost{}, descrita no Algoritmo~\ref{Algo:LCOAddCost}, primeiro realiza um \linkcutEvert{} no último nó da ordem ciclica que contém~$p$.
Dessa forma, o caminho entre os nós \treapFirst($p$) e \treapLast($p$) coincide com a ordem ciclica que contém~$p$.
Em seguida, a rotina \linkcutWeight{} atribui o peso~$w$ para todos os nós entre \treapFirst($p$) e a raiz da sua árvore, isto é, \treapLast($p$).
Logo todos os nós da ordem cíclica que contém $p$ têm seus pesos atualizados para o novo valor~$w$.
A rotina \linkcutEvert{} consome tempo $\O{\lg m}$ amortizado, onde~$m$ é o número de arestas no grafo corrente, enquanto \treapLast{} e \treapFirst{} consomem tempo~$\O{\lg t}$ esperado, onde $t$ é o número de nós de link cut tree na ordem cíclica que contém~$p$ e \linkcutWeight{} consume tempo $\O{t}$.
No nosso caso, a rotina \LCOAddCost{} só é chamada na linha~\ref{Algo:MSFupdate:linha:LCOAddCost} do Algoritmo~\ref{Algo:MSFupdate} e nessa chamada, temos que~$t\leq 2$.

\begin{algorithm}[htb]
\caption{\LCOAddCost($p$, $w$)}
\label{Algo:LCOAddCost}
\begin{algorithmic}[1]
\State \linkcutEvert(\treapLast($p$))
\State \linkcutWeight(\treapFirst($p$), $w$)
\end{algorithmic}
\end{algorithm}

As implementações das rotinas \LCOMax{} e~\LCOMin{} são análogas.
Em ambas, é chamado \linkcutEvert($p$).
Em seguida, é atribuído à variável~$m$ o valor de \linkcutMax($q$) (resp. \linkcutMin($q$)), que corresponde ao nó com maior (resp. menor) peso entre $q$ e a raiz da link cut tree, isto é, $p$.
Então é retornado o valor \LCOFindNode($m$).
Veja os Algoritmos~\ref{Algo:LCOMax} e~\ref{Algo:LCOMin}.
O consumo de tempo das rotinas $\linkcutEvert$ e $\linkcutMax$ é~$\O{\lg m}$ amortizado, enquanto que $\LCOFindNode$ consome tempo esperado $\O{\lg m}$. Assim o consumo de tempo de $\LCOMax$ e $\LCOMin$ é $\O{\lg m}$ amortizado esperado.


\begin{minipage}{0.4\textwidth}
\begin{algorithm}[H]
\caption{\LCOMax($p$, $q$)}
\label{Algo:LCOMax}
\begin{algorithmic}[1]
\State \linkcutEvert($p$)
\State $m$ $\gets$ \linkcutMax($q$)
\State \Return \LCOFindNode($m$)
\end{algorithmic}
\end{algorithm}
\end{minipage}
\hfill
\begin{minipage}{0.4\textwidth}
\begin{algorithm}[H]
\caption{\LCOMin($p$, $q$)}
\label{Algo:LCOMin}
\begin{algorithmic}[1]
\State \linkcutEvert($p$)
\State $m$ $\gets$ \linkcutMin($q$)
\State \Return \LCOFindNode($m$)
\end{algorithmic}
\end{algorithm}
\end{minipage}



A implementação da rotina \LCOMerge{} pode ser vista no Algoritmo~\ref{Algo:LCOMerge}.
O processo de \LCOMerge{} começa chamando a função \linkcutEvert{} invertendo a árvore de modo que o nó~\treapLast($p$) se torne a nova raiz.
Isso é necessário, pois a operação \linkcutAddEdge{} de adição de aresta subsequente assume que \treapLast($p$) é a raiz de sua árvore.
A função \linkcutAddEdge{} é usada para adicionar uma aresta entre~\treapLast($p$) e~\treapFirst($q$), conectando o último nó da ordem cíclica do vértice que contém~$p$ ao primeiro nó da ordem cíclica do vértice que contém~$q$.
Para garantir que a ordem cíclica das arestas seja mantida corretamente, a função \treapJoin{} é chamada para unir as raízes das ABBs associadas aos vértices.
Como \treapFirst{}, \treapLast{}, \treapGetRoot{} e \treapJoin{} consomem tempo esperado~$\O{\lg m}$ e \linkcutEvert{} e \linkcutAddEdge{} consome tempo~$\O{\lg m}$ amortizado, onde $m$ é o número de arestas no grafo corrente, então o consumo de tempo de \LCOMerge{} é~$\O{\lg m}$ esperado amortizado.


\begin{algorithm}[htb]
\caption{\LCOMerge($p$, $q$, $z$)}
\label{Algo:LCOMerge}
\begin{algorithmic}[1]
\State $l$ $\gets$ \treapLast($p$)
\State \linkcutEvert($l$)
\State \linkcutAddEdge($l$, \treapFirst($q$))
\State $T$ $\gets$ \treapJoin(\treapGetRoot($p$), \treapGetRoot($q$))
\State $T$.$\node$ $\gets$ $z$
\end{algorithmic}
\end{algorithm}

A implementação da rotina \LCOSplit{} pode ser vista no Algoritmo~\ref{Algo:LCOSplit}.
Nessa implementação, primeiro, a função \linkcutEvert{} torna \treapLast($p$) a raiz da sua link cut tree, assim o caminho entre \treapFirst($p$) e a raiz corresponde à ordem cíclica de $v$.
Dessa forma \linkcutParent($p$) retorna o sucessor do nó $p$ na ordem cíclica, que é armazenado na variável~$y$ e ao aplicar \linkcutDelEdge($p$) a ordem cíclica de~$v$ é dividida em duas sequências, a primeira de \treapFirst($p$) até $p$ e a segunda de~$y$ até~\treapLast($p$).
Essas sequências correspondem às ordens cíclicas dos vértices resultantes desejados.
Finalmente, a árvore binária de busca que contém o nó~$p$ é então dividida em duas partes: uma que mantém a ordem das arestas até $p$, e outra que contém o restante.
Como \treapLast{} e \treapSplitRight{} consomem tempo esperado~$\O{\lg m}$ e \linkcutEvert{} e \linkcutDelEdge{} consomem tempo~$\O{\lg m}$ amortizado, onde $m$ é o número de arestas no grafo corrente, então o consumo de tempo de \LCOSplit{} é~$\O{\lg m}$ esperado amortizado.

\begin{algorithm}[htb]
\caption{\LCOSplit($p$, $v$, $z$)}
\label{Algo:LCOSplit}
\begin{algorithmic}[1]
\State \linkcutEvert(\treapLast($p$))
\State \linkcutDelEdge($p$)
\State $T$, $T'$ $\gets$ \treapSplitRight($p$)
\State $T$.$\node$ $\gets$ $v$; $T'$.$\node$ $\gets$ $z$
\end{algorithmic}
\end{algorithm}


A implementação da rotina \LCOCycle($p$), descrita no Algoritmo~\ref{Algo:LCOCycle}, começa comparando~$p$ com o último nó da ordem cíclica que o contém.
Caso esses nós sejam iguais, então a função termina imediatamente.
Caso contrário, a função $\linkcutEvert$ é aplicada a \treapLast($p$).
Essa operação faz com que o pai de~$p$ seja o sucessor dele na ordem cíclica.
Então é chamado \linkcutDelEdge($p$) para remover a aresta que liga esses nós.
Em seguida, a função $\linkcutAddEdge$ adiciona uma nova aresta que conecta o último nó da ordem cíclica que contém~$p$ ao primeiro nó dessa ordem, efetivamente ajustando a ordem cíclica.
Finalmente, a árvore binária de busca associada é dividida em partes, reorganizadas, e depois unidas para representar a nova ordem cíclica, garantindo que~$p$ seja, de fato, o último nó na ordem cíclica.
Como \treapFirst{}, \treapLast{}, \treapSplitRight{} e \treapJoin{} consomem tempo esperado~$\O{\lg m}$ e \linkcutEvert{}, \linkcutDelEdge{} e \linkcutAddEdge{} consomem tempo~$\O{\lg m}$ amortizado, onde $m$ é o número de arestas no grafo corrente, então o consumo de tempo de \LCOCycle{} é~$\O{\lg m}$ esperado amortizado.


\begin{algorithm}[h!]
\caption{\LCOCycle($p$)}
\label{Algo:LCOCycle}
\begin{algorithmic}[1]
\State $l$ $\gets$ \treapLast($p$)
\If{$l$ $=$ $p$}
\State\Return
\EndIf
\State \linkcutEvert($l$); \linkcutDelEdge($p$)
\State \linkcutAddEdge($l$, \treapFirst($p$))
\State $T$, $T'$ $\gets$ \treapSplitRight($p$)
\State $T$ $\gets$ \treapJoin($T'$, $T$)
\State $T$.$\node$ $\gets$ $p$
\end{algorithmic}
\end{algorithm}

\newpage
A implementação de \treapPredecessor{} está descrita no Algoritmo~\ref{Algo:treapPredecessor}.
Seu consumo esperado é~$\O{\lg m}$, onde $m$ é o número de arestas no grafo corrente.

\begin{algorithm}[h!]
\caption{\treapPredecessor($p$)}
\label{Algo:treapPredecessor}
\begin{algorithmic}[1]
	\If{$p$ $=$ \treapFirst($p$)}
\State\Return \treapLast($p$)
\EndIf
\If{$p.abb.esq$ $\neq$ $\Nil$}
\State \varname{tmp} $\gets p.abb.esq$
\While {\varname{tmp}.\varname{dir} $\neq$ $\Nil$}
\varname{tmp} $\gets$ \varname{tmp}.\varname{dir}
\EndWhile
\State \Return \varname{tmp}
\EndIf
\State \varname{tmp} $\gets$ $p$
\While {\varname{tmp}.\varname{pai}.\varname{esq} $=$ $tmp$} \varname{tmp} $\gets$ \varname{tmp}.\varname{pai}
\EndWhile
\State \varname{tmp}  $\gets$ \varname{tmp}.\varname{pai}.\varname{esq}
\While {\varname{tmp}.\varname{dir} $\neq$ $\Nil$}
\varname{tmp} $\gets$ \varname{tmp}.\varname{dir}
\EndWhile
\State \Return \varname{tmp}

\end{algorithmic}
\end{algorithm}
