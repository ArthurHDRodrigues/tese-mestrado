%!TeX root=../tese.tex
%("dica" para o editor de texto: este arquivo é parte de um documento maior)
% para saber mais: https://tex.stackexchange.com/q/78101

\chapter[Perguntas frequentes sobre o modelo]{Perguntas frequentes sobre o modelo\footnote{Esta
seção não é de fato um anexo, mas sim um apêndice; ela foi definida desta
forma apenas para servir como exemplo de anexo.}}

\begin{itemize}

\item \textbf{Não consigo decorar tantos comandos!}\\
Use a colinha que é distribuída juntamente com este modelo (\url{gitlab.com/ccsl-usp/modelo-latex/raw/master/pre-compilados/colinha.pdf?inline=false}).

\item \textbf{Por que tantos arquivos?}\\
O preâmbulo \LaTeX{} deste modelo é muito longo; as partes que normalmente não precisam ser modificadas foram colocadas no diretório \texttt{extras}, juntamente com alguns arquivos acessórios.

\item \textbf{As figuras e tabelas são colocadas em lugares ruins.}\\
Veja a discussão a respeito na Seção~\ref{sec:limitations}.

\item \textbf{Estou tendo problemas com caracteres acentuados.}\\
Versões modernas de \LaTeX{} usam UTF-8, mas arquivos antigos podem usar outras codificações (como ISO-8859-1, também conhecido como latin1 ou Windows-1252). Nesses casos, use \textsf{\textbackslash{}usepackage[latin1]\{inputenc\}} no preâmbulo do documento. Você também pode representar os caracteres acentuados usando comandos \LaTeX{}: \textsf{\textbackslash\textquotesingle{}a} para á, \textsf{\textbackslash{}c\{c\}} para cedilha etc., independentemente da codificação usada no texto\footnote{Você pode consultar os comandos desse tipo mais comuns em \url{en.wikibooks.org/wiki/LaTeX/Special_Characters}. Observe que a dica sobre o pingo do i \emph{não} é mais válida atualmente; basta usar \textsf{\textbackslash\textquotesingle{}i}.}.

\item \textbf{Existe algo específico para citações de páginas web?}\\
Biblatex define o tipo ``online'', que deve ser usado para materiais com título, autor etc., como uma postagem ou comentário em um blog, um gráfico ou mesmo uma mensagem de email para uma lista de discussão. Bibtex\index{bibtex}, por padrão, não tem um tipo específico para isso; com ele, normalmente usa-se o campo ``howpublished'' para especificar que se trata de um recurso \textit{online}. Se o que você está citando não é algo determinado com título, autor etc. mas sim um sítio (como uma empresa ou um produto), pode ser mais adequado colocar a referência apenas como nota de rodapé e não na lista de referências; nesses casos, algumas pessoas acrescentam uma segunda lista de referências especificamente para recursos \textit{online} (biblatex\index{biblatex} permite criar múltiplas bibliografias). Já artigos disponíveis \textit{online} mas que fazem parte de uma publicação de formato tradicional (mesmo que apenas \textit{online}), como os anais de um congresso, devem ser citados por seu tipo verdadeiro e apenas incluir o campo ``url'' (não é nem necessário usar o comando \textsf{\textbackslash{}url\{\}}), aceito por todos os tipos de documento do bibtex/biblatex.

\item \textbf{Aparece uma folha em branco entre os capítulos.}\\
Essa característica foi colocada propositalmente, dado que todo capítulo deve (ou deveria) começar em uma página de numeração ímpar (lado direito do documento). Se quiser mudar esse comportamento, acrescente ``openany'' como opção da classe, i.e., \textsf{\textbackslash{}documentclass[openany,\dots]\{book\}}.

\item \textbf{É possível resumir o nome das seções/capítulos que aparece no topo das páginas e no sumário?}\\
Sim, usando a sintaxe \textsf{\textbackslash{}section[mini-titulo]\{titulo enorme\}}. Isso é especialmente útil nas legendas (\textit{captions}\index{Legendas}) das figuras e tabelas, que muitas vezes são demasiadamente longas para a lista de figuras/tabelas.

\item \textbf{Existe algum programa para gerenciar referências em formato bibtex?}\\
Sim, há vários. Uma opção bem comum é o JabRef; outra é usar Zotero\index{Zotero} ou Mendeley\index{Mendeley} e exportar os dados deles no formato .bib.

\item \textbf{Posso usar pacotes \LaTeX{} adicionais aos sugeridos?}\\
Com certeza! Você pode modificar os arquivos o quanto desejar, o modelo serve só como uma ajuda inicial para o seu trabalho.

\item \textbf{Como faço para usar o Makefile (comando make) no Windows?}\\
Lembre-se que a ferramenta recomendada para compilação do documento é o \textsf{latexmk}, então você não precisa do \textsf{make}. Mas, se quiser usá-lo, você pode instalar o MSYS2 (\url{www.msys2.org}) ou o Windows Subsystem for Linux (procure as versões de Linux disponíveis na Microsoft Store). Se você pretende usar algum dos editores sugeridos, é possível deixar a compilação a cargo deles, também dispensando o \textsf{make}.\looseness=-1

\item \textbf{Como eu faço para...}\\
Leia os comentários dos arquivos ``tese.tex'' e outros que compõem este modelo, além do tutorial (Capítulo~\ref{chap:tutorial}) e dos exemplos do Capítulo~\ref{chap:exemplos}; é provável que haja uma dica neles ou, pelo menos, a indicação da \textit{package} relacionada ao que você precisa.

\end{itemize}
