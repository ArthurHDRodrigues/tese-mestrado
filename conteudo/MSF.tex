
\chapter{O problema MSF em grafos planares}








\section{Otimizações}



%\if FALSE
\chapter{Árvores geradoras mínimas}

Retomemos o problema da floresta maximal de peso mínimo em grafos dinâmicos inicialmente apresentado na Seção~\ref{sec:Motivação} e que é o segundo problema que vamos estudar. Ele consiste na busca por uma implementação tão eficiente quanto possível para a seguinte biblioteca:
\begin{itemize}
\item \MSFCreate($n$): cria e devolve um grafo dinâmico com $n$ vértices isolados;
\item \MSFaddEdge($G$, $u$, $v$, $w$): adiciona a aresta $uv$ com peso $w$ ao grafo dinâmico~$G$;
\item \MSFdelEdge($G$, $u$, $v$): remove a aresta $uv$ de $G$; e
\item \MSFweight($G$): devolve o peso de uma MSF de $G$.
\end{itemize}

A solução que estudaremos foi desenvolvida por Holm, de Lichtenberg e Thorup~\cite{poly_log} em 2001. Ela é determinística e possui consumo de tempo amortizado~$\O{\lg^2 n}$ para cada uma das operações listadas acima.

Podemos adaptar a solução para o problema de conexidade em grafos dinâmicos apresentado na Seção~\ref{sec:connDG} para obter uma solução do problema de MSF dinâmico para o caso decremental, isto é, o caso em que a única atualização aplicada ao grafo é a de remoção de arestas do grafo~\cite{poly_log}. Primeiro, evidentemente, a floresta maximal $F_{\lceil \lg n \rceil}$ mantida nessa solução deve ser de peso mínimo. Para tal, ao procurar uma substituta para alguma aresta da árvore na rotina~\dymGraphReplace{}, basta percorrer as arestas de nível $i$ em ordem crescente de peso. A prova de como essa simples mudança funciona será apresentada na dissertação.

A operação de adição de aresta exigirá um pouco mais da nossa estrutura de dados. Dada uma MSF $F$ de um grafo dinâmico~$G$, ao adicionar uma aresta $uv$ com peso $w$ a $G$, se os vértices~$u$ e~$v$ não estiverem na mesma componente conexa de~$G$, então basta adicionar a aresta a~$F$. No caso em que os vértices estão conectados, então precisamos verificar se $uv$ substitui alguma aresta de $F$. Para tal, é necessário considerar o caminho em~$F$ que liga $u$ a~$v$, verificando se~$w$ é menor do que o peso de alguma aresta nesse caminho. Se existir tal aresta com peso maior do que~$w$, então a removemos de~$F$ e adicionamos $uv$ a~$F$. Caso contrário, somente adicionamos $uv$ a $G$.

Notemos que é necessário ter informação sobre a conexidade do grafo dinâmico, o que já é resolvido pela biblioteca de grafos dinâmicos e Euler tour trees como visto nas Seções anteriores. No entanto, Euler tour trees não permitem percorrer caminhos da árvore de forma eficiente, pois muitas vezes o caminho entre dois vértices arbitrários estará fragmentado ao longo da sequência Euleriana que representa a árvore. Para obter caminhos de forma eficiente, usaremos uma estrutura de dados chamada \defi{árvores topológicas} (top tree)~\cite{AHLTMinDiameter}, que será introduzida adequadamente no texto completo de dissertação.

