\chapter{Conclusão}
\label{sec:conclusao}

\section{Análise de estudo experimentais}
\label{sec:conclusao:xp}

Existem alguns estudos experimentais envolvendo alguns dos algoritmos mencionados acima.
Em $1997$, Alberts, Cattaneo e Italiano~\cite{EmpiricalStudy1997} fizeram um estudo experimental envolvendo um algoritmo simples baseado em esparsificação proposto por Eppstein et al.~\cite{Eppstein1992SparsificationaTF} e o algoritmo de Henzinger e King~\cite{HenzingerKing}, sem a melhoria de Henzinger e Thorup~\cite{HenzingerThorup}.
O experimento deles incluiu também uma simplificação proposta por eles do algoritmo de Henzinger e King.
Essa simplificação possui consumo de tempo assintótico pior do que o algoritmo original de Henzinger e King, porém ainda assim ela se saiu bem nos experimentos com grafos aleatórios.
Para grafos não aleatórios, o algoritmo baseado em espasificação se saiu melhor para instâncias com poucas operações de atualização, enquanto que o algoritmo original de Henzinger e King se saiu melhor com mais operações de atualização.
As implementações desenvolvidas para estes experimentos foram feitas em C++, usando a plataforma Leda~\cite{LEDA}.

Em $2002$, Raj Iyer, David Karger, Hariharan Rahul e Mikkel Thorup~\cite{EmpiricalStudy2002} usaram como base o estudo de Alberts et al. para comparar o então recente algoritmo de Holm, de Lichtenberg e Thorup~\cite{poly_log} com o algoritmo de Henzinger e King, considerando algumas variantes destes dois algoritmos no seu estudo.
Entre outras coisas, os autores mostram que uma das variantes de Henzinger e King considerada tem consumo de tempo $\O{\lg^2 n}$ por operação de atualização.
As heurísticas propostas pelos autores para o algoritmo de Holm, de Lichtenberg e Thorup não invalidam as análises do consumo de tempo do algoritmo original.
Uma das heurísticas usa a ideia do algoritmo de Henzinger e King de fazer um sorteio aleatório das primeiras arestas que são testadas como possíveis substitutas, em vez de percorrer sequencialmente as arestas candidatas e ir fazendo os rebaixamentos. A segunda heurística é inspirada na heurística analisada no estudo de Alberts e outros, de truncar o número de níveis usados.
O resultado do estudo experimental em relação a estes algoritmos é que a versão do algoritmo de Holm, de Lichtenberg e Thorup com as duas heurísticas implementadas se sai melhor que o algoritmo original de Henzinger e King.
As implementações desenvolvidas para estes experimentos também foram feitas em C++ usando a plataforma Leda~\cite{LEDA}.

Mais recentemente, Chen e outros apresentaram um outro estudo experimental, envolvendo duas heurísticas que eles propuseram e supostamente o algoritmo de Henzinger e King, entre outros.
A nossa intenção inicial era incluir o algoritmo de Holm, de Lichtenberg e Thorup nesse estudo experimental. Este foi um dos motivos que nos levou a implementar o algoritmo deles em python, que é a linguagem usada neste estudo experimental. No entanto, durante a fase de testes usando como base o estudo experimental de Chen e outros, notamos que a implementação do algoritmo de Henzinger e King incluída no estudo tratava-se da simplificação do algoritmo de Henzinger e King usada no estudo de Alberts e outros, para a qual a análise original não se aplica. Na verdade, a implementação de Chen e outros desta simplificação também executa a escolha aleatória das arestas candidatas a substitutas de uma maneira pouco eficiente, o que resulta em uma implementação com consumo de tempo muito pior que o consumo do algoritmo original de Henzinger e King. Ou seja, não é de fato uma comparação entre as heurísticas deles e o algoritmo de Henzinger e King, como é dito no artigo. Após percebermos estes problemas, desistimos de estender este estudo experimental. Ademais, nesse meio tempo, também encontramos o trabalho de Iyer, Karger, Rahul e Thorup que já apresenta um excelente estudo comparativo entre o algoritmo de Henzinger e King e o algoritmo de Holm, de Lichtenberg e Thorup. 


