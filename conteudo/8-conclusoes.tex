\chapter{Comentários finais}
\label{sec:conclusao}

\section{Análise de estudo experimentais para o problema de conexidade em grafos dinâmicos}
\label{sec:conclusao:xp}

Em $1997$, Alberts, Cattaneo e Italiano~\cite{EmpiricalStudy1997} fizeram um estudo experimental envolvendo um algoritmo simples baseado em esparsificação proposto por Eppstein et al.~\cite{Eppstein1992SparsificationaTF} e o algoritmo de Henzinger e King~\cite{HenzingerKing}, sem a melhoria de Henzinger e Thorup~\cite{HenzingerThorup}.


O algoritmo de Henzinger e King, assim como Holm, de Lichtenberg e Thorup~\cite{poly_log} estudado no Capítulo~\ref{sec:connDG}, associa um nível entre $0$ e $\ceil{\lg n}$ à cada aresta.
Nesse experimento os autores propuseram heuristicamente truncar essa estrutura de níveis, mantendo assim somente os níveis entre~$k$ e~$\ceil{\lg n}$, onde $k$ é um parâmetro pré-definido.
Nessa simplificação, no nível~$k$, ao invés de fazer a busca por aresta substituíra proposta por Henzinger e King, o algoritmo faz uma busca exaustiva. Dessa forma, essa simplificação possui consumo de tempo assintótico pior do que o algoritmo original de Henzinger e King, porém ainda assim ela se saiu bem nos experimentos com grafos aleatórios.

Para grafos não aleatórios, o algoritmo baseado em esparsificação se saiu melhor para instâncias com poucas operações de atualização, enquanto que o algoritmo original de Henzinger e King se saiu melhor com mais operações de atualização.
As implementações desenvolvidas para estes experimentos foram feitas em C++, usando a plataforma Leda~\cite{LEDA}.

Em $2002$, Raj Iyer, David Karger, Hariharan Rahul e Mikkel Thorup~\cite{EmpiricalStudy2002} usaram como base o estudo de Alberts et al. para comparar o então recente algoritmo de Holm, de Lichtenberg e Thorup~\cite{poly_log} com o algoritmo de Henzinger e King, considerando algumas variantes destes dois algoritmos no seu estudo.
Entre outras coisas, os autores mostram que uma das variantes de Henzinger e King considerada tem consumo de tempo $\O{\lg^2 n}$ por operação de atualização.

Os autores também propõe duas heurísticas para o algoritmo de Holm, de Lichtenberg e Thorup. Essas heurísticas não invalidam as análises do consumo de tempo do algoritmo original.

A primeira heurística usa a ideia do algoritmo de Henzinger e King de fazer um sorteio aleatório das primeiras arestas que são testadas como possíveis substitutas, em vez de percorrer sequencialmente as arestas candidatas e ir fazendo os rebaixamentos.

A segunda heurística é inspirada na heurística analisada no estudo de Alberts e outros, de truncar o número de níveis usados.

O resultado do estudo experimental em relação a estes algoritmos é que a versão do algoritmo de Holm, de Lichtenberg e Thorup com as duas heurísticas implementadas se sai melhor que o algoritmo original de Henzinger e King.
As implementações desenvolvidas para estes experimentos também foram feitas em C++ usando a plataforma Leda~\cite{LEDA}.


Em $2019$, David Fernández-Baca e Lei Liu~\cite{xp-Phylogeny}\TODO{Continuar}


Mais recentemente, Chen et al.~\cite{QC22} apresentaram um outro estudo experimental, envolvendo duas heurísticas que eles propuseram e supostamente o algoritmo de Henzinger e King, entre outros.

A nossa intenção inicial era incluir o algoritmo de Holm, de Lichtenberg e Thorup nesse estudo experimental. Este foi o principal motivo que nos levou a implementar o algoritmo deles em Python~3, que é a linguagem usada neste estudo experimental.

No entanto, durante a fase de testes usando como base o estudo experimental de Chen e outros, notamos que a implementação do algoritmo de Henzinger e King incluída no estudo tratava-se de sua versão simplificada com níveis truncados usada no estudo de Alberts e outros, para a qual a análise original não se aplica.
Na verdade, a implementação de Chen e outros desta simplificação também executa a escolha aleatória das arestas candidatas a substitutas de uma maneira pouco eficiente, o que resulta em uma implementação com consumo de tempo muito pior que o consumo do algoritmo original de Henzinger e King.
Ou seja, não é de fato uma comparação entre as heurísticas deles e o algoritmo de Henzinger e King, como é dito no artigo.
Após percebermos estes problemas, desistimos de estender este estudo experimental.
Ademais, nesse meio tempo, também encontramos o trabalho de Iyer, Karger, Rahul e Thorup~\cite{EmpiricalStudy2002} que já apresenta um excelente estudo comparativo entre o algoritmo de Henzinger e King e o algoritmo de Holm, de Lichtenberg e Thorup. 
\begin{comment}
\section{Trabalhos futuros}

Um estudo experimental com boas praticas como~\cite{guideXP1999}
\end{comment}
