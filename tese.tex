% Arquivo LaTeX de exemplo de dissertação/tese a ser apresentada à CPG do IME-USP
%
% Criação: Jesús P. Mena-Chalco
% Revisão: Fabio Kon e Paulo Feofiloff
% Adaptação para UTF8, biblatex e outras melhorias: Nelson Lago
%
% Except where otherwise indicated, these files are distributed under
% the MIT Licence. The example text, which includes the tutorial and
% examples as well as the explanatory comments in the source, are
% available under the Creative Commons Attribution International
% Licence, v4.0 (CC-BY 4.0) - https://creativecommons.org/licenses/by/4.0/


%%%%%%%%%%%%%%%%%%%%%%%%%%%%%%%%%%%%%%%%%%%%%%%%%%%%%%%%%%%%%%%%%%%%%%%%%%%%%%%%
%%%%%%%%%%%%%%%%%%%%%%%%%%%%%%% PREÂMBULO LaTeX %%%%%%%%%%%%%%%%%%%%%%%%%%%%%%%%
%%%%%%%%%%%%%%%%%%%%%%%%%%%%%%%%%%%%%%%%%%%%%%%%%%%%%%%%%%%%%%%%%%%%%%%%%%%%%%%%

% A opção twoside (frente-e-verso) significa que a aparência das páginas pares
% e ímpares pode ser diferente. Por exemplo, as margens podem ser diferentes ou
% os números de página podem aparecer à direita ou à esquerda alternadamente.
% Mas nada impede que você crie um documento "só frente" e, ao imprimir, faça
% a impressão frente-e-verso.
%
% Aqui também definimos a língua padrão do documento
% (a última da lista) e línguas adicionais.
%\documentclass[12pt,twoside,brazilian,english]{book}
\documentclass[12pt,twoside,english,brazilian]{book}

% Ao invés de definir o tamanho das margens, vamos definir os tamanhos do
% texto, do cabeçalho e do rodapé, e deixamos a package geometry calcular
% o tamanho das margens em função do tamanho do papel. Assim, obtemos o
% mesmo resultado impresso, mas com margens diferentes, se o tamanho do
% papel for diferente.
\usepackage[a4paper]{geometry}

\geometry{
  textwidth=152mm,
  hmarginratio=12:17, % 24:34 -> com papel A4, 24mm + 152mm + 34mm = 210mm
  textheight=237mm,
  vmarginratio=8:7, % 32:28 -> com papel A4, 32mm + 237mm + 28mm = 297mm
  headsep=11mm, % distância entre a base do cabeçalho e o texto
  headheight=21mm, % qualquer medida grande o suficiente, p.ex., top - headsep
  footskip=10mm,
  marginpar=20mm,
  marginparsep=5mm,
}

\usepackage{nolbreaks}
% Vários pacotes e opções de configuração genéricos; para personalizar o
% resultado, modifique estes arquivos.
%%%%%%%%%%%%%%%%%%%%%%%%%%%%%%%%%%%%%%%%%%%%%%%%%%%%%%%%%%%%%%%%%%%%%%%%%%%%%%%%
%%%%%%%%%%%%%%%%%%%%%%% CONFIGURAÇÕES E PACOTES BÁSICOS %%%%%%%%%%%%%%%%%%%%%%%%
%%%%%%%%%%%%%%%%%%%%%%%%%%%%%%%%%%%%%%%%%%%%%%%%%%%%%%%%%%%%%%%%%%%%%%%%%%%%%%%%

% Vários comandos auxiliares para o desenvolvimento de packages e classes;
% aqui, usamos em alguns comandos de formatação e condicionais.
\usepackage{etoolbox}
\usepackage{xstring}
\usepackage{xparse}
\usepackage{regexpatch}

% Esta package permite detectar XeTeX, LuaTeX e pdfTeX, mas pode não estar
% disponível em todas as instalações de TeX.
%\usepackage{iftex}
% Por conta disso, usaremos estas (que não detectam pdfTeX):
\usepackage{ifxetex}
\usepackage{ifluatex}

\newbool{unicodeengine}
\ifboolexpr{bool{xetex} or bool{luatex}}
  {\booltrue{unicodeengine}}
  {\boolfalse{unicodeengine}}

% Detecta se estamos produzindo um arquivo PDF ou DVI (lembrando que tanto
% pdfTeX quanto LuaTeX podem gerar ambos)
\usepackage{ifpdf}

% Permite criar "headed lists", ou seja, "listas" de elementos que vão
% aparecendo ao longo do documento (como, por exemplo, teoremas). Podem ser
% também citações a autores específicos, seções de um documento que está
% sendo analisado etc. Precisa ser carregado antes das definições de fontes.
\usepackage{amsthm}

% Outras packages "padrão" da AMS que, embora tratem especificamente
% de recursos matemáticos, são praticamente obrigatórias
\usepackage{amsmath}
\usepackage{amssymb}
\usepackage{mathtools}

\ifunicodeengine
  % Não é preciso carregar fontenc e inputenc com LuaTeX e XeTeX!
  \usepackage{fontspec}
  \usepackage{unicode-math}
\else
  % "fontenc" é um parâmetro interno do LaTeX. Com pdfLaTeX, o fontenc
  % default é OT1, mas ele tem algumas limitações; a mais importante é
  % que, com ele, palavras acentuadas não podem ser hifenizadas. Por
  % conta disso, quase todos os documentos LaTeX utilizam o fontenc T1.
  % A escolha do fontenc tem consequências para as fontes que podem ser
  % usadas no documento; hoje em dia T1 tem mais opções de qualidade,
  % então não se perde nada.
  \usepackage[T1]{fontenc}
  \usepackage[utf8]{inputenc}
\fi

% Internacionalização dos nomes das seções ("chapter" X "capítulo" etc.),
% hifenização e outras convenções tipográficas. babel deve ser um dos
% primeiros pacotes carregados. É possível passar a língua do documento
% como parâmetro aqui, mas já fizemos isso ao carregar a classe, no início
% do documento.
\usepackage{babel}

% É possível personalizar as palavras-chave que babel utiliza, por exemplo:
%\addto\extrasbrazil{\renewcommand{\chaptername}{Chap.}}
% Com BibTeX, isso vale também para a bibliografia; com BibLaTeX, é melhor
% usar o comando "DefineBibliographyStrings".

% Para línguas baseadas no alfabeto latino, como o inglês e o português,
% o pacote babel funciona muito bem, mas com outros alfabetos ele às vezes
% falha. Por conta disso, o pacote polyglossia foi criado para substituí-lo.
% polyglossia só funciona com LuaTeX e XeTeX; como babel também funciona com
% esses sistemas, provavelmente não há razão para usar polyglossia, mas é
% possível que no futuro esse pacote se torne o padrão.
%\usepackage{polyglossia}
%\setdefaultlanguage{brazil}
%\setotherlanguage{english}

% Alguns pacotes (espeficicamente, tikz) usam, além de babel, este pacote
% como auxiliar para a tradução de palavras-chave, como os meses do ano.
\usepackage{translator}

% microajustes no tamanho das letras, espaçamento etc. para melhorar
% a qualidade visual do resultado. LaTeX tradicional não dá suporte a
% nenhum tipo de microajuste; pdfLaTeX dá suporte a todos. LuaLaTeX
% e XeLaTeX dão suporte a alguns:
%
% * expansion não funciona com XeLaTeX
% * tracking não funciona com XeLaTeX; é possível obter o mesmo resultado
%   com a opção "LetterSpace" do pacote fontspec, mas a configuração é
%   totalmente manual. Por padrão, aumenta o afastamento entre caracteres
%   nas fontes "small caps"; o resultado não se presta ao uso na
%   bibliografia ou citações, então melhor desabilitar.
% * kerning e spacing só funcionam com pdfLaTex; ambas são funções
%   consideradas experimentais e nem sempre produzem resultados vantajosos.

\newcommand\microtypeopts{
  protrusion=true,
  tracking=false,
  kerning=false,
  spacing=false
}

% TeXLive 2018 inclui a versão 2.7a da package microtype e a versão
% 1.07 de luatex. Essa combinação faz aparecer um bug:
% https://tex.stackexchange.com/questions/476740/microtype-error-with-lualatex-attempt-to-call-field-warning-a-nil-value
% Aqui, aplicamos a solução sugerida, que não tem "contra-indicações".
\ifluatex
  \usepackage{luatexbase}
\fi

\ifxetex
  \usepackage[expansion=false,\microtypeopts]{microtype}
\else
  \usepackage[expansion=true,\microtypeopts]{microtype}
\fi

% Alguns "truques" (sujos?) para minimizar over/underfull boxes.
%
% Para fazer um texto justificado, é preciso modificar o tamanho dos espaços
% em cada linha para mais ou para menos em relação ao seu tamanho ideal. Para
% escolher as quebras de linha, TeX vai percorrendo o texto procurando lugares
% possíveis para quebrar as linhas considerando essa flexibilidade mas dentro
% de um certo limite mínimo/máximo. Nesse processo, ele associa a cada possível
% linha o valor badness, que é o nível de distorção do tamanho dos espaços
% daquela linha em relação ao ideal, e ignora opções que tenham badness muito
% grande (esse limite é dado por \tolerance). Depois de encontradas todas as
% possíveis quebras de linha e a badness de cada uma, TeX calcula as penalties
% e demerits de cada possibilidade e escolhe a solução que minimiza o demerit
% total do parágrafo.
%
% Para cada fonte, o espaço em TeX tem um tamanho ideal, um tamanho mínimo e um
% tamanho máximo. TeX nunca reduz um espaço para menos que o mínimo da fonte,
% mas pode aumentá-lo para mais que o máximo. Se os espaços de uma linha ficam
% com o tamanho ideal, a badness da linha é 0; se o tamanho é
% reduzido/aumentado 50% do mínimo/máximo, a badness da linha é 12; se o
% tamanho é reduzido/aumentado para o mínimo/máximo, a badness é 100. Se esse
% aumento for de 30% além do máximo, a badness da linha é 200; se for de 45%
% além do máximo, a badness é 300; se for de 60% além do máximo, a badness é
% 400; se for de 100% além do máximo, a badness é 800.
%
% A \tolerance default é 200; aumentar para, digamos, 300 ou 400, permite que
% TeX escolha parágrafos com maior variação no espaçamento. No entanto, no
% cálculo de demerits, a badness de cada linha é elevada ao quadrado, então TeX
% geralmente prefere escolher outras opções no lugar de uma linha ruim. Por
% exemplo, órfãs/viúvas têm demerit de 22.500 e dois hífens seguidos têm
% demerit de 10.000 e, portanto, não é surpreendente que a maioria dos
% parágrafos tenha demerits abaixo de 40.000, quase todos abaixo de 100.000 e
% praticamente nenhum acima de 1.000.000. Isso significa que, para a grande
% maioria dos parágrafos, aumentar \tolerance não faz diferença: uma linha com
% badness 400 (correspondendo a demerit de 160.000) nunca será efetivamente
% escolhida se houver qualquer outra opção com badness menor. Também fica claro
% que não há muita diferença real entre definir \tolerance como 800 ou 9.999.
%
% O problema muda de figura se TeX não consegue encontrar uma solução. Isso
% pode acontecer em dois casos: (1) o parágrafo tem ao menos uma linha que não
% pode ser quebrada com badness < 10.000 e (2) o parágrafo tem ao menos uma
% linha que não pode ser quebrada com badness < tolerance (mas essa badness é
% menor que 10.000).
%
% No primeiro caso, se houver várias possibilidades de linhas que não podem ser
% quebradas, TeX não vai ser capaz de compará-las e escolher a melhor: todas
% têm a badness máxima (10.000) e, portanto, a que gerar menos deméritos no
% restante do parágrafo será a escolhida. Na realidade, no entanto, essas
% linhas *não* são igualmente ruins entre si, o que pode levar TeX a fazer uma
% má escolha. Para evitar isso, TeX tenta novamente aplicando
% \emergencystretch, que "faz de conta" que o tamanho máximo ideal dos espaços
% da linha é maior que o definido na fonte. Isso reduz a badness de todas as
% linhas, o que soa parecido com aumentar \tolerance. Há três diferenças, no
% entanto: (1) essa mudança só afeta os parágrafos que falharam; (2) soluções
% que originalmente teriam badness = 10.000 (e, portanto, seriam vistas como
% equivalentes) podem ser avaliadas e comparadas entre si; e (3) como a badness
% de todas as linhas diminui, a possibilidade de outras linhas que
% originalmente tinham badness alta serem escolhidas aumenta. Esse último ponto
% significa que \emergencystretch pode fazer TeX escolher linhas mais
% espaçadas, fazendo o espaçamento do parágrafo inteiro aumentar e, portanto,
% tornando o resultado mais homogêneo mesmo com uma linha particularmente ruim.
%
% É esse último ponto que justifica o uso de \emergencystretch no segundo caso
% também: apenas aumentar a tolerância, nesse caso, poderia levar TeX a
% diagramar uma linha ruim em meio a um parágrafo bom, enquanto
% \emergencystretch pode fazer TeX aumentar o espaçamento de maneira geral no
% parágrafo, minimizando o contraste da linha problemática com as demais.
% Colocando a questão de outra maneira, aumentar \tolerance para lidar com
% esses parágrafos problemáticos pode fazê-los ter uma linha especialmente
% ruim, enquanto \emergencystretch pode dividir o erro entre várias linhas.
% Assim, definir \tolerance em torno de 800 parece razoável: no caso geral,
% não há diferença e, se um desses casos difíceis não pode ser resolvido com
% uma linha de badness até 800, \emergencystretch deve ser capaz de gerar um
% resultado igual ou melhor.
%
% Penalties & demerits: https://tex.stackexchange.com/a/51264
% Definições (fussy, sloppy etc.): https://tex.stackexchange.com/a/241355
% Mais definições (hfuzz, hbadness etc.): https://tex.stackexchange.com/a/50850
% Donald Arseneau defendendo o uso de \sloppy: https://groups.google.com/d/msg/comp.text.tex/Dhf0xxuQ66E/QTZ7aLYrdQUJ
% Artigo detalhado sobre \emergencystretch: https://www.tug.org/TUGboat/tb38-1/tb118wermuth.pdf
% Esse artigo me leva a crer que algo em torno de 1.5em é suficiente

\tolerance=800
\hyphenpenalty=100 % Default 50; se o texto é em 2 colunas, 50 é melhor
\setlength{\emergencystretch}{2.5em}

% Não gera warnings para Overfull menor que 0.5pt
\hfuzz=.5pt
\vfuzz\hfuzz

% Não gera warnings para Underfull com badness < 1000
\hbadness=1000
\vbadness=1000

% LaTeX às vezes coloca notas de rodapé logo após o final do texto da
% página ao invés de no final da página; este pacote evita isso.
\usepackage[bottom]{footmisc}

% Se uma página está vazia, não imprime número de página ou cabeçalho
\usepackage{emptypage}

% Espaçamento entre linhas configurável (\singlespacing, \onehalfspacing etc.)
\usepackage{setspace}

% Carrega nomes de cores disponíveis (podem ser usados com hyperref e listings)
\usepackage[usenames,svgnames,dvipsnames]{xcolor}

% LaTeX define os comandos "MakeUppercase" e "MakeLowercase", mas eles têm
% algumas limitações; esta package define os comandos MakeTextUppercase e
% MakeTextLowercase que resolvem isso.
\usepackage{textcase}

% Normalmente, LaTeX faz o final da página terminar sempre no mesmo lugar
% (exceto no final dos capítulos). Esse padrão pode ser ativado explicitamente
% com o comando "\flushbottom". Mas se, por alguma razão, o volume de texto na
% página é "pequeno", essa página vai ter espaços verticais artificialmente
% grandes. Uma solução para esse problema é modificar o padrão para
% "\raggedbottom"; isso permite que as páginas terminem em lugares diferentes.
% Outra opção é corrigir manualmente cada página problemática, por exemplo
% com o comando "\enlargethispage".
%\raggedbottom

% Por padrão, LaTeX coloca uma espaço aumentado após sinais de pontuação;
% Isso não é tão bom quanto alguns TeX-eiros defendem :) .
% Esta opção desabilita isso e, consequentemente, evita problemas com
% "id est" (i.e.) e "exempli gratia" (e.g.)
\frenchspacing

% LaTeX procura por arquivos adicionais no diretório atual e nos diretórios
% padrão do sistema. Assim, é preciso usar caminhos relativos para incluir
% arquivos de subdiretórios: "\input{diretorio/arquivo}". No entanto, há
% duas limitações:
%
% 1. É necessário dizer "\input{diretorio/arquivo} mesmo quando o arquivo
%    que contém esse comando já está dentro do subdiretório.
%
% 2. Isso não deve ser usado para packages ("\usepackage{diretorio/package}"),
%    embora na prática funcione.
%
% O modo recomendado de resolver esses problemas é modificando o arquivo
% texmf.cnf ou a variável de ambiente TEXINPUTS ou colocando os arquivos
% compartilhados na árvore TEXMF (geralmente, no diretório texmf dentro do
% diretório do usuário), o que é um tanto complicado para usuários menos
% experientes.
%
% O primeiro problema pode ser solucionado também com a package import,
% mas não há muita vantagem pois é preciso usar outro comando no lugar de
% "\input". O segundo problema é mais importante, pois torna muito difícil
% colocar packages adicionais em um diretório separado. Para contorná-lo,
% vamos usar um truque que é suficiente para nossa necessidade, embora
% *não* seja normalmente recomendado.
%\usepackage{import}

\newcommand\dowithsubdir[2]{
    \csletcs{@oldinput@path}{input@path}
    \csappto{input@path}{{#1}}
    #2
    \csletcs{input@path}{@oldinput@path}
}

%%%%%%%%%%%%%%%%%%%%%%%%%%%%%%%%%%%%%%%%%%%%%%%%%%%%%%%%%%%%%%%%%%%%%%%%%%%%%%%%
%%%%%%%%%%%%%%%%%%%%%%%%%%%%%%%%%%% LÍNGUAS %%%%%%%%%%%%%%%%%%%%%%%%%%%%%%%%%%%%
%%%%%%%%%%%%%%%%%%%%%%%%%%%%%%%%%%%%%%%%%%%%%%%%%%%%%%%%%%%%%%%%%%%%%%%%%%%%%%%%

\makeatletter
\ExplSyntaxOn

% We need to have at least some variant of Portuguese and of English
% loaded to generate the abstract/resumo, palavras-chave/keywords etc.
% We will make sure that both languages are present in the class options
% list by adding them if needed. With this, these options become global
% and therefore are seen by all packages (among them, babel).
%
% babel traditionally uses "portuguese", "brazilian", "portuges", or
% "brazil" to support the Portuguese language, using .ldf files. babel
% is also in the process of implementing a new scheme, using .ini
% files, based on the concept of "locales" instead of "languages". This
% mechanism uses the names "portuguese-portugal", "portuguese-brazil",
% "portuguese-pt", "portuguese-br", "portuguese", "brazilian", "pt",
% "pt-PT", and "pt-BR" (i.e., neither "portuges" nor "brazil"). To avoid
% compatibility problems, let's stick with "brazilian" or "portuguese"
% by substituting portuges and brazil if necessary.

\NewDocumentCommand\@IMEportugueseAndEnglish{m}{

  % Make sure any instances of "portuges" and "brazil" are replaced
  % by "portuguese" e "brazilian"; other options are unchanged.
  \seq_gclear_new:N \l_tmpa_seq
  \seq_gclear_new:N \l_tmpb_seq
  \seq_gset_from_clist:Nc \l_tmpa_seq {#1}

  \seq_map_inline:Nn \l_tmpa_seq{
    \def\@tempa{##1}
    \ifstrequal{portuges}{##1}
      {
        \GenericInfo{sbc2019}{}{Substituting~language~portuges~->~portuguese}
        \def\@tempa{portuguese}
      }
      {}
    \ifstrequal{brazil}{##1}
      {
        \GenericInfo{}{Substituting~language~brazil~->~brazilian}
        \def\@tempa{brazilian}
      }
      {}
    \seq_gput_right:NV \l_tmpb_seq {\@tempa}
  }

  % Remove the leftmost duplicates (default is to remove the rightmost ones).
  % Necessary in case the user did "portuges,portuguese", "brazil,brazilian"
  % or some variation: When we substitute the language, we end up with the
  % exact same language twice, which may mess up the main language selection.
  \seq_greverse:N \l_tmpb_seq
  \seq_gremove_duplicates:N \l_tmpb_seq
  \seq_greverse:N \l_tmpb_seq

  % If the user failed to select some variation of English and Portuguese,
  % we add them here. We also remember which ones of portuguese/brazilian,
  % english/american/british etc. were selected.
  \exp_args:Nnx \regex_extract_all:nnNTF
    {\b(portuguese|brazilian)\b}
    {\seq_use:Nn \l_tmpb_seq {,}}
    \l_tmpa_tl
    {
      \tl_reverse:N \l_tmpa_tl
      \xdef\@IMEpt{\tl_head:N \l_tmpa_tl}
    }
    {
      \seq_gput_left:Nn \l_tmpb_seq {brazilian}
      \gdef\@IMEpt{brazilian}
    }

  \exp_args:Nnx \regex_extract_all:nnNTF
    {\b(english|american|USenglish|canadian|british|UKenglish|australian|newzealand)\b}
    {\seq_use:Nn \l_tmpb_seq {,}}
    \l_tmpa_tl
    {
      \tl_reverse:N \l_tmpa_tl
      \xdef\@IMEen{\tl_head:N \l_tmpa_tl}
    }
    {
      \seq_gput_left:Nn \l_tmpb_seq {english}
      \gdef\@IMEen{english}
    }

  \exp_args:Nc \xdef {#1} {\seq_use:Nn \l_tmpb_seq {,}}
}


% https://tex.stackexchange.com/a/43541
% This message is part of a larger thread that discusses some
% limitations of this method, but it is enough for us here.
\def\@getcl@ss#1.cls#2\relax{\def\@currentclass{#1}}
\def\@getclass{\expandafter\@getcl@ss\@filelist\relax}
\@getclass

% The three class option lists we need to update: \@unusedoptionlist,
% \@classoptionslist and one of \opt@book.cls, \opt@article.cls etc.
% according to the current class. Note that beamer.cls (and maybe
% others) does not use \@unusedoptionlist; with it, we incorrectly
% add "english,brazilian" to \@unusedoptionlist, but that does not
% cause problems.
\@IMEportugueseAndEnglish{@unusedoptionlist}
\@IMEportugueseAndEnglish{@classoptionslist}
\@IMEportugueseAndEnglish{opt@\@currentclass .cls}

\ExplSyntaxOff
\makeatother

% Babel permite definir a língua ou línguas usadas no documento e deve
% ser um dos primeiros pacotes a serem carregados. É possível definir
% as línguas como parâmetro aqui, mas já fizemos isso ao carregar a
% classe, no início do documento.
%
% A escolha da língua afeta quatro coisas:
%
% 1. A internacionalização das palavras criadas automaticamente, como
%    "Capítulo/Chapter", "Sumário/Table of Contents" etc. - babel chama
%    essas palavras de "captions";
%
% 2. A hifenização das palavras;
%
% 3. Algumas convenções tipográficas. Por exemplo, em francês é usual
%    acrescentar um espaço antes de caracteres como "?" e "!"; línguas
%    diferentes usam caracteres diferentes para as aspas tipográficas;
%    com algumas línguas asiáticas, pode ser necessário utilizar uma
%    fonte diferente etc.;
%
% 4. Atalhos (shorthands) - algumas línguas definem "atalhos" (shorthands"),
%    ou seja, tratam alguns caracteres como comandos especiais; por exemplo,
%    em francês o espaço que é colocado antes da exclamação funciona porque
%    o caracter "!" é, na verdade, um comando.
%
%%%% MUDANDO A LÍNGUA E HIFENIZAÇÃO %%%%
%
% Cada documento tem uma língua padrão; quando usamos pequenos trechos em
% outra língua, como por exemplo em citações, queremos alterar apenas os
% aspectos 2, 3 e 4; nesse caso, a troca da língua deve ser feita com
% \foreignlanguage{língua}{texto} ou com \begin{otherlanguage*}{língua}.
% Para alterar todos os quatro aspectos, deve-se usar \selectlanguage{língua}
% (que altera a língua padrão a partir desse ponto) ou
% \begin{otherlanguage}{língua}{texto}, que faz a alteração apenas até
% \end{otherlanguage}. Se você quiser apenas desabilitar a hifenização de
% um trecho de texto, pode usar \begin{hyphenrules}{nohyphenation}.
% Finalmente, com \babeltags é possível definir comandos curtos como
% "\textbr" (para "brazilian") que são equivalentes a \foreignlanguage.
%
%%%% PERSONALIZANDO CAPTIONS %%%%
%
% É possível personalizar os captions. Para versões de babel a partir
% de 3.51 (lançada em outubro de 2020), basta fazer
% \setlocalecaption{english}{contents}{Table of Contents}. Com versões
% anteriores, por razões históricas há dois mecanismos para fazer isso,
% então é preciso checar qual deve ser usado para cada língua (veja a
% documentação de babel ou faça um teste). São eles:
%
%   1. \renewcommand\spanishchaptername{Capítulo}
%
%   2. \addto\captionsenglish{\renewcommand\contentsname{Table of Contents}}
%      (este é o mais comum)
%
% Esses métodos valem também para a bibliografia, mas apenas se você
% estiver usando bibtex; com biblatex, que é o padrão neste modelo, é
% melhor usar o comando "\DefineBibliographyStrings" (veja a documentação
% de biblatex).
%
% Quando babel faz uma troca de língua, ele executa \extraslíngua e, se for
% necessário trocar os "captions", \captionslíngua (ou seja, os comandos
% acima modificam \captionslíngua). Então, se você quiser executar algo a
% mais quando uma língua é selecionada, faça \addto\extrasenglish{\blah}.
\usepackage{babel}
\usepackage{iflang}

% Por padrão, LaTeX utiliza maiúsculas no início de cada palavra nestas
% expressões ("Lista de Figuras"); vamos usar maiúsculas apenas na primeira
% palavra.
\addto\captionsbrazilian{%
  \renewcommand\listfigurename{Lista de figuras}%
  \renewcommand\listtablename{Lista de tabelas}%
  \renewcommand\indexname{Índice remissivo}%
}

% Alguns pacotes (tikz, siunitx) usam, além de babel, este pacote como
% auxiliar para a tradução de palavras-chave, como os meses do ano.
\usepackage{translator}

%%%%%%%%%%%%%%%%%%%%%%%%%%%%%%%%%%%%%%%%%%%%%%%%%%%%%%%%%%%%%%%%%%%%%%%%%%%%%%%%
%%%%%%%%%%%%%%%%%%%%%%%%%%%%%%%%%%% FONTE %%%%%%%%%%%%%%%%%%%%%%%%%%%%%%%%%%%%%%
%%%%%%%%%%%%%%%%%%%%%%%%%%%%%%%%%%%%%%%%%%%%%%%%%%%%%%%%%%%%%%%%%%%%%%%%%%%%%%%%

% LaTeX normalmente usa quatro tipos de fonte:
%
% * uma fonte serifada, para o corpo do texto;
% * uma fonte com design similar à anterior, para modo matemático;
% * uma fonte sem serifa, para títulos ou "entidades". Por exemplo, "a classe
%   \textsf{TimeManager} é responsável..." ou "chamamos \textsf{primos} os
%   números que...". Observe que em quase todos os casos desse tipo é mais
%   adequado usar negrito ou itálico;
% * uma fonte "teletype", para trechos de programas.
%
% A escolha de uma família de fontes para o documento normalmente é feita
% carregando uma package específica que, em geral, seleciona as quatro fontes
% de uma vez.
%
% LaTeX usa por default a família de fontes "Computer Modern". Essas fontes
% precisaram ser re-criadas diversas vezes em formatos diferentes, então há
% diversas variantes dela. Com o fontenc OT1 (default "ruim" do LaTeX), a
% versão usada é a BlueSky Computer Modern, que é de boa qualidade, mas com
% os problemas do OT1. Com fontenc T1 (padrão deste modelo e recomendado), o
% LaTeX usa o conjunto "cm-super". Com fontspec (ou seja, com LuaLaTeX e
% XeLaTeX), LaTeX utiliza a versão "Latin Modern". Ao longo do tempo, versões
% diferentes dessas fontes foram recomendadas como "a melhor"; atualmente, a
% melhor opção para usar a família Computer Modern é a versão "Latin Modern".
%
% Você normalmente não precisa lidar com isso, mas pode ser útil saber: O
% mecanismo tradicionalmente usado por LaTeX para gerir fontes é o NFSS
% (veja "texdoc fntguide"). Ele funciona com todas as versões de LaTeX,
% mas só com fontes que foram adaptadas para funcionar com LaTeX. LuaLaTeX
% e XeLaTeX podem usar NFSS mas também são capazes de utilizar um outro
% mecanismo (através da package fontspec), que permite utilizar quaisquer
% fontes instaladas no computador.

\ifunicodeengine
    % Com LuaLaTex e XeLaTeX, Latin Modern é a fonte padrão. Existem
    % diversas packages e "truques" para melhorar alguns aspectos de
    % Latin Modern, mas eles foram feitos para pdflatex (veja o "else"
    % logo abaixo). Assim, se você pretende usar Latin Modern como a
    % fonte padrão do documento, é melhor usar pdfLaTeX. Deve ser
    % possível implementar essas melhorias com fontspec também, mas
    % este modelo não faz isso, apenas ativamos Small Caps aqui.

    \ifluatex
      % Com LuaTeX, basta indicar o nome de cada fonte; para descobrir
      % o nome "certo", use o comando "otfinfo -i" e veja os itens
      % "preferred family" e "full name"
      \setmainfont{Latin Modern Roman}[
        SmallCapsFont = {LMRomanCaps10-Regular},
        ItalicFeatures = {
          SmallCapsFont = {LMRomanCaps10-Oblique},
        },
        SlantedFont = {LMRomanSlant10-Regular},
        SlantedFeatures = {
          SmallCapsFont = {LMRomanCaps10-Oblique},
          BoldFont = {LMRomanSlant10-Bold}
        },
      ]
    \fi

    \ifxetex
      % Com XeTeX, é preciso informar o nome do arquivo de cada fonte.
      \setmainfont{lmroman10-regular.otf}[
        SmallCapsFont = {lmromancaps10-regular.otf},
        ItalicFeatures = {
          SmallCapsFont = {lmromancaps10-oblique.otf},
        },
        SlantedFont = {lmromanslant10-regular.otf},
        SlantedFeatures = {
          SmallCapsFont = {lmromancaps10-oblique.otf},
          BoldFont = {lmromanslant10-bold.otf}
        },
      ]
    \fi

\else
    % Usando pdfLaTeX

    % Permite utilizar small caps + itálico (e outras pequenas melhorias)
    \usepackage{fontaxes}

    % Ativa Latin Modern como a fonte padrão.
    \usepackage{lmodern}

    % Alguns truques para melhorar a aparência das fontes Latin Modern;
    % eles não funcionam com LuaLaTeX e XeLaTeX.

    % Latin Modern não tem fontes bold + Small Caps, mas cm-super sim;
    % assim, vamos ativar o suporte às fontes cm-super (sem ativá-las
    % como a fonte padrão do documento) e configurar substituições
    % automáticas para que a fonte Latin Modern seja substituída por
    % cm-super quando o texto for bold + Small Caps.
    \usepackage{fix-cm}

    % Com Latin Modern, é preciso incluir substituições para o encoding TS1
    % também por conta dos números oldstyle, porque para inclui-los nas fontes
    % computer modern foi feita uma hack: os dígitos são declarados como sendo
    % os números itálicos da fonte matemática e, portanto, estão no encoding TS1.
    %
    % Primeiro forçamos o LaTeX a carregar a fonte Latin Modern (ou seja, ler
    % o arquivo que inclui "DeclareFontFamily") e, a seguir, definimos a
    % substituição
    \fontencoding{TS1}\fontfamily{lmr}\selectfont
    \DeclareFontShape{TS1}{lmr}{b}{sc}{<->ssub * cmr/bx/n}{}
    \DeclareFontShape{TS1}{lmr}{bx}{sc}{<->ssub * cmr/bx/n}{}

    \fontencoding{T1}\fontfamily{lmr}\selectfont
    \DeclareFontShape{T1}{lmr}{b}{sc}{<->ssub * cmr/bx/sc}{}
    \DeclareFontShape{T1}{lmr}{bx}{sc}{<->ssub * cmr/bx/sc}{}

    % Latin Modern não tem "small caps + itálico", mas tem "small caps + slanted";
    % vamos definir mais uma substituição aqui.
    \fontencoding{T1}\fontfamily{lmr}\selectfont % já feito acima, mas tudo bem
    \DeclareFontShape{T1}{lmr}{m}{scit}{<->ssub * lmr/m/scsl}{}
    \DeclareFontShape{T1}{lmr}{bx}{scit}{<->ssub * lmr/bx/scsl}{}

    % Se fizermos mudanças manuais na fonte Latin Modern, estes comandos podem
    % vir a ser úteis
    %\newcommand\lmodern{%
    %  \renewcommand{\oldstylenums}[1]{{\fontencoding{TS1}\selectfont ##1}}%
    %  \fontfamily{lmr}\selectfont%
    %}
    %
    %\DeclareRobustCommand\textlmodern[1]{%
    %  {\lmodern #1}%
    %}
\fi

% Algumas packages mais novas que tratam de fontes funcionam corretamente
% tanto com fontspec (LuaLaTeX/XeLaTeX) quanto com NFSS (qualquer versão
% de LaTeX, mas menos poderoso que fontspec). No entanto, muitas funcionam
% apenas com NFSS. Nesse caso, em LuaLaTeX/XeLaTeX é melhor usar os
% comandos de fontspec, como exemplificado mais abaixo.

% É possível mudar apenas uma das fontes. Em particular, a fonte
% teletype da família Computer Modern foi criada para simular
% as impressoras dos anos 1970/1980. Sendo assim, ela é uma fonte (1)
% com serifas e (2) de espaçamento fixo. Hoje em dia, é mais comum usar
% fontes sem serifa para representar código-fonte. Além disso, ao imprimir,
% é comum adotar fontes que não são de espaçamento fixo para fazer caber
% mais caracteres em uma linha de texto. Algumas opções de fontes para
% esse fim:
%\usepackage{newtxtt} % Não funciona com fontspec (lualatex / xelatex)
%\usepackage{DejaVuSansMono}
% inconsolata é uma boa fonte, mas não tem variante itálico
%\ifunicodeengine
%  \setmonofont{inconsolatan}
%\else
%  \usepackage[narrow]{inconsolata}
%\fi
\usepackage[scale=.85]{sourcecodepro}

% Ao invés da família Computer Modern, é possível usar outras como padrão.
% Uma ótima opção é a libertine, similar (mas não igual) à Times mas com
% suporte a Small Caps e outras qualidades. A fonte teletype da família
% é serifada, então é melhor definir outra; a opção "mono=false" faz
% o pacote não carregar sua própria fonte, mantendo a escolha anterior.
% Versões mais novas de LaTeX oferecem um fork desta fonte, libertinus.
% As packages libertine/libertinus funcionam corretamente com pdfLaTeX,
% LuaLaTeX e XeLaTeX.
\makeatletter
\IfFileExists{libertinus.sty}
    {
      \usepackage[mono=false]{libertinus}
      % Com LuaLaTeX/XeLaTeX, Libertinus configura também
      % a fonte matemática; aqui só precisamos corrigir \mathit
      \ifunicodeengine
        \ifluatex
          \setmathfontface\mathit{Libertinus Serif Italic}
        \fi
        \ifxetex
          % O nome de arquivo da fonte mudou na versão 2019-04-04
          \@ifpackagelater{libertinus-otf}{2019/04/03}
              {\setmathfontface\mathit{LibertinusSerif-Italic.otf}}
              {\setmathfontface\mathit{libertinusserif-italic.otf}}
        \fi
      \fi
    }
    {
      % Libertinus não está disponível; vamos usar libertine
      \usepackage[mono=false]{libertine}

      % Com Libertine, é preciso modificar também a fonte
      % matemática, além de \mathit
      \ifunicodeengine
        \ifluatex
	  \setmathfont{Libertinus Math}
          \setmathfontface\mathit{Linux Libertine O Italic}
        \fi

        \ifxetex
          \setmathfont{libertinusmath-regular.otf}
          \setmathfontface\mathit{LinLibertine_RI.otf}
        \fi
      \fi
    }
\makeatother

\ifunicodeengine
  \relax
\else
  % A família libertine por padrão não define uma fonte matemática
  % específica para pdfLaTeX; uma opção que funciona bem com ela:
  %\usepackage[libertine]{newtxmath}
  % Outra, provavelmente melhor:
  \usepackage{libertinust1math}
\fi

% Ativa apenas a fonte biolinum, que é a fonte sem serifa da família.
%\IfFileExists{libertinus.sty}
%  \usepackage[sans]{libertinus}
%\else
%  \usepackage{biolinum}
%\fi

% Também é possível usar a Times como padrão; nesse caso, a fonte
% sem serifa usualmente é a Helvetica. Mas provavelmente libertine
% é uma opção melhor.
%\ifunicodeengine
%  % Clone da fonte Times como fonte principal
%  \setmainfont{TeX Gyre Termes}
%  \setmathfont[Scale=MatchLowercase]{TeX Gyre Termes Math}
%  % TeX Gyre Termes Math tem um bug e não define o caracter
%  % \setminus; Vamos contornar esse problema usando apenas
%  % esse caracter da fonte STIX Two Math
%  \setmathfont[range=\setminus]{STIX Two Math}
%  % Clone da fonte Helvetica como fonte sem serifa
%  \setsansfont{TeX Gyre Heros}
%  % Clone da Courier como fonte teletype, mas provavelmente
%  % é melhor utilizar sourcecodepro
%  %\setmonofont{TeX Gyre Cursor}
%\else
%  \usepackage[helvratio=0.95,largesc]{newtxtext}
%  \usepackage{newtxtt} % Fonte teletype
%  \usepackage{newtxmath}
%\fi

% Cochineal é outra opção de qualidade; ela define apenas a fonte
% com serifa.
%
% Com NFSS (recomendado no caso de cochineal):
%\usepackage{cochineal}
%\usepackage[cochineal,vvarbb]{newtxmath}
%\usepackage[cal=boondoxo]{mathalfa}
%
% Com fontspec (até a linha "setmathfontface..."):
%
%\setmainfont{Cochineal}[
%  Extension=.otf,
%  UprightFont=*-Roman,
%  ItalicFont=*-Italic,
%  BoldFont=*-Bold,
%  BoldItalicFont=*-BoldItalic,
%  %Numbers={Proportional,OldStyle},
%]
%
%\DeclareRobustCommand{\lfstyle}{\addfontfeatures{Numbers=Lining}}
%\DeclareTextFontCommand{\textlf}{\lfstyle}
%\DeclareRobustCommand{\tlfstyle}{\addfontfeatures{Numbers={Tabular,Lining}}}
%\DeclareTextFontCommand{\texttlf}{\tlfstyle}
%
%% Cochineal não tem uma fonte matemática; com fontspec, provavelmente
%% o melhor a fazer é usar libertinus.
%\setmathfont{Libertinus Math}
%\setmathfontface\mathit{Cochineal-Italic.otf}

% gentium inclui apenas uma fonte serifada, similar a Garamond, que busca
% cobrir todos os caracteres unicode
%\usepackage{gentium}

% LaTeX normalmente funciona com fontes que foram adaptadas para ele, ou
% seja, ele não usa as fontes padrão instaladas no sistema: para usar
% uma fonte é preciso ativar o pacote correspondente, como visto acima.
% É possível escapar dessa limitação e acessar as fontes padrão do sistema
% com XeTeX ou LuaTeX. Com eles, além dos pacotes de fontes "tradicionais",
% pode-se usar o pacote fontspec para usar fontes do sistema.
%\usepackage{fontspec}
%\setmainfont{DejaVu Serif}
%\setmainfont{Charis SIL}
%\setsansfont{DejaVu Sans}
%\setsansfont{Libertinus Sans}[Scale=1.1]
%\setmonofont{DejaVu Sans Mono}

% fontspec oferece vários recursos interessantes para manipular fontes.
% Por exemplo, Garamond é uma fonte clássica; a versão EBGaramond é muito
% boa, mas não possui versões bold e bold-italic; aqui, usamos
% CormorantGaramond ou Gentium para simular a versão bold.
%\setmainfont{EBGaramond12}[
%  Numbers        = {Lining,} ,
%  Scale          = MatchLowercase ,
%  UprightFont    = *-Regular ,
%  ItalicFont     = *-Italic ,
%  BoldFont       = gentiumbasic-bold ,
%  BoldItalicFont = gentiumbasic-bolditalic ,
%%  BoldFont       = CormorantGaramond Bold ,
%%  BoldItalicFont = CormorantGaramond Bold Italic ,
%]
%
%\newfontfamily\garamond{EBGaramond12}[
%  Numbers        = {Lining,} ,
%  Scale          = MatchLowercase ,
%  UprightFont    = *-Regular ,
%  ItalicFont     = *-Italic ,
%  BoldFont       = gentiumbasic-bold ,
%  BoldItalicFont = gentiumbasic-bolditalic ,
%%  BoldFont       = CormorantGaramond Bold ,
%%  BoldItalicFont = CormorantGaramond Bold Italic ,
%]

% Crimson tem Small Caps, mas o recurso é considerado "em construção".
% Vamos utilizar Gentium para Small Caps
%\setmainfont{Crimson}[
%  Numbers           = {Lining,} ,
%  Scale             = MatchLowercase ,
%  UprightFont       = *-Roman ,
%  ItalicFont        = *-Italic ,
%  BoldFont          = *-Bold ,
%  BoldItalicFont    = *-Bold Italic ,
%  SmallCapsFont     = Gentium Plus ,
%  SmallCapsFeatures = {Letters=SmallCaps} ,
%]
%
%\newfontfamily\crimson{Crimson}[
%  Numbers           = {Lining,} ,
%  Scale             = MatchLowercase ,
%  UprightFont       = *-Roman ,
%  ItalicFont        = *-Italic ,
%  BoldFont          = *-Bold ,
%  BoldItalicFont    = *-Bold Italic ,
%  SmallCapsFont     = Gentium Plus ,
%  SmallCapsFeatures = {Letters=SmallCaps} ,
%]

% Com o pacote fontspec, também é possível usar o comando "\fontspec" para
% selecionar uma fonte temporariamente, sem alterar as fontes-padrão do
% documento.

%%%%%%%%%%%%%%%%%%%%%%%%%%%%%%%%%%%%%%%%%%%%%%%%%%%%%%%%%%%%%%%%%%%%%%%%%%%%%%%%
%%%%%%%%%%%%%%%%%%%%%%%%%%%%% FIGURAS / FLOATS %%%%%%%%%%%%%%%%%%%%%%%%%%%%%%%%%
%%%%%%%%%%%%%%%%%%%%%%%%%%%%%%%%%%%%%%%%%%%%%%%%%%%%%%%%%%%%%%%%%%%%%%%%%%%%%%%%

% Permite importar figuras. LaTeX "tradicional" só é capaz de trabalhar com
% figuras EPS. Hoje em dia não há nenhuma boa razão para usar essa versão;
% pdfTeX, XeTeX, e LuaTeX podem usar figuras nos formatos PDF, JPG e PNG; EPS
% também pode funcionar em algumas instalações mas não é garantido, então é
% melhor evitar.
\usepackage{graphicx}

% A package float é amplamente utilizada; ela permite definir novos tipos
% de float e também acrescenta a possibilidade de definir "H" como opção de
% posicionamento do float, que significa "aqui, incondicionalmente". No
% entanto, ela tem algumas fragilidades e não é atualizada desde 2001.
% floatrow é uma versão aprimorada e com mais recursos da package "float",
% mas também não é atualizada desde 2009. Aqui utilizamos alguns recursos
% disponibilizados por ambas e é possível escolher qual delas utilizar.
% Um dos principais recursos dessas packages é permitir a criação de novos
% tipos de float; veja o arquivo source-code.tex para um exemplo.
%\usepackage{float}
\usepackage{floatrow}

% Em documentos com duas colunas, floats normalmente são colocados como
% parte de uma das colunas. No entanto, é possível usar "\begin{figure*}"
% ou "\begin{table*}" para criar floats que ocupam as duas colunas. Floats
% "duplos" desse tipo têm algumas limitações:
%
% 1. Mesmo que haja espaço disponível na página atual, eles são sempre
%    inseridos na página seguinte ao lugar em que foram definidos (então
%    é comum defini-los antes do lugar "certo" para compensar isso)
%
% 2. Eles só podem aparecer no topo da página ou em uma página de floats,
%    ou seja, nunca "here" nem no pé da página.
%
% 3. Em alguns casos, eles podem aparecer fora da ordem em relação aos
%    demais floats do mesmo tipo (o que não acontece com floats "normais")
%
% Esta package:
%
% 1. Soluciona parcialmente o primeiro problema: floats "duplos" podem
%    aparecer na página em que são definidos se sua definição está contida
%    no texto da coluna da esquerda;
%
% 2. Soluciona o segundo problema: floats "duplos" podem aparecer tanto no
%    topo quanto no pé da página. Observe que eles *não* podem aparecer
%    "here" porque isso não faz sentido: a figura interromperia o fluxo
%    do texto da "outra" coluna.
%
% 3. Soluciona o terceiro problema.
%
\usepackage{stfloats}

% Às vezes é interessante utilizar uma imagem mais larga que o texto.
% Por padrão, \centering *não* vai centralizar a imagem corretamente
% nesse caso. Com esta package, podemos acrescentar a opção "center"
% ao comando \includegraphics para resolver esse problema
% (ou seja, \includegraphics[width=1.2\textwidth,center]{imagem}.
% A package tem muitos outros recursos também
\usepackage[export]{adjustbox}

% Por padrão, LaTeX prefere colocar floats no topo da página que
% onde eles foram definidos; vamos mudar isso. Este comando depende
% do pacote "floatrow", carregado logo acima.
\floatplacement{table}{htbp}
\floatplacement{figure}{htbp}


% Em alguns casos, um float pode aparecer antes do local do texto em que
% foi definido (ou seja, no topo da página ao invés do meio da página).
% Esta package garante que floats (tabelas e figuras) só apareçam após
% o local no texto em que foram definidos; veja os detalhes em
% https://tex.stackexchange.com/a/297580 . Note que, se o float tem a
% opção "h", normalmente LaTeX *não* coloca o float no topo da página
% atual: se o float não pode ser colocado "here", ele é delegado para
% a página seguinte.
\usepackage{flafter}

% Às vezes um float pode ser adiado por muitas páginas; é possível forçar
% LaTeX a imprimir todos os floats pendentes com o comando \clearpage mas,
% para isso, o usuário deve identificar os casos problemáticos e inserir
% \clearpage manualmente. Esta package acrescenta o comando \FloatBarrier,
% que executa \clearpage automaticamente se necessário no local em que é
% chamado. A opção "section" faz o comando ser aplicado automaticamente
% a cada nova seção. "above" e "below" desabilitam a barreira quando os
% floats estão na mesma página. A desvantagem de placeins é que, para
% funcionar, ela gera quebras de página que muitas vezes são inesperadas.
%\usepackage[section,above,below]{placeins}

% LaTeX escolhe automaticamente o "melhor" lugar para colocar cada float.
% Por padrão, ele tenta colocá-los no topo da página e depois no pé da
% página; se não tiver sucesso, vai para a página seguinte e recomeça.
% Se esse algoritmo não tiver sucesso "logo", LaTeX cria uma página só
% com floats. É possível modificar esse comportamento com as opções de
% posicionamento: "tp", por exemplo, instrui LaTeX a considerar apenas
% o topo da página ou uma página só de floats (ignorando o pé da página),
% e "htbp" o instrui para tentar "aqui" como a primeira opção. A ordem
% dessas opções não é relevante: dentre as opções disponíveis, LaTeX
% sempre tenta "aqui, topo, pé, página". Os pacotes "float" e "floatrow"
% acrescentam a opção "H", que significa "aqui, incondicionalmente".
%
% A escolha do "melhor" lugar leva em conta os parâmetros abaixo, mas é
% possível ignorá-los com a opção de posicionamento "!". Dado que os
% valores default não são muito bons para floats "grandes" ou documentos
% com muitos floats, é muito comum usar "!" ou "H". No entanto, modificando
% estes parâmetros o algoritmo automático tende a funcionar melhor. Ainda
% assim, vale ler a discussão a respeito na seção "Limitações do LaTeX"
% deste modelo.

% Fração da página que pode ser ocupada por floats no topo. Default: 0.7
\renewcommand{\topfraction}{.85}
% Idem para documentos em colunas e floats que tomam as 2 colunas. Default: 0.7
\renewcommand{\dbltopfraction}{.66}
% Fração da página que pode ser ocupada por floats no pé. Default: 0.3
\renewcommand{\bottomfraction}{.7}
% Fração mínima da página que deve conter texto. Default: 0.2
\renewcommand{\textfraction}{.15}
% Numa página só de floats, fração mínima que deve ser ocupada. Default: 0.5
% floatpagefraction *deve* ser menor que topfraction.
\renewcommand{\floatpagefraction}{.66}
% Idem para documentos em colunas e floats que tomam as 2 colunas. Default: 0.5
\renewcommand{\dblfloatpagefraction}{.66}
% Máximo de floats no topo da página. Default: 2
\setcounter{topnumber}{9}
% Idem para documentos em colunas e floats que tomam as 2 colunas. Default: 2
\setcounter{dbltopnumber}{9}
% Máximo de floats no pé da página. Default: 1
\setcounter{bottomnumber}{9}
% Máximo de floats por página. Default: 3
\setcounter{totalnumber}{20}

% Define o ambiente "\begin{landscape} -- \end{landscape}"; o texto entre
% esses comandos é impresso em modo paisagem, podendo se estender por várias
% páginas. A rotação não inclui os cabeçalhos e rodapés das páginas.
% O principal uso desta package é em conjunto com a package longtable: se
% você precisa mostrar uma tabela muito larga (que precisa ser impressa em
% modo paisagem) e longa (que se estende por várias páginas), use
% "\begin{landscape}" e "\begin{longtable}" em conjunto. Note que o modo
% landscape entra em ação imediatamente, ou seja, "\begin{landscape}" gera
% uma quebra de página no local em que é chamado. Na maioria dos casos, o
% que se quer não é isso, mas sim um "float paisagem"; isso é o que a
% package rotating oferece (veja abaixo).
\usepackage{pdflscape}

% Define dois novos tipos de float: sidewaystable e sidewaysfigure, que
% imprimem a figura ou tabela sozinha em uma página em modo paisagem. Além
% disso, permite girar elementos na página de diversas outras maneiras.
\usepackage[figuresright,clockwise]{rotating}

% Captions com fonte menor, indentação normal, corpo do texto
% negrito e nome do caption itálico
\usepackage[
  font=small,
  format=plain,
  labelfont=bf,up,
  textfont=it,up]{caption}

% Em geral, a package caption é capaz de "adivinhar" se o caption
% está acima ou abaixo da figura/tabela, mas isso não funciona
% corretamente com longtable. Aqui, forçamos a package a considerar
% que os captions ficam abaixo das tabelas.
\captionsetup*[longtable]{position=bottom}

% Sub-figuras (e seus captions) - observe que existe uma package chamada
% "subfigure", mas ela é obsoleta; use esta no seu lugar.
\usepackage{subcaption}

% Permite criar imagens com texto ao redor
\usepackage{wrapfig}

% Permite incorporar um arquivo PDF como uma página adicional. Útil se
% for necessário importar uma imagem ou tabela muito grande ou ainda
% para definir uma capa personalizada.
\usepackage{pdfpages}

% Caixas de texto coloridas
%\usepackage{tcolorbox}


%%%%%%%%%%%%%%%%%%%%%%%%%%%%%%%%%%%%%%%%%%%%%%%%%%%%%%%%%%%%%%%%%%%%%%%%%%%%%%%%
%%%%%%%%%%%%%%%%%%%%%%%%%%%%%%%%%% TABELAS %%%%%%%%%%%%%%%%%%%%%%%%%%%%%%%%%%%%%
%%%%%%%%%%%%%%%%%%%%%%%%%%%%%%%%%%%%%%%%%%%%%%%%%%%%%%%%%%%%%%%%%%%%%%%%%%%%%%%%

% Tabelas simples são fáceis de fazer em LaTeX; tabelas com alguma sofisticação
% são trabalhosas, pois é difícil controlar alinhamento, largura das colunas,
% distância entre células etc. Ou seja, é muito comum que a tabela final fique
% "torta". Por isso, em muitos casos, vale mais a pena gerar a tabela em uma
% planilha, como LibreOffice calc ou excel, transformar em PDF e importar como
% figura, especialmente se você quer controlar largura/altura das células
% manualmente etc. No entanto, se você quiser fazer as tabelas em LaTeX para
% garantir a consistência com o tipo e o tamanho das fontes, é possível e o
% resultado é muito bom. Aqui há alguns pacotes que incrementam os recursos de
% tabelas do LaTeX e alguns comandos pré-prontos que podem facilitar um pouco
% seu uso.

% LaTeX por padrão não permite notas de rodapé dentro de tabelas. De maneira
% geral, notas de rodapé em tabelas são consideradas "ruins" em termos de
% tipografia, mas às vezes são necessárias. Se esse é o caso, o recomendado
% é que as notas de rodapé apareçam no "rodapé" da tabela, com numeração
% própria, e não no rodapé da página. Você pode fazer isso com esta package:
\usepackage{threeparttable}
% Formatação personalizada das notas de threeparttable:
\appto{\TPTnoteSettings}{\footnotesize\itshape}
\def\TPTtagStyle{\textit}

% Se você realmente quer notas de rodapé em tabelas que aparecem como as
% demais notas de rodapé (no final da página e mantendo a sequência numérica),
% você pode usar a package abaixo. No entanto, ela não funciona com floats
% duplos (floats que ocupam toda a largura da página em um documento de duas
% colunas) e, em alguns casos, a nota pode desaparecer ou aparecer em uma
% página diferente da tabela (mova o lugar do texto em que ela é definida
% para resolver esse problema).
\usepackage{tablefootnote}

% Por padrão, cada coluna de uma tabela tem a largura do maior texto contido
% nela, ou seja, se uma coluna contém uma célula muito larga, LaTeX não
% força nenhuma quebra de linha e a tabela "estoura" a largura do papel. A
% solução simples, nesses casos, é inserir uma ou mais quebras de linha
% manualmente, o que além de deselegante não é totalmente trivial (é preciso
% usar \makecell).
% Esta package estende o ambiente tabular para permitir definir um tamanho
% fixo para uma ou mais colunas; nesse caso, LaTeX quebra as linhas se uma
% célula é larga demais para a largura definida. Encontrar valores "bons"
% para as larguras das colunas, no entanto, também é um trabalho manual
% um tanto penoso. As packages tabularx e tabulary permitem configurar
% algumas colunas como "largura automática", evitando a necessidade da
% definição manual. Finalmente, ltxtable permite utilizar tabularx e
% longtable juntas. Neste modelo, não usamos tabularx/tabulary, mas você
% pode carregá-las se quiser.
\usepackage{array}

% Se você quer ter um pouco mais de controle sobre o tamanho de cada coluna da
% tabela, utilize estes tipos de coluna (criados com base nos recursos do pacote
% array). É só usar algo como M{número}, onde "número" (por exemplo, 0.4) é a
% fração de \textwidth que aquela coluna deve ocupar. "M" significa que o
% conteúdo da célula é centralizado; "L", alinhado à esquerda; "J", justificado;
% "R", alinhado à direita. Obviamente, a soma de todas as frações não pode ser
% maior que 1, senão a tabela vai ultrapassar a linha da margem.
\newcolumntype{M}[1]{>{\centering}m{#1\textwidth}}
\newcolumntype{L}[1]{>{\RaggedRight}m{#1\textwidth}}
\newcolumntype{R}[1]{>{\RaggedLeft}m{#1\textwidth}}
\newcolumntype{J}[1]{m{#1\textwidth}}

% Permite alinhar os elementos de uma coluna pelo ponto decimal; dê
% preferência à package siunitx (carregada em utils.tex), que também
% oferece esse recurso e muitos outros.
\usepackage{dcolumn}

% Define tabelas do tipo "longtable", similares a "tabular" mas que podem ser
% divididas em várias páginas. "longtable" também funciona corretamente com
% notas de rodapé. Note que, como uma longtable pode se estender por várias
% páginas, não faz sentido colocá-las em um float "table". Por conta disso,
% longtable define o comando "\caption" internamente.
\usepackage{longtable}

% Permite agregar linhas de tabelas, fazendo colunas "compridas"
\usepackage{multirow}

% Cria comando adicional para possibilitar a inserção de quebras de linha
% em uma célula de tabela, entre outros
\usepackage{makecell}

% Às vezes a tabela é muito larga e não cabe na página. Se os cabeçalhos da
% tabela é que são demasiadamente largos, uma solução é inclinar o texto das
% células do cabeçalho. Para fazer isso, use o comando "\rothead".
\renewcommand{\rothead}[2][60]{\makebox[11mm][l]{\rotatebox{#1}{\makecell[c]{#2}}}}

% Se quiser criar uma linha mais grossa no meio de uma tabela, use
% o comando "\thickhline".
\newlength\savedwidth
\newcommand\thickhline{
  \noalign{
    \global\savedwidth\arrayrulewidth
    \global\arrayrulewidth 1.5pt
  }
  \hline
  \noalign{\global\arrayrulewidth\savedwidth}
}

% Modifica (melhora) o layout default das tabelas e acrescenta os comandos
% \toprule, \bottomrule, \midrule e \cmidrule
\usepackage{booktabs}

% Permite colorir linhas, colunas ou células
\usepackage{colortbl}

% Ao invés de digitar os dados de uma tabela dentro do seu documento,
% você pode fazer LaTeX ler os dados de um arquivo CSV e criar uma
% tabela automaticamente com uma destas duas packages:
%\usepackage{csvsimple}     % mais simples
%\usepackage{pgfplotstable} % mais complexa

% Você também pode se interessar pelo ambiente "tabbing", que permite
% criar tabelas simples com algumas vantagens em relação a "tabular",
% ou por esta package, que permite criar tabulações.
%\usepackage{tabto-ltx}

%%%%%%%%%%%%%%%%%%%%%%%%%%%%%%%%%%%%%%%%%%%%%%%%%%%%%%%%%%%%%%%%%%%%%%%%%%%%%%%%
%%%%%%%%%%%%%%% CAPA E PÁGINAS PRELIMINARES (TESE/DISSERTAÇÃO)  %%%%%%%%%%%%%%%%
%%%%%%%%%%%%%%%%%%%%%%%%%%%%%%%%%%%%%%%%%%%%%%%%%%%%%%%%%%%%%%%%%%%%%%%%%%%%%%%%

% Formatação de datas de acordo com a língua
\usepackage[useregional]{datetime2}
\DTMusemodule{brazilian}{portuges}
\DTMnewdatestyle{month-year}{%
  \renewcommand*{\DTMdisplaydate}[4]{##2,\space##1}%
  \renewcommand*{\DTMDisplaydate}{\DTMdisplaydate}%
}

\makeatletter

%%%%%%%%%%%%%%%%%%%%%%%%%%%%%%%%%%%%%%%%%%%%%%%%%%%%%%%%%%%%%%%%%%%%%%%%%%%%%%%%
%%%%%%%%%%%%%%%%%%%%% TEXTOS PADRÃO EM PT E EN PARA A CAPA %%%%%%%%%%%%%%%%%%%%%
%%%%%%%%%%%%%%%%%%%%%%%%%%%%%%%%%%%%%%%%%%%%%%%%%%%%%%%%%%%%%%%%%%%%%%%%%%%%%%%%

% \extrasLANGUAGE vs \captionsLANGUAGE: https://tex.stackexchange.com/a/354197/217608

% Palavras fixas a serem traduzidas
\providecommand\keywordsname{} % Keywords / Palavras-chave
\providecommand\programname{} % Program / Programa
\providecommand\committeename{} % Examining committee / Comissão julgadora
\providecommand\advisorname{} % Advisor / Orientador(a)
\providecommand\coadvisorname{} % Co-advisor / Coorientador(a)
\providecommand\workname{} % Report, Thesis / Tese, Dissertação, Monografia
\providecommand\degreename{} % Masters, Doctorate, Bachelor / Mestrado, Doutorado, Bacharelado
\providecommand\titlename{} % Master, Doctor, Bachelor / Mestre(a), Doutor(a), Bacharel
\providecommand\@licenseboilerplate{O conteúdo deste trabalho é publicado sob a licença}

% Textos longos a serem traduzidos
\providecommand\@coverTCCText{}
\providecommand\@coverQualiText{}
\providecommand\@coverThesisText{}
\providecommand\@institutionBlockText{} % Só para TCC
\providecommand\@provisionalFrontmatterText{}
\providecommand\@finalFrontmatterText{}
\providecommand\@institution{}

% Este não precisa ser traduzido, o texto em inglês não utiliza
\providecommand\@bywhom{%
  \ifdefstring{\@authorGender}{masc}
    {pelo candidato \@author}
    {pela candidata \@author}%
}

%%%%%%%%%% PORTUGUÊS %%%%%%%%%%
\expandafter\addto\csname captions\@IMEpt\endcsname{%
  \DTMrenewdatestyle{month-year}{%
    \renewcommand*{\DTMdisplaydate}[4]
      {\DTMportugesmonthname{##2}\space de\space##1}%
  }%
  \renewcommand\keywordsname{Palavras-chave}%
  \renewcommand\programname{Programa}%
  \renewcommand\committeename{Comissão julgadora}%
  \renewcommand\advisorname{%
    \iftoggle{@tcc}{%
      \ifdefstring{\@advisorGender}{masc}
        {Supervisor}
        {Supervisora}%
    }{%
      \ifdefstring{\@advisorGender}{masc}
        {Orientador}
        {Orientadora}%
    }%
  }%
  \renewcommand\coadvisorname[1]{%
    \iftoggle{@tcc}{%
      \ifcsstring{@coadvisor#1Gender}{masc}
        {Cossupervisor}
        {Cossupervisora}%
    }{%
      \ifcsstring{@coadvisor#1Gender}{masc}
        {Coorientador}
        {Coorientadora}%
    }%
  }%
  \renewcommand\workname{%
    \iftoggle{@tcc}
      {Monografia}
      {\iftoggle{@qualificacao}
        {Exame de Qualificação}
        {\iftoggle{@doutorado}
          {Tese}
          {Dissertação}%
        }%
      }%
  }%
  \renewcommand\degreename{%
    \iftoggle{@doutorado}
      {Doutorado}
      {\iftoggle{@mestrado}
        {Mestrado}
        {\iftoggle{@tcc}
          {Bacharelado}
          {Nível não definido!}%
        }%
      }%
  }%
  \renewcommand\titlename{%
    \iftoggle{@doutorado}
      {\ifdefstring{\@authorGender}{masc}{Doutor}{Doutora}}
      {\iftoggle{@mestrado}
        {\ifdefstring{\@authorGender}{masc}{Mestre}{Mestra}}
        {\iftoggle{@tcc}
          {Bacharel}{Nível não definido!}%
        }%
      }%
  }%
  %
  %
  \renewcommand\@coverTCCText{%
    Monografia Final\vspace{.5\baselineskip}\\
    \@macCDXCIX{} --- Trabalho de\\
    Formatura Supervisionado%
  }%
  \renewcommand\@coverQualiText{%
    Relatório apresentado ao\\
    Instituto de Matemática e Estatística\\
    da Universidade de São Paulo\\
    para exame de qualificação de\\
    \degreename{} em Ciências%
  }%
  \renewcommand\@coverThesisText{%
    \workname{} apresentada ao\\
    Instituto de Matemática e Estatística\\
    da Universidade de São Paulo\\
    para obtenção do título de\\
    \titlename{} em Ciências%
  }%
  \renewcommand\@institutionBlockText{%
    Universidade de São Paulo\\
    Instituto de Matemática e Estatística\\
    Bacharelado em Ciência da Computação%
  }%
  \renewcommand\@provisionalFrontmatterText{%
    \iftoggle{@qualificacao}{%
      Esta é a versão original do texto de qualificação elaborado
      \@bywhom{}, tal como submetido à Comissão Julgadora.%
    }{%
      Esta é a versão original da \MakeLowercase{\workname} elaborada
      \@bywhom{}, tal como submetida à Comissão Julgadora.%
    }%
  }%
  \renewcommand\@finalFrontmatterText{%
    Esta versão da \MakeLowercase{\workname} contém as correções e alterações
    sugeridas pela Comissão Julgadora durante a defesa da versão
    original do trabalho, realizada em \DTMusedate{@defensedate}.\\[1\baselineskip]
    Uma cópia da versão original está disponível no Instituto de
    Matemática e Estatística da Universidade de São Paulo.%
  }%
  \renewcommand\@institution{%
    Instituto de Matemática e Estatística,
    Universidade de São Paulo%
  }%
  \renewcommand\@licenseboilerplate{O conteúdo deste trabalho é publicado sob a licença}%
}


%%%%%%%%%% INGLÊS %%%%%%%%%%
\expandafter\addto\csname captions\@IMEen\endcsname{%
  \DTMrenewdatestyle{month-year}{%
    \renewcommand*{\DTMdisplaydate}[4]
      {\DTMenglishmonthname{##2},\space##1}%
  }%
  \renewcommand\keywordsname{Keywords}%
  \renewcommand\programname{Program}%
  \renewcommand\committeename{Examining Committee}
  \renewcommand\advisorname{%
    \iftoggle{@tcc}{Supervisor}{Advisor}%
  }%
  \renewcommand\coadvisorname[1]{%
    \iftoggle{@tcc}{Co-supervisor}{Coadvisor}%
  }%
  % "Tese" e "dissertação" têm sentido contrário em língua inglesa:
  % http://guides.lib.berkeley.edu/dissertations_theses
  % https://www.grad.ubc.ca/handbook-graduate-supervision/graduate-thesis
  % Como "Thesis" é o nome genérico, vamos usar para mestrado e doutorado
  %
  %%%%%
  %
  % Nomes possíveis para o TCC em inglês:
  %
  % * monograph/monography
  %     usado para trabalho de alto nível de um autor "senior",
  %     então não faz sentido para um trabalho de graduação.
  %
  % * undergraduate thesis / bachelor's thesis
  %     plausível, mas no nosso caso report parece melhor.
  %
  % * senior project / senior thesis / honor thesis
  %     usado para "TCCs" de caráter fortemente acadêmico;
  %     não é o caso aqui.
  %
  % * essay / report
  %     razoável, porque trata-se de um texto/relato
  %     sobre o projeto de TCC.
  \renewcommand\workname{%
    \iftoggle{@tcc}
      {Capstone Project Report}
      {\iftoggle{@qualificacao}
        {Qualifying Exam}
        {Thesis}%
      }%
  }%
  \renewcommand\degreename{%
    \iftoggle{@doutorado}
      {Doctorate}
      {\iftoggle{@mestrado}
        {Master's}
        {\iftoggle{@tcc}
          {Bachelor}
          {Nível não definido!}%
        }%
      }%
  }%
  \renewcommand\titlename{%
    \iftoggle{@doutorado}
      {Doctor}
      {\iftoggle{@mestrado}
        {Master}
        {\iftoggle{@tcc}
          {Bachelor}%
          {Nível não definido!}%
        }%
      }%
  }%
  %
  %
  \renewcommand\@coverTCCText{%
    Final Essay\vspace{.5\baselineskip}\\
    \@macCDXCIX{} --- Capstone Project%
  }%
  \renewcommand\@coverQualiText{%
    Report presented to the\\
    Institute of Mathematics and Statistics\\
    of the University of São Paulo\\
    for the \titlename{} of Science\\
    qualifying examination\\%
  }%
  \renewcommand\@coverThesisText{%
    \workname{} presented to the\\
    Institute of Mathematics and Statistics\\
    of the University of São Paulo\\
    in partial fulfillment\\
    of the requirements\\
    for the degree of\\
    \titlename{} of Science%
  }%
  \renewcommand\@institutionBlockText{%
    University of São Paulo\\
    Institute of Mathematics and Statistics\\
    Bachelor of Computer Science%
  }%
  \renewcommand\@provisionalFrontmatterText{%
    \iftoggle{@qualificacao}{%
      This is the original version of the qualifying text prepared
      by candidate \@author, as submitted to the Examining Committee.%
    }{%
      This is the original version of the \MakeLowercase{\workname} prepared
      by candidate \@author, as submitted to the Examining Committee.%
    }%
  }%
  \renewcommand\@finalFrontmatterText{%
    This version of the \MakeLowercase{\workname} includes the corrections
    and modifications suggested by the Examining Committee during
    the defense of the original version of the work, which took
    place on \DTMusedate{@defensedate}.\\[1\baselineskip]
    A copy of the original version is available at the Institute of
    Mathematics and Statistics of the University of São Paulo.%
  }%
  \renewcommand\@institution{%
    Institute of Mathematics and Statistics,
    University of São Paulo%
  }%
  \renewcommand\@licenseboilerplate{The content of this work is published under the}%
}


%%%%%%%%%%%%%%%%%%%%%%%%%%%%%%%%%%%%%%%%%%%%%%%%%%%%%%%%%%%%%%%%%%%%%%%%%%%%%%%%
%%%%%%%%%%%%%%%%%%%%%%% COLETA E DEFINIÇÃO DE METADADOS %%%%%%%%%%%%%%%%%%%%%%%%
%%%%%%%%%%%%%%%%%%%%%%%%%%%%%%%%%%%%%%%%%%%%%%%%%%%%%%%%%%%%%%%%%%%%%%%%%%%%%%%%

\renewcommand\author[2][masc]{
  \gdef\@author{#2}
  \gdef\@authorGender{#1}
  \hypersetup{pdfauthor={\@author}}
}

\NewDocumentCommand{\orientador}{O{masc} m}{
  \gdef\@advisor{#2}
  \gdef\@advisorGender{#1}
}

% Mais de um coorientador é raro, mas acontece
\ExplSyntaxOn
\newcounter{numberOfCoadvisors}
\NewDocumentCommand\coorientador{O{masc} m}{
    \stepcounter{numberOfCoadvisors}
    \tl_gclear_new:c {@coadvisor\Roman{numberOfCoadvisors}}
    \tl_gclear_new:c {@coadvisor\Roman{numberOfCoadvisors}Gender}

    \tl_set:cn {@coadvisor\Roman{numberOfCoadvisors}} {#2}
    \tl_set:cn {@coadvisor\Roman{numberOfCoadvisors}Gender} {#1}
}

\seq_gclear_new:N \@committeeMembers

\newtoggle{@mestrado}
\newtoggle{@doutorado}
\newtoggle{@tcc}
\newtoggle{@qualificacao}
\newtoggle{@finalversion}

% Opções usando LaTeX3 (veja texdoc l3keys).
\keys_define:nn { IME / defense }
  {
    % Chaves à esquerda definem as variáveis à direita
    data .code:n= {\DTMsavedate{@defensedate}{#1}},
    data .value_required:n = true,
    nivel .choice:,
    nivel / mestrado .code:n = {\@mestrado},
    nivel / masters .code:n = {\@mestrado},
    nivel / dissertacao .code:n = {\@mestrado},
    nivel / doutorado .code:n = {\@doutorado},
    nivel / phd .code:n = {\@doutorado},
    nivel / tese .code:n = {\@doutorado},
    nivel / graduacao .code:n = {\@tcc},
    nivel / bachelor .code:n = {\@tcc},
    nivel / tcc .code:n = {\@tcc},
    nivel .value_required:n = true,
    quali .code:n = {\ifstrequal{#1}{true}{\toggletrue{@qualificacao}}{\togglefalse{@qualificacao}}},
    quali .default:n = {true},
    definitiva .code:n = {\ifstrequal{#1}{true}{\toggletrue{@finalversion}}{\togglefalse{@finalversion}}},
    definitiva .default:n = {true},
    provisoria .code:n = {\ifstrequal{#1}{true}{\togglefalse{@finalversion}}{\toggletrue{@finalversion}}},
    provisoria .default:n = {true},
    programa .tl_gset:N = \@program,
    program .value_required:n = true,
    apoio .tl_gset:N = \@financing,
    apoio .value_required:n = true,
    local .tl_gset:N = \@defenselocation,
    local .value_required:n = true,
    direitos .tl_gset:N = \@license,
    direitos .value_required:n = true,
    fichacatalografica .tl_gset:N = \@catalogingData,
    fichacatalografica .value_required:n = true,
    membrobanca .code:n = {\seq_gput_right:Nn \@committeeMembers {#1}},
    membrobanca .value_required:n = true,
  }

\NewDocumentCommand\defesa{m}{%
  \keys_set:nn {IME/defense}{#1}

  \exp_args:NV \str_case:nnF \@license
    {
      {CC-BY}{\gdef\@license{\@licenseboilerplate\space CC~BY~4.0\\
        \href{https\c_colon_str//creativecommons.org/licenses/by/4.0/}{%
        (Creative~Commons~Attribution~4.0~International~License)}}
        \hypersetup{pdflicenseurl={https://creativecommons.org/licenses/by/4.0/}}
      }

      {CC-BY-NC}{\gdef\@license{\@licenseboilerplate\space CC~BY-NC~4.0\\
        \href{https\c_colon_str//creativecommons.org/licenses/by-nc/4.0/}{%
        (Creative~Commons~Attribution-NonCommercial~4.0~International~License)}}
        \hypersetup{pdflicenseurl={https://creativecommons.org/licenses/by-nc/4.0/}}
      }

      {CC-BY-ND}{\gdef\@license{\@licenseboilerplate\space CC~BY-ND~4.0\\
        \href{https\c_colon_str//creativecommons.org/licenses/by-nd/4.0/}{%
        (Creative~Commons~Attribution-NoDerivatives~4.0~International~License)}}
        \hypersetup{pdflicenseurl={https://creativecommons.org/licenses/by-nc-nd/4.0/}}
      }

      {CC-BY-SA}{\gdef\@license{\@licenseboilerplate\space CC~BY-SA~4.0\\
        \href{https\c_colon_str//creativecommons.org/licenses/by-sa/4.0/}{%
        (Creative~Commons~Attribution-ShareAlike~4.0~International~License)}}
        \hypersetup{pdflicenseurl={https://creativecommons.org/licenses/by-sa/4.0/}}
      }

      {CC-BY-NC-SA}{\gdef\@license{\@licenseboilerplate\space CC~BY-NC-SA~4.0\\
        \href{https\c_colon_str//creativecommons.org/licenses/by-nc-sa/4.0/}{%
        (Creative~Commons~Attribution-NonCommercial-ShareAlike~4.0~International~License)}}
        \hypersetup{pdflicenseurl={https://creativecommons.org/licenses/by-nc-sa/4.0/}}
      }

      {CC-BY-NC-ND}{\gdef\@license{\@licenseboilerplate\space CC~BY-NC-ND~4.0\\
        \href{https\c_colon_str//creativecommons.org/licenses/by-nc-nd/4.0/}{%
        (Creative~Commons~Attribution-NonCommercial-NoDerivatives~4.0~International~License)}}
        \hypersetup{pdflicenseurl={https://creativecommons.org/licenses/by-nc-nd/4.0/}}
      }
    }
    % If there is no match, use the user-supplied text
    {}
}

\seq_gclear_new:N \@seqkeywordspt
\seq_gclear_new:N \@seqkeywordsen
\newcommand*{\palavrachave}[1]{\seq_gput_right:Nn \@seqkeywordspt {#1}}
\newcommand*{\keyword}[1]{\seq_gput_right:Nn \@seqkeywordsen {#1}}

% Na impressão, as palavras-chave são separadas por pontos
\newcommand*{\@keywordspt}{\seq_use:Nn \@seqkeywordspt {.\space}.}
\newcommand*{\@keywordsen}{\seq_use:Nn \@seqkeywordsen {.\space}.}

% Para inclusão nos metadados com hyperxmp, são separadas por vírgulas
\newcommand*{\@commakeywordspt}{\seq_use:Nn \@seqkeywordspt {,}}
\newcommand*{\@commakeywordsen}{\seq_use:Nn \@seqkeywordsen {,}}

\newcommand*{\@currlangkeywords}{%
    \IfLanguagePatterns{brazilian}
      {\@keywordspt}
      {\@keywordsen}%
}

\newcommand*{\@currlangtitle}{%
    \IfLanguagePatterns{brazilian}
      {\@titlept}
      {\@titleen}%
}

\newcommand*{\@currlangsubtitle}{%
    \IfLanguagePatterns{brazilian}
      {\@subtitlept}
      {\@subtitleen}%
}

\ExplSyntaxOff

\NewDocumentCommand{\@doutorado}{}{
  \toggletrue{@doutorado}
  \togglefalse{@mestrado}
  \togglefalse{@tcc}
}

\NewDocumentCommand{\@mestrado}{}{
  \togglefalse{@doutorado}
  \toggletrue{@mestrado}
  \togglefalse{@tcc}
}

\NewDocumentCommand{\@tcc}{}{
  \togglefalse{@mestrado}
  \togglefalse{@doutorado}
  \toggletrue{@tcc}
}

% Defaults quando o usuário não define alguma dessas variáveis.
% Não podemos usar \title ou \author aqui porque esses comandos
% dependem de hyperref, que ainda não foi carregada.
\providecommand\@author{Autor não definido!}
\orientador{Orientador não definido!}
\DTMsavedate{@defensedate}{1970-01-01}
\providecommand\@program{Programa não definido!}
\providecommand\@financing{}
\providecommand\@defenselocation{Local não definido!}
\providecommand\@license{Direitos não definidos!}
\providecommand\@title{Título não definido!}
\providecommand\@titlept{Título em português não definido!}
\providecommand\@titleen{Título em inglês não definido!}
\providecommand\@resumo{Resumo não definido!}
\providecommand\@abstract{Abstract não definido!}


%%%%%%%%%%%%%%%%%%%%%%%%%%%%%%%%%%%%%%%%%%%%%%%%%%%%%%%%%%%%%%%%%%%%%%%%%%%%%%%%
%%%%%%%%%%%%%%%%%%%%%%%%%%%%%% TÍTULO E SUBTÍTULO %%%%%%%%%%%%%%%%%%%%%%%%%%%%%%
%%%%%%%%%%%%%%%%%%%%%%%%%%%%%%%%%%%%%%%%%%%%%%%%%%%%%%%%%%%%%%%%%%%%%%%%%%%%%%%%

\ExplSyntaxOn

% Opções usando LaTeX3 (veja texdoc l3keys).
\keys_define:nn { IME / title }
  {
    % Chaves à esquerda definem as variáveis à direita
    titlept .tl_gset:N = \@titlept,
    titlept .value_required:n = true,
    titleen .tl_gset:N = \@titleen,
    titleen .value_required:n = true,
    subtitlept .tl_gset:N = \@subtitlept,
    subtitlept .value_required:n = true,
    subtitleen .tl_gset:N = \@subtitleen,
    subtitleen .value_required:n = true,
  }

\RenewDocumentCommand\title{m}{
  \keys_set:nn {IME/title}{#1}

  \bgroup
  % O título deve existir nas duas línguas; o subtítulo é opcional,
  % mas se houver também deve existir nas duas línguas.
  \IfLanguagePatterns{brazilian}
    {
      \global\let\@title\@titlept
      \ifdefvoid{\@subtitlept}
        {}
        {\global\let\@subtitle\@subtitlept}

      \let\@mainlangtitle\@titlept
      \let\@mainlangsubtitle\@subtitlept
      \let\@otherlangtitle\@titleen
      \let\@otherlangsubtitle\@subtitleen
      \def\@otherlangname{en}
    }
    {
      \global\let\@title\@titleen
      \ifdefvoid{\@subtitlept}
        {}
        {\global\let\@subtitle\@subtitlept}

      \let\@mainlangtitle\@titleen
      \let\@mainlangsubtitle\@subtitleen
      \let\@otherlangtitle\@titlept
      \let\@otherlangsubtitle\@subtitlept
      \def\@otherlangname{pt}
    }

  % Insere os metadados XMP no arquivo PDF final.
  % \@IMEremoveLinebreaksEtc está definida em hyperlinks.tex.

  \@IMEremoveLinebreaksEtc{\@mainlangtitle}
  \@IMEremoveLinebreaksEtc{\@mainlangsubtitle}
  \@IMEremoveLinebreaksEtc{\@otherlangtitle}
  \@IMEremoveLinebreaksEtc{\@otherlangsubtitle}

  \hypersetup{
    pdftitle={\@mainlangtitle
              \ifdefvoid{\@mainlangsubtitle}{}{:~\@mainlangsubtitle}%
             },
  }

  \XMPLangAlt{\@otherlangname}{
      pdftitle={\@otherlangtitle
                \ifdefvoid{\@otherlangsubtitle}{}{:~\@otherlangsubtitle}%
               },
  }

  % XMPLangAlt undefines "\do"; this may cause
  % problems with biblatex, but we do not need
  % to worry here because we are in a group.
  % https://github.com/plk/biblatex/issues/1105
  \egroup
}

\ExplSyntaxOff


%%%%%%%%%%%%%%%%%%%%%%%%%%%%%%%%%%%%%%%%%%%%%%%%%%%%%%%%%%%%%%%%%%%%%%%%%%%%%%%%
%%%%%%%%%%%%%%%%%%%%%%%%%%%%%%%%%% DEDICATÓRIA %%%%%%%%%%%%%%%%%%%%%%%%%%%%%%%%%
%%%%%%%%%%%%%%%%%%%%%%%%%%%%%%%%%%%%%%%%%%%%%%%%%%%%%%%%%%%%%%%%%%%%%%%%%%%%%%%%

% A dedicatória vai em uma página separada, sem numeração,
% com o texto alinhado à direita e margens esquerda e
% superior muito grandes. Vamos fazer isso com uma minipage.
\newenvironment{dedicatoria} {
  \hypersetup{pageanchor=false} % Veja comentário em \maketitle

  \if@openright\cleardoublepage\else\clearpage\fi

  \thispagestyle{empty}
  \vspace*{140mm plus 0mm minus 100mm}
  \noindent
  \begin{FlushRight}
     \begin{minipage}[b][100mm][b]{100mm}
       \begin{FlushRight}
         \itshape
} {
       \end{FlushRight}
     \end{minipage}\hspace*{3em}
  \end{FlushRight}
  \vspace*{50mm plus 0mm minus 10mm}
  \if@openright\cleardoublepage\else\clearpage\fi

  \hypersetup{pageanchor=true}
}


%%%%%%%%%%%%%%%%%%%%%%%%%%%%%%%%%%%%%%%%%%%%%%%%%%%%%%%%%%%%%%%%%%%%%%%%%%%%%%%%
%%%%%%%%%%%%%%%%%%%%%%%%%%%%%%%%%%% RESUMO %%%%%%%%%%%%%%%%%%%%%%%%%%%%%%%%%%%%%
%%%%%%%%%%%%%%%%%%%%%%%%%%%%%%%%%%%%%%%%%%%%%%%%%%%%%%%%%%%%%%%%%%%%%%%%%%%%%%%%

% A página de resumo deve existir em português e inglês; ambas as versões
% utilizam o mesmo environment.

\NewDocumentCommand{\resumo}{+m}{%
  \long\gdef\@resumo{#1}%

  \bgroup
  \@IMEremoveLinebreaksEtc{\@resumo}
  \IfLanguagePatterns{brazilian}
    {
      \hypersetup{
        pdfsubject={\@resumo},
        pdfkeywords={\@commakeywordspt},
      }
    }
    {
      \XMPLangAlt{pt}{pdfsubject={\@resumo}}
      % o item "keywords" não pode ser traduzido

      % XMPLangAlt undefines "\do"; this may cause
      % problems with biblatex, but we do not need
      % to worry here because we are in a group.
      % https://github.com/plk/biblatex/issues/1105
    }
  \egroup

  \bgroup\bgroup % Dois grupos aninhados, veja a documentação da package babel
  \expandafter\selectlanguage\expandafter{\@IMEpt}
  \begin{IMEabstract}\@resumo\end{IMEabstract}
  \egroup\egroup
}

\DeclareDocumentCommand{\abstract}{+m}{%
  \long\gdef\@abstract{#1}%

  \bgroup
  \@IMEremoveLinebreaksEtc{\@abstract}
  \IfLanguagePatterns{brazilian}
    {
      \XMPLangAlt{en}{pdfsubject={\@abstract}}
      % o item "keywords" não pode ser traduzido

      % XMPLangAlt undefines "\do"; this may cause
      % problems with biblatex, but we do not need
      % to worry here because we are in a group.
      % https://github.com/plk/biblatex/issues/1105
    }
    {
      \hypersetup{
        pdfsubject={\@abstract},
        pdfkeywords={\@commakeywordsen},
      }
    }
  \egroup

  \bgroup\bgroup % Dois grupos aninhados, veja a documentação da package babel
  \expandafter\selectlanguage\expandafter{\@IMEen}
  \begin{IMEabstract}\@abstract\end{IMEabstract}
  \egroup\egroup
}

\NewDocumentEnvironment{IMEabstract}{} {
  \if@openright\cleardoublepage\else\clearpage\fi
  \thispagestyle{empty}

    \begin{center}\Large\bfseries\abstractname\end{center}

  \vspace*{2em plus 1em minus 1em}

  \footnotesize

  % Esse é o jeito mais simples de mudar as margens de um parágrafo:
  % faz de conta que é uma lista
  \begin{list}{}{\rightmargin 4em \leftmargin 4em}
    \item\@selfReference
  \end{list}

  \vspace*{1em plus 1em minus 0em}
} {
  % Impede uma quebra de página entre esta linha e a próxima, ou seja,
  % entre a última linha do resumo/abstract e as palavras-chave.
  \@afterheading

  \vspace*{1em plus 1em minus .5em}

  \begingroup

      \setlength{\leftmargini}{\widthof{\textbf{\keywordsname:}\quad}}
      \setlength{\labelwidth}{\widthof{\textbf{\keywordsname:}}}
      \setlength{\labelsep}{\widthof{\quad}}

      \begin{description}\item[\keywordsname:]\@currlangkeywords\end{description}

  \endgroup
}


%%%%%%%%%%%%%%%%%%%%%%%%%%%%%%%%%%%%%%%%%%%%%%%%%%%%%%%%%%%%%%%%%%%%%%%%%%%%%%%%
%%%%%%%%%%%%%%%%%%%%%% IMPRIME A CAPA E A FOLHA DE ROSTO %%%%%%%%%%%%%%%%%%%%%%%
%%%%%%%%%%%%%%%%%%%%%%%%%%%%%%%%%%%%%%%%%%%%%%%%%%%%%%%%%%%%%%%%%%%%%%%%%%%%%%%%

\RenewDocumentCommand\maketitle{}{
  % Embora as páginas iniciais *pareçam* não ter numeração, a numeração
  % existe, só não é impressa. Os comandos \frontmatter, \mainmatter,
  % \pagenumbering etc. reiniciam a contagem de páginas quando os números
  % passam a ser impressos. Isso significa que há mais de uma página com
  % o número "1". O pacote hyperref não lida bem com essa situação, então
  % vamos desabilitar hyperlinks para números de páginas aqui.
  \hypersetup{pageanchor=false}
  \bgroup
  \onehalfspacing

  \@IMEcover
  \iftoggle{@tcc}{}{\@IMEtitlePage}
  \@IMEversoPage

  \egroup
  \if@openright\cleardoublepage\else\clearpage\fi
  \hypersetup{pageanchor=true}
}

% Layout da capa
\NewDocumentCommand{\@IMEcover}{} {
  \cleardoublepage
  \thispagestyle{empty}

  \begin{hyphenrules}{nohyphenation}
      \iftoggle{@tcc}{\@institutionBlock}{}
      \@titleBlock
      \vspace*{\fill}
      \@detailsBlock
  \end{hyphenrules}
}

% Layout para a página de rosto (duas versões, de acordo
% com a Resolução CoPGr 6018 de 13/10/2011)
\NewDocumentCommand{\@IMEtitlePage}{} {
  \if@openright\cleardoublepage\else\clearpage\fi
  \thispagestyle{empty}

  \begin{hyphenrules}{nohyphenation}
      \@titleBlock
      \vspace*{2cm plus 2cm minus 1cm}
      \@versionInfoBlock
      \vspace*{3.5cm plus 3cm minus 3.5cm}
      \iftoggle{@finalversion}{\@committeeBlock}{}
      \vspace*{2cm plus 2cm minus 2cm}
  \end{hyphenrules}
}

\NewDocumentCommand{\@IMEversoPage}{}{
  \clearpage
  \thispagestyle{empty}
  \vspace*{4cm plus 4cm minus 2cm}
  \@versoPageBlock
  \vspace*{8cm plus 5cm minus 6cm}
}


%%%%%%%%%%%%%%%%%%%%%%%%%%%%%%%%%%%%%%%%%%%%%%%%%%%%%%%%%%%%%%%%%%%%%%%%%%%%%%%%
%%%%%%%%%%%%%%%%%%%%%%%% POSIÇÃO DOS ELEMENTOS NA CAPA %%%%%%%%%%%%%%%%%%%%%%%%%
%%%%%%%%%%%%%%%%%%%%%%%%%%%%%%%%%%%%%%%%%%%%%%%%%%%%%%%%%%%%%%%%%%%%%%%%%%%%%%%%

% O IME usa uma capa padrão de cartolina para todas as teses/dissertações.
% Essa capa tem uma janela recortada por onde se vê o título e o autor do
% trabalho. Ela fica centralizada na página, tem 100m de largura, 60mm de
% altura e começa 47mm abaixo do topo da página. Como o documento já tem
% margens definidas pelo usuário, precisamos calcular quanto precisamos
% acrescentar ou subtrair dessas margens para colocar o título e autor
% na posição exata (na verdade, com uma pequena folga: 49mm abaixo do topo
% da página, 96mm de largura e 56mm de altura).
%
% Para centralizar horizontalmente, poderíamos pensar em usar "\center",
% mas isso não funciona porque ele centraliza o texto em relação à coluna
% de texto, não à página. Assim, como as margens esquerda e direita do
% documento podem ser diferentes, a janela não ficaria na posição correta.
% O que faremos, então, é colocar essa janela em uma minipage e calcular
% a margem esquerda para que essa minipage fique centralizada.
%
% Além disso, outros elementos da capa também não podem ser centralizados
% com "\center", porque eles ficariam desalinhados em relação à janela
% com o título e autor. Vamos colocar esses outros elementos em uma
% minipage também, mas de tamanho diferente da anterior.
%
% Então, precisamos calcular três valores: a margem adicional em relação ao
% topo da página, a margem esquerda da janela com título e autor e a margem
% esquerda para os demais elementos centralizados da página.

\AtEndPreamble{
  % Calcula o valor das margens (após geometry ser carregada)

  % A distância entre o topo da página e o início do texto (fora o cabeçalho)
  % é dada por (1in + \voffset + \headsep + \topmargin + \headheight).
  % Queremos colocar a caixa com o título 49mm abaixo do topo, então:
  \dimgdef\@topTitleBlockMargin{49mm - (1in + \voffset + \headsep + \topmargin + \headheight)}

  % Quando \vspace é usado no início da página, ele não tem efeito; como
  % não é isso que queremos, vamos usar \vspace*. No entanto, \vspace*
  % é implementado inserindo uma \hrule de espessura zero e depois
  % acrescentando o espaço solicitado. O resultado não é exatamente
  % o esperado, pois \topskip, \parskip e \baselineskip interagem com
  % \vspace* de maneira um tanto complexa:
  % https://tex.stackexchange.com/a/247516/183146
  %
  % Aqui, vamos compensar essa diferença. Note que, se a primeira linha
  % da página tivesse um tamanho de fonte especial, seria necessário
  % usar o valor de \baselineskip correspondente a essa fonte. Além
  % disso, definimos espaçamento simples porque o \vspace* mencionado
  % acima é executado com espaçamento simples.
  \bgroup
  \setstretch {\setspace@singlespace}% \singlespacing adds \baselineskip
  \dimgdef\@topTitleBlockMargin{\@topTitleBlockMargin - \baselineskip - \parskip}
  \egroup

  % Queremos colocar a caixa com o título centralizada na página. "\center"
  % centraliza em função da área de texto, não da página inteira, então
  % não podemos usá-lo, pois as margens esquerda e direita podem ser
  % diferentes. A distância entre a borda esquerda/interna do papel e o
  % início do texto é dada por (1in + \hoffset + \oddsidemargin), então:
  \dimgdef\@leftTitleBlockMargin{(\paperwidth - 96mm)/2 - (1in + \hoffset + \oddsidemargin)}
  \dimgdef\@coverLeftMargin{(\paperwidth - 160mm)/2 - (1in + \hoffset + \oddsidemargin)}
}


%%%%%%%%%%%%%%%%%%%%%%%%%%%%%%%%%%%%%%%%%%%%%%%%%%%%%%%%%%%%%%%%%%%%%%%%%%%%%%%%
%%%%%%%%%%%%% OS ELEMENTOS QUE COMPÕEM A CAPA E A FOLHA DE ROSTO %%%%%%%%%%%%%%%
%%%%%%%%%%%%%%%%%%%%%%%%%%%%%%%%%%%%%%%%%%%%%%%%%%%%%%%%%%%%%%%%%%%%%%%%%%%%%%%%

% Com fontspec (ou seja, lualatex/xelatex), o comando \oldstylenums funciona
% com qualquer fonte que tenha suporte a números old-style. Já com pdflatex,
% o comando para escolher números old style depende da fonte em uso. Nesse
% caso, se não soubermos qual a fonte atual (ou seja, não é nem libertine
% nem libertinus), vamos usar latin modern e torcer para o resultado não ser
% muito discrepante do restante do texto.

% 499 = CDXCIX
\@ifpackageloaded{fontspec}
  {\providecommand{\@macCDXCIX}{mac~\oldstylenums{499}}}
  {
    \providecommand{\@macCDXCIX}{{\fontfamily{lmr}\selectfont mac~\oldstylenums{499}}}

    \@ifpackageloaded{libertinus}
      {\renewcommand{\@macCDXCIX}{{\LibertinusSerifOsF mac~499}}}
      {}

    \@ifpackageloaded{libertine}
      {\renewcommand{\@macCDXCIX}{{\libertineOsF mac~499}}}
      {}
  }

\newcommand{\@coverText}{
  \bgroup
  \setstretch{.9}

  \iftoggle{@tcc}
    {\@coverTCCText}
    {\iftoggle{@qualificacao}{\@coverQualiText}{\@coverThesisText}}
  \par
  \egroup
}

\ExplSyntaxOn
\newcounter{@IMEtmpcnt}
\newcommand*{\@coverPeople} {%
  \begin{tabular}{rl}
    \iftoggle{@tcc}{}{\programname : & \@program \tabularnewline}
    \advisorname : & \@advisor \tabularnewline
    \setcounter{@IMEtmpcnt}{0}%
    \int_while_do:nNnn {\value{@IMEtmpcnt}} < {\value{numberOfCoadvisors}} {%
      \stepcounter{@IMEtmpcnt}%
      \coadvisorname{\Roman{@IMEtmpcnt}}: & \csuse{@coadvisor\Roman{@IMEtmpcnt}} \tabularnewline
    }%
  \end{tabular}
}
\ExplSyntaxOff

\newcommand{\@selfReference} {%
  \bgroup
  \edef\@tempa{\@currlangtitle}
  \let\@currlangtitle\@tempa
  \edef\@tempa{\@currlangsubtitle}
  \let\@currlangsubtitle\@tempa
  \@IMEremoveLinebreaksEtc{\@currlangtitle}%
  \@IMEremoveLinebreaksEtc{\@currlangsubtitle}%
  \@author.
  \textbf{\@currlangtitle\ifdefvoid{\@currlangsubtitle}{}{: \textit{\@currlangsubtitle}}}.
  \workname{} (\degreename).
  \@institution,
  São Paulo, \DTMfetchyear{@defensedate}.%
  \egroup
}

\NewDocumentCommand{\@versoPageBlock}{} {
  \bgroup
  \onehalfspacing
  \begin{list}{}{\rightmargin 3em \leftmargin 3em}
    \item
      % Cuidado com este bug: https://github.com/schlcht/microtype/issues/10
      \begingroup\centering\footnotesize\itshape\@license\par\endgroup

      \ifcsvoid{@catalogingData} {} {
        \vspace*{3cm plus 3cm minus 1cm}
        \setlength{\fboxsep}{20pt}
        \begin{center}
        \fbox{
          \begin{minipage}[t]{120mm}
            \setlength\parskip{1em}

            \@catalogingData

          \end{minipage}
        }
        \end{center}
      }
  \end{list}
  \egroup
}

% Só para TCC
\newcommand{\@institutionBlock}{

    % A posição do quadro de título é fixa em relação à página;
    % a posição deste quadro é definida em função da posição do
    % quadro de título. Assim, primeiro vamos encontrar onde
    % deve começar o quadro do título. Veja os comentários em
    % \@titleBlock para entender o mecanismo.
    \bgroup
    \hfuzz=60pt % Não precisa avisar que estamos invadindo a margem direita aqui

    \setstretch {\setspace@singlespace}% \singlespacing adds \baselineskip

    \vspace*{\@topTitleBlockMargin}
    \ifdeflength{\@normalstrutheight}
      {}
      {\newlength{\@normalstrutheight}}
    \settoheight{\@normalstrutheight}{\strut}
    \vspace{-\@normalstrutheight}

    % Estamos alinhados com o quadro do título do trabalho,
    % mas não é isso que queremos: a parte inferior deste
    % quadro deve ficar 15mm acima do quadro de título e
    % este quadro tem 20mm de altura, então precisamos subir:
    \vspace{-20mm} % Espaço ocupado por este quadro
    \vspace{-15mm} % Espaço entre este quadro e o quadro de título

    \noindent\strut%
    \hspace*{\@coverLeftMargin}%
%    \fbox{%
      \begin{minipage}[t][20mm][s]{160mm}
        \vspace{0pt plus 20mm}

        \centering\large

        \textsc{\@institutionBlockText}

        \vspace{0pt plus 20mm}
      \end{minipage}
%    }% fbox
    \par

    % Agora precisamos voltar o "cursor" para o começo da página
    % para que o quadro de título seja inserido no lugar certo.
    % Para isso, vamos:
    %
    % 1. Chegar novamente ao início do quadro de título e
    %
    % 2. Retroceder o tamanho da margem superior

    % compensa o espaço inserido por \par logo acima
    \vspace{-\parskip}
    \egroup

    % A altura da minipage já compensou o \vspace{-20mm} acima;
    % ainda precisamos compensar o \vspace{-15mm}
    \vspace{15mm}

    % Agora estamos no início do quadro de título, então
    % podemos recuar exatamente o tamanho da margem superior.
    \vspace{-\@topTitleBlockMargin}
}

% O quadro com o título e o autor que deve ser visível
% através da janela na capa.
\NewDocumentCommand{\@titleBlock}{} {

    \bgroup
    \setstretch {\setspace@singlespace}% \singlespacing adds \baselineskip

    % Este espaço coloca o topo da próxima linha
    % na posição que queremos:
    \vspace*{\@topTitleBlockMargin}

    % No entanto, a próxima linha contém apenas
    % uma minipage, e definir o topo de uma linha
    % desse tipo é complicado. Assim, vamos:
    %
    % 1. Acrescentar um \strut a essa linha;
    %
    % 2. mover o baseline dessa linha para o topo do \strut;
    %
    % 3. Alinhar o topo da minipage ao baseline da linha.
    %
    % Sobre alinhamento de minipages:
    % https://en.wikibooks.org/wiki/LaTeX/Boxes

    \ifdeflength{\@normalstrutheight}
      {}
      {\newlength{\@normalstrutheight}}
    \settoheight{\@normalstrutheight}{\strut}
    \vspace{-\@normalstrutheight}

    \noindent\strut
    \hspace*{\@leftTitleBlockMargin}%
%    \fbox{%
      \begin{minipage}[t][56mm][s]{96mm}
          \vspace*{2cm plus 1.5cm minus 1.8cm}

          \centering\large

          \textbf{\@title}

          \vspace{0.3cm plus 0.2cm minus 0.1cm}

          \ifdefvoid{\@subtitle}{}{\textbf{\textit{\@subtitle}}}

          \vspace{1cm plus 1cm minus 0.6cm}

          \@author

          \vspace*{2cm plus 1.5cm minus 1.8cm}
      \end{minipage}%
%    }% fbox
    \par
    \egroup
}

% As demais informações da capa
\NewDocumentCommand{\@detailsBlock}{} {

  \bgroup
  \hfuzz=60pt % Não precisa avisar que estamos invadindo a margem direita aqui
  \onehalfspacing
  \noindent
  \hspace*{\@coverLeftMargin}%
%  \fbox{%
    \begin{minipage}[t][130mm][s]{160mm}
      \begin{center}
        \Large

        \vspace*{0.3cm plus 0.5cm minus 0.3cm}

        \textsc{\@coverText}

        \vspace*{1.5cm plus 0.5cm minus 0.5cm}

        \large\@coverPeople

        \vspace*{2.5cm plus 1cm minus 1cm}

        \normalsize

        \bgroup
        \singlespacing
        \@financing\par
        \egroup

        \vspace*{1cm plus 1cm minus 0.3cm}

        \@defenselocation

        \iftoggle{@tcc}
          {\DTMfetchyear{@defensedate}}
          {\DTMsetdatestyle{month-year}\DTMusedate{@defensedate}}

      \end{center}
    \end{minipage}%
%  }% fbox
  \par
  \egroup
}

% As informações da banca que vão apenas na versão definitiva
% da página de rosto
\ExplSyntaxOn
\NewDocumentCommand{\@committeeBlock}{} {
    \bgroup
    \onehalfspacing
    \begin{minipage}[t][][t]{\textwidth}
      \begin{quote}
        \normalsize\noindent\committeename :\par
        \begin{list}{}
        {
          \setlength{\leftmargin}{0pt}
          \setlength{\itemsep}{.1\baselineskip}
          \setlength{\topsep}{\baselineskip}
        }
          \item[] \seq_use:Nn \@committeeMembers {\item[]}
        \end{list}
      \end{quote}
    \end{minipage}
    \par
    \egroup
}
\ExplSyntaxOff

% A informação sobre a versão provisória ou definitiva
\NewDocumentCommand{\@versionInfoBlock}{} {%
  % As diretrizes dizem que "A natureza do trabalho, o grau pretendido, o
  % nome da instituição a que é submetido e a área de concentração devem
  % ser alinhados a partir do meio da parte impressa da página para a
  % margem direita, tanto na folha de rosto como na folha de avaliação."
  %
  % Assim, queremos alinhar o texto à direita com uma grande margem
  % à esquerda. Uma solução simples é alinhar o texto à direita
  % e inserir uma minipage. Dentro dela, definimos o texto
  % também alinhado à direita.

  \bgroup
  \onehalfspacing
  \begin{FlushRight}
    %\fbox{
      % Margem direita + 80mm de largura significa que a minipage
      % começa mais ou menos no meio da página.
      \begin{minipage}[t][50mm][s]{80mm}
        \begin{FlushRight}
          \normalsize
          \iftoggle{@finalversion}{%
            \@finalFrontmatterText%
          } {%
            \@provisionalFrontmatterText%
          }%
        \end{FlushRight}
        \vspace*{0pt plus 50mm}
      \end{minipage}
      \par
    %} % fbox
  \end{FlushRight}
  \egroup
}


%%%%%%%%%%%%%%%%%%%%%%%%%%%%%%%%%%%%%%%%%%%%%%%%%%%%%%%%%%%%%%%%%%%%%%%%%%%%%%%%
%%%%%%%%%%%%%%%%%%%%%%%%%%%%%% METADADOS XMP %%%%%%%%%%%%%%%%%%%%%%%%%%%%%%%%%%%
%%%%%%%%%%%%%%%%%%%%%%%%%%%%%%%%%%%%%%%%%%%%%%%%%%%%%%%%%%%%%%%%%%%%%%%%%%%%%%%%

% TODO: Com versões recentes de hyperxmp (final de 2020), não é
%       recomendado definir pdflang; no futuro, isto deve ser mudado.
\AtEndPreamble{
\IfLanguagePatterns{brazilian}
  {
    \hypersetup{
      pdflang={pt},
      pdfmetalang={pt},
    }
  }
  {
    \hypersetup{
      pdflang={en},
      pdfmetalang={en},
    }
  }
}


%%%%%%%%%%%%%%%%%%%%%%%%%%%%%%%%%%%%%%%%%%%%%%%%%%%%%%%%%%%%%%%%%%%%%%%%%%%%%%%%
%%%%%%%%%%%%%%%%%%%%%%%%%%%%% SUMÁRIO E SEÇÕES %%%%%%%%%%%%%%%%%%%%%%%%%%%%%%%%%
%%%%%%%%%%%%%%%%%%%%%%%%%%%%%%%%%%%%%%%%%%%%%%%%%%%%%%%%%%%%%%%%%%%%%%%%%%%%%%%%

% titlesec permite definir formatação personalizada de títulos, seções etc.
% Observe que titlesec é incompatível com os comandos refsection
% e refsegment do pacote biblatex!
% Esta package utiliza titlesec e implementa a possibilidade de incluir
% uma imagem no título dos capítulos com o comando \imgchapter (leia
% os comentários no arquivo da package).
\usepackage{imagechapter} % carregado do diretório extras (veja basics.tex)

\makeatother
 % capa, páginas preliminares e alguns detalhes
%%%%%%%%%%%%%%%%%%%%%%%%%%%%%%%%%%%%%%%%%%%%%%%%%%%%%%%%%%%%%%%%%%%%%%%%%%%%%%%%
%%%%%%%%%%%%%%%%%%%%%%% SUMÁRIO, CABEÇALHOS, SEÇÕES %%%%%%%%%%%%%%%%%%%%%%%%%%%%
%%%%%%%%%%%%%%%%%%%%%%%%%%%%%%%%%%%%%%%%%%%%%%%%%%%%%%%%%%%%%%%%%%%%%%%%%%%%%%%%

% Formatação personalizada do sumário, lista de tabelas/figuras etc.
%\usepackage{titletoc}

% titlesec permite definir formatação personalizada de títulos, seções etc.
% Observe que titlesec é incompatível com os comandos refsection
% e refsegment do pacote biblatex!
% Vamos usar titlesec apenas
% para fazer títulos, seções etc. não serem justificados.
\usepackage[raggedright]{titlesec}

% Permite saber o número total de páginas; útil para colocar no
% rodapé algo como "página 3 de 10" com "\thepage\ de \zpageref{LastPage}"
%\usepackage{zref-lastpage,zref-user}

% Permite definir cabeçalhos e rodapés
%\usepackage{fancyhdr}

% Personalização de cabeçalhos e rodapés com o estilo deste modelo
\usepackage{imeusp-headers} % carregado do diretório extras (veja basics.tex)

% biblatex pode ser configurado para inserir a bibliografia no sumário;
% bibtex não oferece essa possibilidade. Com esta package, resolvemos
% esse problema.
\usepackage[nottoc,notlot,notlof]{tocbibind}

% Só olha até o nível 2 (subseções) para gerar o sumário e os
% cabeçalhos, ou seja, não coloca nomes de subsubseções (nível 3)
% no sumário nem nos cabeçalhos.
\setcounter{tocdepth}{2}

% Só numera até o nível 2 (subseções, como 2.3.1), ou seja, não numera
% sub-subseções (como 2.3.1.1). Veja que isso afeta referências
% cruzadas: se você fizer \ref{uma-sub-subsecao} sem que ela seja
% numerada, a referência vai apontar para a seção um nível acima.
\setcounter{secnumdepth}{2}

% Normalmente, o capítulo de introdução não deve ser numerado, mas
% deve aparecer no sumário. Por padrão, LaTeX não oferece uma solução
% para isso, então criamos aqui os comandos \unnumberedchapter,
% \unnumberedsection e \unnumberedsubsection.
\newcommand{\unnumberedchapter}[2][]{
  \ifblank{#1}
    {
      \chapter*{#2}
      \phantomsection
      \addcontentsline{toc}{chapter}{#2}
      \chaptermark{#2}
    }
    {
      \chapter*{#2}
      \phantomsection
      \addcontentsline{toc}{chapter}{#1}
      \chaptermark{#1}
    }
}

\newcommand{\unnumberedsection}[2][]{
  \ifblank{#1}
    {
      \section*{#2}
      \phantomsection
      \addcontentsline{toc}{section}{#2}
      \sectionmark{#2}
    }
    {
      \section*{#2}
      \phantomsection
      \addcontentsline{toc}{section}{#1}
      \sectionmark{#1}
    }
}

\newcommand{\unnumberedsubsection}[2][]{
  \ifblank{#1}
    {
      \subsection*{#2}
      \phantomsection
      \addcontentsline{toc}{subsection}{#2}
    }
    {
      \subsection*{#2}
      \phantomsection
      \addcontentsline{toc}{subsection}{#1}
    }
}


%%%%%%%%%%%%%%%%%%%%%%%%%%%%%%%%%%%%%%%%%%%%%%%%%%%%%%%%%%%%%%%%%%%%%%%%%%%%%%%%
%%%%%%%%%%%%%%%%%%%%%%%%%% ESPAÇAMENTO E ALINHAMENTO %%%%%%%%%%%%%%%%%%%%%%%%%%%
%%%%%%%%%%%%%%%%%%%%%%%%%%%%%%%%%%%%%%%%%%%%%%%%%%%%%%%%%%%%%%%%%%%%%%%%%%%%%%%%

% LaTeX por default segue o estilo americano e não faz a indentação da
% primeira linha do primeiro parágrafo de uma seção; este pacote ativa essa
% indentação, como é o estilo brasileiro
\usepackage{indentfirst}

% A primeira linha de cada parágrafo costuma ter um pequeno recuo para
% tornar mais fácil visualizar onde cada parágrafo começa. Além disso, é
% possível colocar um espaço em branco entre um parágrafo e outro. Esta
% package coloca o espaço em branco e desabilita o recuo; como queremos
% o espaço *e* o recuo, é preciso guardar o valor padrão do recuo e
% redefini-lo depois de carregar a package.
% TODO: depois que ubuntu 18.04 se tornar obsoleta (abril/2023), remover
%       as linhas "oldparindent" e carregar a package com a opção "indent".
\newlength\oldparindent
\setlength\oldparindent\parindent
\usepackage[parfill]{parskip}
\setlength\parindent\oldparindent


%%%%%%%%%%%%%%%%%%%%%%%%%%%%%%%%%%%%%%%%%%%%%%%%%%%%%%%%%%%%%%%%%%%%%%%%%%%%%%%%
%%%%%%%%%%%%%%%%%%%%%%%%%% EPÍGRAFE E NOTAS DE RODAPÉ %%%%%%%%%%%%%%%%%%%%%%%%%%
%%%%%%%%%%%%%%%%%%%%%%%%%%%%%%%%%%%%%%%%%%%%%%%%%%%%%%%%%%%%%%%%%%%%%%%%%%%%%%%%

% O formato padrão do pacote epigraph é bem feinho...
% Outra opção para epígrafes é o pacote quotchap
\usepackage{epigraph}

\setlength\epigraphwidth{.85\textwidth}

% Sem linha entre o texto e o autor
\setlength{\epigraphrule}{0pt}

% Ambiente auxiliar para colocar margem à direita da epígrafe
% (como sempre, o modo mais simples de mudar as margens de um
% pagrágrafo é fazer de conta que é uma lista)
\newenvironment{epShiftLeft}
  {
    \par\begin{list}{}
      {
        \leftmargin 0pt
        \labelwidth 0pt
        \labelsep 0pt
        \itemsep 0pt
        \topsep 0pt
        \partopsep 0pt
        \rightmargin 2em
      }
    \item\FlushRight
  }
  {
    \end{list}
    % O espaço padrão que epigraph coloca entre a citação
    % e o autor é muito pequeno; vamos aumentar um pouco
    \vspace*{.3\baselineskip}
  }

\renewcommand\textflush{epShiftLeft}
\renewcommand\sourceflush{epShiftLeft}

\newcommand{\epigrafe}[2] {%
  \ifstrempty{#2}{
    \epigraph{\itshape #1}{}
  }{
    \epigraph{\itshape #1}{--- #2}
  }
}

% Formato personalizado para as notas de rodapé. Copiado quase
% literalmente do exemplo na documentação das classes-padrão de
% LaTeX2e (texdoc classes). Seria possível fazer algo similar
% usando list{} com um único item usando \@thefnmark como label.

% \footnotesep não é um espaço adicional, mas sim um strut que
% existe no começo de cada nota. É por isso que o valor é "grande"
% (\baselineskip) mas a separação de fato é pequena.
\makeatletter
\renewcommand\@makefntext[1]{%
    \setlength{\footnotesep}{1\baselineskip}%
    \@setpar{%
        \@@par
        \@tempdima = \hsize
        \advance\@tempdima-4pt\relax
        \parshape \@ne 4pt \@tempdima
    }%
    \par
    \parindent 1em\noindent
    \parskip .3\baselineskip
    \hbox to \z@{\hss\@makefnmark\,}#1%
}
\makeatother

% \maketitle redefine as notas de rodapé (\thanks) para usar símbolos
% ao invés de números, mas essa não é a única mudança. \maketitle
% também muda \@makefnmark para que a indicação de nota de rodapé
% não ocupe espaço horizontal (isso é feito com \rlap). Isso é feito
% porque a lista de autores em geral é similar a
% \author{Fulano\thanks{instituição 1}, Ciclano\thanks{instituição 2}}.
% Com essa mudança, a nota aparece acima da vírgula entre os autores.
% Mas isso significa que\maketitle precisa também modificar \@makefntext
% para que esse efeito aconteça apenas na lista de autores e não na
% nota em si. Assim, como criamos um novo formato para as notas de
% rodapé, precisamos mudar o formato em \maketitle também.
\makeatletter
\newcommand\@maketitlemakefntext[1]{%
    \setlength{\footnotesep}{1\baselineskip}%
    \@setpar{%
        \@@par
        \@tempdima = \hsize
        \advance\@tempdima-4pt\relax
        \parshape \@ne 4pt \@tempdima
    }%
    \par
    \parindent 1em\noindent
    \parskip .3\baselineskip
    \hbox to \z@{\hss\@textsuperscript{\normalfont\@thefnmark}\,}#1%
}

\patchcmd\maketitle
  {\long\def\@makefntext}
  {\let\@makefntext\@maketitlemakefntext\long\def\@disabledmakefntext}
  {}{}

\makeatother

%%%%%%%%%%%%%%%%%%%%%%%%%%%%%%%%%%%%%%%%%%%%%%%%%%%%%%%%%%%%%%%%%%%%%%%%%%%%%%%%
%%%%%%%%%%%%%%%%%%%%%%%%%%%%%% ÍNDICE REMISSIVO %%%%%%%%%%%%%%%%%%%%%%%%%%%%%%%%
%%%%%%%%%%%%%%%%%%%%%%%%%%%%%%%%%%%%%%%%%%%%%%%%%%%%%%%%%%%%%%%%%%%%%%%%%%%%%%%%

% Cria índice remissivo. Este pacote precisa ser carregado antes de hyperref.
% A criação do índice remissivo depende de um programa auxiliar, que pode ser
% o "makeindex" (default) ou o xindy. xindy é mais poderoso e lida melhor com
% línguas diferentes e caracteres acentuados, mas o programa não está mais
% sendo mantido e índices criados com xindy não funcionam em conjunto com
% hyperref. Se quiser utilizar xindy mesmo assim, é possível contornar esse
% segundo problema configurando hyperref para *não* gerar hyperlinks no
% índice (mais abaixo) e configurando xindy para que ele gere esses hyperlinks
% por conta própria. Para isso, modifique a chamada ao pacote imakeidx (aqui)
% e altere as opções do pacote hyperref.
\providecommand\theindex{} % evita erros de compilação se a classe não tem index
%\usepackage[xindy]{imakeidx} % usando xindy
\usepackage{imakeidx} % usando makeindex

% Cria o arquivo de configuração para xindy lidar corretamente com hyperlinks.
\begin{filecontents*}{hyperxindy.xdy}
(define-attributes ("emph"))
(markup-locref :open "\hyperpage{" :close "}" :attr "default")
(markup-locref :open "\textbf{\hyperpage{" :close "}}" :attr "textbf")
(markup-locref :open "\textit{\hyperpage{" :close "}}" :attr "textit")
(markup-locref :open "\emph{\hyperpage{" :close "}}" :attr "emph")
\end{filecontents*}

% Cria o arquivo de configuração para makeindex colocar um cabeçalho
% para cada letra do índice.
\begin{filecontents*}{mkidxhead.ist}
headings_flag 1
heading_prefix "{\\bfseries "
heading_suffix "}\\nopagebreak\n"
\end{filecontents*}

% Por padrão, o cabeçalho das páginas do índice é feito em maiúsculas;
% vamos mudar isso e deixar fancyhdr definir a formatação
\indexsetup{
  othercode={\chaptermark{\indexname}},
}

\makeindex[
  noautomatic,
  intoc,
  % Estas opções são usadas por xindy
  % "-C utf8" ou "-M lang/latin/utf8.xdy" são truques para contornar este
  % bug, que existe em outras distribuições tambem:
  % https://bugs.launchpad.net/ubuntu/+source/xindy/+bug/1735439
  % Se "-C utf8" não funcionar, tente "-M lang/latin/utf8.xdy"
  %options=-C utf8 -M hyperxindy.xdy,
  %options=-M lang/latin/utf8.xdy -M hyperxindy.xdy,
  % Estas opções são usadas por makeindex
  options=-s mkidxhead.ist -l -c,
]

%%%%%%%%%%%%%%%%%%%%%%%%%%%%%%%%%%%%%%%%%%%%%%%%%%%%%%%%%%%%%%%%%%%%%%%%%%%%%%%%
%%%%%%%%%%%%%%%%%%%%%%%%%%%%%%% BIBLIOGRAFIA %%%%%%%%%%%%%%%%%%%%%%%%%%%%%%%%%%%
%%%%%%%%%%%%%%%%%%%%%%%%%%%%%%%%%%%%%%%%%%%%%%%%%%%%%%%%%%%%%%%%%%%%%%%%%%%%%%%%

% Tradicionalmente, bibliografias no LaTeX são geradas com uma combinação entre
% LaTeX (muitas vezes usando o pacote natbib) e um programa auxiliar chamado
% bibtex. Nesse esquema, LaTeX e natbib são responsáveis por formatar as
% referências ao longo do texto e a formatação da bibliografia fica por conta
% do programa bibtex. A configuração dessa formatação é feita através de um
% arquivo auxiliar de "estilo", com extensão ".bst". Vários journals etc.
% fornecem o arquivo .bst que corresponde ao formato esperado da bibliografia.
%
% bibtex e natbib funcionam bem e, se você tiver alguma boa razão para usá-los,
% obterá bons resultados. No entanto, bibtex tem dois problemas: não lida
% corretamente com caracteres acentuados (embora, na prática, funcione com
% os caracteres usados em português) e o formato .bst, que define a formatação
% da bibliografia, é complexo e pouco flexível.
%
% Por conta disso, a comunidade está migrando para um novo sistema chamado
% biblatex. No biblatex, as formatações da bibliografia e das citações são
% feitas pelo próprio pacote biblatex, dentro do LaTeX. Assim, é bem mais fácil
% modificar e personalizar o estilo da bibliografia. biblatex usa o mesmo
% formato de arquivo de dados do bibtex (".bib") e, portanto, não é difícil
% migrar de um para o outro. biblatex também usa um programa auxiliar (biber),
% mas não para realizar a formatação da bibliografia. A maior desvantagem de
% biblatex é que ele é significativamente mais lento que bibtex.
%
% Observe que biblatex pode criar bibliografias independentes por capítulo
% ou outras divisões do texto. Normalmente é preciso indicar essas seções
% manualmente, mas as opções "refsection" e "refsegment" fazem biblatex
% identificar cada capítulo/seção/etc como uma nova divisão desse tipo.
% No entanto, refsection e refsegment são incompatíveis com o pacote
% titlesec, mencionado em imeusp-formatting.tex. Se você pretende criar
% bibliografias independentes por seções, há duas soluções: (1) desabilitar
% o pacote titlesec; (2) indicar as seções manualmente.
%
% Algumas dicas de configuração:
% https://tex.stackexchange.com/questions/12806/guidelines-for-customizing-biblatex-styles
% https://github.com/PaulStanley/biblatex-tutorial/releases

\PassOptionsToPackage{
  natbib=true, % Reconhece a sintaxe de natbib (\citet, \citep)
  hyperref=true, % Ativa o suporte ao pacote hyperref
  % Se um item da bibliografia tem língua definida (com langid), permite
  % hifenizar com base na língua selecionada.
  autolang=hyphen,
  % Inclui, em cada item da bibliografia, links para as páginas onde o
  % item foi citado
  backref=true,
}{biblatex}

% Este arquivo é executado antes de carregarmos biblatex, então precisamos
% adiar a execução deste comando. Não carregamos biblatex neste arquivo
% porque o usuário pode querer modificar o estilo bibliográfico, que é
% definido por um parâmetro na hora da carga da package.
\AtEndPreamble{
  % Impede que um item da bibliografia seja dividido em duas páginas.
  % À parte a estética, isso contorna este bug, que afeta links na
  % úlima página do trabalho, ou seja, pode afetar a bibliografia
  % (atenddvi pode ser carregada por hyperxmp):
  % https://github.com/ho-tex/atenddvi/issues/1
  \AtBeginBibliography{\interlinepenalty=10000\raggedbottom}
}

\makeatletter
\AtEndPreamble{
  % backrefs só fazem sentido com documentos grandes
  \ifboolexpr{test {\@ifclassloaded{book}} or test {\@ifclassloaded{report}}}
    {}
    {\ExecuteBibliographyOptions{backref=false,}}

  % em apresentações e posters, a bibliografia deve ser o mais compacta possível
  \@ifclassloaded{beamer}
    {\ExecuteBibliographyOptions{maxbibnames=2,maxcitenames=2}}
    {}
}
\makeatother

%%%%%%%%%%%%%%%%%%%%%%%%%%%%%%%%%%%%%%%%%%%%%%%%%%%%%%%%%%%%%%%%%%%%%%%%%%%%%%%%
%%%%%%%%%%%%%%%%%%%%%%%%%% HIPERLINKS E REFERÊNCIAS %%%%%%%%%%%%%%%%%%%%%%%%%%%%
%%%%%%%%%%%%%%%%%%%%%%%%%%%%%%%%%%%%%%%%%%%%%%%%%%%%%%%%%%%%%%%%%%%%%%%%%%%%%%%%

% O comando \ref por padrão mostra apenas o número do elemento a que se
% refere; assim, é preciso escrever "veja a Figura~\ref{grafico}" ou
% "como visto na Seção~\ref{sec:introducao}". Usando o pacote hyperref
% (carregado mais abaixo), esse número é transformado em um hiperlink.
%
% Se você quiser mudar esse comportamento, ative as packages varioref
% e cleveref e também as linhas "labelformat" e "crefname" mais abaixo.
% Nesse caso, você deve escrever apenas "veja a \ref{grafico}" ou
% "como visto na \ref{sec:introducao}" etc. e o nome do elemento será
% gerado automaticamente como hiperlink.
%
% Se, além dessa mudança, você quiser usar os recursos de varioref ou
% cleveref, mantenha as linhas labelformat comentadas e use os comandos
% \vref ou \cref, conforme sua preferência, também sem indicar o nome do
% elemento, que é inserido automaticamente. Vale lembrar que o comando
% \vref de varioref pode causar problemas com hyperref, impedindo a
% geração do PDF final.
%
% ATENÇÃO: varioref, hyperref e cleveref devem ser carregadas nessa ordem!
%\usepackage{varioref}

%\labelformat{figure}{Figura~#1}
%\labelformat{table}{Tabela~#1}
%\labelformat{equation}{Equação~#1}
%% Isto não funciona corretamente com os apêndices; o comando seguinte
%% contorna esse problema
%%\labelformat{chapter}{Capítulo~#1}
%\makeatletter
%\labelformat{chapter}{\@chapapp~#1}
%\makeatother
%\labelformat{section}{Seção~#1}
%\labelformat{subsection}{Seção~#1}
%\labelformat{subsubsection}{Seção~#1}

% XMP (eXtensible Metadata Platform) is a mechanism proposed by Adobe for
% embedding document metadata within the document itself. The package
% integrates seamlessly with hyperref and requires virtually no modifications
% to documents that already exploit hyperref's mechanisms for specifying PDF
% metadata. It should be loaded before hyperref.
\usepackage{hyperxmp}

% HACK ALERT! hyperxmp usa atenddvi, que tem este bug:
% https://github.com/ho-tex/atenddvi/issues/1 . Aparentemente, ele não
% afeta xelatex (cf https://github.com/latex3/latex2e/issues/94 ), então
% só precisamos nos preocupar com pdflatex e lualatex. Com pdflatex,
% hyperxmp não deveria utilizar atenddvi, mas às vezes usa. Com lualatex,
% me parece que atenddvi também não é necessária, pois hyperxmp não usa
% \special ou \write com lualatex. Então, vamos deixar hyperxmp carregar
% atenddvi mas (1) vamos impedi-la de funcionar e (2) vamos garantir que
% hyperxmp sempre use AtEndDocument com pdflatex e lualatex. Há ainda um
% outro truque para contornar esse problema na configuração da bibliografia,
% mas não podemos ter certeza que apenas a bibliografia pode ser afetada
% por este bug.

% TODO: remover isto quando o bug for corrigido
\makeatletter
\ifpdf
  \ifxetex
    \relax
  \else
    \let\AtEndDvi@AtBeginShipout\relax
    \let\AtEndDvi@CheckImpl\relax
    \let\hyxmp@at@end\AtEndDocument
  \fi
\fi
\makeatother

% Usamos "PassOptions" aqui porque outras packages definem opções para
% hyperref também e chamar a package com opções diretamente gera conflitos.
\PassOptionsToPackage{
  unicode=true,
  plainpages=false,
  pdfpagelabels,
  colorlinks=true,
  %citecolor=black,
  %linkcolor=black,
  %urlcolor=black,
  %filecolor=black,
  citecolor=DarkGreen,
  linkcolor=NavyBlue,
  urlcolor=DarkRed,
  filecolor=green,
  bookmarksopen=true,
  % hyperref não gera hyperlinks corretos em índices remissivos criados com
  % xindy; assim, é possível desabilitar essa função aqui e gerar os
  % hyperlinks através da configuração de xindy definida anteriormente. Com
  % makeindex (o default), quem precisa criar os hyperlinks é hyperref.
  %hyperindex=false,
}{hyperref}

% Cria hiperlinks para capítulos, seções, \ref's, URLs etc.
\usepackage{hyperref}

%\usepackage[nameinlink,noabbrev,capitalise]{cleveref}
%% cleveref não tem tradução para o português
%\crefname{figure}{Figura}{Figuras}
%\crefname{table}{Tabela}{Tabelas}
%\crefname{chapter}{Capítulo}{Capítulos}
%\crefname{section}{Seção}{Seções}
%\crefname{subsection}{Seção}{Seções}
%\crefname{subsubsection}{Seção}{Seções}
%\crefname{appendix}{Apêndice}{Apêndices}
%\crefname{subappendix}{Apêndice}{Apêndices}
%\crefname{subsubappendix}{Apêndice}{Apêndices}
%\crefname{line}{Linha}{Linhas}
%\crefname{subfigure}{Figura}{Figuras}
%\crefname{equation}{Equação}{Equações}
%\crefname{listing}{Código-fonte}{Códigos-fonte}
%\crefname{lstlisting}{Código-fonte}{Códigos-fonte}
%\crefname{lstnumber}{Linha}{Linhas}
%\crefrangelabelformat{chapter}{#3#1#4~a~#5#2#6}
%\crefrangelabelformat{section}{#3#1#4~a~#5#2#6}
%\newcommand{\crefrangeconjunction}{ e }
%\newcommand{\crefpairconjunction}{ e }
%\newcommand{\crefmiddleconjunction}{, }
%\newcommand{\creflastconjunction}{ e }
%\crefmultiformat{type}{first}{second}{middle}{last}
%\crefrangemultiformat{type}{first}{second}{middle}{last}

% ao criar uma referência hyperref para um float, a referência aponta
% para o final do caption do float, o que não é muito bom. Este pacote
% faz a referência apontar para o início do float (é possível personalizar
% também). Esta package é incompatível com a classe beamer (usada para
% criar posters e apresentações), então testamos a compatibilidade antes
% de carregá-la.
\ifboolexpr{
  test {\ifcsdef{figure}} and
  test {\ifcsdef{figure*}} and
  test {\ifcsdef{table}} and
  test {\ifcsdef{table*}}
}{\usepackage[all]{hypcap}}{}

% hyperref detecta url's definidas com \url que começam com "http" e
% "www" e cria links adequados. No entanto, quando a url não começa
% com essas strings (por exemplo, "usp.br"), hyperref considera que
% se trata de um link para um arquivo local. Isto força todas as
% \url's que não tem esquema definido a serem do tipo http.
\hyperbaseurl{http://}

%\nocolorlinks % para impressão em P&B
% Para formatar código-fonte (ex. em Java). listings funciona bem mas
% tem algumas limitações (https://tex.stackexchange.com/a/153915 ).
% Se isso for um problema, a package minted pode oferecer resultados
% (muito) melhores; a desvantagem é que ela depende de um programa
% externo, o pygments (escrito em python).
%
% listings também não tem suporte específico a pseudo-código, mas
% incluímos uma configuração para isso que deve ser suficiente.
% Caso contrário, há diversas packages específicas para a criação
% de pseudocódigo:
%
%  * a mais comum é algorithmicx ("\usepackage{algpseudocode}");
%
%  * algorithm2e é bastante flexível, mas um tanto complexa;
%
%  * clrscode3e foi usada no livro "Introduction to Algorithms",
%    de Cormen, Leiserson, Rivert e Stein;
%
%  * pseudocode foi usada no livro "Combinatorial Algorithms",
%    de Kreher e Stinson;
%
%  * algpseudocodex é uma package relativamente nova similar
%    a algorithmicx/algpseudocode mas com diversas melhorias;
%
%  * pseudo também é relativamente nova; ela funciona de forma
%    um pouco diferente das demais e é bastante customizável.
%
% A diferença entre essas packages e listings/minted é que estas
% últimas "entendem" o código e aplicam a formatação automaticamente,
% enquanto com as packages acima o usuário precisa usar comandos LaTeX
% para definir a formatação.
%
% algorithmicx/algpseudocode, algorithm2e, clrscode3e, pseudocode
% e algpseudocodex usam uma "linguagem" própria baseada em comandos
% LaTeX que pode ser facilmente modificada pelo usuário (ou seja,
% é fácil fazer pseudocódigo em português). Segue um exemplo com
% algpseudocodex, provavelmente a opção mais interessante dentre
% este grupo (note o comando "\While", que imprime automaticamente
% a palavra-chave "while" e ajusta a indentação):
%
% \begin{algorithmic}[1]
%   \Function{Euclid}{$a, b$} \Comment{The g.c.d. of a and b}
%     \State $r\gets a\bmod b$
%     \While{$r\not=0$} \Comment{We have the answer if r is 0}
%       \State $a\gets b$
%       \State $b\gets r$
%       \State $r\gets a\bmod b$
%     \EndWhile
%     \State \textbf{return} $b$ \Comment{The gcd is b}
%   \EndFunction
% \end{algorithmic}
%
% pseudo não usa uma "linguagem" própria desse tipo; ao invés disso,
% ela oferece comandos para a formatação direta de palavras-chave,
% variáveis, indentação etc. Um exemplo ("\\", "\\+" e "\\-" controlam
% as quebras de linha e a indentação):
%
% \begin{pseudo}
%    \kw{Function} \fn{Euclid}(a, b) \ct{The g.c.d. of a and b} \\+
%      $r\gets a\bmod b$ \\
%      \kw{while} $r\not=0$ \ct{We have the answer if r is 0} \\+
%        $a\gets b$ \\
%        $b\gets r$ \\
%        $r\gets a\bmod b$ \\-
%      \kw{end} \\
%      \kw{return} $b$ \ct{The gcd is b} \\-
%    \kw{end}
% \end{pseudo}

\usepackage{listings}
\usepackage{lstautogobble}
% Carrega a "linguagem" pseudocode para listings
\appto{\lstaspectfiles}{,lstpseudocode.sty}
\appto{\lstlanguagefiles}{,lstpseudocode.sty}
% Estes dois são carregados do diretório extras (veja basics.tex)
\lstloadaspects{simulatex,invisibledelims,pseudocode}
\lstloadlanguages{[base]pseudocode,[english]pseudocode,[brazilian]pseudocode}

% O pacote listings não lida bem com acentos! No caso dos caracteres acentuados
% usados em português é possível contornar o problema com a tabela abaixo.
% From https://en.wikibooks.org/wiki/LaTeX/Source_Code_Listings#Encoding_issue
\lstset{literate=
  {á}{{\'a}}1 {é}{{\'e}}1 {í}{{\'i}}1 {ó}{{\'o}}1 {ú}{{\'u}}1
  {Á}{{\'A}}1 {É}{{\'E}}1 {Í}{{\'I}}1 {Ó}{{\'O}}1 {Ú}{{\'U}}1
  {à}{{\`a}}1 {è}{{\`e}}1 {ì}{{\`i}}1 {ò}{{\`o}}1 {ù}{{\`u}}1
  {À}{{\`A}}1 {È}{{\'E}}1 {Ì}{{\`I}}1 {Ò}{{\`O}}1 {Ù}{{\`U}}1
  {ä}{{\"a}}1 {ë}{{\"e}}1 {ï}{{\"i}}1 {ö}{{\"o}}1 {ü}{{\"u}}1
  {Ä}{{\"A}}1 {Ë}{{\"E}}1 {Ï}{{\"I}}1 {Ö}{{\"O}}1 {Ü}{{\"U}}1
  {â}{{\^a}}1 {ê}{{\^e}}1 {î}{{\^i}}1 {ô}{{\^o}}1 {û}{{\^u}}1
  {Â}{{\^A}}1 {Ê}{{\^E}}1 {Î}{{\^I}}1 {Ô}{{\^O}}1 {Û}{{\^U}}1
  {Ã}{{\~A}}1 {ã}{{\~a}}1 {Õ}{{\~O}}1 {õ}{{\~o}}1
  {œ}{{\oe}}1 {Œ}{{\OE}}1 {æ}{{\ae}}1 {Æ}{{\AE}}1 {ß}{{\ss}}1
  {ű}{{\H{u}}}1 {Ű}{{\H{U}}}1 {ő}{{\H{o}}}1 {Ő}{{\H{O}}}1
  {ç}{{\c c}}1 {Ç}{{\c C}}1 {ø}{{\o}}1 {å}{{\r a}}1 {Å}{{\r A}}1
  {€}{{\euro}}1 {£}{{\pounds}}1 {«}{{\guillemotleft}}1
  {»}{{\guillemotright}}1 {ñ}{{\~n}}1 {Ñ}{{\~N}}1 {¿}{{?`}}1
}

% Opções default para o pacote listings
% Ref: http://en.wikibooks.org/wiki/LaTeX/Packages/Listings
\lstset{
  columns=[l]fullflexible,            % do not try to align text with proportional fonts
  basicstyle=\footnotesize\ttfamily,  % the font that is used for the code
  numbers=left,                       % where to put the line-numbers
  numberstyle=\footnotesize\ttfamily, % the font that is used for the line-numbers
  stepnumber=1,                       % the step between two line-numbers. If it's 1 each line will be numbered
  numbersep=20pt,                     % how far the line-numbers are from the code
  autogobble,                         % ignore irrelevant indentation
  commentstyle=\color{Brown}\upshape,
  stringstyle=\color{black},
  identifierstyle=\color{DarkBlue},
  keywordstyle=\color{cyan},
  showspaces=false,                   % show spaces adding particular underscores
  showstringspaces=false,             % underline spaces within strings
  showtabs=false,                     % show tabs within strings adding particular underscores
  %frame=single,                       % adds a frame around the code
  framerule=0.6pt,
  tabsize=2,                          % sets default tabsize to 2 spaces
  captionpos=b,                       % sets the caption-position to bottom
  breaklines=true,                    % sets automatic line breaking
  breakatwhitespace=false,            % sets if automatic breaks should only happen at whitespace
  escapeinside={\%*}{*)},             % if you want to add a comment within your code
  backgroundcolor=\color[rgb]{1.0,1.0,1.0}, % choose the background color.
  rulecolor=\color{darkgray},
  extendedchars=true,
  inputencoding=utf8,
  xleftmargin=30pt,
  xrightmargin=10pt,
  framexleftmargin=25pt,
  framexrightmargin=5pt,
  framesep=5pt,
}

% Um exemplo de estilo personalizado para listings (tabulações maiores)
\lstdefinestyle{wider} {
  tabsize = 4,
  numbersep=15pt,
  xleftmargin=25pt,
  framexleftmargin=20pt,
}

% Outro exemplo de estilo personalizado para listings (sem cores)
\lstdefinestyle{nocolor} {
  commentstyle=\color{darkgray}\upshape,
  stringstyle=\color{black},
  identifierstyle=\color{black},
  keywordstyle=\color{black}\bfseries,
}

% Um exemplo de definição de linguagem para listings (XML)
\lstdefinelanguage{XML}{
  morecomment=[s]{<!--}{-->},
  morecomment=[s]{<!-- }{ -->},
  morecomment=[n]{<!--}{-->},
  morecomment=[n]{<!-- }{ -->},
  morestring=[b]",
  morestring=[s]{>}{<},
  morecomment=[s]{<?}{?>},
  morekeywords={xmlns,version,type}% list your attributes here
}

% Estilo padrão para a "linguagem" pseudocode
\lstdefinestyle{pseudocode}{
  basicstyle=\rmfamily\small,
  commentstyle=\itshape,
  keywordstyle=\bfseries,
  identifierstyle=\itshape,
  % as palavras "function" e "procedure"
  procnamekeystyle=\bfseries\scshape,
  % funções precedidas por function/procedure ou com \func{}
  procnamestyle=\ttfamily,
  specialidentifierstyle=\ttfamily\bfseries,
}
\lstset{defaultdialect=[english]{pseudocode}}

% A package listings tem seu próprio mecanismo para a criação de
% captions, lista de programas etc. Neste modelo não usamos esses
% recursos (veja mais abaixo), mas utilizamos estes nomes:
\addto\extrasbrazil{%
  \gdef\lstlistlistingname{Lista de Programas}%
  \gdef\lstlistingname{Programa}%
}
\addto\extrasbrazilian{%
  \gdef\lstlistlistingname{Lista de Programas}%
  \gdef\lstlistingname{Programa}%
}
\addto\extrasenglish{%
  \gdef\lstlistlistingname{List of Programs}%
  \gdef\lstlistingname{Program}%
}

% Novo tipo de float para programas, possível graças à package float
% ou floatrow.
% Observe que a lista de floats de cada tipo é criada automaticamente
% pela package float/floatrow, mas precisamos:
%  1. Definir o nome do comando ("\begin{program}")
%  2. Definir o nome do float em cada língua ("Figura X", "Programa X")
%  3. Definir a extensão do arquivo temporário a ser usada. Pode ser
%     qualquer coisa, desde que não haja repetições. Aqui, usamos "lop";
%     lembre-se que LaTeX já usa várias outras, como "lof", "lot" etc.,
%     então seja cuidadoso na escolha!
%  4. Acrescentar os comandos correspondentes em paginas-preliminares.tex

\makeatletter
\@ifpackageloaded{floatrow}
  {
    \ltx@IfUndefined{chapter}
        % O novo ambiente se chama "program" ("\begin{program}") e a extensão
        % temporária é "lop"
        {\DeclareNewFloatType{program}{placement=htbp,fileext=lop}}
        {\DeclareNewFloatType{program}{placement=htbp,fileext=lop,within=chapter}}

    % Ajusta ligeiramente o espaçamento do estilo "ruled".
    \DeclareFloatVCode{customrule}{{\kern 0pt\hrule\kern 2.5pt\relax}}
    \floatsetup[program]{style=ruled,precode=customrule}
  }
  {
    % Não temos a package floatrow; vamos assumir que temos a package float.

    % O estilo padrão do novo float a ser criado (veja mais sobre isso na
    % documentação da package float). Para "program" usamos "ruled", mas
    % para outros floats provavelmente é melhor usar o mesmo formato de
    % Figuras e Tables (plain).
    \floatstyle{ruled}

    \ltx@IfUndefined{chapter}
        % O novo ambiente se chama "program" ("\begin{program}") e a extensão
        % temporária é "lop"
        {\newfloat{program}{htbp}{lop}}
        {\newfloat{program}{htbp}{lop}[chapter]}

    % Retorna o estilo dos floats para o padrão
    \floatstyle{plain}
  }
\makeatother

\captionsetup*[program]{style=ruled,position=top}

% "Program X / Programa X" e "Lista de Programas / List of Programs"
\floatname{program}{\lstlistingname}
\gdef\programlistname{\lstlistlistingname}

% Se um programa é maior que uma página, ele não pode ser inserido em
% um float. Nesse caso, vamos criar o ambiente "programruledcaption",
% que cria a mesma estrutura visual e os mesmos captions que os floats
% do tipo "program", mas sem ser um float. Para isso, vamos usar recursos
% da package framed (a package tcolorbox poderia ter sido usada também).
%
% Observe que "programruledcaption" funciona *apenas* para os floats do
% tipo "program". Se quiser criar algo similar para outro tipo de float,
% você vai precisar criar um novo comando ("myfloatruledcaption")
% copiando os comandos abaixo e modificando-os conforme necessário.
\newsavebox{\programCaptionTextBox}
\usepackage{framed}
\newenvironment{programruledcaption}[2][]{
  % All spacing measurements were adjusted to visually reproduce
  % the float captions
  \setlength\fboxsep{0pt}

  % topsep means space before AND after
  \setlength\topsep{.28\baselineskip plus .3\baselineskip minus 0pt}

  \vspace{.3\baselineskip} % Some extra top space

  % For whatever reason, the framed package actually calls "\captionof"
  % multiple times, messing up the counter. We need to prevent this,
  % so we put the caption in a box once and reuse the box.

  \savebox{\programCaptionTextBox}{%
    \parbox[b]{\textwidth}{%
      \ifstrempty{#1}
        {\captionof{program}[#2]{#2}}%
        {\captionof{program}[#1]{#2}}%
    }
  }

  \def\fullcaption{
    \vspace*{-.325\baselineskip}
    \noindent\usebox{\programCaptionTextBox}%
    \vspace*{-.56\baselineskip}%
    \kern 2pt\hrule\kern 2pt\relax
  }

  \def\FrameCommand{
    \hspace{-.007\textwidth}%
    \CustomFBox
      {\fullcaption}
      {\vspace{.13\baselineskip}}
      {.8pt}{.4pt}{0pt}{0pt}
  }

  \def\FirstFrameCommand{
    \hspace{-.007\textwidth}%
    \CustomFBox
      {\fullcaption}
      {\hfill\textit{cont}\enspace$\longrightarrow$}
      {.8pt}{0pt}{0pt}{0pt}
  }

  \def\MidFrameCommand{
    \hspace{-.007\textwidth}%
    \CustomFBox
      {$\longrightarrow$\enspace\textit{cont}\par\vspace*{.3\baselineskip}}
      {\hfill\textit{cont}\enspace$\longrightarrow$}
      {0pt}{0pt}{0pt}{0pt}
  }

  \def\LastFrameCommand{
    \hspace{-.007\textwidth}%
    \CustomFBox
      {$\longrightarrow$\enspace\textit{cont}\par\vspace*{.3\baselineskip}}
      {\vspace{.13\baselineskip}}
      {0pt}{.4pt}{0pt}{0pt}
  }

  \MakeFramed{\FrameRestore}

}{
  \endMakeFramed
}

%%%%%%%%%%%%%%%%%%%%%%%%%%%%%%%%%%%%%%%%%%%%%%%%%%%%%%%%%%%%%%%%%%%%%%%%%%%%%%%%
%%%%%%%%%%%%%%%%%%%%%%%%%%%% OUTROS PACOTES ÚTEIS %%%%%%%%%%%%%%%%%%%%%%%%%%%%%%
%%%%%%%%%%%%%%%%%%%%%%%%%%%%%%%%%%%%%%%%%%%%%%%%%%%%%%%%%%%%%%%%%%%%%%%%%%%%%%%%

% Você provavelmente vai querer ler a documentação de alguns destes pacotes
% para personalizar algum aspecto do trabalho ou usar algum recurso específico.

% A classe Book inclui o comando \appendix, que (obviamente) permite inserir
% apêndices no documento. No entanto, não há suporte similar para anexos. Esta
% package (que não é padrão do LaTeX, foi criada para este modelo) define o
% comando \annex. Ela deve ser carregada depois de hyperref.
\dowithsubdir{extras/}{\usepackage{annex}}

% Formatação personalizada das listas "itemize", "enumerate" e
% "description", além de permitir criar novos tipos de listas
%\usepackage{paralist}
% Esta package tem a mesma finalidade, mas é mais "moderna" e tem mais
% recursos:
%  * É possível personalizar labels, espaçamento etc.
%  * É possível definir novos tipos de lista, que podem ou não ser
%    baseados nos tipos padrão
%  * É possível "interromper" uma lista e retornar a ela depois sem
%    perder a numeração
%  * Com a opção "inline", a package define os ambientes "itemize*",
%    "description*" e "enumerate*", que fazem os itens da lista como
%    parte de um único parágrafo
%\usepackage[inline]{enumitem}

% Trechos de texto "puro" (tabs, quebras de linha etc. não são modificados)
\usepackage{verbatim}

% Sublinhado e outras formas de realce de texto
\usepackage{soul}
\usepackage{soulutf8}

% Recursos e símbolos adicionais para o modo matemático
% para evitar problemas de compatibilidade com algumas fontes, o pacote
% amsthm já foi carregado mais acima
\usepackage{latexsym}
\usepackage{amsmath}
\usepackage{amssymb}
\usepackage{mathtools}
%\usepackage{MnSymbol}
%\usepackage{stmaryrd}

% Notação bra-ket
%\usepackage{braket}

%\num \SI and \SIrange. For example, \SI{10}{\hertz} \SIrange{10}{100}{\hertz}
%\usepackage[binary-units]{siunitx}

% Citações melhores; se você pretende fazer citações de textos
% relativamente extensos, vale a pena ler a documentação. biblatex
% utiliza recursos deste pacote.
\usepackage{csquotes}

\usepackage{url}
% URL com fonte sem serifa ao invés de teletype
\urlstyle{sf}

% Permite inserir comentários, muito bom durante a escrita do texto;
% você também pode se interessar pela package pdfcomment.
\usepackage[textsize=scriptsize,colorinlistoftodos,textwidth=2.5cm]{todonotes}
\presetkeys{todonotes}{color=orange!40!white}{}

% Comando para fazer notas com highlight no texto correspondente:
% \hltodo[texto][opções]{comentário}
\makeatletter
\if@todonotes@disabled
  \NewDocumentCommand{\hltodo}{O{} O{} +m}{#1}
\else
  \NewDocumentCommand{\hltodo}{O{} O{} +m}{
    \ifstrempty{#1}{}{\texthl{#1}}%
    \todo[#2]{#3}{}%
  }
\fi
\makeatother

% Vamos reduzir o espaçamento entre linhas nas notas/comentários
\makeatletter
\xpatchcmd{\@todo}
  {\renewcommand{\@todonotes@text}{#2}}
  {\renewcommand{\@todonotes@text}{\begin{spacing}{0.5}#2\end{spacing}}}
  {}
  {}
\makeatother

% Símbolos adicionais, como \celsius, \ohm, \perthousand etc.
%\usepackage{gensymb}

% Símbolos adicionais, como \textrightarrow, \texteuro etc.
\usepackage{textcomp}

% Permite criar listas como glossários, listas de abreviaturas etc.
% https://en.wikibooks.org/wiki/LaTeX/Glossary
%\usepackage{glossaries}

% Permite formatar texto em colunas
\usepackage{multicol}

% Gantt charts; útil para fazer o cronograma para o exame de
% qualificação, por exemplo.
\usepackage{pgfgantt}

% Na versão 5 do pacote pgfgantt, a opção "compress calendar"
% deixou de existir, sendo substituída por "time slot unit=month".
% Aqui, um truque para funcionar com ambas as versões.
\makeatletter
\@ifpackagelater{pgfgantt}{2018/01/01}
  {\ganttset{time slot unit=month}}
  {\ganttset{compress calendar}}
\makeatother

% Estes parâmetros definem a aparência das gantt charts e variam
% em função da fonte do documento.
\ganttset{
    time slot format=isodate-yearmonth,
    vgrid,
    x unit=1.7em,
    y unit title=3ex,
    y unit chart=4ex,
    % O "strut" é necessário para alinhar o baseline dos nomes dos meses
    title label font=\strut\footnotesize,
    group label font=\footnotesize\bfseries,
    bar label font=\footnotesize,
    milestone label font=\footnotesize\itshape,
    % "align=right" é necessário para \ganttalignnewline funcionar
    group label node/.append style={align=right},
    bar label node/.append style={align=right},
    milestone label node/.append style={align=right},
    group incomplete/.append style={fill=black!50},
    bar/.append style={fill=black!25,draw=black},
    bar incomplete/.append style={fill=white,draw=black},
    % Não é preciso imprimir "0%"
    progress label text=\ifnumequal{#1}{0}{}{(#1\%)},
    % Formato e tamanho dos elementos
    title height=.9,
    group top shift=.4,
    group left shift=0,
    group right shift=0,
    group peaks tip position=0,
    group peaks width=.2,
    group peaks height=.3,
    milestone height=.4,
    milestone top shift=.4,
    milestone left shift=.8,
    milestone right shift=.2,
}

% Em inglês, tanto o nome completo quanto a abreviação do mês de maio
% são "May"; por conta disso, na tradução em português LaTeX erra a
% abreviação. Como talvez usemos o nome inteiro do mês em outro lugar,
% ao invés de forçar a tradução para "Mai" globalmente, fazemos isso
% apenas em ganttchart.
\AtBeginEnvironment{ganttchart}{\deftranslation[to=Portuguese]{May}{Mai}}

% Ilustrações, diagramas, gráficos etc. criados diretamente em LaTeX.
% Também é útil se você quiser importar gráficos gerados com GnuPlot.
\usepackage{tikz}

% Gráficos gerados diretamente em LaTeX; é possível usar tikz para
% isso também.
\usepackage{pgfplots}

% Importação direta de arquivos gerados por gnuplot com o
% driver/terminal "lua tikz"; esta package não faz parte da
% instalação padrão do LaTeX, mas sim do gnuplot.
%\usepackage{gnuplot-lua-tikz}

% Os comandos \TeX e \LaTeX são nativos do LaTeX; esta package acrescenta os
% comandos \XeLaTeX e \LuaLaTeX. Você provavelmente não precisa desse recurso
% e, portanto, pode removê-la.
\usepackage{metalogo}


% Diretórios onde estão as figuras; com isso, não é preciso colocar o caminho
% completo em \includegraphics (e nem a extensão).
\graphicspath{{figuras/},{logos/}}

% Comandos rápidos para mudar de língua:
% \en -> muda para o inglês
% \br -> muda para o português
% \texten{blah} -> o texto "blah" é em inglês
% \textbr{blah} -> o texto "blah" é em português
\babeltags{br = brazilian, en = english}

% Bibliografia
\usepackage[
  style=extras/plainnat-ime, % variante de autor-data, similar a plainnat
  %style=alphabetic, % similar a alpha
  %style=numeric, % comum em artigos
  %style=authoryear-comp, % autor-data "padrão" do biblatex
  %style=apa, % variante de autor-data, muito usado
  %style=abnt,
]{biblatex}




\usepackage{color}
\newcommand{\red}{\color{red}}
\newcommand{\blue}{\color{blue}}

\usepackage{slashbox}



\definecolor{ccqqqq}{rgb}{1,0,0}
\definecolor{qqqqff}{rgb}{0,0,1}


%%%%%%%%%%%%%%%%%%%%%%%%%%%%%%%%%%%%%
%  Defini o estlo no algorithmic
%%%%%%%%%%%%%%%%%%%%%%%%%%%%%%%%%%%%%
\usepackage[Algoritmo]{algorithm}
\usepackage[noend]{algpseudocode}
\algrenewcommand\alglinenumber[1]{\footnotesize #1}
\algrenewcommand\algorithmicif{\textbf{se}}
\algrenewcommand\algorithmicfor{\textbf{para}}
\algrenewcommand\algorithmicelse{\textbf{senão}}
\algrenewcommand\algorithmicwhile{\textbf{enquanto}}
\algrenewcommand\algorithmicdo{\textbf{faça}}
\algrenewcommand\algorithmicend{\textbf{fim}}
\algrenewcommand\algorithmicthen{\textbf{então}}
\algrenewcommand\algorithmicreturn{\textbf{retorne}}


%\DeclareFloatStyle{rulednotop}{capposition=top,font=normalsize,heightadjust=all}
%\DeclareFloatStyle{rulednotop}{midcode=rule,postcode=lowrule,capposition=top,heightadjust=all}
%\floatsetup[algorithm]{style=rulednotop}
%\captionsetup[algorithm]{textfont=normalsize}


\makeatletter
\def\BState{\State\hskip-\ALG@thistlm}
\def\Nil{\text{NIL}}
\makeatother

\newtheorem{theorem}{Teorema}
\newtheorem{lemma}[theorem]{Lema}



%%%%%%%%%%%%%%%%%%%%%%%%%%%%%%%%%%%%%
%     Comandos de observações       %
%%%%%%%%%%%%%%%%%%%%%%%%%%%%%%%%%%%%%
\newcommand{\TODO}[1]{{{\red #1}}}
%\newcommand{\NOTE}[1]{{{\blue #1}}}
\newcommand{\NEW}[1]{{\blue #1}}

%\newcommand{\defi}[1]{\index{#1}\textbf{#1}} % defini estilo ao definir algo no texto
\NewDocumentCommand{\defi}{ O{#2} m }{\index{#1}\textbf{#2}}


\renewcommand{\leq}{\leqslant}
\newcommand{\ceil}[1]{\lceil{#1}\rceil}

%%%%%%%%%%%%%%%%%%%%%%%%%%%%%%%%%%%%%
% Comandos de notação assintotica   %
%%%%%%%%%%%%%%%%%%%%%%%%%%%%%%%%%%%%%
\renewcommand{\O}[1]{\mathrm{O}(#1)}
\newcommand{\Otil}[1]{\text{Õ}(#1)}
\newcommand{\OTheta}[1]{\Theta(#1)}


\newcommand{\AlgoName}[1]{\text{\scshape #1}}

%%%%%%%%%%%%%%%%%%%%%%%%%%%%%%
%     Métodos de C           %
%%%%%%%%%%%%%%%%%%%%%%%%%%%%%%
\newcommand{\malloc}{\text{\TODO{alloc}}}  %(int)
%\newcommand{\free}{\text{\TODO{delete}}}    %(ptn)
\newcommand{\random}{\AlgoName{sorteie}} %(max_rand) escolhe um inteiro não negativo menor que max_rand
\newcommand{\order}{\AlgoName{ordene}} %(max_rand) escolhe um inteiro não negativo menor que max_rand
\newcommand{\sizeof}{\text{\TODO{sizeof}}}  %(int)


%%%%%%%%%%%%%%%%%%%%%%%%%%%%%%
%    Nomes de váriaveis      %
%%%%%%%%%%%%%%%%%%%%%%%%%%%%%%
\newcommand{\varname}[1]{\textit{#1}}
\newcommand{\altvarname}[1]{$#1$}
\newcommand{\node}{\mathit{n\acute{o}}}
\newcommand{\var}{\mathit{var}}


\newcommand{\R}{\mathbb{R}}


%%%%%%%%%%%%%%%%%%%%%%%%%%%%%%
%     Métodos de Grafos      %
%%%%%%%%%%%%%%%%%%%%%%%%%%%%%%
\newcommand{\graphCreate}{\AlgoName{novoGrafo}}
\newcommand{\graphAdd}{\AlgoName{ligueGLA}}
\newcommand{\graphDel}{\AlgoName{removaGLA}}

%%%%%%%%%%%%%%%%%%%%%%%%%%%%%%
%     Métodos de Treap       %
%%%%%%%%%%%%%%%%%%%%%%%%%%%%%%
\newcommand{\treapCreate}{\AlgoName{novoNó}}
\newcommand{\treapSearch}{\AlgoName{busca}}
\newcommand{\treapGetLast}{\AlgoName{último}}
\newcommand{\treapGetRoot}{\AlgoName{raiz}}
\newcommand{\treapOrder}{\AlgoName{ordem}}
\newcommand{\treapJoin}{\AlgoName{junta}}
\newcommand{\treapSplit}{\AlgoName{corta}}
\newcommand{\treapSplitRight}{\AlgoName{cortaDireita}}
\newcommand{\treapGetSize}{\AlgoName{tamanho}}
\newcommand{\treapGetEdgesLevel}{\AlgoName{arestasDeNível}}

\newcommand{\treapFirst}{\AlgoName{primeiro}}
\newcommand{\treapLast}{\AlgoName{último}}
\newcommand{\treapPredecessor}{\AlgoName{\AlgoName{Pred}}}    %(F, u, v)

%%%%%%%%%%%%%%%%%%%%%%%%%%%%%%
% Métodos de Euler Tour Tree %
%%%%%%%%%%%%%%%%%%%%%%%%%%%%%%
\newcommand{\ETTCreate}{\AlgoName{novoETT}}     % (v)
\newcommand{\ETTAddEdge}{\AlgoName{ligueETT}} % ($F$, $uu$, $vv$)
\newcommand{\ETTDelEdge}{\AlgoName{removaETT}} % ($F$, $uu$, $vv$)
\newcommand{\ETTQuery}{\AlgoName{conectadoETT}} % ($F$, $uu$, $vv$)
\newcommand{\ETmovetofront}{\AlgoName{movaInício}} % ($F$, $uu$)


%%%%%%%%%%%%%%%%%%%%%%%%%%%%%%%%%%
% Métodos da tabela hash         %
%%%%%%%%%%%%%%%%%%%%%%%%%%%%%%%%%%
\newcommand{\dymForestHash}{$H$}  %simbolo que identifica a matriz/hash da floresta
\newcommand{\nivel}{\AlgoName{nível}} 
\newcommand{\hashCreate}{\AlgoName{novoDicio}}     

%%%%%%%%%%%%%%%%%%%%%%%%%%%%%%%%%%
% Métodos de Florestas dinamicas %
%%%%%%%%%%%%%%%%%%%%%%%%%%%%%%%%%%
\newcommand{\dymForestCreate}{\AlgoName{novaFD}}  %(n)
\newcommand{\dymForestAddEdge}{\AlgoName{ligueFD}}    %(F, u, v)
\newcommand{\dymForestDelEdge}{\AlgoName{removaFD}}    %(F, u, v)
\newcommand{\dymForestQuery}{\AlgoName{\AlgoName{conectadoFD}}}    %(F, u, v)

%%%%%%%%%%%%%%%%%%%%%%%%%%%%%%%%%%
% Métodos de Grafos dinamicas %
%%%%%%%%%%%%%%%%%%%%%%%%%%%%%%%%%%
\newcommand{\dymGraphCreate}{\AlgoName{novoGD}}    %(n)
\newcommand{\dymGraphAddEdge}{\AlgoName{ligueGD}} %(G, u, v)
\newcommand{\dymGraphDelEdge}{\AlgoName{removaGD}} %(G, u, v)
\newcommand{\dymGraphQuery}{\AlgoName{\AlgoName{conectadoGD}}} %(G, u, v)
\newcommand{\dymGraphReplace}{\AlgoName{\AlgoName{substituaGD}}} %(G, u, v, i)
\newcommand{\dymGraphHash}{$H$}  %simbolo que identifica a matriz/hash da floresta

%%%%%%%%%%%%%%%%%%%%%%%%%%%%%%
% Métodos de Link/Cut Tree   %
%%%%%%%%%%%%%%%%%%%%%%%%%%%%%%

\newcommand{\linkcutCreate}{\AlgoName{newLCT}}
\newcommand{\linkcutAddEdge}{\AlgoName{link}}    %(F, u, v)
\newcommand{\linkcutDelEdge}{\AlgoName{cut}}    %(F, u, v)
\newcommand{\linkcutEvert}{\AlgoName{\AlgoName{evert}}}    %(v)
\newcommand{\linkcutMax}{\AlgoName{\AlgoName{max}}}    %(F, u, v)
\newcommand{\linkcutMin}{\AlgoName{\AlgoName{min}}}    %(F, u, v)
\newcommand{\linkcutParent}{\AlgoName{\AlgoName{parent}}}    %(F, u, v)
\newcommand{\linkcutQuery}{\AlgoName{\AlgoName{conectadoLC}}}    %(F, u, v)
\newcommand{\linkcutPath}{\AlgoName{\AlgoName{caminho}}}    %(F, u, v)
\newcommand{\linkcutWeight}{\AlgoName{\AlgoName{set weight}}}    %(F, u, v)

%%%%%%%%%%%%%%%%%%%%%%%%%%%%%%%%%%%%%%%
% Métodos de Link/Cut Tree com ordem  %
%%%%%%%%%%%%%%%%%%%%%%%%%%%%%%%%%%%%%%%

\newcommand{\hashNP}{\AlgoName{node Path}} 
\newcommand{\hashEdges}{\AlgoName{hashEdges}} 

\newcommand{\LCOMakeOcto}{\AlgoName{Create Octo}}
\newcommand{\LCODestroyOcto}{\AlgoName{Destroy Octo}}

\newcommand{\LCOMakeNode}{\AlgoName{Make edge}}
\newcommand{\LCODestroyNode}{\AlgoName{Make edge}}
\newcommand{\LCOLink}{\AlgoName{Link}}
\newcommand{\LCOMerge}{\AlgoName{Merge}}
\newcommand{\LCOSplit}{\AlgoName{Split}}
\newcommand{\LCOCycle}{\AlgoName{Cycle}}
\newcommand{\LCOParent}{\AlgoName{\AlgoName{Parent}}}    %(F, u, v)
\newcommand{\LCORoot}{\AlgoName{Root}}
\newcommand{\LCOAddCost}{\AlgoName{Set weight}}
\newcommand{\LCOMax}{\AlgoName{\AlgoName{Find max}}}    %(F, u, v)
\newcommand{\LCOMin}{\AlgoName{\AlgoName{Find min}}}    %(F, u, v)
\newcommand{\LCOEvert}{\AlgoName{\AlgoName{Evert}}}    %(F, u, v)
\newcommand{\LCOConnected}{\AlgoName{\AlgoName{Connected}}}    %(F, u, v)
\newcommand{\LCOFindNode}{\AlgoName{Find node}}

%\newcommand{\LCOMakeEdge}{\AlgoName{make edge}}
%\newcommand{\LCOSplice}{\AlgoName{splice}}
%%%%%%%%%%%%%%%%%%%%%%%%%%%%%%
%     Métodos de MSF         %
%%%%%%%%%%%%%%%%%%%%%%%%%%%%%%
\newcommand{\MSFCreate}{\AlgoName{novoGDP}} %(n)
\newcommand{\MSFupdate}{\AlgoName{mudaPesoGDP}} %(n)
\newcommand{\MSFaddEdge}{\AlgoName{ligueGDP}}    %(G, u, v, w)
\newcommand{\MSFdelEdge}{\AlgoName{removaGDP}}    %(G, u, v)
\newcommand{\MSFweight}{\AlgoName{pesoGDP}}    %(G)

\newcommand{\dymGraphReplaceMSF}{\AlgoName{substituaGDP}} %(G, u, v, i)
\newcommand{\treapGetEdgeMinWeight}{\AlgoName{arestaMinPesoGDP}}


%%%%%%%%%%%%%%%%%%%%%%%%%%%%%%
%     Métodos de VPSP
% Verify partial sum of permutations%
%%%%%%%%%%%%%%%%%%%%%%%%%%%%%%

\newcommand{\VPSPconvert}{\AlgoName{converta}}
\newcommand{\VPSPupdate}{\AlgoName{substitua}}
\newcommand{\VPSPverify}{\AlgoName{verifique}}


%%%%%%%%%%%%%%%%%%%%%%%%%%%%%%
%     Nomes de Algoritmos
%%%%%%%%%%%%%%%%%%%%%%%%%%%%%%

\newcommand{\HDT}{HDT}
\newcommand{\HK}{HK}
\newcommand{\CLHB}{CLHB}


%%%%%%%%%%%%%%%%%%%%%%%%%%%%%%%%%%%%%%%%%%%%%%%%%%%%%%%%%%%%%%%%%%%%%%%%%%%%%%%%
%%%%%%%%%%%%%%%%%%%%%%%%%%%%%%%%%% METADADOS %%%%%%%%%%%%%%%%%%%%%%%%%%%%%%%%%%%
%%%%%%%%%%%%%%%%%%%%%%%%%%%%%%%%%%%%%%%%%%%%%%%%%%%%%%%%%%%%%%%%%%%%%%%%%%%%%%%%

% O arquivo com os dados bibliográficos para biblatex; você pode usar
% este comando mais de uma vez para acrescentar múltiplos arquivos
\addbibresource{bibliografia.bib}

% Este comando permite acrescentar itens à lista de referências sem incluir
% uma referência de fato no texto (pode ser usado em qualquer lugar do texto)
%\nocite{bronevetsky02,schmidt03:MSc, FSF:GNU-GPL, CORBA:spec, MenaChalco08}
% Com este comando, todos os itens do arquivo .bib são incluídos na lista
% de referências
%\nocite{*}

% É possível definir como determinadas palavras podem (ou não) ser
% hifenizadas; no entanto, a hifenização automática geralmente funciona bem
\babelhyphenation{documentclass latexmk soft-ware clsguide} % todas as línguas
\babelhyphenation[brazilian]{Fu-la-no}
\babelhyphenation[english]{what-ever}

% Estes comandos definem o título e autoria do trabalho e devem sempre ser
% definidos, pois além de serem utilizados para criar a capa, também são
% armazenados nos metadados do PDF.
\title{
    % Obrigatório nas duas línguas
    titlept={Algoritmos para conexidade em\\ grafos dinâmicos},
    titleen={Algorithms for dynamic graph connectivity},
    % Opcional, mas se houver deve existir nas duas línguas
    %subtitlept={um subtítulo},
    %subtitleen={a subtitle},
}

\author{Arthur Henrique Dias Rodrigues}

% Para TCCs, este comando define o supervisor
\orientador[fem]{Profª. Drª. Cristina Gomes Fernandes}

% Se não houver, remova; se houver mais de um, basta
% repetir o comando quantas vezes forem necessárias
%\coorientador{Prof. Dr. Ciclano de Tal}
%\coorientador[fem]{Profª. Drª. Beltrana de Tal}

% A página de rosto da versão para depósito (ou seja, a versão final
% antes da defesa) deve ser diferente da página de rosto da versão
% definitiva (ou seja, a versão final após a incorporação das sugestões
% da banca).
\defesa{
  nivel=mestrado, % mestrado, doutorado ou tcc
  % É a versão para defesa ou a versão definitiva?
  %definitiva,
  % É qualificação?
  %quali,
  programa={Ciência da Computação},
  membrobanca={Profª. Drª. Fulana de Tal (orientadora) -- IME-USP [sem ponto final]},
  % Em inglês, não há o "ª"
  %membrobanca{Prof. Dr. Fulana de Tal (advisor) -- IME-USP [sem ponto final]},
  membrobanca={Prof. Dr. Ciclano de Tal -- IME-USP [sem ponto final]},
  membrobanca={Profª. Drª. Convidada de Tal -- IMPA [sem ponto final]},
  % Se não houve bolsa, remova
  %
  % Norma sobre agradecimento por auxílios da FAPESP:
  % https://fapesp.br/11789/referencia-ao-apoio-da-fapesp-em-todas-as-formas-de-divulgacao
  %
  % Norma sobre agradecimento por auxílios da CAPES (Portaria 206,
  % de 4 de Setembro de 2018):
  % https://www.in.gov.br/materia/-/asset_publisher/Kujrw0TZC2Mb/content/id/39729251/do1-2018-09-05-portaria-n-206-de-4-de-setembro-de-2018-39729135
  %
  %apoio={O presente trabalho foi realizado com apoio da Coordenação
  %       de Aperfeiçoamento\\ de Pessoal de Nível Superior -- Brasil
  %       (CAPES) -- Código de Financiamento 001}, % o código é sempre 001
  %
  %apoio={This study was financed in part by the Coordenação de
  %       Aperfeiçoamento\\ de Pessoal de Nível Superior -- Brasil
  %       (CAPES) -- Finance Code 001}, % o código é sempre 001
  %
  %apoio={Durante o desenvolvimento deste trabalho, o autor recebeu\\
  %       auxílio financeiro da FAPESP -- processo nº aaaa/nnnnn-d},
  %
  %apoio={During the development if this work, the author received\\
  %       financial support from FAPESP -- grant \#aaaa/nnnnn-d},
  %
  %apoio={Durante o desenvolvimento deste trabalho o autor
  %       recebeu auxílio financeiro da XXXX},
  local={São Paulo},
  data=2024-09-02, % YYYY-MM-DD
  % A licença do seu trabalho. Use CC-BY, CC-BY-NC, CC-BY-ND, CC-BY-SA,
  % CC-BY-NC-SA ou CC-BY-NC-ND para escolher a licença Creative Commons
  % correspondente (o sistema insere automaticamente o texto da licença).
  % Se quiser estabelecer regras diferentes para o uso de seu trabalho,
  % converse com seu orientador e coloque o texto da licença aqui, mas
  % observe que apenas TCCs sob alguma licença Creative Commons serão
  % acrescentados ao BDTA. Se você tem alguma intenção de publicar o
  % trabalho comercialmente no futuro, sugerimos a licença CC-BY-NC-ND.
  direitos={CC-BY}, % Creative Commons Attribution 4.0 International License
  %direitos={CC-BY-NC-ND}, % Creative Commons Attribution / NonCommercial /
                           % NoDerivatives 4.0 International License
  %direitos={Autorizo a reprodução e divulgação total ou parcial
  %          deste trabalho, por qualquer meio convencional ou
  %          eletrônico, para fins de estudo e pesquisa, desde que
  %          citada a fonte.},
  %direitos={I authorize the complete or partial reproduction and disclosure
  %          of this work by any conventional or electronic means for study
  %          and research purposes, provided that the source is acknowledged.}
  % Para gerar a ficha catalográfica, acesse https://fc.ime.usp.br/,
  % preencha o formulário e escolha a opção "Gerar Código LaTeX".
  % Basta copiar e colar o resultado aqui.
  fichacatalografica={},
}

%%%%%%%%%%%%%%%%%%%%%%%%%%%%%%%%%%%%%%%%%%%%%%%%%%%%%%%%%%%%%%%%%%%%%%%%%%%%%%%%
%%%%%%%%%%%%%%%%%%%%%%% AQUI COMEÇA O CONTEÚDO DE FATO %%%%%%%%%%%%%%%%%%%%%%%%%
%%%%%%%%%%%%%%%%%%%%%%%%%%%%%%%%%%%%%%%%%%%%%%%%%%%%%%%%%%%%%%%%%%%%%%%%%%%%%%%%

\begin{document}

%%%%%%%%%%%%%%%%%%%%%%%%%%% CAPA E PÁGINAS INICIAIS %%%%%%%%%%%%%%%%%%%%%%%%%%%%

% Aqui começa o conteúdo inicial que aparece antes do capítulo 1, ou seja,
% página de rosto, resumo, sumário etc. O comando frontmatter faz números
% de página aparecem em algarismos romanos ao invés de arábicos e
% desabilita a contagem de capítulos.
\frontmatter

\pagestyle{plain}

\onehalfspacing % Espaçamento 1,5 na capa e páginas iniciais

\maketitle % capa e folha de rosto

%%%%%%%%%%%%%%%% DEDICATÓRIA, AGRADECIMENTOS, RESUMO/ABSTRACT %%%%%%%%%%%%%%%%%%


\begin{dedicatoria}
Dedico esse trabalho a minha mãe e a minha irmã, por todo amor e apoio.
\end{dedicatoria}

% Reinicia o contador de páginas (a próxima página recebe o número "i") para
% que a página da dedicatória não seja contada.
\pagenumbering{roman}

% Agradecimentos:
% Se o candidato não quer fazer agradecimentos, deve simplesmente eliminar
% esta página. A epígrafe, obviamente, é opcional; é possível colocar
% epígrafes em todos os capítulos. O comando "\chapter*" faz esta seção
% não ser incluída no sumário.
\chapter*{Agradecimentos}
%\epigrafe{Do. Or do not. There is no try.}{Mestre Yoda}
Gostaria de expressar minha profunda gratidão a todas e todos que me apoiaram durante esta jornada.
Primeiramente, à minha mãe, Lourdes, à minha irmã, Bruna, e ao meu namorado Eugênio por estarem sempre ao meu lado, oferecendo amor, paciência e incentivo incondicionais.

Agradeço imensamente à minha orientadora, Cris, que me recebeu de braços abertos no mestrado, me orientou com excelência e dedicação, me ensinando com meus erros e tendo muita compreensão nos meus momentos mais dificeis.
Sua orientação foi essencial para o meu crescimento.

Ao meu amigo Coelho, pela inspiração nessa jornada pela Teoria dos Grafos, por todos os ensinamentos e sabedoria que vou levar pro resto da minha vida e por todos os cafés que tomamos juntos.

Aos amigos, William Gnann, Hercules Freire, Matheus Pereira, Matheus Polkorny, Rodrigo Dias e Rodrigo Ribeiro, meu sincero agradecimento pela amizade e apoio ao longo desta jornada.
Obrigado por estarem ao meu lado, compreendendo minhas ansiedades e cansaço.
Um agradecimento especial a Rodrigo Ribeiro, Hercules Freire e Rodrigo Dias, por compartilharem comigo os desafios de conciliar nossas ocupações duplas — o mestrado/doutorado e a atuação como pesquisadores no Instituto de Pesquisas Tecnológicas.

Aos amigos do IME, Antônio Kaique, Ariana, César, Colucci, Gabriel Morete, Heloísa, Hugo, Juliane, Pedro e Thiago Oliveira, sou muito grato pela camaradagem que tornou este percurso muito mais agradável.

A todos, meu sincero agradecimento!


% As palavras-chave são obrigatórias, em português e em inglês, e devem ser
% definidas antes do resumo/abstract. Acrescente quantas forem necessárias.
\palavrachave{Palavra-chave1}
\palavrachave{Palavra-chave2}
\palavrachave{Palavra-chave3}

\keyword{Keyword1}
\keyword{Keyword2}
\keyword{Keyword3}

% O resumo é obrigatório, em português e inglês. Estes comandos também
% geram automaticamente a referência para o próprio documento, conforme
% as normas sugeridas da USP.
\resumo{
Elemento obrigatório, constituído de uma sequência de frases concisas e
objetivas, em forma de texto.  Deve apresentar os objetivos, métodos empregados,
resultados e conclusões.  O resumo deve ser redigido em parágrafo único, conter
no máximo 500 palavras e ser seguido dos termos representativos do conteúdo do
trabalho (palavras-chave). Deve ser precedido da referência do documento.
Texto texto texto texto texto texto texto texto texto texto texto texto texto
texto texto texto texto texto texto texto texto texto texto texto texto texto
texto texto texto texto texto texto texto texto texto texto texto texto texto
texto texto texto texto texto texto texto texto texto texto texto texto texto
texto texto texto texto texto texto texto texto texto texto texto texto texto
texto texto texto texto texto texto texto texto.
Texto texto texto texto texto texto texto texto texto texto texto texto texto
texto texto texto texto texto texto texto texto texto texto texto texto texto
texto texto texto texto texto texto texto texto texto texto texto texto texto
texto texto texto texto texto texto texto texto texto texto texto texto texto
texto texto.
}

\abstract{
Elemento obrigatório, elaborado com as mesmas características do resumo em
língua portuguesa. De acordo com o Regimento da Pós-Graduação da USP (Artigo
99), deve ser redigido em inglês para fins de divulgação. É uma boa ideia usar
o sítio \url{www.grammarly.com} na preparação de textos em inglês.
Text text text text text text text text text text text text text text text text
text text text text text text text text text text text text text text text text
text text text text text text text text text text text text text text text text
text text text text text text text text text text text text.
Text text text text text text text text text text text text text text text text
text text text text text text text text text text text text text text text text
text text text.
}



%%%%%%%%%%%%%%%%%%%%%%%%%%% LISTAS DE FIGURAS ETC. %%%%%%%%%%%%%%%%%%%%%%%%%%%%%

% Como as listas que se seguem podem não incluir uma quebra de página
% obrigatória, inserimos uma quebra manualmente aqui.
\makeatletter
\if@openright\cleardoublepage\else\clearpage\fi
\makeatother

% Todas as listas são opcionais; Usando "\chapter*" elas não são incluídas
% no sumário. As listas geradas automaticamente também não são incluídas por
% conta das opções "notlot" e "notlof" que usamos para a package tocbibind.

% Normalmente, "\chapter*" faz o novo capítulo iniciar em uma nova página, e as
% listas geradas automaticamente também por padrão ficam em páginas separadas.
% Como cada uma destas listas é muito curta, não faz muito sentido fazer isso
% aqui, então usamos este comando para desabilitar essas quebras de página.
% Se você preferir, comente as linhas com esse comando e des-comente as linhas
% sem ele para criar as listas em páginas separadas. Observe que você também
% pode inserir quebras de página manualmente (com \clearpage, veja o exemplo
% mais abaixo).
\newcommand\disablenewpage[1]{{\let\clearpage\par\let\cleardoublepage\par #1}}

% Nestas listas, é melhor usar "raggedbottom" (veja basics.tex). Colocamos
% a opção correspondente e as listas dentro de um grupo para ativar
% raggedbottom apenas temporariamente.
\bgroup
\raggedbottom

%%%%% Listas criadas manualmente

%\chapter*{Lista de abreviaturas}
%\disablenewpage{\chapter*{Lista de abreviaturas}}
%
%\begin{tabular}{rl}
%   ETT & Euler Tour Tree\\
%   ABB & Árvore binária de busca\\
%   IME & Instituto de Matemática e Estatística\\
%   USP & Universidade de São Paulo
%\end{tabular}

%\chapter*{Lista de símbolos}
%\disablenewpage{\chapter*{Lista de símbolos}}

%\begin{tabular}{rl}
%  $\omega$ & Frequência angular\\
%    $\psi$ & Função de análise \emph{wavelet}\\
%    $\Psi$ & Transformada de Fourier de $\psi$\\
%\end{tabular}

% Quebra de página manual
\clearpage

%%%%% Listas criadas automaticamente

% Você pode escolher se quer ou não permitir a quebra de página
%\listoffigures
%\disablenewpage{\listoffigures}

% Você pode escolher se quer ou não permitir a quebra de página
%\listoftables
%\disablenewpage{\listoftables}

% Esta lista é criada "automaticamente" pela package float quando
% definimos o novo tipo de float "program" (em utils.tex)
% Você pode escolher se quer ou não permitir a quebra de página
%\listof{program}{\programlistname}
%\disablenewpage{\listof{program}{\programlistname}}

% Sumário (obrigatório)
\tableofcontents

\egroup % Final de "raggedbottom"

% Referências indiretas ("x", veja "y") para o índice remissivo (opcionais,
% pois o índice é opcional). É comum colocar esses itens no final do documento,
% junto com o comando \printindex, mas em alguns casos isso torna necessário
% executar texindy (ou makeindex) mais de uma vez, então colocar aqui é melhor.
%\index{Inglês|see{Língua estrangeira}}
%\index{Figuras|see{Floats}}
%\index{Tabelas|see{Floats}}
%\index{Código-fonte|see{Floats}}
%\index{Subcaptions|see{Subfiguras}}
%\index{Sublegendas|see{Subfiguras}}
%\index{Equações|see{Modo matemático}}
%\index{Fórmulas|see{Modo matemático}}
%\index{Rodapé, notas|see{Notas de rodapé}}
%\index{Captions|see{Legendas}}
%\index{Versão original|see{Tese/Dissertação, versões}}
%\index{Versão corrigida|see{Tese/Dissertação, versões}}
%\index{Palavras estrangeiras|see{Língua estrangeira}}
%\index{Floats!Algoritmo|see{Floats, ordem}}
%

%%%%%%%%%%%%%%%%%%%%%%%%%%%%%%%% CAPÍTULOS %%%%%%%%%%%%%%%%%%%%%%%%%%%%%%%%%%%%%

% Aqui vai o conteúdo principal do trabalho, ou seja, os capítulos que compõem
% a dissertação/tese. O comando mainmatter reinicia a contagem de páginas,
% modifica a numeração para números arábicos e ativa a contagem de capítulos.
\mainmatter

\pagestyle{mainmatter}

% Espaçamento simples
\singlespacing

\nocite{*}
\chapter{Introdução}

\section{Motivação}
\label{sec:Motivação}

Problemas em grafos servem para modelar uma série de aplicações do dia a dia,
e há uma vasta e clássica literatura que aborda vários problemas centrais em grafos.
Estes problemas clássicos geralmente consideram um modelo estático da situação. Ou seja, o grafo dado a priori não sofre alterações enquanto estamos resolvendo o problema.

Há no entanto aplicações em que o grafo modela uma situação menos estática. Por exemplo, em redes de dispositivos de internet das coisas, chuvas, ventos fortes ou falha na fonte de energia podem prejudicar a conexão entre dispositivos, o que pode ser representado pela remoção de uma aresta no grafo que abstrai a rede.

\defi{Algoritmos em grafos dinâmicos} é o termo usado para se referir à área de projeto de algoritmos que se concentra em resolver problemas clássicos de forma eficiente nesse contexto em que o grafo está sofrendo alterações. A literatura nessa área tem mais de $40$ anos e muito progresso significativo tem ocorrido recentemente. No entanto, há uma carência de material escrito $-$ especialmente em português $-$ nos livros de algoritmos sobre essa área tão atraente e atual.

Formalmente, um \defi[grafo!dinâmico]{grafo dinâmico} de ordem~$n$ é uma sequência de grafos~$(G_0, G_1,\ldots, G_T)$, onde~$G_0$ é um grafo com $n$ vértices e
cada $G_t$ para $1\leq t\leq T$ é obtido a partir de $G_{t-1}$ pela adição ou remoção de uma aresta.
Ou seja,~$E(G_t) := E(G_{t-1})\cup \{uv\}$, para alguma aresta~$uv\notin E(G_{t-1})$;
ou $E(G_t) := E(G_{t-1})\setminus \{uv\}$, onde~$uv\in E(G_{t-1})$, respectivamente.
As operações de adição e remoção de arestas são chamadas de \defi{modificações} ou \defi{atualizações} do grafo dinâmico.

Há aplicações em que as conexões possuem um custo ou peso associado,
representando a latência de comunicação entre dispositivos ou o custo de construção de uma infraestrutura cabeada, por exemplo. 
Para lidar com tais aplicações, associa-se um valor real a cada aresta do grafo.
Nesse caso, o grafo resultante é chamado de \defi[grafo!ponderado]{ponderado}.
Em um problema com grafos dinâmicos ponderados também é considerada a operação
de mudança de peso de uma determinada aresta como uma operação de atualização válida.

Um problema em grafos dinâmicos envolve verificar se o grafo corrente~$G_t$ satisfaz alguma determinada propriedade.
A operação de verificação dessa propriedade é chamada de \defi{consulta}.
Solucionar um tal problema envolve desenvolver um algoritmo ou estrutura de dados capaz de dar suporte às modificações e consultas de forma \nolbreaks{eficiente}.

Em alguns casos, restringimos cada~$G_t$ a uma família de grafos, como, por exemplo, florestas ou~\defi[grafo!plano]{grafos planos}, isto é, um grafo que é planar, com uma imersão específica \nolbreaks{no plano}.

Trabalhar com essas classes mais restritas de grafos permite o desenvolvimento de algoritmos mais simples que podem servir de etapa intermediária para se obter uma solução para o problema geral ou ser de interesse para alguma aplicação específica.
Usaremos a primeira estratégia no estudo do nosso primeiro problema:
O \defi[problema!de conexidade em!grafos dinâmicos]{problema de conexidade em grafos dinâmicos}, que consiste em, dado um grafo dinâmico submetido a uma sequência de inserções e remoções de arestas, responder a consultas do tipo “Os vértices $u$ e $v$ estão conectados por um caminho?”.

Vamos primeiro tratar o caso em que o grafo dinâmico é uma floresta, isto é, trabalharemos com o \defi[problema!de conexidade em!florestas dinâmicas]{problema de conexidade em florestas dinâmicas},
para em seguida usar as estruturas de dados desenvolvidas nesse problema para solucionar o caso geral.

O segundo problema estudado também será restrito a uma classe específica de grafos.
Estudaremos o \defi[problema!da floresta maximal de peso mínimo em grafos ponderados dinâmicos]{problema da floresta maximal de peso mínimo em grafos ponderados dinâmicos} restrito a grafos planos.
Esse problema visa manter uma \defi{floresta maximal de peso mínimo} (MSF) de um grafo dinâmico plano ponderado ao longo de uma sequência de inserções e remoções de arestas e de modificações nos pesos das arestas.

Começaremos esse estudo na próxima seção, em que faremos uma revisão histórica dos resultados relacionados aos problemas que serão abordados nos próximos capítulos.
No Capítulo~\ref{sec:connDF}, estudaremos o problema de conexidade em florestas dinâmicas e uma de suas soluções, proposta por Holm, de Lichtenberg e Thorup~\cite{poly_log}, que envolve Euler tour trees.
\NEW{Como pode ser visto no diagrama da Figura~\ref{fig:roadmap},} essa estrutura de dados utiliza \NEW{treaps implícitas, que são} árvores binárias de busca de chave implícita, cuja implementação será elaborada no Capítulo~\ref{sec:TreapDeChaveImplicita}.
No Capítulo~\ref{sec:connDG}, expandiremos nosso estudo sobre conexidade estudando o problema de conexidade em grafos dinâmicos.
A solução estuda também foi proposta por Holm, de Lichtenberg e Thorup~\cite{poly_log}.
Em nossos estudos, implementamos essa solução em Python3 e disponibilizamos o código no repositório git~\cite{github}.

No Capítulo~\ref{sec:MSF}, apresentaremos um algoritmo proposto por Eppstein et al.~\cite{EPPSTEIN-planar} para solucionar o problema da floresta maximal de peso mínimo em grafos dinâmicos planos ponderados.

No Capítulo~\ref{sec:lim}, mostraremos a demonstração de um limitante inferior de~$\Omega(\lg n)$ por operação para todos os problemas estudados.
Esse limitante, obtido por Patrascu e Demaine~\cite{lowerBoundPatrascu}, mostra que os algoritmos apresentados para o problema de conexidade em florestas dinâmicas e para o problema da floresta maximal de peso mínimo em grafos dinâmicos planos ponderados são assintoticamente ótimos.
No Capítulo~\ref{sec:conclusao} faremos considerações finais sobre nossos estudos.


\begin{figure}[htb]
\centering
\begin{tikzpicture}[x=0.75pt,y=0.75pt,yscale=-1,xscale=1]
%uncomment if require: \path (0,390); %set diagram left start at 0, and has height of 390

%Rounded Rect [id:dp5638261819516392] 
\draw   (89.47,271.31) .. controls (89.47,268.07) and (92.1,265.43) .. (95.35,265.43) -- (205.25,265.43) .. controls (208.5,265.43) and (211.13,268.07) .. (211.13,271.31) -- (211.13,288.95) .. controls (211.13,292.2) and (208.5,294.83) .. (205.25,294.83) -- (95.35,294.83) .. controls (92.1,294.83) and (89.47,292.2) .. (89.47,288.95) -- cycle ;
%Rounded Rect [id:dp9959376432936062] 
\draw   (75.37,142.93) .. controls (75.37,138.52) and (78.95,134.93) .. (83.37,134.93) -- (216.37,134.93) .. controls (220.78,134.93) and (224.37,138.52) .. (224.37,142.93) -- (224.37,166.93) .. controls (224.37,171.35) and (220.78,174.93) .. (216.37,174.93) -- (83.37,174.93) .. controls (78.95,174.93) and (75.37,171.35) .. (75.37,166.93) -- cycle ;
%Rounded Rect [id:dp7099894103599613] 
\draw   (83.5,30.67) .. controls (83.5,26.25) and (87.08,22.67) .. (91.5,22.67) -- (207.83,22.67) .. controls (212.25,22.67) and (215.83,26.25) .. (215.83,30.67) -- (215.83,54.67) .. controls (215.83,59.08) and (212.25,62.67) .. (207.83,62.67) -- (91.5,62.67) .. controls (87.08,62.67) and (83.5,59.08) .. (83.5,54.67) -- cycle ;
%Rounded Rect [id:dp9694794812846669] 
\draw   (258.33,144.25) .. controls (258.33,140.7) and (261.2,137.83) .. (264.75,137.83) -- (334.92,137.83) .. controls (338.46,137.83) and (341.33,140.7) .. (341.33,144.25) -- (341.33,163.49) .. controls (341.33,167.03) and (338.46,169.9) .. (334.92,169.9) -- (264.75,169.9) .. controls (261.2,169.9) and (258.33,167.03) .. (258.33,163.49) -- cycle ;
%Rounded Rect [id:dp08232296692986518] 
\draw   (381.83,142.93) .. controls (381.83,139.1) and (384.94,136) .. (388.77,136) -- (471.9,136) .. controls (475.73,136) and (478.83,139.1) .. (478.83,142.93) -- (478.83,163.73) .. controls (478.83,167.56) and (475.73,170.67) .. (471.9,170.67) -- (388.77,170.67) .. controls (384.94,170.67) and (381.83,167.56) .. (381.83,163.73) -- cycle ;
%Rounded Rect [id:dp3985250366724754] 
\draw   (367.5,28.53) .. controls (367.5,22.72) and (372.22,18) .. (378.03,18) -- (493.47,18) .. controls (499.28,18) and (504,22.72) .. (504,28.53) -- (504,60.13) .. controls (504,65.95) and (499.28,70.67) .. (493.47,70.67) -- (378.03,70.67) .. controls (372.22,70.67) and (367.5,65.95) .. (367.5,60.13) -- cycle ;
%Straight Lines [id:da5589289981455373] 
\draw    (149.7,134.4) -- (149.31,66.4) ;
\draw [shift={(149.3,64.4)}, rotate = 89.67] [color={rgb, 255:red, 0; green, 0; blue, 0 }  ][line width=0.75]    (10.93,-3.29) .. controls (6.95,-1.4) and (3.31,-0.3) .. (0,0) .. controls (3.31,0.3) and (6.95,1.4) .. (10.93,3.29)   ;
%Straight Lines [id:da8087039160610074] 
\draw    (149.3,205.6) -- (149.67,178) ;
\draw [shift={(149.7,176)}, rotate = 90.77] [color={rgb, 255:red, 0; green, 0; blue, 0 }  ][line width=0.75]    (10.93,-3.29) .. controls (6.95,-1.4) and (3.31,-0.3) .. (0,0) .. controls (3.31,0.3) and (6.95,1.4) .. (10.93,3.29)   ;
%Straight Lines [id:da19920966114360028] 
\draw    (257.5,154.75) -- (226.5,155.17) ;
\draw [shift={(224.5,155.2)}, rotate = 359.22] [color={rgb, 255:red, 0; green, 0; blue, 0 }  ][line width=0.75]    (10.93,-3.29) .. controls (6.95,-1.4) and (3.31,-0.3) .. (0,0) .. controls (3.31,0.3) and (6.95,1.4) .. (10.93,3.29)   ;
%Straight Lines [id:da38139604958901885] 
\draw    (319,138.25) -- (376.72,72.17) ;
\draw [shift={(378.03,70.67)}, rotate = 131.14] [color={rgb, 255:red, 0; green, 0; blue, 0 }  ][line width=0.75]    (10.93,-3.29) .. controls (6.95,-1.4) and (3.31,-0.3) .. (0,0) .. controls (3.31,0.3) and (6.95,1.4) .. (10.93,3.29)   ;
%Straight Lines [id:da5633794503104798] 
\draw    (427.5,135.33) -- (427.5,72.67) ;
\draw [shift={(427.5,70.67)}, rotate = 90] [color={rgb, 255:red, 0; green, 0; blue, 0 }  ][line width=0.75]    (10.93,-3.29) .. controls (6.95,-1.4) and (3.31,-0.3) .. (0,0) .. controls (3.31,0.3) and (6.95,1.4) .. (10.93,3.29)   ;
%Straight Lines [id:da7918857844120134] 
\draw    (276,137.75) -- (212.8,64.02) ;
\draw [shift={(211.5,62.5)}, rotate = 49.4] [color={rgb, 255:red, 0; green, 0; blue, 0 }  ][line width=0.75]    (10.93,-3.29) .. controls (6.95,-1.4) and (3.31,-0.3) .. (0,0) .. controls (3.31,0.3) and (6.95,1.4) .. (10.93,3.29)   ;
%Rounded Rect [id:dp21131335155672382] 
\draw   (36.83,88.11) .. controls (36.83,82.93) and (41.03,78.73) .. (46.21,78.73) -- (110.46,78.73) .. controls (115.64,78.73) and (119.83,82.93) .. (119.83,88.11) -- (119.83,116.23) .. controls (119.83,121.4) and (115.64,125.6) .. (110.46,125.6) -- (46.21,125.6) .. controls (41.03,125.6) and (36.83,121.4) .. (36.83,116.23) -- cycle ;
%Straight Lines [id:da7631806191728931] 
\draw    (110.46,78.73) -- (110.63,65.93) ;
\draw [shift={(110.66,63.93)}, rotate = 90.77] [color={rgb, 255:red, 0; green, 0; blue, 0 }  ][line width=0.75]    (10.93,-3.29) .. controls (6.95,-1.4) and (3.31,-0.3) .. (0,0) .. controls (3.31,0.3) and (6.95,1.4) .. (10.93,3.29)   ;
%Curve Lines [id:da8410667922534024] 
\draw    (211,279.5) .. controls (270.37,279.17) and (367.69,202.93) .. (384.42,73.13) ;
\draw [shift={(384.67,71.17)}, rotate = 96.96] [color={rgb, 255:red, 0; green, 0; blue, 0 }  ][line width=0.75]    (10.93,-3.29) .. controls (6.95,-1.4) and (3.31,-0.3) .. (0,0) .. controls (3.31,0.3) and (6.95,1.4) .. (10.93,3.29)   ;
%Rounded Rect [id:dp3745338241823657] 
\draw   (88.63,211.37) .. controls (88.63,208.13) and (91.26,205.5) .. (94.5,205.5) -- (204.43,205.5) .. controls (207.67,205.5) and (210.3,208.13) .. (210.3,211.37) -- (210.3,228.97) .. controls (210.3,232.21) and (207.67,234.83) .. (204.43,234.83) -- (94.5,234.83) .. controls (91.26,234.83) and (88.63,232.21) .. (88.63,228.97) -- cycle ;
%Straight Lines [id:da6552331155919412] 
\draw    (149.97,265.27) -- (150.34,237.67) ;
\draw [shift={(150.37,235.67)}, rotate = 90.77] [color={rgb, 255:red, 0; green, 0; blue, 0 }  ][line width=0.75]    (10.93,-3.29) .. controls (6.95,-1.4) and (3.31,-0.3) .. (0,0) .. controls (3.31,0.3) and (6.95,1.4) .. (10.93,3.29)   ;

% Text Node
\draw (93.8,270.43) node [anchor=north west][inner sep=0.75pt]   [align=left] {Treaps ímplicitas};
% Text Node
\draw (81.37,145.6) node [anchor=north west][inner sep=0.75pt]   [align=left] {Florestas dinâmicas};
% Text Node
\draw (93.5,34.33) node [anchor=north west][inner sep=0.75pt]   [align=left] {Grafos dinâmicos};
% Text Node
\draw (262.5,145) node [anchor=north west][inner sep=0.75pt]   [align=left] {Dicionários};
% Text Node
\draw (389.83,144) node [anchor=north west][inner sep=0.75pt]   [align=left] {Link cut trees};
% Text Node
\draw (375.5,24) node [anchor=north west][inner sep=0.75pt]   [align=left] {MSTs dinâmicas\\planas};
% Text Node
\draw (41.3,83.4) node [anchor=north west][inner sep=0.75pt]   [align=left] {Listas de\\Adjacências};
% Text Node
\draw (95.3,210.5) node [anchor=north west][inner sep=0.75pt]   [align=left] {Euler tour trees};
\end{tikzpicture}

\caption{Roadmap com estruturas de dados usadas. Uma seta de A para B significa que a estrutura de dados B usa a estrutura de dados A.}
\label{fig:roadmap}
\end{figure}

\section{Historiografia}

Essa seção é inspirada na historiografia apresentada em \cite{HHSRecentAdvances2022, Zaroliagis2002}.

Soluções para os problemas de conexidade dinâmica e de floresta maximal de peso mínimo se desenvolveram em paralelo ao longo dos anos. Em $1975$, Spira e Pan~\cite{SP1975} atacaram o problema MSF propondo um algoritmo cuja complexidade é $\O{n}$ para inserção e $\O{n^2}$ para remoção de arestas, onde $n$ é o número de vértices do grafo. Três anos depois, Chin e Houck~\cite{CH1978} apresentaram uma solução mais simples para inserção e remoção de arestas que possui o mesmo consumo de tempo que o algoritmo de Spira e Pan.

Em $1985$, Frederickson~\cite{frederickson1983data} reduziu a complexidade de ambas as operações de modificação para $\O{\!\sqrt{m}}$, onde~$m$ é o número de arestas do grafo no momento em que a operação é aplicada.
Em $1992$, Eppstein et al.~\cite{Eppstein1992SparsificationaTF,Eppstein1997SparsificationaTF} melhoraram o consumo de tempo do algoritmo de Frederickson para~$\O{\!\sqrt{n}}$.

O primeiro algoritmo poli-logarítmico para o problema de conectividade dinâmica foi apresentado por Henzinger e King~\cite{HenzingerKing} em~$1995$ e possui consumo amortizado esperado $\O{\lg^3 n}$ para cada operação. Nesse artigo, elas propuseram a utilização de uma estrutura de dados chamada Euler tour trees como uma maneira eficiente de representar florestas dinâmicas.
Em seguida, em $1997$, o consumo amortizado esperado por operação foi reduzido a $\O{\lg^2 n}$~\cite{HenzingerThorup} e, nesse mesmo ano, foi realizado o primeiro estudo experimental, avaliando diferentes versões do algoritmo de Frederickson~\cite{xpAnalyGiuseppe}.

Ainda em~$1995$, Nikoletseas, Reif, Spirakis e Yung~\cite{NikoletseasRSY} propuseram uma estrutura de dados para o problema \defi[grafo!estocástico]{estocástico} de conexidade em grafos dinâmicos, em que a sequência de atualizações e consultas é escolhida ao acaso com probabilidade uniforme.
Nessas condições, essa estrutura de dados permite fazer atualizações em tempo~$\O{\lg^3 n}$ esperado amortizado e consulta em tempo~$\O{1}$ esperado amortizado.

Em $1999$, Fatourou, Spirakis, Zarafidis, e Zoura~\cite{Fatourou} implementaram os algoritmos de Henzinger e King~\cite{HenzingerKing} e de Nikoletseas et al.~\cite{NikoletseasRSY} e os compararam experimentalmente.
Eles concluíram que a estrutura de dados de Nikoletseas et al. tem uma performance melhor do que a de Henzinger e King quando o grafo dinâmico estocástico é denso.

Inspirados na solução de Henzinger e King para o problema de conexidade dinâmica, em $1998$, Holm, de Lichtenberg e Thorup~\cite{poly_log} atacaram ambos os problemas de conectividade dinâmica e de MSF usando Euler tour trees.
A solução deles para o problema de conexidade empata com o consumo assintótico da solução de Henzinger e King de~$\O{\lg^2 n}$.
No entanto, se a implementação das Euler tour trees for determinística, então o algoritmo para conexidade dinâmica de Holm et al. é completamente determinístico e seu consumo é somente amortizado,
enquanto que o consumo na solução de Henzinger e King é esperado e amortizado. 
Apresentaremos a solução de Holm et al. nesse texto com uma implementação aleatorizada de Euler tour trees.
Avaliações experimentais foram feitas sobre essa solução~\cite{EmpiricalStudy1997, EmpiricalStudy2002,xp-Phylogeny}.
Comentaremos esses experimentos no Capítulo~\ref{sec:conclusao}.

O algoritmo de Holm et al. para MSF possui consumo amortizado $\O{\lg^4 n}$ para cada operação, sendo assim a primeira estrutura de dados determinística com consumo de tempo poli-logarítmico para o problema MSF. Uma implementação eficiente deste algoritmo foi apresentada por Cattaneo et al. \cite{xpstudy2002} junto a um outro algoritmo mais simples e assintoticamente pior, mas que teve bom desempenho nos experimentos práticos realizados.

Em $2000$, Thorup~\cite{Thorup2000} complementou o algoritmo de Holm et al.~\cite{poly_log}, adicionando uma estrutura de dados auxiliar chamada floresta estrutural, que reduz o consumo de tempo de cada operação para~$\O{\lg n\, (\lg\lg n)^3}$ esperado amortizado e~$\O{\frac{\lg n}{\lg \lg \lg n}}$ para a consulta de conexidade.
Ela só não é ótima por um fator de $\O{(\lg\lg n)^3}$, devido ao limitante inferior de~$\Omega(\lg n)$ provado em~$2006$ por Patrascu e Demaine~\cite{lowerBoundPatrascu}.

Em $2010$, Tarjan e Werneck~\cite{tarjanWerneck2010} fizeram um estudo experimental com diversas estruturas de dados para árvores dinâmicas que resolvem ambos os problemas de conexidade em florestas dinâmicas e da floresta maximal de peso mínimo,
destacando qualidades e deficiências de cada estrutura de dados quando sujeitas a diferentes tipos de cargas de trabalho. 

Desde $1997$, os algoritmos com consumo de tempo amortizado foram predominantes na literatura, porém todos possuem alguma instância com consumo de tempo $\OTheta{n}$ no pior caso.
O problema de melhorar o tempo assintótico no pior caso continuou a permear a literatura sem avanços até $2013$, quando Kapron, King e Mountjoy \cite{bruceM} apresentaram uma estrutura de dados para o problema de conexidade em grafos dinâmicos baseada no problema \textit{cutset} que proporciona consumo $\O{\lg^4 n}$ para a inserção de arestas, $\O{\lg^5 n}$ para remoção e responde à consulta de conexidade em tempo~$\O{\frac{\lg n}{\lg\lg n}}$.
A estrutura responde a consultas corretamente quando a resposta é “sim” e com alta probabilidade de acerto quando a resposta é “não”. A probabilidade de falso positivo, mesmo que baixa, continuou incentivando a pesquisa na área. 

\newpage
Em $2015$, Kejlberg-Rasmussen et al.~\cite{kejlbergrasmussen_et_al} apresentaram uma estrutura de dados para o problema de conexidade em grafos dinâmicos que permite fazer consultas em tempo constante, mas que possui consumo de tempo para as atualizações de~$\mathrm{O}\!\left(\sqrt{\frac{n\left(\lg \lg n\right)^2}{\lg n}}\right)$, um consumo de tempo assintótico consideravelmente maior do que os últimos algoritmos citados.

Em $2016$, Wulff-Nilsen~\cite{Wulff-Nilsen2016} apresentou um novo algoritmo para o problema de conexidade em grafos dinâmicos que dá suporte às atualizações com consumo~$\O{\frac{\lg^2 n}{\lg \lg n}}$ amortizado e~$\O{\frac{\lg n}{\lg \lg n}}$ para a consulta de conexidade.

Em $2017$, Huang et al.~\cite{fastestConn} reorganizam a solução de Thorup~\cite{Thorup2000} obtendo uma solução com consumo de tempo de~$\O{\lg n\, (\lg\lg n)^2}$ esperado amortizado para cada atualização e~$\O{\frac{\lg n}{\lg \lg \lg n}}$ para as consultas de conexidade.
Essa é atualmente a estrutura de dados com menor consumo de tempo assintótico conhecida para solucionar o problema de conexidade em grafos dinâmicos.

Em $2022$, Chen et al. \cite{QC22} propuseram uma heurística baseada no diâmetro do grafo dinâmico e com ela desenvolveram um algoritmo que possui bom desempenho prático quando aplicado a grafos dinâmicos extraídos de situações reais.
O consumo de tempo dessa heurística é linear em função do diâmetro do grafo, portanto as operações de atualização e de consulta possuem consumo de tempo de~$\O{n}$ no pior dos casos.
Veja um resumo dos resultados mencionados nesta seção na Tabela~\ref{tab:historiografia}
\begin{table}[h!]
\centering
\begin{tabular}{||c | c | c | c | c | c||} 
 \hline
 ano & inserção & remoção & consulta & tipo & ref. \\ [0.5ex] 
 \hline\hline
	$1985$ & $\O{\!\sqrt{m}}$ & $\O{\!\sqrt{m}}$ & $\O{1}$  & {\small determinístico; pior caso} & \cite{frederickson1983data} \\ 
 \hline
	$1992$ & $\O{\!\sqrt{n}}$ & $\O{\!\sqrt{n}}$ & $\O{1}$  & {\small determinístico; pior caso} & \cite{Eppstein1992SparsificationaTF} \\ 
 \hline
	$1995$ & $\O{\lg^3 n}$ & $\O{\lg^3 n}$   & $\O{\lg n}$  & {\small aleatorizado; amortizado} & \cite{HenzingerKing} \\ 
 \hline
	$1997$ & $\O{\lg^2 n}$ & $\O{\lg^2 n}$   & $\O{\lg n}$  & {\small aleatorizado; amortizado} & \cite{HenzingerThorup} \\ 
 \hline
	$1998$ & $\O{\lg^2 n}$ & $\O{\lg^2 n}$   & $\O{\lg n}$  & {\small determinístico; amortizado} & \cite{poly_log} \\ 
 \hline
	$2000$ & $\O{{\scriptstyle \lg n(\lg  \lg n)^3}}$ &  $\O{{\scriptstyle \lg n(\lg \lg n)^3}}$      & $\O{\frac{\lg n}{\lg\lg\lg n}}$  & {\small aleatorizado; amortizado }& \cite{Thorup2000} \\ 
 \hline
	$2013$ & $\O{\lg^4 n}$ & $\O{\lg^5 n}$ & $\O{{\scriptstyle \lg n / \lg \lg n }}$ & {\small determinístico; pior caso} & \cite{bruceM} \\
 \hline
	$2015$ & $ \mathrm{O}\!\left(\sqrt{\frac{n\left(\lg \lg n\right)^2}{\lg n}}\right)  $ & $\mathrm{O}\!\left(\sqrt{\frac{n\left(\lg \lg n\right)^2}{\lg n}}\right)$ & $\O{1}$ & {\small determinístico; pior caso} & \cite{kejlbergrasmussen_et_al} \\
 \hline
	$2016$ & $\O{  \frac{\lg^2 n}{\lg \lg n}}$ & $\O{ \frac{\lg^2 n}{\lg \lg n}} $   & $\O{ \frac{\lg n}{\lg \lg n}} $  & {\small determinístico; amortizado} & \cite{Wulff-Nilsen2016} \\ 
 \hline
	$2017$ & $\O{{\scriptstyle \lg n(\lg  \lg n)^2}}$ &  $\O{{\scriptstyle \lg n(\lg \lg n)^2}}$      & $\O{\frac{\lg n}{\lg\lg\lg n}}$  & {\small aleatorizado; amortizado }& \cite{fastestConn} \\ 
 \hline
	$2022$ & $\O{n}$ & $\O{n} $   & $\O{n} $  & {\small determinístico; pior caso } & \cite{QC22} \\ 
 \hline
\end{tabular}
\caption{Consumo de tempo de estruturas de dados para o problema de conexidade em grafos dinâmicos.}
\label{tab:historiografia}
\end{table}

%%%%%%%%%%%%%%%%%%%%%%%%%%%%%%%%%%%%%%%%%%%%%%%%
% Conectividade em florestas dinamicas         %
%%%%%%%%%%%%%%%%%%%%%%%%%%%%%%%%%%%%%%%%%%%%%%%%
\chapter{Conexidade em florestas dinâmicas}
\label{sec:connDF}

O problema de conexidade em grafos dinâmicos, descrito na Seção \ref{sec:Motivação}, pode ser reduzido ao caso em que o grafo é uma floresta, originando assim o \defi[problema!de conexidade em florestas dinâmicas]{problema de conexidade em florestas dinâmicas}. Detalharemos como essa redução é feita no Capítulo~\ref{sec:connDG}. Esse problema pode ser apresentado como a implementação da seguinte biblioteca da forma mais eficiente possível: 

\begin{itemize}
\item \dymForestCreate($n$): cria uma floresta dinâmica com $n$ vértices isolados;
\item \dymForestAddEdge($F$, $u$, $v$): adiciona a aresta $uv$ à floresta dinâmica~$F$;
\item \dymForestDelEdge($F$, $u$, $v$): remove a aresta $uv$ de $F$; e
\item \dymForestQuery($F$, $u$, $v$): retorna verdadeiro se $u$ e $v$ estão na mesma componente conexa de $F$ e falso, caso contrário.
\end{itemize}
 
Há na literatura uma estrutura de dados bem conhecida chamada \defi{link-cut trees} \cite{SleatroTarjanLinkCutTree1983} que resolve uma versão direcionada desse problema, em que as árvores da floresta são enraizadas.
Com essa estrutura de dados e uma rotina adicional, que permite mudar a raiz de uma dada árvore para um de seus outros nós, as link-cut trees resolvem também a versão não direcionada do problema, ou seja, o problema de conexidade em florestas dinâmicas.
\NEW{
Na Seção~\ref{sec:linkcuttree} introduziremos as link-cut trees como solução para um outro problema envolvendo grafos dinâmicos.
} 
Nessa seção apresentaremos as Euler tour trees, uma solução mais simples e tão eficiente quanto.


\section{Euler tour trees} 

Tarjan e Vishkin~\cite{tarjan1985} propuseram a \defi{representação por trilha Euleriana} de uma árvore (originalmente nomeada \textit{Euler tour technique}).
Essa representação é obtida de uma árvore~$T$ substituindo-se cada aresta por dois arcos em sentidos opostos e adicionando-se um laço a cada vértice, como pode ser visto na Figura~\ref{fig:exemploSeqEuler}. O digrafo resultante é \defi[grafo!Euleriano]{Euleriano}, ou seja, é conexo e possui uma trilha que começa e termina num mesmo vértice e que passa por todos os arcos do digrafo exatamente uma vez. Uma tal trilha é chamada de \defi{trilha Euleriana} do digrafo.


\begin{figure}[htb]
\centering
\begin{tikzpicture}[dot/.style={draw,circle,fill,inner sep=1.5pt},line width=1pt,x=1.5cm,y=1.5cm]
\clip(.5,1.3) rectangle (3.3,3.5);
\draw [line width=1pt] (1.7420645075484014,2.9548225063123694)-- (0.6109449986613309,3.1228105521866865);
\draw [line width=1pt] (1.7420645075484014,2.9548225063123694)-- (2,2);
\draw [line width=1pt] (2,2)-- (0.9693194965265414,1.7453085760172875);
\draw [line width=1pt] (2.5036103155119798,2.742037648204901)-- (3,2);
\draw [line width=1pt] (2,2)-- (2.5036103155119798,2.742037648204901);
\begin{scriptsize}
\draw [fill=black,bend left] (1.7420645075484014,2.9548225063123694) circle (1.5pt);
\draw[color=black] (1.8316581320147085,3.195605372065557) node {$1$};
\draw [fill=black] (0.6109449986613309,3.1228105521866865) circle (1.5pt);
\draw[color=black] (0.7005386231276353,3.363593417939874) node {$2$};
\draw [fill=black] (2,2) circle (1.5pt);
\draw[color=black] (1.9,1.8) node {$4$};
\draw [fill=black] (0.9693194965265414,1.7453085760172875) circle (1.5pt);
\draw[color=black] (1,2) node {$5$};
\draw [fill=black] (3,2) circle (1.5pt);
\draw[color=black] (3.0859688745429477,2.2357448368651927) node {$3$};
\draw [fill=black] (2.5036103155119798,2.742037648204901) circle (1.5pt);
\draw[color=black] (2.5932039399782822,2.9828205139580892) node {$0$};
\end{scriptsize}
\end{tikzpicture}

%\documentclass[border=5pt,tikz]{standalone}
%\usetikzlibrary{positioning}
%\begin{document}
\begin{tikzpicture}[dot/.style={draw,circle,fill,inner sep=1.5pt},line width=.7pt,x=1.5cm,y=1.5cm]
\clip(-1,1.3) rectangle (4,3.5);
\begin{scriptsize}
\node[label=right:0] (r0) at (2.7,2.4) [dot] {};

\node[label=above:1] (r1) at (1.7420645075484014,2.9548225063123694) [dot] {};
%\draw[color=black] (1.8316581320147085,3.195605372065557) node {$1$};


\node[label=above:2] (r2) at (0.6109449986613309,3.1228105521866865) [dot] {};
%\draw[color=black] (0.7005386231276353,3.363593417939874) node {$2$};

\node[label=below:3] (r3) at (3,1.6) [dot] {};
%\draw[color=black] (2.6,1.2357448368651927) node {$3$};

\node (r4) at (2,2) [dot] {};
\draw[color=black] (1.9,1.8) node {$4$};

\node[label=above:5] (r5) at (0.9693194965265414,1.7453085760172875) [dot] {};
%\draw[color=black] (1,2) node {$5$};

\draw[->] (r0) to[out=30,in=90,looseness=30] (r0);
\draw[->] (r1) to[out=-30,in=30,looseness=30] (r1);
\draw[->] (r2) to[out=150,in=210,looseness=30] (r2);
\draw[->] (r3) to[out=-30,in=30,looseness=30] (r3);
\draw[->] (r4) to[out=-30,in=-90,looseness=30] (r4);
\draw[->] (r5) to[out=150,in=210,looseness=30] (r5);

\draw[->] (r0) to[bend left] (r4);
\draw[->] (r4) to[bend left] (r0);

\draw[->] (r0) to[bend right] (r3);
\draw[->] (r3) to[bend right] (r0);

\draw[->] (r5) to[bend left] (r4);
\draw[->] (r4) to[bend left] (r5);

\draw[->] (r1) to[bend left] (r4);
\draw[->] (r4) to[bend left] (r1);

\draw[->] (r1) to[bend left] (r2);
\draw[->] (r2) to[bend left] (r1);

\end{scriptsize}
\end{tikzpicture}
%\end{document}

\caption{Um exemplo de árvore e sua transformação como um digrafo a ser usado para sua representação por trilha Euleriana.}
\label{fig:exemploSeqEuler}
\end{figure}

A representação da árvore~$T$ é essencialmente a sequência de arcos que forma uma trilha Euleriana de~$T$.
Denotamos cada arco pelo par de vértices que o compõe.
\NEW{
Isto é, o arco com origem no vértice~$u$ e destino no vértice~$v$ é escrito como $uv$.
Dessa forma, um laço em~$v$ é escrito como~$vv$.
Utilizaremos esses laços como representantes dos vértices na sequência. 
Por exemplo, a sequência~\eqref{eq:eulerSeq} é uma trilha Euleriana da árvore da Figura \ref{fig:exemploSeqEuler}.
}
\begin{equation}
30~00~04~41~12~22~21~11~14~44~45~55~54~40~03~33.\label{eq:eulerSeq}  
\end{equation}

Note que a sequência depende do vértice inicial e da ordem em que cada vizinho de cada vértice é visitado. Chamaremos uma tal sequência de \defi{sequência Euleriana} da árvore~$T$.

Henzinger e King \cite{HenzingerKing} propuseram armazenar uma sequência Euleriana em uma \defi[arvore@\'arvore!binária de busca]{árvore binária de busca} (ABB), que é uma árvore binária composta por nós que possuem  quatro campos: chave, pai, filho esquerdo e filho direito~\cite{CLRS}.

Os campos pai e filhos atribuem a estrutura de árvore binária à ABB. Isto é, cada nó~$N$ possui até dois filhos (esquerdo e direito), o campo \defi{pai} de cada um dos filhos aponta para~$N$. Nenhum nó é filho de dois outros nós. Pode ocorrer de um nó não possuir algum dos filhos; nesse caso, os campos correspondentes a filhos inexistentes contêm~$\Nil$. Somente um nó não possui pai: este é chamado de \defi{raiz} da ABB.

Para uma árvore binária ser considerada de busca, é necessário que, para todo nó $N$, todas as chaves da subárvore esquerda sejam menores do que a chave de $N$ e, simetricamente, todas as chaves da subárvore direita sejam maiores do que a chave de $N$.

Cada nó da ABB guarda um elemento da sequência Euleriana, ou seja, um par de vértices da árvore $T$, em um campo adicional \defi{info}, e armazena, no campo \defi{chave}, \NEW{um valor entre $1$ e $n$ correspondente ao índice desse elemento na sequência, onde $n$ é o comprimento da sequência.}

\begin{figure}[htb]
%\scalebox{.6}{
\centering
\begin{tikzpicture}[line cap=round,line join=round,>=triangle 45,x=1cm,y=1cm]
\clip(-.6,-1) rectangle (14.1,4.6);
\draw [line width=1pt] (1,1) circle (0.5cm);
\draw [line width=1pt] (2,2) circle (0.5cm);
\draw [line width=1pt] (3,1) circle (0.5cm);
\draw [line width=1pt] (5.5,2) circle (0.5cm);
\draw [line width=1pt] (6.5,1) circle (0.5cm);
\draw [line width=1pt] (3.5,3) circle (0.5cm);
\draw [line width=1pt] (7,4) circle (0.5cm);
\draw [line width=1pt] (10.5,3) circle (0.5cm);
\draw [line width=1pt] (9.5,2) circle (0.5cm);
\draw [line width=1pt] (8.5,1) circle (0.5cm);
\draw [line width=1pt] (11.5,2) circle (0.5cm);
\draw [line width=1pt] (10.5,1) circle (0.5cm);
\draw [line width=1pt] (12.5,1) circle (0.5cm);
\draw [line width=1pt] (4.5,1) circle (0.5cm);
\draw [line width=1pt] (7.5,0) circle (0.5cm);
\draw [line width=1pt] (9.5,0) circle (0.5cm);
\draw [line width=1pt] (1.3535533905932737,1.3535533905932737)-- (1.646446609406726,1.646446609406726);
\draw [line width=1pt] (2.353553390593274,1.646446609406726)-- (2.646446609406726,1.353553390593274);
i%\draw [line width=1pt] (2.646446609406726,0.646446609406726)-- (2.353553390593274,0.35355339059327395);
\draw [line width=1pt] (2.416025147168923,2.2773500981126156)-- (3.0839748528310773,2.722649901887385);
\draw [line width=1pt] (3.947213595499958,2.776393202250021)-- (5.052786404500042,2.223606797749979);
\draw [line width=1pt] (5.146446609406726,1.646446609406726)-- (4.853553390593274,1.353553390593274);
\draw [line width=1pt] (5.853553390593274,1.646446609406726)-- (6.146446609406726,1.353553390593274);
\draw [line width=1pt] (10.01923802617959,3.137360563948689)-- (7.480761973820412,3.8626394360513108);
\draw [line width=1pt] (6.519238026179588,3.8626394360513108)-- (3.9807619738204116,3.137360563948689);
\draw [line width=1pt] (10.146446609406727,2.646446609406727)-- (9.853553390593273,2.353553390593273);
\draw [line width=1pt] (9.146446609406727,1.646446609406727)-- (8.853553390593273,1.353553390593273);
\draw [line width=1pt] (8.146446609406727,0.6464466094067269)-- (7.853553390593274,0.35355339059327395);
\draw [line width=1pt] (8.853553390593273,0.6464466094067269)-- (9.146446609406727,0.35355339059327306);
\draw [line width=1pt] (10.853553390593273,1.353553390593273)-- (11.146446609406727,1.646446609406727);
\draw [line width=1pt] (11.146446609406727,2.353553390593273)-- (10.853553390593273,2.646446609406727);
\draw [line width=1pt] (11.853553390593273,1.646446609406727)-- (12.146446609406727,1.353553390593273);


\draw[color=black] (1,1) node {$30$};
\draw[color=black] (1,.2) node {1};
\draw[color=black] (2,2) node {$00$};
\draw[color=black] (2,1.2) node {2};
\draw[color=black] (3,1) node {$04$};
\draw[color=black] (3,.2) node {3};
\draw[color=black] (3.5,3) node {$41$};
\draw[color=black] (3.5,2.2) node {4};
\draw[color=black] (4.5,1) node {$12$};
\draw[color=black] (4.5,.2) node {5};
\draw[color=black] (5.5,2) node {$22$};
\draw[color=black] (5.5,1.2) node {6};
\draw[color=black] (6.5,1) node {$21$};
\draw[color=black] (6.5,.2) node {7};
\draw[color=black] (7,4) node {$11$};
\draw[color=black] (7,3.2) node {8};
\draw[color=black] (7.5,0) node {$14$};
\draw[color=black] (7.5,-.8) node {9};
\draw[color=black] (8.5,1) node {$44$};
\draw[color=black] (8.5,.2) node {10};
\draw[color=black] (9.5,0) node {$45$};
\draw[color=black] (9.5,-.8) node {11};
\draw[color=black] (9.5,2) node {$55$};
\draw[color=black] (9.5,1.2) node {12};
\draw[color=black] (10.5,3) node {$54$};
\draw[color=black] (10.5,2.2) node {13};
\draw[color=black] (10.5,1) node {$40$};
\draw[color=black] (10.5,.2) node {14};
\draw[color=black] (11.5,2) node {$03$};
\draw[color=black] (11.5,1.2) node {15};
\draw[color=black] (12.5,1) node {$33$};
\draw[color=black] (12.5,.2) node {16};


\end{tikzpicture}
%}
\caption{Sequência~\eqref{eq:eulerSeq}  armazenada em uma ABB. Dentro do círculo mostramos o arco armazenado no nó e abaixo do círculo está sua chave.}
\label{fig:seq-treap-indices}
\end{figure}

Por exemplo, a ABB na Figura~\ref{fig:seq-treap-indices} está armazenando a sequência Euleriana~\eqref{eq:eulerSeq}.
Uma tal ABB é chamada de \defi{Euler tour tree}. 
Henzinger e King propuseram representar uma floresta por uma coleção de Euler tour trees: uma para cada componente da floresta. 
Dessa forma, como veremos na Seção~\ref{sec:impleDF-ETT}, é possível obter uma implementação de uma floresta dinâmica em que as operações de consulta de conexidade e de inserção e remoção de aresta têm custo~$\O{\lg n}$, onde $n$ é o número de vértices da floresta.


\section{Implementação de floresta dinâmica com Euler tour trees}
\label{sec:impleDF-ETT}

\NEW{
Para implementar a biblioteca de florestas dinâmicas com Euler tour trees, consideremos inicialmente a seguinte biblioteca.
}

\begin{itemize}
\item  \treapCreate($u$, $v$): cria e retorna uma ABB com somente um nó que armazena o par de vértices $uv$;
\item \treapGetRoot($\node$): retorna a raiz da ABB que contém $\node$;
\end{itemize}

A implementação dessas rotinas será detalhada no Capítulo~\ref{sec:TreapDeChaveImplicita}. Veremos que o consumo esperado de \treapCreate{} e \treapGetRoot{} será, respectivamente,~$\O{1}$ e~$\O{\lg n}$, onde~$n$ é o número de nós na ABB.

\NEW{
Precisaremos também de uma tabela de símbolos que, para cada par de vértices $(u,v)$, 
armazena um ponteiro para o nó da ABB que contém esse par, ou $\Nil$ se um tal nó não existir.
Notemos que para cada par de vértices $(u,v)$, existe no máximo um nó nas ABBs contendo $uv$.
Portanto não há ambiguidade sobre qual nó cada ponteiro deve indicar.
}
Para manusear a tabela, usaremos a seguinte biblioteca.
\begin{itemize}
    \item $F \gets \hashCreate(n)$: cria e retorna uma tabela de símbolos~$F$;
    \item $F[u,v] \gets uv$: insere o par~$(u,v)$ com valor associado~$uv$ na tabela~$F$.
    Se o par~$(u,v)$ já estiver presente na tabela de símbolos, então seu valor associado é substituído por~$uv$;
    \item $F[u,v] \gets \Nil{}$: remove o par~$(u,v)$ e seu valor associado da tabela de símbolos~$F$;
    \item $\var \gets F[u,v]$: atribui o valor associado \emph{var}  ao par~$(u,v)$ à variável~$\var$; Caso o par~$(u,v)$ não esteja presente em~$F$, atribui~$\Nil$ a~$\var$.
\end{itemize}
Assumiremos que a primeira rotina consome~$\O{n}$ e as demais consomem tempo esperado~$\O{1}$~\cite{CLRS}.

No Algoritmo~\ref{Algo:dymForestCreate} apresentamos a implementação de~\dymForestCreate{}, que cria e retorna uma nova floresta dinâmica que possui $n$ vértices e nenhuma aresta.
Já no algoritmo~\ref{Algo:dymForestQuery}, mostramos a implementação de~\dymForestQuery{}, que responde a consulta de conexidade entre dois vértices~$u$ e~$v$ na floresta~$F$.


\begin{algorithm}[htb]
\caption{\dymForestCreate($n$)}
\label{Algo:dymForestCreate}
\begin{algorithmic}[1]
\State $F~\gets~\hashCreate(n)$
\For {$v$ $\gets$ 1 até $n$}\label{Algo:dymForestCreate:for}
\State $F[v,v]~\gets$ \treapCreate($v$, $v$)
\EndFor
\State \Return $F$
\end{algorithmic}
\end{algorithm}

Com essa implementação, em uma floresta com~$n$ vértices, \dymForestCreate{} consumirá tempo~$\O{n}$. A rotina \dymForestQuery{}, descrita no Algoritmo~\ref{Algo:dymForestQuery}, consumirá tempo esperado $\O{\lg n}$.


\begin{algorithm}[htb]
\caption{\dymForestQuery($F$, $u$, $v$)}
\label{Algo:dymForestQuery}
\begin{algorithmic}[1]
\State $uu$ $\gets$ $F[u,u]$
\State $vv$ $\gets$ $F[v,v]$
\State \Return \treapGetRoot($uu$) = \treapGetRoot($vv$)
\end{algorithmic}
\end{algorithm}

Para implementar \dymForestAddEdge{} e \dymForestDelEdge{}, precisaremos das seguintes rotinas da biblioteca de Euler tour trees. 
\begin{itemize}
\item \treapSplit($\node$): recebe um ponteiro para um nó~$\node$, remove~$\node$ de sua ABB e retorna as raízes das duas ABBs resultantes; e
\item \treapJoin($T_1, T_2, \ldots, T_k$): junta as ABBs $T_1, T_2, \ldots, T_{k-1}$ e $T_k$ de modo que a sequência armazenada na árvore resultante é a concatenação das sequências armazenada em~$T_1, T_2, \ldots, T_k$ e retorna a raiz dessa árvore resultante.
\end{itemize}

Na Seção~\ref{sec:imple-treap}, mostraremos a implementação dessas rotinas e que~\treapJoin{} consome~$\O{h}$, onde~$h$ é a soma das alturas de~$T_1, T_2, \ldots, T_k$ e \treapSplit{} consome~$\O{\lg n}$, onde~$n$ é número de nós da árvore que contém~$\node$.

Para implementar a operação \dymForestDelEdge($F$, $u$, $v$), descrita no Algoritmo~\ref{Algo:dymForestDelEdge}, 
primeiro utilizaremos a tabela de símbolos para obter ponteiros para os nós que armazenam os arcos $uv$ e $vu$.

Em seguida, aplicamos~\treapSplit($uv$) dividindo essa sequência em duas partes, nomeadas~$A$ e~$B$, como pode ser visto na Figura~\ref{fig:algorit-cut-seqxy}.
\begin{figure}[htb]
\centering
\definecolor{ccqqqq}{rgb}{1,0,0}
\definecolor{qqqqcc}{rgb}{0,0,1}
\begin{tikzpicture}[line cap=round,line join=round,>=triangle 45,x=1cm,y=1cm]
\clip(-.1,-.6) rectangle (10.1,1.1);
\draw (-.1,0.8) node[anchor=north west] {\Large $w$};
\draw [line width=1pt] (0,0)-- (3,0);
\draw [line width=1pt] (0,1)-- (3,1);

\draw [line width=1pt] (0,0)-- (0,1);
\draw (2.45,0.8) node[anchor=north west] {\Large $u$};
\draw [line width=1pt] (3,1)-- (3,0);
\draw (1.2,0.85) node[anchor=north west] {\Large $A$};


\draw (4.05,0.8) node[anchor=north west] {\Large $uv$};
\draw [line width=1pt] (4,0)-- (5,0);
\draw [line width=1pt] (4,1)-- (5,1);
\draw [line width=1pt] (5,1)-- (5,0);
\draw [line width=1pt] (4,1)-- (4,0);


\draw [line width=1pt] (6,0)-- (6,1);
\draw [line width=1pt] (6,0)-- (10,0);
\draw [line width=1pt] (6,1)-- (10,1);
\draw [line width=1pt] (10,1)-- (10,0);
\draw (9.3,0.8) node[anchor=north west] {\Large $w$};

\draw (6.05,0.8) node[anchor=north west] {\Large $v$};
\draw (7.5,0.85) node[anchor=north west] {\Large $B$};
\end{tikzpicture}

\caption{Sequências $A$ e $B$ após a chamada de \treapSplit($uv$).}
\label{fig:algorit-cut-seqxy}
\end{figure}

Como a sequência original é Euleriana, sabemos que o primeiro e último vértice dessa sequência coincidem, na Figura~\ref{fig:algorit-cut-seqxy} chamaremos esse vértice de~$w$.
Dessa forma~$A$ representa um caminho de~$w$ até~$u$ e~$B$ representa um caminho de~$v$ até~$w$.

Note que não sabemos se $vu$ está na sequência~$A$ ou na sequência~$B$, mas não precisaremos dessa informação, pois podemos fazer a concatenação de~$B$ com~$A$,
chamando \treapJoin($B$, $A$), obtendo um caminho de~$v$ até~$u$ que passa pelo arco~$vu$ em algum ponto.

Por fim, para concluir esse algoritmo, basta chamar~\treapSplit($vu$) para dividir essa sequência em duas.
A primeira sendo a sequência Euleriana que representa a árvore que contém o vértice~$v$ e a segunda sendo a sequência Euleriana que contém o vértice~$u$.

\begin{algorithm}[htb]
\caption{\dymForestDelEdge($F$, $u$, $v$)}
\label{Algo:dymForestDelEdge}
\begin{algorithmic}[1]
\State $uv$ $\gets$ $F[u,v]$
\State $vu$ $\gets$ $F[v,u]$ 
\State $A$, $B$ $\gets$ \treapSplit($uv$)
\State \treapJoin($B$, $A$)
\State \treapSplit($vu$)
\State $F[u,v]$ $\gets$ $\Nil{}$
\State $F[v,u]$ $\gets$ $\Nil{}$
\end{algorithmic}
\end{algorithm}

Como exemplo, veremos o que ocorre com a sequência~\eqref{eq:eulerSeq} durante a execução da chamada \dymForestDelEdge($F$, $1$, $4$).
Primeiro, após a chamada de \treapSplit($14$) obtemos as seguinte duas sequências:
\begin{equation}
A = 30~00~04~41~12~22~21~11~~~~~~~~~~~~B = 44~45~55~54~40~03~33.\nonumber
\end{equation}
Nesse caso, temos que $w=3$. Após \treapJoin($B$, $A$) temos
\begin{equation}
 44~45~55~54~40~03~33~30~00~04~41~12~22~21~11.\nonumber
\end{equation}
Por fim, após \treapSplit($41$) temos: 
\begin{equation}
 44~45~55~54~40~03~33~30~00~04~~~~~~~~~~~~12~22~21~11.\label{eq:apos-remocao}
\end{equation}
Note que as duas sequências resultantes representam a floresta obtida pela remoção da aresta~$14$, que está ilustrada na Figura

\begin{figure}[htb]
\centering
\begin{tikzpicture}[dot/.style={draw,circle,fill,inner sep=1.5pt},line width=1pt,x=1.5cm,y=1.5cm]
\clip(.5,1.3) rectangle (3.3,3.5);
\draw [line width=1pt] (1.7420645075484014,2.9548225063123694)-- (0.6109449986613309,3.1228105521866865);
%\draw [line width=1pt] (1.7420645075484014,2.9548225063123694)-- (2,2);
\draw [line width=1pt] (2,2)-- (0.9693194965265414,1.7453085760172875);
\draw [line width=1pt] (2.5036103155119798,2.742037648204901)-- (3,2);
\draw [line width=1pt] (2,2)-- (2.5036103155119798,2.742037648204901);
\begin{scriptsize}
\draw [fill=black,bend left] (1.7420645075484014,2.9548225063123694) circle (1.5pt);
\draw[color=black] (1.8316581320147085,3.195605372065557) node {$1$};
\draw [fill=black] (0.6109449986613309,3.1228105521866865) circle (1.5pt);
\draw[color=black] (0.7005386231276353,3.363593417939874) node {$2$};
\draw [fill=black] (2,2) circle (1.5pt);
\draw[color=black] (1.9,1.8) node {$4$};
\draw [fill=black] (0.9693194965265414,1.7453085760172875) circle (1.5pt);
\draw[color=black] (1,2) node {$5$};
\draw [fill=black] (3,2) circle (1.5pt);
\draw[color=black] (3.0859688745429477,2.2357448368651927) node {$3$};
\draw [fill=black] (2.5036103155119798,2.742037648204901) circle (1.5pt);
\draw[color=black] (2.5932039399782822,2.9828205139580892) node {$0$};
\end{scriptsize}
\end{tikzpicture}

\caption{Floresta resultante de~\dymForestDelEdge($F$, $1$, $4$), representada pelas sequências~\eqref{eq:apos-remocao}.}
\label{fig:algorit-del-pos}
\end{figure}

Para implementar \dymForestAddEdge{}, descrita no Algoritmo~\ref{Algo:dymForestAddEdge}, utilizamos a rotina auxiliar \ETmovetofront{}.
Essa rotina recebe uma floresta~$F$ e um vértice~$u$ e restrutura a ABB que contém $uu$ de forma que este se torne o primeiro elemento de sua sequência Euleriana e retorna a raiz da ABB resultante. 

Por exemplo, se aplicarmos \ETmovetofront($F$,2) e \ETmovetofront($F$,5), onde $F$ é a floresta dinâmica ilustrada na Figura~\ref{fig:algorit-del-pos}, obtemos as sequências:
\begin{equation}
55~54~40~03~33~30~00~04~44~45~~~~~~~~~~~~22~21~11~12.\label{eq:apos-moveToFront}
\end{equation}

Para implementar~\ETmovetofront($F$, $u$), basta cortar a sequência em $uu$ chamando \treapSplit($uu$) e concatenar as sequências resultantes da forma apropriada com~\treapJoin{}. Note que como \treapSplit{} remove $uu$ da sequência, temos que adicioná-lo novamente a sequência como pode ser visto no Algoritmo~\ref{Algo:ETmovetofront}.

\begin{algorithm}[htb]
\caption{\ETmovetofront($F$, $u$)}
\label{Algo:ETmovetofront}
\begin{algorithmic}[1]
\State $uu$ $\gets$ $F[u,u]$
\State $A$, $B$ $\gets$ \treapSplit($uu$)
\State \Return \treapJoin($uu$, $B$, $A$)
\end{algorithmic}
\end{algorithm}


Com a rotina \ETmovetofront{} implementada, podemos elaborar~\dymForestAddEdge{}, descrita no Algoritmo~\ref{Algo:dymForestAddEdge}.


\begin{algorithm}[htb]
\caption{\dymForestAddEdge($F$, $u$, $v$)}
\label{Algo:dymForestAddEdge}
\begin{algorithmic}[1]
\State $U$ $\gets$ \ETmovetofront($F$, $u$)
\State $V$ $\gets$ \ETmovetofront($F$, $v$)
\State $uv$ $\gets$ \treapCreate($u$, $v$)
\State $vu$ $\gets$ \treapCreate($v$, $u$)
\State $F[u,v]$ $\gets$ $uv$
\State $F[v,u]$ $\gets$ $vu$
\State \treapJoin($U$, $uv$, $V$, $vu$)
\end{algorithmic}
\end{algorithm}

Primeiro, usamos \ETmovetofront{} para mover~$uu$ e~$vv$ para o início de suas sequências.
Em seguida, criamos novos nós~$uv$ e~$vu$; os adicionamos à tabela de símbolos e usamos \treapJoin{} pra unir todas as sequências de tal forma que a sequência resultante seja Euleriana.

\newpage
Dessa forma, se quisermos adicionar uma aresta ligando os vértices~$2$ e~$5$ na floresta da Figura~\ref{fig:algorit-del-pos}, obtendo assim a floresta da Figura~\ref{fig:algorit-add-pos}, primeiro temos que mover $22$ e $55$ para o início de suas sequências, como fizemos com as sequências \eqref{eq:apos-moveToFront}. Em seguida criamos os nós contendo $25$ e $52$ e unimos essas sequências, obtendo assim a sequência:
\begin{equation}
55~54~40~03~33~30~00~04~44~45~52~22~21~11~12~25.\label{eq:add}
\end{equation}
O consumo esperado de \dymForestAddEdge{} também será~$\O{\lg n}$.

\begin{figure}[htb]
\centering
\begin{tikzpicture}[dot/.style={draw,circle,fill,inner sep=1.5pt},line width=1pt,x=1.5cm,y=1.5cm]
\clip(.5,1.3) rectangle (3.3,3.5);
\draw [line width=1pt] (1.7420645075484014,2.9548225063123694)-- (0.6109449986613309,3.1228105521866865);
\draw [line width=1pt] (2,2)-- (0.9693194965265414,1.7453085760172875);
\draw [line width=1pt] (2.5036103155119798,2.742037648204901)-- (3,2);
\draw [line width=1pt] ( 0.9693194965265414,1.7453085760172875         )-- (0.6109449986613309,3.1228105521866865  );
\draw [line width=1pt] (2,2)-- (2.5036103155119798,2.742037648204901);
\begin{scriptsize}
\draw [fill=black,bend left] (1.7420645075484014,2.9548225063123694) circle (1.5pt);
\draw[color=black] (1.8316581320147085,3.195605372065557) node {$1$};
\draw [fill=black] (0.6109449986613309,3.1228105521866865) circle (1.5pt);
\draw[color=black] (0.7005386231276353,3.363593417939874) node {$2$};
\draw [fill=black] (2,2) circle (1.5pt);
\draw[color=black] (1.9,1.8) node {$4$};
\draw [fill=black] (0.9693194965265414,1.7453085760172875) circle (1.5pt);
\draw[color=black] (1,2) node {$5$};
\draw [fill=black] (3,2) circle (1.5pt);
\draw[color=black] (3.0859688745429477,2.2357448368651927) node {$3$};
\draw [fill=black] (2.5036103155119798,2.742037648204901) circle (1.5pt);
\draw[color=black] (2.5932039399782822,2.9828205139580892) node {$0$};
\end{scriptsize}
\end{tikzpicture}

\caption{Floresta resultante de~\dymForestAddEdge($F$, $2$, $5$), representada pelas sequências~\eqref{eq:add}.}
\label{fig:algorit-add-pos}
\end{figure}

%%%%%%%%%%%%%%%%%%%%%%%%%%%%%%%%%%%%%%%%%%%%%%%%
%        Treaps com chave implícita            %
%%%%%%%%%%%%%%%%%%%%%%%%%%%%%%%%%%%%%%%%%%%%%%%%
\chapter{Árvore binária de busca com chave implícita}
\label{sec:TreapDeChaveImplicita}

O objetivo desse capítulo é mostrar uma implementação eficiente para a biblioteca das Euler tour trees:
\begin{itemize}
\item  \treapCreate($u$, $v$): cria e retorna uma ABB com somente um nó que armazena o par de vértices $uv$;
\item \treapGetRoot($\node$): retorna a raiz da ABB que contém $\node$;
\item \treapSplit($\node$): corta a ABB que contém um nó~$\node$ em três ABBs. A primeira ABB contém todos os nós com chave estritamente menor do que a chave de~$\node$, a segunda contém somente~$\node$ e a última contém todos os nós com chave estritamente maior do que a chave de~$\node$. Essa rotina retorna as raízes dessas três ABBs; e
\item \treapJoin($T, R$): junta as ABBs~$T$ e~$R$ de modo que a sequência armazenada na árvore resultante é a concatenação das sequências armazenada em~$T$ e~$R$ e retorna a raiz dessa árvore resultante.
\end{itemize}

Note que cada uma das ABBs envolvidas nas Euler tour trees representa uma sequência Euleriana e os nós de cada ABB têm como chaves os inteiros de~$1$ a~$t$, onde~$t$ é o número de nós da ABB.

A operação \treapJoin{} terá o efeito de alterar a chave dos nós da árvore~$R$ para os inteiros de~$t+1,\ldots,t+r$, onde~$t$ e~$r$ são, respectivamente, o número de nós nas árvores~$T$ e~$R$.
Já a operação \treapSplit{} terá o efeito de alterar as chaves da segunda e da terceira ABBs devilvidas, para que fiquem entre~$1$ e~$t$, onde~$t$ é o número de nós da ABB.

Se as chaves forem armazenadas explicitamente nos nós das ABBs, não há como garantir uma implementação muito eficiente.
Aqui apresentaremos uma implementação uma implementação que omite as chaves dos nós, e assim admite implementações para \treapJoin{} e \treapSplit{} que consomem tempo esperado~$\O{\lg n}$, onde~$n$ é o número de nós nas ABBs envolvidas nas operações.

%Com uma implementação tradicional de ABB, em que o campo \varname{chave} $\mathit{chave}$ dos nós de uma ABB~$T$ com $t$ nós armazena explicitamente os números de $1$ a $t$, é necessário, na operação \treapJoin($T$, $R$), atualizar todas as chaves de~$R$ para que possuam os valores $1+t,2+t, \ldots, r+t$, onde $r$ é o número de nós em~$R$.

%Analogamente, com essa implementação tradicional, ao realizar \treapSplit($T$, \varname{nó}) é necessário atualizar todos os nós com chave maior do que a chave de \varname{nó}.
%No pior dos casos, essas atualizações de chaves podem custar tempo $\OTheta{n}$, onde $n$ é o tamanho das árvores envolvidas.

Para reduzir o consumo de tempo de \treapJoin{} e \treapSplit{}, na biblioteca de Euler tour trees, as ABBs utilizadas possuirão \defi{chave implícita}.
Especificamente, substituímos o campo $chave$ de cada nó pelo campo \defi{tam} que armazena o tamanho da subárvore enraizada naquele nó, isto é, o número de nós nessa subárvore.
Com esse novo campo, a chave de cada nó pode ser determinada em tempo proporcional à profundidade do nó na ABB.

\begin{figure}[htb]
\centering
\begin{tikzpicture}[line cap=round,line join=round,>=triangle 45,x=1cm,y=1cm]
\clip(-.6,-1) rectangle (14.1,4.6);
\draw [line width=1pt] (1,1) circle (0.5cm);
\draw [line width=1pt] (2,2) circle (0.5cm);
\draw [line width=1pt] (3,1) circle (0.5cm);
\draw [line width=1pt] (5.5,2) circle (0.5cm);
\draw [line width=1pt] (6.5,1) circle (0.5cm);
\draw [line width=1pt] (3.5,3) circle (0.5cm);
\draw [line width=1pt] (7,4) circle (0.5cm);
\draw [line width=1pt] (10.5,3) circle (0.5cm);
\draw [line width=1pt] (9.5,2) circle (0.5cm);
\draw [line width=1pt] (8.5,1) circle (0.5cm);
\draw [line width=1pt] (11.5,2) circle (0.5cm);
\draw [line width=1pt] (10.5,1) circle (0.5cm);
\draw [line width=1pt] (12.5,1) circle (0.5cm);
\draw [line width=1pt] (4.5,1) circle (0.5cm);
\draw [line width=1pt] (7.5,0) circle (0.5cm);
\draw [line width=1pt] (9.5,0) circle (0.5cm);
\draw [line width=1pt] (1.3535533905932737,1.3535533905932737)-- (1.646446609406726,1.646446609406726);
\draw [line width=1pt] (2.353553390593274,1.646446609406726)-- (2.646446609406726,1.353553390593274);
i%\draw [line width=1pt] (2.646446609406726,0.646446609406726)-- (2.353553390593274,0.35355339059327395);
\draw [line width=1pt] (2.416025147168923,2.2773500981126156)-- (3.0839748528310773,2.722649901887385);
\draw [line width=1pt] (3.947213595499958,2.776393202250021)-- (5.052786404500042,2.223606797749979);
\draw [line width=1pt] (5.146446609406726,1.646446609406726)-- (4.853553390593274,1.353553390593274);
\draw [line width=1pt] (5.853553390593274,1.646446609406726)-- (6.146446609406726,1.353553390593274);
\draw [line width=1pt] (10.01923802617959,3.137360563948689)-- (7.480761973820412,3.8626394360513108);
\draw [line width=1pt] (6.519238026179588,3.8626394360513108)-- (3.9807619738204116,3.137360563948689);
\draw [line width=1pt] (10.146446609406727,2.646446609406727)-- (9.853553390593273,2.353553390593273);
\draw [line width=1pt] (9.146446609406727,1.646446609406727)-- (8.853553390593273,1.353553390593273);
\draw [line width=1pt] (8.146446609406727,0.6464466094067269)-- (7.853553390593274,0.35355339059327395);
\draw [line width=1pt] (8.853553390593273,0.6464466094067269)-- (9.146446609406727,0.35355339059327306);
\draw [line width=1pt] (10.853553390593273,1.353553390593273)-- (11.146446609406727,1.646446609406727);
\draw [line width=1pt] (11.146446609406727,2.353553390593273)-- (10.853553390593273,2.646446609406727);
\draw [line width=1pt] (11.853553390593273,1.646446609406727)-- (12.146446609406727,1.353553390593273);


\draw[color=black] (1,1) node {$30$};
\draw[color=black] (1,.2) node {1};
\draw[color=black] (2,2) node {$00$};
\draw[color=black] (2,1.2) node {3};
\draw[color=black] (3,1) node {$04$};
\draw[color=black] (3,.2) node {1};
\draw[color=black] (3.5,3) node {$41$};
\draw[color=black] (3.5,2.2) node {7};
\draw[color=black] (4.5,1) node {$12$};
\draw[color=black] (4.5,.2) node {1};
\draw[color=black] (5.5,2) node {$22$};
\draw[color=black] (5.5,1.2) node {3};
\draw[color=black] (6.5,1) node {$21$};
\draw[color=black] (6.5,.2) node {1};
\draw[color=black] (7,4) node {$11$};
\draw[color=black] (7,3.2) node {16};
\draw[color=black] (7.5,0) node {$14$};
\draw[color=black] (7.5,-.8) node {1};
\draw[color=black] (8.5,1) node {$44$};
\draw[color=black] (8.5,.2) node {3};
\draw[color=black] (9.5,0) node {$45$};
\draw[color=black] (9.5,-.8) node {1};
\draw[color=black] (9.5,2) node {$55$};
\draw[color=black] (9.5,1.2) node {4};
\draw[color=black] (10.5,3) node {$54$};
\draw[color=black] (10.5,2.2) node {8};
\draw[color=black] (10.5,1) node {$40$};
\draw[color=black] (10.5,.2) node {1};
\draw[color=black] (11.5,2) node {$03$};
\draw[color=black] (11.5,1.2) node {3};
\draw[color=black] (12.5,1) node {$33$};
\draw[color=black] (12.5,.2) node {1};


\end{tikzpicture}

\caption{Árvore da Figura~\ref{fig:seq-treap-indices} exibindo o valor do campo $tam$ abaixo de cada nó.}
\label{fig:seq-treap-size}
\end{figure}

Com essa mudança, em \treapJoin{} e \treapSplit{}, será necessário ajustar apenas o campo $tam$ de um número bem mais reduzido de nós, o que resultará em uma implementação com o consumo de tempo desejado.

Como manipularemos muitos ponteiros a ABBs que podem conter $\Nil$, é conveniente a adição da rotina interna \treapGetSize($T$), descrita no Algoritmo~\ref{Algo:treapGetSize}, que retorna $0$ caso $T$ seja~$\Nil$ e, no caso em que $T$ aponte para uma ABB não vazia, retorna seu tamanho. Essa rotina consome tempos~$\O{1}$.

\begin{algorithm}[!htb]
\caption{\treapGetSize($T$)}
\label{Algo:treapGetSize}
\begin{algorithmic}[1]
\If { $T$ = \Nil}
\State \Return $0$
\EndIf
\State \Return $T$.$tam$
\end{algorithmic}
\end{algorithm}

Com chaves implícitas, a alteração de uma subárvore enraizada num nó $x$ torna necessário somente a atualização do campo $tam$ dos nós do caminho entre o nó $x$ e a raiz da ABB que contém~$x$, ou seja, o consumo de tempo dessas atualizações será assintoticamente proporcional à altura da ABB. Resta então o desafio de manter a ABB balanceada. Na próxima seção apresentaremos a estrutura de dados treap, que resolve esse desafio sem onerar o custo assintótico das operações ou adicionar demasiada complexidade aos algoritmos.

\section{Treaps}

\defi{Heaps} são árvores binárias quase completas constituídas por nós que possuem quatro campos: prioridade, pai e filhos esquerdo e direito~\cite{CLRS}.
Os campos referentes aos pais e filhos dão a estrutura de árvore binária ao heap e a \defi{prioridade} de um nó é um número real não negativo.
Não apresentaremos a definição de árvore binária quase completa, pois ela é desnecessária para a definição de treaps. Para que uma árvore binária quase completa seja considerada um heap é necessário que a prioridade de cada nó seja maior do que a de seus filhos.

\defi{Treaps} são uma mescla entre árvores binárias de busca e heaps. Seus nós possuem cinco campos: \varname{pai}, \varname{esq}, \varname{dir}, $chave$ e $prio$. Os campos \varname{esq} e \varname{dir} representam os filhos esquerdo e direito de cada nó, respectivamente, e junto ao campo \varname{pai} descrevem a estrutura de uma árvore binária. O campo $chave$ satisfaz a propriedade de uma ABB enquanto o campo $prio$ armazena a prioridade do nó e satisfaz a propriedade de um heap. Diferentemente de heaps, treaps não precisam ser quase completas.


Tais árvores foram inicialmente nomeadas \defi[arvore@\'arvore!cartesiana]{árvores cartesianas} \cite{Vuillemin1980AUL}, pois podemos representar cada nó como um par ordenado em que a primeira coordenada é a chave do nó e a segunda coordenada é sua prioridade. Ao visualizar esses pares imersos no plano cartesiano, como feito na Figura \ref{fig:TREAP}, ambas as estruturas de ABB e heap são bem representadas. Isto é, nós com maior prioridade ficam ilustrados acima de nós com menor prioridade e os nós ficam ordenados de forma crescente, da esquerda para a direita, em função de suas chaves.

\begin{figure}[htb]
\centering
\begin{tikzpicture}[line cap=round,line join=round,>=triangle 45,x=1cm,y=1cm]
\begin{axis}[
x=1.5cm,y=1.5cm,
axis lines=middle,
ymajorgrids=true,
xmajorgrids=true,
xmin=-.5,
xmax=7.5,
ymin=-.5,
ymax=7.5,
xtick={-3,-2,...,16},
ytick={-1,0,...,10},]
\clip(-1,-1) rectangle (16.11970291004026,10.724908935002318);
\draw [line width=2pt] (3.383319601128083,6.426379568244459)-- (2.0132670659467835,4.457693392208796);
\draw [line width=2pt] (3.383319601128083,6.426379568244459)-- (4.630423800125734,4.5424562038011995);
\draw [line width=2pt] (2.0132670659467835,4.457693392208796)-- (1.1677195880152356,2.990209488027528);
\draw [line width=2pt] (2.0132670659467835,4.457693392208796)-- (2.9057052270438763,3.295352004803549);
\draw [line width=2pt] (4.630423800125734,4.5424562038011995)-- (3.449654930862001,2.8177376307193422);
\draw [line width=2pt] (4.630423800125734,4.5424562038011995)-- (5.399913625039178,3.189215477229281);
\draw [line width=2pt] (3.449654930862001,2.8177376307193422)-- (4.325281283349713,0.9072801343825158);
\draw [line width=2pt] (5.399913625039178,3.189215477229281)-- (5.943863328857303,1.3981615744135059);
\draw [line width=2pt] (1.1677195880152356,2.990209488027528)-- (1.6586010280462258,1.1593543873714025);
\begin{scriptsize}
\draw [fill=black] (3.383319601128083,6.426379568244459) circle (2.5pt);
\draw[color=black] (4.073207030360826,6.711621486100305) node {$(3.38, 6.43)$};
\draw [fill=black] (2.0132670659467835,4.457693392208796) circle (2.5pt);
\draw[color=black] (1.3,4.748095725976344) node {$(2.01, 4.46)$};
\draw [fill=black] (4.630423800125734,4.5424562038011995) circle (2.5pt);
\draw[color=black] (5.320311229358477,4.827698121657045) node {$(4.63, 4.54)$};
\draw [fill=black] (1.1677195880152356,2.990209488027528) circle (2.5pt);
\draw[color=black] (.65,3.275451405883374) node {$(1.17, 3)$};
\draw [fill=black] (2.9057052270438763,3.295352004803549) circle (2.5pt);
\draw[color=black] (3.3,3.5805939226593946) node {$(2.91, 3.3)$};
\draw [fill=black] (3.449654930862001,2.8177376307193422) circle (2.5pt);
\draw[color=black] (4.2,2.8) node {$(3.45, 2.82)$};
\draw [fill=black] (5.399913625039178,3.189215477229281) circle (2.5pt);
\draw[color=black] (6.036732790484788,3.4744573950851265) node {$(5.4, 3.19)$};
\draw [fill=black] (4.325281283349713,0.9072801343825158) circle (2.5pt);
\draw[color=black] (5.015168712582456,1.1925220522383615) node {$(4.33, 0.91)$};
\draw [fill=black] (5.943863328857303,1.3981615744135059) circle (2.5pt);
\draw[color=black] (6.580682494302912,1.6834034922693517) node {$(5.94, 1.4)$};
\draw [fill=black] (1.6586010280462258,1.1593543873714025) circle (2.5pt);
\draw[color=black] (2.3484884572789686,1.4445963052272481) node {$(1.66, 1.16)$};
\end{scriptsize}
\end{axis}
\end{tikzpicture}
\caption{Uma treap imersa no plano cartesiano.}
\label{fig:TREAP}
\end{figure}

\section{Treaps implícitas}
\label{sec:imple-treap}

\defi{Treaps implícitas} são treaps com chaves implícitas~\cite{treap}.

A implementação da rotina \treapGetRoot{}, apresentada no Algoritmo~\ref{Algo:treapGetRoot}, independe da técnica que usaremos para balancear a treap,
pois é uma consulta e somente usa a estrutura da ABB sem fazer modificações.
Seu consumo de tempo é~$\O{h}$ onde~$h$ é a altura da árvore.

\begin{algorithm}[htb]
\caption{\treapGetRoot(\varname{nó})}
\label{Algo:treapGetRoot}
\begin{algorithmic}[1]
\State \varname{raiz} $\gets$ \varname{nó}
\While {\varname{raiz}.\varname{pai} $\neq \Nil$}
\State  \varname{raiz} $\gets$  \varname{raiz}.\varname{pai}
\EndWhile
\State \Return \varname{raiz}
\end{algorithmic}
\end{algorithm}

Para balancear uma treap e assim garantir consumo de tempo logarítmico, Aragon e Seidel~\cite{AragonSeidel1989, AragonSeidel1996} propuseram escolher a prioridade de cada nó de forma aleatória com distribuição de probabilidade uniforme em um universo suficientemente grande para que a probabilidade de haver nós com a mesma prioridade seja próxima de~$0$.


Para representar essa escolha aleatória, usaremos uma função auxiliar \random() que retorna um número real entre $0$ e~$1$ escolhido com probabilidade uniforme.
Aproximações para esse tipo de função está presente nativamente em diversas linguagens de programação e a elaboração de sua implementação foge do escopo desse texto. Consideraremos que seu consumo de tempo é~$\O{1}$.

Com essa técnica de balanceamento de treaps em mãos, podemos apresentar a implementação dos demais algoritmos que compõem essa biblioteca.
O primeiro desses é \treapCreate($u$, $v$), que recebe vértices~$u$ e~$v$, cria um nó de treap chamado \varname{nó}, inicializa seus campos apropriadamente e o retorna.
Essa rotina consome tempo~$\O{1}$. 

\begin{algorithm}
\caption{\treapCreate($u$, $v$)}
\label{Algo:TREAPbuild}
\begin{algorithmic}[1]
\State $\node$.\varname{tam} $\gets 1$
\State $\node$.\varname{prio} $\gets$ \random() 
\State $\node$.\varname{info} $\gets$ ($u$, $v$)
\State $\node$.\varname{esq} $\gets$ $\node$.\varname{dir} $\gets$ $\node$.\varname{pai} $\gets$ \Nil
\State \Return $\node$
\end{algorithmic}
\end{algorithm}

\NEW{
A implementação da rotina $\treapJoin$ pode ser vista no Algoritmo~\ref{Algo:TREAPjoin}.
Essa rotina junta as ABBs $T$ e~$R$ de modo que a sequência armazenada na árvore resultante seja a concatenação das sequências armazenada em~$T_1, T_2, \ldots, T_k$ e retorna a raiz dessa árvore resultante.
}

\begin{algorithm}
\caption{\treapJoin($T$, $R$)}
\label{Algo:TREAPjoin}
\begin{algorithmic}[1]
\If { $T$ = \Nil} \Return $R$
\EndIf
\If { $R$ = \Nil} \Return $T$
\EndIf

\If { $T$.$prio$ $>$ $R$.$prio$}\label{algo:TREAPjoin:if}
  \State $T$.\varname{dir} $\gets$ \treapJoin($T$.\varname{dir}, $R$)
  \State $T$.\varname{dir}.\varname{pai} $\gets$ $T$
  \State $T$.$tam$ $\gets$ $T$.$tam$ + $R$.$tam$
  \State \Return $T$
\Else 
  \State $R$.\varname{esq} $\gets$ \treapJoin($T$, $R$.\varname{esq})
  \State $R$.\varname{esq}.\varname{pai} $\gets$ $R$
  \State $R$.$tam$ $\gets$ $T$.$tam$ + $R$.$tam$
  \State \Return $R$
\EndIf
\end{algorithmic}
\end{algorithm}

Para compreender o funcionamento de~\treapJoin{} podemos utilizar a ideia da imersão das treaps no plano cartesiano para visualizar melhor suas estruturas.
Na Figura~\ref{fig:treap-join}, podemos ver um exemplo de treaps~$T$ e~$R$.

Após tratar os casos em que~$T$ ou~$R$ podem ser~$\Nil$, precisamos definir o nó que será a raiz da junção das duas árvores.
Este nó será aquele com maior prioridade, assim mantendo a propriedade de heap.
Essa comparação é feita na linha~\ref{algo:TREAPjoin:if}.

Se a prioridade da raiz de~$T$ for maior do que a de~$R$, como ilustrado na Figura~\ref{fig:treap-join}, então devemos juntar~$R$ a alguma subárvore de~$T$.
Como usamos chaves implícitas e a sequência contida em~$R$ ficará após a sequência contida em~$T$ depois da junção dessas árvores,
temos que todas as chaves de~$R$ são estritamente maiores do que as chaves de~$T$, logo temos que recursivamente juntar~$R$ com a subárvore enraizada no nó $T$.\varname{dir}.
Em seguida, corrigimos o campo \varname{pai} de $T$.\varname{dir} e o campo~$tam$ de~$T$.


\begin{figure}[htb]
\begin{subfigure}{0.4\textwidth}
\scalebox{0.55}{
\begin{tikzpicture}[line cap=round,line join=round,>=triangle 45,x=1cm,y=1cm]
\begin{axis}[
x=1cm,y=1cm,
axis lines=middle,
ymajorgrids=true,
xmajorgrids=true,
xmin=0,
xmax=10.5,
ymin=-.5,
ymax=6.5,
xtick={-4,-3,...,21},
ytick={-3,-2,...,12},]
\clip(-4.171746300211414,-3.220558139534887) rectangle (21.671805496828764,12.052169133192377);
\draw [line width=2pt] (4,6)-- (2,2);
\draw [line width=2pt] (2,2)-- (1,0);
\draw [line width=2pt] (2,2)-- (3,1);
\draw [line width=2pt] (4,6)-- (6,3);
\draw [line width=2pt] (6,3)-- (5,1);
\draw [line width=2pt] (10,5)-- (8,4);
\draw [line width=2pt] (8,4)-- (7,2);
\draw [line width=2pt] (8,4)-- (9.01218604651163,1.486418604651162);
\begin{scriptsize}
\draw [fill=black] (1,0) circle (2.5pt);
\draw [fill=black] (2,2) circle (2.5pt);
\draw [fill=black] (3,1) circle (2.5pt);
\draw [fill=black] (4,6) circle (2.5pt);
\draw [fill=black] (5,1) circle (2.5pt);
\draw [fill=black] (6,3) circle (2.5pt);
\draw [fill=black] (7,2) circle (2.5pt);
\draw [fill=black] (8,4) circle (2.5pt);
\draw [fill=black] (9.01218604651163,1.486418604651162) circle (2.5pt);
\draw [fill=black] (10,5) circle (2.5pt);
\draw (4,6) node[anchor=south] {$T$};
\draw (10,5) node[anchor=south] {$R$};
\end{scriptsize}
\end{axis}
\end{tikzpicture}

}
\caption{Exemplo de duas treaps imersas no plano cartesiano.}
\label{fig:treap-join}
\end{subfigure}
\hspace{1cm}
\begin{subfigure}{0.4\textwidth}
\scalebox{0.55}{
\begin{tikzpicture}[line cap=round,line join=round,>=triangle 45,x=1cm,y=1cm]
\begin{axis}[
x=1cm,y=1cm,
axis lines=middle,
ymajorgrids=true,
xmajorgrids=true,
xmin=0,
xmax=10.5,
ymin=-.5,
ymax=6.5,
xtick={-4,-3,...,21},
ytick={-3,-2,...,12},]
\clip(-4.171746300211414,-3.220558139534887) rectangle (21.671805496828764,12.052169133192377);
\draw [line width=2pt] (4,6)-- (2,2);
\draw [line width=2pt] (2,2)-- (1,0);
\draw [line width=2pt] (2,2)-- (3,1);
\draw [line width=2pt,color=red] (4,6)-- (10,5);
\draw [line width=2pt] (6,3)-- (5,1);
\draw [line width=2pt] (10,5)-- (8,4);
\draw [line width=2pt,color=red] (8,4)-- (6,3);
\draw [line width=2pt,color=red] (7,2)-- (6,3);
\draw [line width=2pt] (8,4)-- (9.01218604651163,1.486418604651162);
\begin{scriptsize}
\draw [fill=black] (1,0) circle (2.5pt);
\draw [fill=black] (2,2) circle (2.5pt);
\draw [fill=black] (3,1) circle (2.5pt);
\draw [fill=black] (4,6) circle (2.5pt);
\draw [fill=black] (5,1) circle (2.5pt);
\draw [fill=black] (6,3) circle (2.5pt);
\draw [fill=black] (7,2) circle (2.5pt);
\draw [fill=black] (8,4) circle (2.5pt);
\draw [fill=black] (9.01218604651163,1.486418604651162) circle (2.5pt);
\draw [fill=black] (10,5) circle (2.5pt);
\end{scriptsize}
\end{axis}
\end{tikzpicture}

}
\caption{Resultado da junção das treaps~$T$ e~$R$. Ponteiros modificados estão destacados em vermelho.}
\label{fig:treap-join-depois}
\end{subfigure}
\caption{Antes e depois da junção das treaps~$T$ e~$R$.}
\end{figure}


Caso a prioridade de~$R$ seja maior do que a de~$T$, então temos que juntar recursivamente $T$ com a subárvore enraizada em~$R$.\varname{esq}. Esse caso é simétrico ao anterior.
Podemos ver o resultado da junção das treaps~$T$ e~$R$ da Figura~\ref{fig:treap-join} na Figura~\ref{fig:treap-join-depois}.


%\begin{figure}[htb]
%\centering
%\begin{tikzpicture}[line cap=round,line join=round,>=triangle 45,x=1cm,y=1cm]
\begin{axis}[
x=1cm,y=1cm,
axis lines=middle,
ymajorgrids=true,
xmajorgrids=true,
xmin=0,
xmax=10.5,
ymin=-.5,
ymax=6.5,
xtick={-4,-3,...,21},
ytick={-3,-2,...,12},]
\clip(-4.171746300211414,-3.220558139534887) rectangle (21.671805496828764,12.052169133192377);
\draw [line width=2pt] (4,6)-- (2,2);
\draw [line width=2pt] (2,2)-- (1,0);
\draw [line width=2pt] (2,2)-- (3,1);
\draw [line width=2pt,color=red] (4,6)-- (10,5);
\draw [line width=2pt] (6,3)-- (5,1);
\draw [line width=2pt] (10,5)-- (8,4);
\draw [line width=2pt,color=red] (8,4)-- (6,3);
\draw [line width=2pt,color=red] (7,2)-- (6,3);
\draw [line width=2pt] (8,4)-- (9.01218604651163,1.486418604651162);
\begin{scriptsize}
\draw [fill=black] (1,0) circle (2.5pt);
\draw [fill=black] (2,2) circle (2.5pt);
\draw [fill=black] (3,1) circle (2.5pt);
\draw [fill=black] (4,6) circle (2.5pt);
\draw [fill=black] (5,1) circle (2.5pt);
\draw [fill=black] (6,3) circle (2.5pt);
\draw [fill=black] (7,2) circle (2.5pt);
\draw [fill=black] (8,4) circle (2.5pt);
\draw [fill=black] (9.01218604651163,1.486418604651162) circle (2.5pt);
\draw [fill=black] (10,5) circle (2.5pt);
\end{scriptsize}
\end{axis}
\end{tikzpicture}

%\caption{Resultado da junção das treaps~$T$ e~$R$ da Figura~\ref{fig:treap-join}. Ponteiros modificados estão destacados em vermelho.}
%\label{fig:treap-join-depois}
%%\end{figure}

Notemos que \treapJoin{} consome $\O{h}$ onde~$h$ é a soma das alturas das duas árvores que são unidas.
Como a altura esperada dessas árvores é $\O{\lg n}$, temos que o consumo esperado de tempo dessa rotina também será logarítmico.


A implementação da rotina~\treapSplit($\node$) pode ser vista no Algoritmo~\ref{Algo:TREAPsplit}.
Nesse algoritmo percorremos o caminho de~$\node$ até a raiz de sua árvore reatribuindo a relação de parentesco entre os nós percorridos de forma a cortar a treap em duas:
a primeira contendo todos os nós com chave menor do que a chave de~$\node$ e a segunda com todos os nós com chave maior do que~$\node$.
Para fazer isso mantemos três ponteiros: $L$, $R$ e~\varname{tmp}. 
Os dois primeiros apontam para as raízes das árvores resultantes desse corte e o terceiro aponta para o nó cuja relação de parentesco será modificada.

\begin{algorithm}
\caption{\treapSplit($\node$)}
\label{Algo:TREAPsplit}
\begin{algorithmic}[1]
\State $R$ $\gets $ $\node$.\varname{dir}; $L$ $\gets $ $\node$.\varname{esq}; \varname{tmp} $\gets$ $\node$
\While { \varname{tmp}.\varname{pai} $\neq \Nil$}
  \If { \varname{tmp}.\varname{pai}.\varname{esq} = \varname{tmp}}
    \State \varname{tmp}.\varname{pai}.\varname{esq} $\gets$ $R$
    \State \varname{tmp}.\varname{pai}.$tam$ $\gets$ \varname{tmp}.\varname{pai}.$tam$ $\mathit{-}$ \treapGetSize($L$)
    \If { $R \neq \Nil$}
    \State $R$.\varname{pai} $\gets$ \varname{tmp}.\varname{pai}
    \EndIf
    \State $R$ $\gets$ \varname{tmp}.\varname{pai}
  \Else
    \State \varname{tmp}.\varname{pai}.\varname{dir} $\gets$ $L$ 
    \State \varname{tmp}.\varname{pai}.$tam$ $\gets$ \varname{tmp}.\varname{pai}.$tam$ $\mathit{-}$ \treapGetSize($R$)
    \If { $L$ $\neq$ \Nil}
    \State $L$.\varname{pai} $\gets$ \varname{tmp}.\varname{pai}
    \EndIf
    \State $L$ $\gets$ \varname{tmp}.\varname{pai}
  \EndIf
  \State \varname{tmp} $\gets$ \varname{tmp}.\varname{pai}
\EndWhile
\If { $L$ $\neq \Nil$} $L$.\varname{pai} $\gets\Nil$\EndIf
\If { $R$ $\neq \Nil$} $R$.\varname{pai} $\gets\Nil$\EndIf
\State $\node$.\varname{dir} $\gets $ $\node$.\varname{esq} $\gets$ $\node$.\varname{pai} $\gets$ $\Nil$
\State\Return $L$, $\node$, $R$
\end{algorithmic}
\end{algorithm}

Simularemos a execução de $\treapSplit(X)$, onde $X$ é o nó de coordenadas $(6,3)$ da Figura~\ref{fig:treap-join-depois}.
O estado inicial dos ponteiros~$L$ e~$R$ são os filhos de~$X$ e~\varname{tmp} aponta para~$X$, como mostra a Figura~\ref{fig:treap-split-1}.

\begin{figure}[htb]
\begin{subfigure}{0.35\textwidth}
\scalebox{.76}{
\begin{tikzpicture}[line cap=round,line join=round,>=triangle 45,x=1cm,y=1cm]
\begin{axis}[
x=1cm,y=1cm,
axis lines=middle,
ymajorgrids=true,
xmajorgrids=true,
xmin=4,
xmax=8.5,
ymin=.5,
ymax=5.5,
xtick={-4,-3,...,21},
ytick={-3,-2,...,12},]
\clip(4.171746300211414,-3.220558139534887) rectangle (21.671805496828764,12.052169133192377);
\draw [line width=2pt] (4,6)-- (2,2);
\draw [line width=2pt] (2,2)-- (1,0);
\draw [line width=2pt] (2,2)-- (3,1);
\draw [line width=2pt] (4,6)-- (10,5);
\draw [line width=2pt] (6,3)-- (5,1);
\draw [line width=2pt] (10,5)-- (8,4);
\draw [line width=2pt] (8,4)-- (6,3);
\draw [line width=2pt] (7,2)-- (6,3);
\draw [line width=2pt] (8,4)-- (9.01218604651163,1.486418604651162);
\begin{scriptsize}
\draw [fill=black] (1,0) circle (2.5pt);
\draw [fill=black] (2,2) circle (2.5pt);
\draw [fill=black] (3,1) circle (2.5pt);
\draw [fill=black] (4,6) circle (2.5pt);
\draw [fill=black] (5,1) circle (2.5pt);
\draw [fill=black] (6,3) circle (2.5pt);
\draw [fill=black] (7,2) circle (2.5pt);
\draw [fill=black] (8,4) circle (2.5pt);
\draw [fill=black] (9.01218604651163,1.486418604651162) circle (2.5pt);
\draw [fill=black] (10,5) circle (2.5pt);
\draw (6,3) node[anchor=south east] {$X=tmp$};
\draw (5,1) node[anchor=south east] {$L$};
\draw (7,2) node[anchor=south west] {$R$};
\end{scriptsize}
\end{axis}
\end{tikzpicture}

	}
\caption{Estado inicial dos ponteiros $L$, $R$ e \varname{tmp}.}
\label{fig:treap-split-1}
\end{subfigure}
\hspace{1cm}
\begin{subfigure}{0.35\textwidth}
\scalebox{.76}{
\begin{tikzpicture}[line cap=round,line join=round,>=triangle 45,x=1cm,y=1cm]
\begin{axis}[
x=1cm,y=1cm,
axis lines=middle,
ymajorgrids=true,
xmajorgrids=true,
xmin=4.5,
xmax=10.5,
ymin=.5,
ymax=5.5,
xtick={-4,-3,...,21},
ytick={-3,-2,...,12},]
\clip(4.171746300211414,-3.220558139534887) rectangle (21.671805496828764,12.052169133192377);
\draw [line width=2pt] (4,6)-- (2,2);
\draw [line width=2pt] (2,2)-- (1,0);
\draw [line width=2pt] (2,2)-- (3,1);
\draw [line width=2pt] (4,6)-- (10,5);
\draw [line width=2pt] (6,3)-- (5,1);
\draw [line width=2pt] (10,5)-- (8,4);
\draw [line width=2pt] (7,2)-- (8,4);
\draw [line width=2pt] (8,4)-- (9.01218604651163,1.486418604651162);
\begin{scriptsize}
\draw [fill=black] (1,0) circle (2.5pt);
\draw [fill=black] (2,2) circle (2.5pt);
\draw [fill=black] (3,1) circle (2.5pt);
\draw [fill=black] (4,6) circle (2.5pt);
\draw [fill=black] (5,1) circle (2.5pt);
\draw [fill=black] (6,3) circle (2.5pt);
\draw [fill=black] (7,2) circle (2.5pt);
\draw [fill=black] (8,4) circle (2.5pt);
\draw [fill=black] (9.01218604651163,1.486418604651162) circle (2.5pt);
\draw [fill=black] (10,5) circle (2.5pt);
\draw (6,3) node[anchor=south east] {$X$};
\draw (5,1) node[anchor=south east] {$L$};
\draw (8,4) node[anchor=south east] {$tmp=R$};
\end{scriptsize}
\end{axis}
\end{tikzpicture}

	}
\caption{Estado após atualização dos ponteiros de parentesco e \varname{tmp}.}
\label{fig:treap-split-2}
\end{subfigure}
\caption{Dois estados da simulação de~\treapSplit(X).}
\end{figure}
Se~\varname{tmp} for filho esquerdo de seu pai, isto é, \varname{tmp}.\varname{pai}.\varname{esq} $=$ \varname{tmp}, então a chave de \varname{tmp}.\varname{pai} é maior do que a chave de~$X$ e ele logo deve ser um nó de~$R$.
Além disso, sabemos que a prioridade e chave de \varname{tmp}.\varname{pai} são maiores do que a prioridade e chave de~$R$,
logo~$R$ torna-se o filho esquerdo de~\varname{tmp}.\varname{pai} e~\varname{tmp}.\varname{pai} deve ser a nova raiz de~$R$. 
Após atualizar os campos de parentesco entre~$R$, \varname{tmp} e \varname{tmp}.\varname{pai}, atualizamos o valor de~\varname{tmp} para \varname{tmp}.\varname{pai}, obtendo o estado mostrado na Figura~\ref{fig:treap-split-2}.
Então repetimos esse processo. 


Na próxima iteração desse laço, repetimos o processo descrito no parágrafo anterior, pois temos que o nó apontado por~\varname{tmp} é filho esquerdo de seu pai.
Já na próxima iteração, fazemos as operações simétricas, pois teremos que o nó apontado por \varname{tmp} é filho direito de seu pai.
Ao término dessa iteração, obtemos as treaps da Figura~\ref{fig:treap-split-final}.

\begin{figure}[htb]
\centering
\scalebox{.7}{
\begin{tikzpicture}[line cap=round,line join=round,>=triangle 45,x=1cm,y=1cm]
\begin{axis}[
x=1cm,y=1cm,
axis lines=middle,
ymajorgrids=true,
xmajorgrids=true,
xmin=0,
xmax=10.5,
ymin=-.5,
ymax=6.5,
xtick={-4,-3,...,21},
ytick={-3,-2,...,12},]
\clip(-4.171746300211414,-3.220558139534887) rectangle (21.671805496828764,12.052169133192377);
\draw [line width=2pt] (4,6)-- (2,2);
\draw [line width=2pt] (2,2)-- (1,0);
\draw [line width=2pt] (2,2)-- (3,1);
\draw [line width=2pt] (4,6)-- (5,1);
\draw [line width=2pt] (10,5)-- (8,4);
\draw [line width=2pt] (7,2)-- (8,4);
\draw [line width=2pt] (8,4)-- (9.01218604651163,1.486418604651162);
\begin{scriptsize}
\draw [fill=black] (1,0) circle (2.5pt);
\draw [fill=black] (2,2) circle (2.5pt);
\draw [fill=black] (3,1) circle (2.5pt);
\draw [fill=black] (4,6) circle (2.5pt);
\draw [fill=black] (5,1) circle (2.5pt);
\draw [fill=black] (6,3) circle (2.5pt);
\draw [fill=black] (7,2) circle (2.5pt);
\draw [fill=black] (8,4) circle (2.5pt);
\draw [fill=black] (9.01218604651163,1.486418604651162) circle (2.5pt);
\draw [fill=black] (10,5) circle (2.5pt);
\draw (6,3) node[anchor=south west] {$X$};
\draw (4,6) node[anchor=south] {$L$};
\draw (10,5) node[anchor=south] {$R$};
\end{scriptsize}
\end{axis}
\end{tikzpicture}

}
\caption{Resultado de~\treapSplit(X).}
\label{fig:treap-split-final}
\end{figure}

Notemos que \treapSplit{} consiste em um laço que percorre os nós de uma treap até sua raiz e em cada nó realiza operações de custo constante, logo seu consumo de tempo é proporcional à altura da treap e, como essa é balanceada, temos que seu consumo esperado é $\O{\lg n}$.


\chapter{Conexidade em grafos dinâmicos}
\label{sec:connDG}
Retomemos o problema de conexidade em grafos dinâmicos inicialmente apresentado na Seção~\ref{sec:Motivação} e que é um dos problemas principais que vamos estudar. Ele consiste na busca por uma implementação tão eficiente quanto possível para a seguinte biblioteca: 
\begin{itemize}
\item \dymGraphCreate($n$): cria um grafo dinâmico com $n$ vértices isolados;
\item \dymGraphAddEdge($G$, $u$, $v$): adiciona a aresta $uv$ ao grafo dinâmico $G$;
\item \dymGraphDelEdge($G$, $u$, $v$): remove a aresta $uv$ de $G$; e
\item \dymGraphQuery($G$, $u$, $v$): retorna verdadeiro se $u$ e $v$ estão na mesma componente conexa de $G$ e falso, caso contrário.
\end{itemize}

Note que dois vértices estão na mesma componente conexa de~$G$ se e somente se estão na mesma componente de alguma floresta maximal de~$G$, pois, por definição, por ser maximal, não é possível haver um caminho entre dois vértices em~$G$ de forma que não haja um caminho entre esses mesmo dois vértices na floresta maximal. Dessa forma, a ideia que usaremos para responder a consulta de conexidade em nosso grafo dinâmico é manter, ao longo da sequência de inserções e remoções de arestas, uma floresta dinâmica~$F$ que seja maximal em~$G$ e quando ocorrer a consulta \dymGraphQuery{}, chamamos a consulta de conexidade na floresta~$F$.

A rotina \dymGraphCreate{} retorna um grafo composto por $n$ vértices isolados. Esse grafo já é uma floresta maximal.

Em uma chamada~\dymGraphAddEdge($G$, $u$, $v$), primeiro testamos a conexidade de $u$ com $v$. Se esses vértices não estiverem conectados em~$G$, então inserimos a aresta $uv$ na floresta maximal que estamos mantendo, assim ligando as duas árvores que contém~$u$ e~$v$ nessa floresta. Essas arestas serão chamadas de \defi{arestas da floresta} ou \defi{arestas titulares}.

Já no caso em que $u$ e $v$ estiverem conectados em~$G$, então a aresta sobressalente~$uv$ será chamada de \defi{aresta reserva} e iremos armazená-la em uma estrutura de dados auxiliar~$R$ dada por listas de adjacências. Para manusear~$R$, usamos a seguinte biblioteca:
\begin{itemize}
    \item \graphCreate($n$): cria e devolve um grafo mantido com listas de adjacências com $n$ vértices isolados.
    \item \graphAdd($G$, $u$, $v$): adiciona $u$ na lista de adjacências de $v$ em $G$ e vice-versa.
    \item \graphDel($G$, $u$, $v$): remove $u$ da lista de adjacências de $v$ em $G$ e vice-versa.
\end{itemize}

 Vimos, na Seção~\ref{sec:connDF}, que a inserção de arestas em uma floresta dinâmica custa tempo~$\O{\lg n}$. Para implementar as listas de adjacências, usaremos tabelas de símbolos que permitem inserção em tempo esperado~$\O{1}$. Assim o custo de~\dymGraphAddEdge{} é~$\O{\lg n}$.

Para remover uma aresta reserva, basta remove-la das listas de adjacências, que também será feito em tempo esperado~$\O{1}$. Já remover uma aresta~$uv$ da floresta é demasiadamente mais complexo, pois sua remoção gera duas árvores~$T_u$ e~$T_v$, que contêm os vértices~$u$ e~$v$, respectivamente. Esse cenário pode ser visto na Figura~\ref{fig:DG-exemploTu-Tv}. Para manter a propriedade de~$F$ ser maximal em~$G$, é necessário verificar se existe alguma aresta reserva em~$R$ que liga~$T_u$ a~$T_v$. Tal aresta é chamada de \defi{aresta substituta}.

Sem perda de generalidade, podemos supor que $|T_u|\leqslant |T_v|$. Então, para encontrar uma aresta substituta se tal aresta exista, podemos percorrer cada vértice $t$ de $T_u$ verificando se existe algum vértice $w$ na lista de adjacências de $t$ que não seja vértice de $T_u$. Caso $w$ não seja vértice de $T_u$, como $F$ era maximal, teremos que $w$ necessariamente é um vértice de $T_v$, assim a aresta $tw$ é uma substituta para a aresta $uv$ que foi removida.

\begin{figure}[htb]
%\scalebox{.6}{
\centering
\begin{tikzpicture}[line cap=round,line join=round,>=triangle 45,x=1cm,y=1cm]
\clip(0,-.3) rectangle (10,4.2);
\draw [shift={(2.9763805811410124,1.951267892855442)},line width=1pt,dash pattern=on 1pt off 5pt]  plot[domain=-4.559320170779205:1.417727517189411,variable=\t]({1*1.9514108400425971*cos(\t r)+0*1.9514108400425971*sin(\t r)},{0*1.9514108400425971*cos(\t r)+1*1.9514108400425971*sin(\t r)});
\draw [line width=1pt,color=ccqqqq] (2,1)-- (2,3);
\draw [line width=1pt,color=ccqqqq] (2,3)-- (4,3);
\draw [line width=1pt,color=ccqqqq] (4,3)-- (4,1);
\draw [line width=1pt] (4,1)-- (2,3);
\draw [line width=1pt] (4,1)-- (2,1);
\draw [line width=1pt,color=ccqqqq] (7,1)-- (7,2);
\draw [line width=1pt,color=ccqqqq] (7,2)-- (7,3);
\draw [line width=1pt,color=ccqqqq] (7,3)-- (9,3);
\draw [line width=1pt,color=ccqqqq] (9,3)-- (9,1);
\draw [line width=1pt,dash pattern=on 1pt off 2pt,color=ccqqqq] (4,3)-- (7,2);
\draw [line width=1pt] (4,3)-- (7,1);
\draw [line width=1pt] (4,3)-- (7,3);
\draw [line width=1pt] (4,1)-- (7,2);
\draw [shift={(7.960093054094084,1.9512678928554428)},line width=1pt,dash pattern=on 1pt off 5pt]  plot[domain=-4.559320170779203:1.417727517189412,variable=\t]({1*1.951410840042597*cos(\t r)+0*1.951410840042597*sin(\t r)},{0*1.951410840042597*cos(\t r)+1*1.951410840042597*sin(\t r)});
\draw (2.65,4.2) node[anchor=north west] {$T_u$};
\draw (7.65,4.2) node[anchor=north west] {$T_v$};
\draw (3.7,3.5) node[anchor=north west] {$u$};
\draw (7,2.3) node[anchor=north west] {$v$};
\draw (7,-0.7) node[anchor=north west] {$v$};
\begin{scriptsize}
\draw [fill=black] (2,1) circle (1.5pt);
\draw [fill=black] (2,3) circle (1.5pt);
\draw [fill=black] (4,3) circle (1.5pt);
\draw [fill=black] (4,1) circle (1.5pt);
\draw [fill=black] (7,1) circle (1.5pt);
\draw [fill=black] (7,2) circle (1.5pt);
\draw [fill=black] (7,3) circle (1.5pt);
\draw [fill=black] (9,1) circle (1.5pt);
\draw [fill=black] (9,3) circle (1.5pt);
\end{scriptsize}
\end{tikzpicture}%}
\caption{Exemplo de grafo dinâmico com as arestas da árvore~$T$ coloridas de vermelho, enquanto que as arestas reservas estão pintadas de preto. A aresta~$uv$ removida está pontilhada.}
\label{fig:DG-exemploTu-Tv}
\end{figure}

Para testar rapidamente se~$t$ e~$w$ estão na mesma árvore em~$F$ depois da remoção de~$uv$, basta acionar a rotina \dymForestQuery($F$, $t$, $w$), que vimos que consome tempo esperado~$\O{\lg n}$. A rotina~\dymForestQuery{}, no pior dos casos, em que não há aresta substituta, será chamada~$\Theta(n^2)$ vezes, o que implica em um consumo esperado de tempo de~$\O{n^2\lg n}$ para o algoritmo implementado dessa forma.

Notemos que somente a remoção de arestas de~$F$ é lenta, isso se deve à necessidade de busca por uma aresta substituta. Para obter uma implementação melhor, é necessário reduzir o número de arestas testadas para encontrar uma substituta. A técnica que apresentaremos no restante dessa seção, deve-se a Jacob Holm, Kristian De Lichtenberg e Mikkel Thorup~\cite{poly_log} e tem como objetivo a redução deste número de forma amortizada. Após uma rápida introdução de notação, poderemos implementar \dymGraphCreate{}, \dymGraphAddEdge{} e~\dymGraphQuery{} como descrito anteriormente, sendo necessários mais artifícios para obtermos uma implementação eficiente de~\dymGraphDelEdge{}.


\section{Fatiamento do grafo em níveis}
Cada aresta do grafo possuirá um \defi{nível} entre $1$ e $\lceil \lg n \rceil$, onde $n$ é o número de vértices do grafo. O valor inicial do nível de uma aresta recém inserida é $\lceil \lg n \rceil$ e é decrementado toda vez que percorremos a aresta em busca de uma aresta substituta. O nível de uma aresta nunca aumentá, somente diminuirá. 

Dado um grafo $G$, podemos definir o grafo~$G_{\leqslant i}$ como o grafo induzido pelas arestas de nível menor ou igual a~$i$. Para cada nível $i$, manteremos uma floresta maximal~$F_{\leqslant i}$ de~$G_{\leqslant i}$ e o grafo~$R_i$, guardado com listas de adjacências, dado pelas arestas reservas de nível~$i$. Trivialmente temos que $G = G_{\leqslant \lceil \lg n \rceil}$ e portanto $F_{\leqslant \lceil \lg n \rceil}$ é uma floresta maximal de~$G$. Ao longo da sequência de modificações realizadas no grafo, manteremos algumas invariantes importantes:
\begin{enumerate}[label=(\roman*)]
    \item $F_{\leqslant i}$ é uma floresta maximal de~$G_{\leqslant i}$ para todo $1\leqslant i \leqslant \lceil \lg n \rceil$; e\label{invar:SF}
    \item $F_{\leqslant i}\subseteq F_{\leqslant i+1}$, para todo $1\leqslant i \leqslant \lceil \lg n \rceil-1$; \label{invar:contida}
    \item Cada componente de $F_{\leqslant i}$ possui no máximo $2^i$ vértices.\label{invar:tamanho}
\end{enumerate}

Para cada rotina da biblioteca de grafos dinâmicos, vamos primeiramente descrever o que a rotina faz, em seguida vamos mostrar que tal rotina preserva as invariantes acima e por fim calcular seu consumo de tempo.

\section{Implementação}
\subsection{Criação de um grafo dinâmico}

A implementação de \dymGraphCreate{} pode ser vista no Algoritmo~\ref{Algo:dymGraphCreate}. Nessa implementação, simplesmente inicializamos cada $F_{\leqslant i}$ e~$R_i$ de~$G$ usando \dymForestCreate{} e~\graphCreate{}, respectivamente. Notemos que essas rotinas retornam a floresta e o grafo dados por~$n$ vértices isolados, dessa forma~$F_{\leqslant \lceil \lg n \rceil}$ e~$R_{\lceil \lg n \rceil}$ juntos representam um grafo dinâmico vazio, que é exatamente o grafo que queremos construir com \dymGraphCreate{}. Além disso, será útil obter o nível de uma aresta $uv$ em tempo esperado constante, assim manteremos um dicionário $G$.\nivel{} que relaciona cada aresta a seu nível. Para manusear esse dicionário, usaremos uma biblioteca análoga à usada na Seção~\ref{sec:impleDF-ETT}, que \NEW{relembramos a seguir. Quando for claro, referenciaremos esse dicionario no pseudo-código somente por \nivel:
\begin{itemize}
    \item $\nivel~\gets~\hashCreate(n^2)$: cria e retorna um dicionário~\nivel{} capaz de armazenar $n^2$ valores;
    \item $\nivel[u,v]~\gets~i$: associa o nível~$i$ à aresta $uv$. Se já houver um nível associado a~$uv$, então esse valor é substituído por~$i$;
    \item $var~\gets~\nivel[u,v]$: atribui o nível da aresta $uv$ à variável $var$.
\end{itemize}}

\begin{algorithm}
\caption{\dymGraphCreate($n$)}
\label{Algo:dymGraphCreate}
\begin{algorithmic}[1]
\For { $i$ $\gets$ 1 até $\lceil \lg n \rceil$}
\State $G.F_{\leqslant i} \gets$ \dymForestCreate($n$)
\State $G.R_i \gets$ \graphCreate($n$)
\EndFor
\State $G$.\nivel{} $\gets$ \hashCreate($n^2$)
\State \Return $G$ 
\end{algorithmic}
\end{algorithm}


Notemos que as invariantes são mantidas por~\dymGraphCreate{}, pois o grafo gerado por essa rotina são~$n$ vértices isolados. Assumindo que o consumo de tempo de~\graphCreate{} seja~$\O{n}$, então o consumo de tempo dessa rotina será $\O{n\lg n}$.

\subsection{Consulta de conexidade}

Para realizar a rotina \dymGraphQuery{}, apresentada no Algoritmo~\ref{Algo:dymGraphQuery}, somente retornamos a resposta da consulta de conexidade feita em~$F_{\lceil \lg n \rceil}$. A corretude desse algoritmo se deve ao invariante~\ref{invar:SF}, pois esse invariante garante que~$F_{\lceil \lg n \rceil}$ é uma floresta maximal de~$G$, logo consultas de conectividade entre os vértices~$u$ e~$v$ em~$G$ e em~$F_{\lceil \lg n \rceil}$ devem possuir a mesma resposta.

\begin{algorithm}
\caption{\dymGraphQuery($G$, $u$, $v$)}
\label{Algo:dymGraphQuery}
\begin{algorithmic}[1]
\State \Return \dymForestQuery($G$.$F_{\leqslant\lceil \lg n \rceil}$, $u$, $v$)
\end{algorithmic}
\end{algorithm}

A rotina \dymGraphQuery{} claramente não interfere nas invariantes, já que  é somente uma consulta que não modifica as estruturas de dados do grafo.

Como o consumo esperado de \dymForestQuery{} é~$\O{\lg n}$, é imediato ver que o consumo de tempo esperado de \dymGraphQuery{} também é~$\O{\lg n}$.

\subsection{Adição de arestas}

Para inserir uma nova aresta~$uv$ em~$G$ usando a rotina \dymGraphAddEdge{}, implementada no Algoritmo~\ref{Algo:dymGraphAddEdge}, primeiro verificamos se os vértices~$u$ e~$v$ estão conectados em~$G$ usando a rotina \dymForestQuery($G.F_{\leqslant \lceil \lg n \rceil}$, $u$, $v$). Como já comentamos, garantimos que esse teste funciona graças ao invariante~\ref{invar:SF}. Caso~$u$ e~$v$ estiverem conectados em~$G$, então a aresta $uv$ é uma aresta reserva e será inserida em $R_{\lceil \lg n \rceil}$. Caso~$u$ e~$v$ não estiverem conectados, então a aresta $uv$ deve ser inserida em~$F_{\leqslant \lceil \lg n \rceil}$.

\begin{algorithm}
\caption{\dymGraphAddEdge($G$, $u$, $v$)}
\label{Algo:dymGraphAddEdge}
\begin{algorithmic}[1]
\State \nivel[$u$,$v$] $\gets$ $\lceil \lg n \rceil$
\If {\dymForestQuery($G.F_{\leqslant\lceil \lg n \rceil}$, $u$, $v$)}
\State \graphAdd($G$.$R_{\lceil \lg n \rceil}$, $u$, $v$)
\Else 
\State \dymForestAddEdge($G.F_{\leqslant\lceil \lg n \rceil}$, $u$, $v$)
\EndIf
\end{algorithmic}
\end{algorithm}

Note que a invariante~\ref{invar:SF} é mantida para $i = \lceil \lg n \rceil$ e as demais invariantes também se mantêm, pois somente o nível $\lceil \lg n \rceil$ da nossa estrutura de dados foi modificado, já que a nova aresta é inserida no nível~$\lceil \lg n \rceil$. 

O custo esperado de tempo de \dymForestQuery{} e \dymGraphAddEdge{} são~$\O{\lg n}$ e o custo esperado de \graphAdd{} é~$\O{1}$. Logo o custo esperado de \dymGraphAddEdge{} também é~$\O{\lg n}$.

\subsection{Remoção de arestas}

\newcommand{\ceil}[1]{\lceil{#1}\rceil}

A complexidade da remoção de uma aresta em um grafo dinâmico vem da busca por uma aresta substituta para a aresta removida. Podemos encapsular essa busca em uma rotina própria chamada \defi{\dymGraphReplace{}}. A rotina \dymGraphReplace{} recebe um grafo dinâmico~$G$, um inteiro $i$ com $1 \leq i \leq \ceil{\lg n}$, e dois vértices $u$ e $v$, extremidades de uma aresta de nível $i$ que acabou de ser removida de $G$ e das florestas~$F_{\leqslant j}$ para~$j \geq i$, e encontra, caso exista, uma aresta substituta em~$G$ com nível mínimo, e a insere na floresta deste nível e nas de nível acima. Dessa forma, a implementação de \dymGraphDelEdge{}, descrita no Algoritmo~\ref{Algo:dymGraphDelEdge}, pode ser feita em poucas linhas.

\begin{algorithm}
\caption{\dymGraphDelEdge($G$, $u$, $v$)}
\label{Algo:dymGraphDelEdge}
\begin{algorithmic}[1]
\State $i$ $\gets$ \nivel[$u,v$]
\If {$uv$ $\in G.F_{\leqslant\lceil \lg n \rceil}$}\label{Algo:dymGraphDelEdge:linha:if}
\For {$j$ $\gets$ $i$ até $\lceil \lg n \rceil$}\label{linha2}
\State \dymForestDelEdge($G$.$F_j$, $u$, $v$))
\EndFor
\State \dymGraphReplace($G$, $u$, $v$, $i$)
\Else
  \State \graphDel($G$.$R_i$, $u$, $v$)\label{Algo:dymGraphDelEdge:linha:removeLA}
\EndIf
\end{algorithmic}
\end{algorithm}

Para remover uma aresta~$uv$ de nível~$i$ de~$G$, primeiro precisamos determinar se ela pertence à floresta $F_{\leqslant\lceil \lg n \rceil}$ ou não, o que é feito na linha~\ref{Algo:dymGraphDelEdge:linha:if}. Para tal, o teste da linha~\ref{Algo:dymGraphDelEdge:linha:if} consulta se a tabela de símbolos $F_{\leqslant\lceil \lg n \rceil}$.\dymForestHash{} possui alguma entrada associada à chave~$(u,v)$. Caso a aresta $uv$ não seja titular, 
ela é uma aresta de~$R_i$ e somente a removemos de~$R_i$, o que é feito na linha~\ref{Algo:dymGraphDelEdge:linha:removeLA}.
Caso $uv$ seja titular, então precisamos removê-la de todos os níveis em que ela está presente.  Devido ao invariante~\ref{invar:contida}, sabemos que $uv$ está nos níveis de $i$ até~$\lceil \lg n \rceil$, então essa remoção é feita no laço da linha~\ref{linha2}.  
A rotina \dymGraphReplace{} faz as alterações devidas à busca de uma aresta substituta, e sua eventual inclusão nas florestas apropriadas de forma a manter os invariantes.
Veremos que o \dymGraphReplace{} garante que cada $F_{\leqslant i}$ é maximal em~$G_{\leqslant i}$, isto é, preserva a invariante~\ref{invar:SF} e preserva também as invariantes~\ref{invar:contida} e~\ref{invar:tamanho}.

O laço da linha~\ref{linha2} terá custo esperado~$\O{\lg^2 n}$, pois executa no pior dos casos $\lceil \lg n \rceil$ vezes a rotina \dymForestDelEdge{}, que possui custo esperado~$\O{\lg n}$. Veremos que \dymGraphReplace{}, descrito no Algoritmo~\ref{Algo:dymGraphReplace}, possui custo amortizado~$\O{\lg n^2}$. Assim concluímos que o custo de tempo esperado amortizado de \dymGraphDelEdge{} será~$\O{\lg^2 n}$.

\medskip

Para entender melhor o funcionamento do algoritmo~\dymGraphReplace{}, descrito no Algoritmo~\ref{Algo:dymGraphReplace}, vamos retomar o processo de busca de uma aresta substituta que foi comentado no início dessa seção, mas agora agregando a estrutura de níveis e as invariantes que definimos. Para tal, ilustramos, na Figura~\ref{fig:DG-antes-de-rebaixar}, o grafo da Figura~\ref{fig:DG-exemploTu-Tv} com a estrutura de níveis e com a aresta $uv$ já removida. Consideramos que todas as arestas desse grafo são de nível $i$ e também que não há aresta de nível~$i-1$ imediatamente antes da chamada de \dymGraphReplace{}.
\begin{figure}[htb]
%\scalebox{.6}{
\centering
\begin{tikzpicture}[line cap=round,line join=round,>=triangle 45,x=1cm,y=1cm]
\clip(-.5,-2.1) rectangle (9.1,3.4);
\draw [line width=1pt,color=ccqqqq] (2,1)-- (2,3);
\draw [line width=1pt,color=ccqqqq] (2,3)-- (4,3);
\draw [line width=1pt,color=ccqqqq] (4,3)-- (4,1);
\draw [line width=1pt] (4,1)-- (2,3);
\draw [line width=1pt] (4,1)-- (2,1);
\draw [line width=1pt,color=ccqqqq] (7,1)-- (7,2);
\draw [line width=1pt,color=ccqqqq] (7,2)-- (7,3);
\draw [line width=1pt,color=ccqqqq] (7,3)-- (9,3);
\draw [line width=1pt,color=ccqqqq] (9,3)-- (9,1);
\draw [line width=1pt] (4,3)-- (7,1);
\draw [line width=1pt] (4,3)-- (7,3);
\draw [line width=1pt] (4,1)-- (7,2);
\draw (3.7,3.5) node[anchor=north west] {$u$};
\draw (7,2.3) node[anchor=north west] {$v$};
\draw (-.5,2.1264069182918597) node[anchor=north west] {nível $i$};
\draw (-.5,-.7) node[anchor=north west] {nível $i-1$};
\begin{scriptsize}
\draw [fill=black] (2,1) circle (1.5pt);
\draw [fill=black] (2,3) circle (1.5pt);
\draw [fill=black] (4,3) circle (1.5pt);
\draw [fill=black] (4,1) circle (1.5pt);
\draw [fill=black] (7,1) circle (1.5pt);
\draw [fill=black] (7,2) circle (1.5pt);
\draw [fill=black] (7,3) circle (1.5pt);
\draw [fill=black] (9,1) circle (1.5pt);
\draw [fill=black] (9,3) circle (1.5pt);
\draw [fill=black] (2,0) circle (1.5pt);
\draw [fill=black] (2,-2) circle (1.5pt);
\draw [fill=black] (4,0) circle (1.5pt);
\draw [fill=black] (4,-2) circle (1.5pt);
\draw [fill=black] (7,0) circle (1.5pt);
\draw [fill=black] (7,-1) circle (1.5pt);
\draw [fill=black] (7,-2) circle (1.5pt);
\draw [fill=black] (9,0) circle (1.5pt);
\draw [fill=black] (9,-2) circle (1.5pt);
\end{scriptsize}
\end{tikzpicture}%}
\caption{Grafo da Figura~\ref{fig:DG-exemploTu-Tv} imerso no nível~$i$ com aresta $uv$ removida.}
\label{fig:DG-antes-de-rebaixar}
\end{figure}

\begin{algorithm}
\caption{\dymGraphReplace($G$, $u$, $v$, $niv$)}
\label{Algo:dymGraphReplace}
\begin{algorithmic}[1]
\For {$i$ $\gets$ $niv$ até $\lceil \lg n \rceil$}\label{Algo:dymGraphReplace:linha:primeira}
\State $T_v$ $\gets$  \treapGetRoot($F_i[v,v]$)
\State $T_u$ $\gets$  \treapGetRoot($F_i[u,u]$)
\If {$\treapGetSize(T_v) < \treapGetSize(T_u)$}\Comment{Garantimos que $|T_v|\geqslant |T_u|$}
\State $u$ $\leftrightarrow$ $v$
\State $T_u \leftrightarrow T_v$
\EndIf
\For {$xy$ em $T_u$ com nível = $i$}\label{Algo:dymGraphReplace:linha:moveTu}\Comment{Move $T_u$ para o nível $i-1$}
\State \nivel$[x,y]$ $\gets$ $i-1$
\State \dymForestAddEdge($G$.$F_{i-1}$, $x$, $y$) 
\EndFor
\For {$xy$ em $G$.$R_i$ com $x$ em $T_u$}\label{Algo:dymGraphReplace:linha:achaSub}\Comment{Procura substituta para $uv$}
\State \graphDel($G$.$R_i$, $x$, $y$)
\If {$y \in T_v$}
\For {$j \gets i$ até $\lceil \lg n \rceil$}\label{Algo:dymGraphReplace:linha:inseresub}
\State \dymForestAddEdge($G$.$F_j$, $x$, $y$)
\EndFor
\State \Return
\Else
\State \nivel$[x,y]$ $\gets$ $i-1$
\State \graphAdd($G$.$R_{i-1}$, $x$, $y$)
\EndIf
\EndFor
\EndFor\label{Algo:dymGraphReplace:linha:ultima}
\end{algorithmic}
\end{algorithm}


No que segue, $|T|$ denota o número de vértices numa árvore~$T$.
Após a remoção de uma aresta $uv$ de nível $i$ de uma componente $T$ da floresta $F_{\leqslant i}$, a árvore $T$ é dividida em duas árvores, $T_u$ e $T_v$, que contém $u$ e $v$, respectivamente. Podemos supor que~$|T_u|\leqslant |T_v|$. Pela invariante~\ref{invar:tamanho}, vale que $|T| \leq 2^i$ e, como ${|T_u| + |T_v| = |T|}$, concluímos que $|T_u| \leq 2^{i-1}$. Logo podemos mover todas as arestas de nível~$i$ de~$T_u$ para o nível $i-1$ sem infringir o invariante~\ref{invar:tamanho} para $i-1$.  Esse rebaixamento é feito no laço da linha~\ref{Algo:dymGraphReplace:linha:moveTu} e ilustramos a estrutura resultante dele na Figura~\ref{fig:DG-depois-de-rebaixar}.

\begin{figure}[htb]
%\scalebox{.6}{
\centering
\begin{tikzpicture}[line cap=round,line join=round,>=triangle 45,x=1cm,y=1cm]
\clip(-.5,-2.1) rectangle (9.1,3.4);
\draw [line width=1pt,color=ccqqqq] (2,1)-- (2,3);
\draw [line width=1pt,color=ccqqqq] (2,3)-- (4,3);
\draw [line width=1pt,color=ccqqqq] (4,3)-- (4,1);
\draw [line width=1pt] (4,1)-- (2,3);
\draw [line width=1pt] (4,1)-- (2,1);
\draw [line width=1pt,color=ccqqqq] (7,1)-- (7,2);
\draw [line width=1pt,color=ccqqqq] (7,2)-- (7,3);
\draw [line width=1pt,color=ccqqqq] (7,3)-- (9,3);
\draw [line width=1pt,color=ccqqqq] (9,3)-- (9,1);
%\draw [line width=1pt,dash pattern=on 1pt off 2pt,color=ccqqqq] (4,3)-- (7,2);
\draw [line width=1pt] (4,3)-- (7,1);
\draw [line width=1pt] (4,3)-- (7,3);
\draw [line width=1pt] (4,1)-- (7,2);
\draw (3.7,3.5) node[anchor=north west] {$u$};
\draw (7,2.3) node[anchor=north west] {$v$};
\draw (3.7,0.5) node[anchor=north west] {$u$};
\draw (7,-0.7) node[anchor=north west] {$v$};
\draw (-.5,2.1264069182918597) node[anchor=north west] {nível $i$};
\draw (-.5,-.7) node[anchor=north west] {nível $i-1$};

\draw [line width=1pt,color=ccqqqq] (4,0)-- (2,0);
\draw [line width=1pt,color=ccqqqq] (4,0)-- (4,-2);
\draw [line width=1pt,color=ccqqqq] (2,0)-- (2,-2);

\begin{scriptsize}
\draw [fill=black] (2,1) circle (1.5pt);
\draw [fill=black] (2,3) circle (1.5pt);
\draw [fill=black] (4,3) circle (1.5pt);
\draw [fill=black] (4,1) circle (1.5pt);
\draw [fill=black] (7,1) circle (1.5pt);
\draw [fill=black] (7,2) circle (1.5pt);
\draw [fill=black] (7,3) circle (1.5pt);
\draw [fill=black] (9,1) circle (1.5pt);
\draw [fill=black] (9,3) circle (1.5pt);
\draw [fill=black] (2,0) circle (1.5pt);
\draw [fill=black] (2,-2) circle (1.5pt);
\draw [fill=black] (4,0) circle (1.5pt);
\draw [fill=black] (4,-2) circle (1.5pt);
\draw [fill=black] (7,0) circle (1.5pt);
\draw [fill=black] (7,-1) circle (1.5pt);
\draw [fill=black] (7,-2) circle (1.5pt);
\draw [fill=black] (9,0) circle (1.5pt);
\draw [fill=black] (9,-2) circle (1.5pt);
\end{scriptsize}
\end{tikzpicture}%}
\caption{Grafo dinâmico após o rebaixamento de $T_u$.}
\label{fig:DG-depois-de-rebaixar}
\end{figure}

Agora percorreremos as arestas reservas em busca de uma substituta. Notemos que, como consequência das invariantes~\ref{invar:SF} e~\ref{invar:contida}, temos que se há uma aresta substituta para~$uv$, então seu nível é maior ou igual a~$i$. Provaremos esse fato por contradição, supondo que exista uma aresta~$xy$ substituta a~$uv$ com nível $j<i$ e, sem perda de generalidade, supondo que~$x$ é um vértice de~$T_u$ e~$y$ de~$T_v$. É imediato notar que $xy$ é uma aresta de $G_{\leqslant j}$, mas que não é uma aresta de~$F_{\leqslant j}$, pois, sendo uma aresta reserva, $xy\notin F_{\leqslant \lceil \lg n \rceil}$ e como~$F_{\leqslant j} \subseteq F_{\leqslant \lceil \lg n \rceil}$  pela invariante~\ref{invar:contida}, temos que~$xy\notin F_{\leqslant j}$. Como $xy$ é uma aresta de $G_{\leqslant j}$, temos que $x$ e $y$ estão na mesma componente de $G_{\leqslant j}$, e portanto deveriam estar na mesma componente de $F_{\leqslant j}$ pela invariante~\ref{invar:SF}. Mas isso contraria a invariante~\ref{invar:contida}, dado que $x$ e $y$ estão em componentes distintas, $T_u$ e $T_v$, de $F_{\leqslant i}$.

Percorremos as arestas reservas de nível~$i$ incidentes a~$T_u$ procurando uma aresta substituta. Essas arestas estão pintadas de preto na Figura~\ref{fig:DG-depois-de-rebaixar}. Cada aresta percorrida desta forma e que não seja uma substituta de~$uv$ tem seus dois extremos em $T_u$ e também será rebaixada.
Rebaixamos tais arestas de nível, pois elas deixam de ser candidatas a substitutas de arestas de nível~$i$, uma vez que rebaixamos as arestas de~$T_u$ de nível $i$ para nível $i-1$. 
Ademais esse rebaixamento não infringe o invariante~\ref{invar:SF}.
Caso encontremos uma substituta no nível~$i$, então a inserimos nas florestas dos níveis~$i$ até~$\lceil \lg n \rceil$. Caso não encontremos uma substituta no nível~$i$, teremos rebaixado para o nível $i-1$ todas as arestas incidentes a $T_u$, e repetimos a busca no nível~$i+1$, eventualmente rebaixando arestas de nível $i+1$ para o nível $i$, até encontrarmos uma aresta substituta ou terminarmos a busca no nível~$\lceil \lg n \rceil$. Podemos ver o resultado desses rebaixamentos na Figura~\ref{fig:DG-depois-achou-sub}.
\begin{figure}[htb]
%\scalebox{.6}{
\centering
\begin{tikzpicture}[line cap=round,line join=round,>=triangle 45,x=1cm,y=1cm]
\clip(-.5,-2.1) rectangle (9.5,3.4);
\draw [line width=1pt,color=ccqqqq] (2,1)-- (2,3);
\draw [line width=1pt,color=ccqqqq] (2,3)-- (4,3);
\draw [line width=1pt,color=ccqqqq] (4,3)-- (4,1);
%\draw [line width=1pt] (4,1)-- (2,3);
%\draw [line width=1pt] (4,1)-- (2,1);
\draw [line width=1pt,color=ccqqqq] (7,1)-- (7,2);
\draw [line width=1pt,color=ccqqqq] (7,2)-- (7,3);
\draw [line width=1pt,color=ccqqqq] (7,3)-- (9,3);
\draw [line width=1pt,color=ccqqqq] (9,3)-- (9,1);
%\draw [line width=1pt,dash pattern=on 1pt off 2pt,color=ccqqqq] (4,3)-- (7,2);
\draw [line width=1pt,color=ccqqqq] (4,3)-- (7,1);
\draw [line width=1pt] (4,3)-- (7,3);
\draw [line width=1pt] (4,1)-- (7,2);
\draw (3.6,3.5) node[anchor=north west] {$u$};
\draw (7,2.3) node[anchor=north west] {$v$};
\draw (1.7,3.5) node[anchor=north west] {$t$};
\draw (1.6,1) node[anchor=north west] {$r$};
\draw (7,1) node[anchor=north west] {$w$};

\draw (3.6,1) node[anchor=north west] {$s$};
\draw (6.9,3.5) node[anchor=north west] {$x$};
\draw (9,3.5) node[anchor=north west] {$y$};
\draw (8.9,1) node[anchor=north west] {$z$};



\draw (3.7,0.5) node[anchor=north west] {$u$};
\draw (7,-0.7) node[anchor=north west] {$v$};
\draw (-.5,2.1264069182918597) node[anchor=north west] {nível $i$};
\draw (-.5,-.7) node[anchor=north west] {nível $i-1$};

\draw [line width=1pt,color=ccqqqq] (4,0)-- (2,0);
\draw [line width=1pt,color=ccqqqq] (4,0)-- (4,-2);
\draw [line width=1pt,color=ccqqqq] (2,0)-- (2,-2);

\draw [line width=1pt] (2,0)-- (4,-2);
\draw [line width=1pt] (2,-2)-- (4,-2);

\begin{scriptsize}
\draw [fill=black] (2,1) circle (1.5pt);
\draw [fill=black] (2,3) circle (1.5pt);
\draw [fill=black] (4,3) circle (1.5pt);
\draw [fill=black] (4,1) circle (1.5pt);
\draw [fill=black] (7,1) circle (1.5pt);
\draw [fill=black] (7,2) circle (1.5pt);
\draw [fill=black] (7,3) circle (1.5pt);
\draw [fill=black] (9,1) circle (1.5pt);
\draw [fill=black] (9,3) circle (1.5pt);
\draw [fill=black] (2,0) circle (1.5pt);
\draw [fill=black] (2,-2) circle (1.5pt);
\draw [fill=black] (4,0) circle (1.5pt);
\draw [fill=black] (4,-2) circle (1.5pt);
\draw [fill=black] (7,0) circle (1.5pt);
\draw [fill=black] (7,-1) circle (1.5pt);
\draw [fill=black] (7,-2) circle (1.5pt);
\draw [fill=black] (9,0) circle (1.5pt);
\draw [fill=black] (9,-2) circle (1.5pt);
\end{scriptsize}
\end{tikzpicture}%}
\caption{Grafo dinâmico após encontrar uma aresta substituta para $uv$.}
\label{fig:DG-depois-achou-sub}
\end{figure}

A rotina \dymGraphReplace{} preserva cada uma das invariantes que definimos. 
Como somente rebaixamos arestas da árvore e arestas reservas cujas duas extremidades estão em~$T_u$, a componente resultante de $F_{\leqslant i-1}$ é maximal em~$G_{\leqslant i-1}$. Portanto a invariante~\ref{invar:SF} é preservada.
A invariante~\ref{invar:contida} é preservada, pois rebaixar arestas de nível mantém essa invariante e ao inserir uma aresta que foi descoberta como substituta, garantimos que ela foi inserida em todos os níveis maiores do que~$i$ (laço da linha~\ref{Algo:dymGraphReplace:linha:inseresub}).  
Como garantimos que $|T_u| \leqslant 2^{i-1}$ antes de rebaixar as arestas de $T_u$, a invariante~\ref{invar:tamanho} também é preservada.

Antes de analisar o consumo de tempo de \dymGraphReplace{}, é necessário elaborar mais alguns detalhes sobre a implementação dessa rotina. Especificamente, vamos detalhar como realizamos os laços das linhas~\ref{Algo:dymGraphReplace:linha:moveTu} e~\ref{Algo:dymGraphReplace:linha:achaSub}.

No laço da linha~\ref{Algo:dymGraphReplace:linha:moveTu}, percorremos o conjunto das arestas de nível~$i$ de~$T_u$. Para acessar esse conjunto eficientemente, adicionaremos dois novos campos aos nós da Euler Tour Tree que armazena~$T_u$. O primeiro será um campo booleano, chamado~$niv$, que valerá~$1$ somente se a aresta representada pelo nó for de nível~$i$. Caso contrário, esse campo valerá $0$. O segundo campo, chamado~$cniv$, armazena o número de arestas de nível $i$ na subárvore enraizada pelo nó. Podemos ver na Figura~\ref{fig:DG-TREAP-niv}, como exemplo, a floresta $F_{\leqslant i}$ da Figura~\ref{fig:DG-depois-achou-sub} representada por uma Euler tour tree.

\begin{figure}[htb]
\scalebox{.61}{
\centering
\begin{tikzpicture}[line cap=round,line join=round,>=triangle 45,x=1cm,y=1cm]
\clip(.25,-.52) rectangle (24.5,6);

\draw [line width=1pt] (0.79855,1) circle (0.5cm);
\draw (0.79855,1) node[anchor=center] {$uu$};

\draw [line width=1pt] (1.75,2) circle (0.5cm);
\draw (1.75,2) node[anchor=center] {$ut$};

\draw [line width=1pt] (2.7,1) circle (0.5cm);
\draw (2.7,1) node[anchor=center] {$tt$};

\draw [line width=1pt] (3.25,3) circle (0.5cm);
\draw (3.25,3) node[anchor=center] {$tr$};

\draw [line width=1pt] (3.8,1) circle (0.5cm);
\draw (3.8,1.01113) node[anchor=center] {$rr$};

\draw [line width=1pt] (4.75,2) circle (0.5cm);
\draw (4.75,2) node[anchor=center] {$rt$};

\draw [line width=1pt] (5.7,1) circle (0.5cm);
\draw (5.7,1) node[anchor=center] {$tu$};

\draw [line width=1pt] (6.25,4) circle (0.5cm);
\draw (6.25,4) node[anchor=center] {$us$};
\draw (6.25,3.3) node[anchor=center] {3};

\draw [line width=1pt] (6.8,1) circle (0.5cm);
\draw (6.8,1) node[anchor=center] {$ss$};

\draw [line width=1pt] (7.75,2) circle (0.5cm);
\draw (7.75,2) node[anchor=center] {$su$};
\draw (7.75,1.3) node[anchor=center] {1};

\draw [line width=1pt,color=ccqqqq] (8.7,1) circle (0.5cm);
\draw (8.7,1) node[anchor=center] {$uw$};
\draw (8.7,.3) node[anchor=center] {1};

\draw [line width=1pt] (9.25,3) circle (0.5cm);
\draw (9.25,3) node[anchor=center] {$ww$};
\draw (9.25,2.3) node[anchor=center] {3};

\draw [line width=1pt,color=ccqqqq] (9.8,1) circle (0.5cm);
\draw (9.8,1) node[anchor=center] {$wv$};
\draw (9.8,.3) node[anchor=center] {1};

\draw [line width=1pt] (10.75,2) circle (0.5cm);
\draw (10.75,2) node[anchor=center] {$vv$};
\draw (10.75,1.3) node[anchor=center] {2};

\draw [line width=1pt,color=ccqqqq] (11.7,1) circle (0.5cm);
\draw (11.7,1) node[anchor=center] {$vx$};
\draw (11.7,0.3) node[anchor=center] {1};

\draw [line width=1pt] (12.25,5) circle (0.5cm);
\draw (12.25,5) node[anchor=center] {$xx$};
\draw (12.25,4.3) node[anchor=center] {10};

\draw [line width=1pt,color=ccqqqq] (12.8,1) circle (0.5cm);
\draw (12.8,1) node[anchor=center] {$xy$};
\draw (12.8,0.3) node[anchor=center] {1};

\draw [line width=1pt] (13.75,2) circle (0.5cm);
\draw (13.75,2) node[anchor=center] {$yy$};
\draw (13.75,1.3) node[anchor=center] {2};

\draw [line width=1pt,color=ccqqqq] (14.7,1) circle (0.5cm);
\draw (14.7,1) node[anchor=center] {$yz$};
\draw (14.7,.3) node[anchor=center] {1};

\draw [line width=1pt] (15.25,3) circle (0.5cm);
\draw (15.25,3) node[anchor=center] {$zz$};
\draw (15.25,2.3) node[anchor=center] {3};

\draw [line width=1pt,color=ccqqqq] (16.75,2) circle (0.5cm);
\draw (16.75,2) node[anchor=center] {$zy$};
\draw (16.75,1.3) node[anchor=center] {1};

\draw [line width=1pt,color=ccqqqq] (18.25,4) circle (0.5cm);
\draw (18.25,4) node[anchor=center] {$yx$};
\draw (18.25,3.3) node[anchor=center] {7};

\draw [line width=1pt,color=ccqqqq] (18.8,1) circle (0.5cm);
\draw (18.8,1) node[anchor=center] {$xv$};
\draw (18.8,.3) node[anchor=center] {1};

\draw [line width=1pt,color=ccqqqq] (19.75,2) circle (0.5cm);
\draw (19.75,2) node[anchor=center] {$vw$};
\draw (19.75,1.3) node[anchor=center] {2};

\draw [line width=1pt,color=ccqqqq] (21.25,3) circle (0.5cm);
\draw (21.25,3) node[anchor=center] {$wu$};
\draw (21.25,2.3) node[anchor=center] {3};



\draw [line width=1pt] (2.833974852831077,2.7226499018873844)-- (2.166025147168922,2.2773500981126147);
\draw [line width=1pt] (2.0943747309573464,1.6375002832027934)-- (2.3556252690426533,1.362499716797207);
\draw [line width=1pt] (1.4053492879009357,1.6377626653013146)-- (1.1432007120990646,1.3622373346986858);
\draw [line width=1pt] (4.403604166294537,1.6394311072880825)-- (4.146395833705462,1.3716988927119165);
\draw [line width=1pt] (5.094374730957346,1.6375002832027934)-- (5.355625269042655,1.362499716797206);
\draw [line width=1pt] (3.666025147168926,2.722649901887383)-- (4.333974852831075,2.2773500981126173);
\draw [line width=1pt] (7.144374730957346,1.3624997167972064)-- (7.405625269042653,1.637500283202792);
\draw [line width=1pt] (8.09437473095735,1.6375002832027892)-- (8.355625269042648,1.3624997167972117);
\draw [line width=1pt] (8.166025147168908,2.277350098112605)-- (8.833974852831064,2.7226499018873764);
\draw [line width=1pt] (9.666025147168892,2.7226499018874044)-- (10.333974852831105,2.2773500981125965);
\draw [line width=1pt] (10.405625269042655,1.6375002832027938)-- (10.14437473095734,1.3624997167972);
\draw [line width=1pt] (6.72434164902525,3.841886116991583)-- (8.775658350974732,3.1581138830084226);
\draw [line width=1pt] (5.775658350974748,3.841886116991583)-- (3.7243416490252534,3.1581138830084177);
\draw [line width=1pt] (11.094374730957348,1.637500283202792)-- (11.355625269042662,1.3624997167971986);
\draw [line width=1pt] (6.743196961916072,4.082199493652679)-- (11.756803038083929,4.917800506347321);
\draw [line width=1pt] (12.743196961916096,4.917800506347318)-- (17.75680303808388,4.082199493652687);
\draw [line width=1pt] (17.77565835097466,3.8418861169915526)-- (15.724341649025257,3.158113883008419);
\draw [line width=1pt] (18.724341649025213,3.8418861169915965)-- (20.775658350974826,3.158113883008392);
\draw [line width=1pt] (20.83397485283108,2.7226499018873866)-- (20.166025147168863,2.277350098112576);
\draw [line width=1pt] (19.40562526904266,1.6375002832027987)-- (19.14437473095733,1.3624997167971906);
\draw [line width=1pt] (15.666025147168863,2.722649901887424)-- (16.333974852831133,2.277350098112576);
\draw [line width=1pt] (14.833974852831076,2.722649901887384)-- (14.166025147168892,2.2773500981125947);
\draw [line width=1pt] (14.094374730957343,1.637500283202797)-- (14.355625269042644,1.3624997167972168);
\draw [line width=1pt] (13.40562526904265,1.6375002832027887)-- (13.144374730957344,1.3624997167972026);
\end{tikzpicture}
}
\caption{Euler tour tree que armazena a floresta $F_{\leqslant i}$ da Figura~\ref{fig:DG-depois-achou-sub}. Os nós que representam arestas de nível~$i$ de $F_{\leqslant i}$ estão pintados de vermelho e o valor do campo $cniv$ está denotado abaixo de cada nó, quando esse campo não for nulo.}
\label{fig:DG-TREAP-niv}
\end{figure}

Com esses dois campos, podemos adicionar uma nova rotina, chamada \treapGetEdgesLevel{} e descrita no Algoritmo~\ref{Algo:treapGetEdgesLevel}, que percorre a Euler Tour Tree que armazena~$T_u$ evitando subárvores que não possuam arestas de nível~$i$ e constrói o conjunto de todas as arestas de nível~$i$ de~$T_u$.

\begin{algorithm}
\caption{\treapGetEdgesLevel($\node$, $k$)}
\label{Algo:treapGetEdgesLevel}
\begin{algorithmic}[1]
\If {$\node$ $=$ \Nil{} ou $\node$.$cniv$ $=0$}
\State \Return \Nil
\EndIf
\If {$\node.esq$ $\neq$ \Nil{} e $\node.esq$.$cniv$ $\geqslant k$}
\State  \Return \treapGetEdgesLevel($\node.esq$, $k$)
\EndIf
\If {$\node.esq$ $\neq$ \Nil{} e $\node.esq$.$cniv +1$ $=k$ e $\node.niv$}
\State  \Return $\node$
\EndIf
\State \Return \treapGetEdgesLevel($\node.dir$, $k-\node.esq$.$cniv-\node.niv$)

\end{algorithmic}
\end{algorithm}

Notemos que \treapGetEdgesLevel{} não altera as estruturas de dados, logo preserva as invariantes. Seu consumo de tempo esperado é $\O{k \lg n}$, onde $k$ é o número de arestas de nível~$i$ na árvore de $F_{\leqslant i}$ que contém o $\node$, que corresponde aos $k$ percursos da raiz da Euler Tour Tree até cada nó com campo $niv$ sendo igual a~$1$, como a Euler Tour Tree é balanceada, cada percurso possui custo de percurso~$\O{\lg n}$.

Com essa nova rotina em mãos, realizar o laço da linha~\ref{Algo:dymGraphReplace:linha:moveTu} se torna fácil, primeiro construímos esse conjunto de arestas titulares de nível~$i$ e então a percorremos, rebaixando cada aresta para o nível~$i-1$.

A técnica utilizada para o laço~\ref{Algo:dymGraphReplace:linha:achaSub} é completamente análoga. Só que agora a informação que queremos extrair da Euler Tour Tree é o conjunto de vértices incidentes à alguma aresta reserva de nível~$i$. Logo, adicionaremos dois outros novos campos aos nós das nossas Euler Tour Trees. O primeiro, chamado $res$, que é igual a~$1$ se o nó é a ocorrência ativa de um vértice $v$, isto é, contém a ocorrência~$vv$ apontada pela tabela de símbolos da floresta e $v$ é incidente alguma aresta reserva de nível~$i$, caso contrário, valerá~$0$. O segundo será um contador, chamado de~$cres$, que armazena o número de nós que satisfazem $res$ $=$ $1$ na subárvore enraizada pelo nó. Podemos ver, na Figura~\ref{fig:DG-TREAP-res}, o valor desses campos para o grafo dinâmico de nível~$i$ da Figura~\ref{fig:DG-depois-achou-sub}. A manutenção desses campos é análoga à manutenção feita nos campos $niv$ e~$cniv$.

\begin{figure}[htb]
\scalebox{.61}{
\centering
\begin{tikzpicture}[line cap=round,line join=round,>=triangle 45,x=1cm,y=1cm]
\clip(.25,-.52) rectangle (24.5,6);

\draw [line width=1pt,color=qqqqff] (0.79855,1) circle (0.5cm);
\draw (0.79855,1) node[anchor=center] {$uu$};
\draw (0.8,.3) node[anchor=center] {1};

\draw [line width=1pt] (1.75,2) circle (0.5cm);
\draw (1.75,2) node[anchor=center] {$ut$};
\draw (1.75,1.3) node[anchor=center] {1};

\draw [line width=1pt] (2.7,1) circle (0.5cm);
\draw (2.7,1) node[anchor=center] {$tt$};

\draw [line width=1pt] (3.25,3) circle (0.5cm);
\draw (3.25,3) node[anchor=center] {$tr$};
\draw (3.25,2.3) node[anchor=center] {1};

\draw [line width=1pt] (3.8,1) circle (0.5cm);
\draw (3.8,1.01113) node[anchor=center] {$rr$};

\draw [line width=1pt] (4.75,2) circle (0.5cm);
\draw (4.75,2) node[anchor=center] {$rt$};

\draw [line width=1pt] (5.7,1) circle (0.5cm);
\draw (5.7,1) node[anchor=center] {$tu$};

\draw [line width=1pt] (6.25,4) circle (0.5cm);
\draw (6.25,4) node[anchor=center] {$us$};
\draw (6.25,3.3) node[anchor=center] {3};


\draw [line width=1pt,color=qqqqff] (6.8,1) circle (0.5cm);
\draw (6.8,1) node[anchor=center] {$ss$};
\draw (6.8,.3) node[anchor=center] {1};

\draw [line width=1pt] (7.75,2) circle (0.5cm);
\draw (7.75,2) node[anchor=center] {$su$};
\draw (7.75,1.3) node[anchor=center] {1};

\draw [line width=1pt] (8.7,1) circle (0.5cm);
\draw (8.7,1) node[anchor=center] {$uw$};

\draw [line width=1pt] (9.25,3) circle (0.5cm);
\draw (9.25,3) node[anchor=center] {$ww$};
\draw (9.25,2.3) node[anchor=center] {2};

\draw [line width=1pt] (9.8,1) circle (0.5cm);
\draw (9.8,1) node[anchor=center] {$wv$};

\draw [line width=1pt,color=qqqqff] (10.75,2) circle (0.5cm);
\draw (10.75,2) node[anchor=center] {$vv$};
\draw (10.75,1.3) node[anchor=center] {1};

\draw [line width=1pt] (11.7,1) circle (0.5cm);
\draw (11.7,1) node[anchor=center] {$vx$};

\draw [line width=1pt,color=qqqqff] (12.25,5) circle (0.5cm);
\draw (12.25,5) node[anchor=center] {$xx$};
\draw (12.25,4.3) node[anchor=center] {4};

\draw [line width=1pt] (12.8,1) circle (0.5cm);
\draw (12.8,1) node[anchor=center] {$xy$};

\draw [line width=1pt] (13.75,2) circle (0.5cm);
\draw (13.75,2) node[anchor=center] {$yy$};

\draw [line width=1pt] (14.7,1) circle (0.5cm);
\draw (14.7,1) node[anchor=center] {$yz$};

\draw [line width=1pt] (15.25,3) circle (0.5cm);
\draw (15.25,3) node[anchor=center] {$zz$};

\draw [line width=1pt] (16.75,2) circle (0.5cm);
\draw (16.75,2) node[anchor=center] {$zy$};

\draw [line width=1pt] (18.25,4) circle (0.5cm);
\draw (18.25,4) node[anchor=center] {$yx$};

\draw [line width=1pt] (18.8,1) circle (0.5cm);
\draw (18.8,1) node[anchor=center] {$xv$};

\draw [line width=1pt] (19.75,2) circle (0.5cm);
\draw (19.75,2) node[anchor=center] {$vw$};

\draw [line width=1pt] (21.25,3) circle (0.5cm);
\draw (21.25,3) node[anchor=center] {$wu$};



\draw [line width=1pt] (2.833974852831077,2.7226499018873844)-- (2.166025147168922,2.2773500981126147);
\draw [line width=1pt] (2.0943747309573464,1.6375002832027934)-- (2.3556252690426533,1.362499716797207);
\draw [line width=1pt] (1.4053492879009357,1.6377626653013146)-- (1.1432007120990646,1.3622373346986858);
\draw [line width=1pt] (4.403604166294537,1.6394311072880825)-- (4.146395833705462,1.3716988927119165);
\draw [line width=1pt] (5.094374730957346,1.6375002832027934)-- (5.355625269042655,1.362499716797206);
\draw [line width=1pt] (3.666025147168926,2.722649901887383)-- (4.333974852831075,2.2773500981126173);
\draw [line width=1pt] (7.144374730957346,1.3624997167972064)-- (7.405625269042653,1.637500283202792);
\draw [line width=1pt] (8.09437473095735,1.6375002832027892)-- (8.355625269042648,1.3624997167972117);
\draw [line width=1pt] (8.166025147168908,2.277350098112605)-- (8.833974852831064,2.7226499018873764);
\draw [line width=1pt] (9.666025147168892,2.7226499018874044)-- (10.333974852831105,2.2773500981125965);
\draw [line width=1pt] (10.405625269042655,1.6375002832027938)-- (10.14437473095734,1.3624997167972);
\draw [line width=1pt] (6.72434164902525,3.841886116991583)-- (8.775658350974732,3.1581138830084226);
\draw [line width=1pt] (5.775658350974748,3.841886116991583)-- (3.7243416490252534,3.1581138830084177);
\draw [line width=1pt] (11.094374730957348,1.637500283202792)-- (11.355625269042662,1.3624997167971986);
\draw [line width=1pt] (6.743196961916072,4.082199493652679)-- (11.756803038083929,4.917800506347321);
\draw [line width=1pt] (12.743196961916096,4.917800506347318)-- (17.75680303808388,4.082199493652687);
\draw [line width=1pt] (17.77565835097466,3.8418861169915526)-- (15.724341649025257,3.158113883008419);
\draw [line width=1pt] (18.724341649025213,3.8418861169915965)-- (20.775658350974826,3.158113883008392);
\draw [line width=1pt] (20.83397485283108,2.7226499018873866)-- (20.166025147168863,2.277350098112576);
\draw [line width=1pt] (19.40562526904266,1.6375002832027987)-- (19.14437473095733,1.3624997167971906);
\draw [line width=1pt] (15.666025147168863,2.722649901887424)-- (16.333974852831133,2.277350098112576);
\draw [line width=1pt] (14.833974852831076,2.722649901887384)-- (14.166025147168892,2.2773500981125947);
\draw [line width=1pt] (14.094374730957343,1.637500283202797)-- (14.355625269042644,1.3624997167972168);
\draw [line width=1pt] (13.40562526904265,1.6375002832027887)-- (13.144374730957344,1.3624997167972026);
\end{tikzpicture}
}
\caption{Euler tour tree que armazena o grafo de nível~$i$ da Figura~\ref{fig:DG-depois-achou-sub}. Os nós que representam vértices incidentes a arestas reservas de nível~$i$ estão pintados de azul e o valor do campo~$cres$ está denotado abaixo de cada nó, quando esse campo não for nulo.}
\label{fig:DG-TREAP-res}
\end{figure}

Podemos fazer a manutenção desses novos campos sem onerar o custo das outras rotinas. Toda nova aresta inserida em algum nível possui o campo $niv$ igual a $1$, esse campo se torna $0$ somente quando rebaixamos essa aresta de nível. Toda manutenção que o campo $cniv$ pede é a sua atualização toda vez que atualizamos o campo $.tam$ das treaps. Isso se deve ao fato de atualizarmos $.tam$ quando a estrutura das subárvores de um nó terem suas estruturas modificadas, nesse caso, essa modificação de estrutura infere também na necessidade de atualização dos demais contadores. O segundo contador, $cres$, exige atualizações em situações adicionais. Toda vez que adicionamos ou removemos uma aresta reserva ou quando um nó se torna ocorrência ativa, devemos percorrer o caminho desse nó até a raiz atualizando o campo $cres$, o que consome tempo esperado~$\O{\lg n}$.


Para concluir essa seção, argumentaremos como ocorre a amortização de custo de \dymGraphDelEdge{}. Notemos que é difícil calcular o número exato de arestas percorridas em uma única execução de \dymGraphReplace{}. No entanto, podemos calcular tal número ao longo de uma sequência de remoções de arestas. Suponhamos que um grafo dinâmico possua $m$ arestas e que iremos remover todas elas uma a uma. No pior dos casos, cada aresta é rebaixada $\lceil \lg n \rceil$ vezes, percorrendo assim todos os níveis da nossa estrutura de dados, e como cada rebaixamento custa $\O{\lg n}$, teremos que o custo total de todas as remoções das $m$ arestas é $\O{m\lg^2 n}$. Logo o custo amortizado de cada remoção é $\O{\lg^2 n}$.






\section{Soluções ótimas para classes específicas de grafos}

O limitante inferior elaborado no Capítulo~\ref{sec:lim} delimita um horizonte de~$\O{\lg n}$ para as soluções do problema de conexidade dinâmica. Vimos no Capítulo~\ref{sec:connDF} que esse horizonte foi alcançado quando restringimos o problema a florestas. Também vimos que a solução ingênua para o problema geral, desenvolvida no início do Capítulo~\ref{sec:connDG}, permite inserção de aresta e consulta de conexidade em tempo ótimo.

Voltamos a destacar que a dificuldade de solucionar o problema de conexidade em grafos dinâmicos é substituir uma aresta que foi removida da MSF que mantemos para realizar a consulta de conexidade.

Em $1992$, Eppstein \textit{et al.} desenvolvem uma maneira de encontrar essa aresta substituta para uma MSF de um grafo planar em~$\O{\lg n}$~\cite{EPPSTEIN-planar}. Consequentemente obtendo assim uma solução ótima para conexidade dinâmica para essa família de grafos.



\chapter{Floresta maximal de peso mínimo em grafos planos}
\label{sec:MSF}


Retomemos o problema da floresta maximal de peso mínimo em grafos dinâmicos inicialmente apresentado na Seção~\ref{sec:Motivação} e que é o segundo problema que iremos apresentar.
Ele consiste na busca por uma implementação tão eficiente quanto possível para a seguinte biblioteca:

\begin{itemize}
\item \MSFCreate($G_0$): recebe um grafo plano ponderado~$G_0$ dado por listas de adjacências e devolve um grafo dinâmico plano ponderado isomorfo a~$G_0$; Para cada aresta~$uv$ em~$G_0$ também é informado a aresta dual~$uv^\star$, dessa forma fixando uma imersão no plano.
\item \MSFupdate($G$, $u$, $v$, $w$): atribui o peso~$w$ à aresta~$uv$ do grafo dinâmico~$G$.
\item \MSFweight($G$): devolve o peso de uma MSF de $G$.
\end{itemize}


\section{Subdivisão planar e suas representações}

Definir:
\begin{itemize}
\item grafo planar
\item imersão no plano
\item grafo plano
\item grafo dual
\item corte induzido por uma árvore e uma aresta da árvore.
\end{itemize}

\begin{theorem}
\label{teo:MSFdual}
Dado uma floresta maximal~$F$ de um grafo planar~$G$, então o conjunto
$$
F^\star = \{e^\star|e\notin F\}
$$
é uma floresta maximal de~$G^\star$.
Além disso, se~$G$ for ponderado e adotarmos $w(e) = w(e^\star)$, então~$F$ será de peso mínimo se e somente se~$T^\star$ for de peso máximo e vice versa.
\end{theorem}


Dado um grafo~$G$, um \defi{corte} é um conjunto de arestas cuja remoção aumenta o conjunto de componentes conexas de~$G$.
Dado $F$ uma floresta maximal de~$G$ e~$uv$ uma aresta de~$F$, então existe um corte associado ao par $(F, uv)$ dado pela união de $uv$ com as arestas que reconectam as duas árvores geradas pela remoção de $uv$.


\begin{theorem}
\label{teo:cutset}
Seja~$G$ um grafo planar, $F$ uma floresta maximal de~$G$ e~$uv$ uma aresta de~$F$, então o conjunto
$$
(F, uv)^\star := \{e^\star|e\in (F, uv)\}
$$
forma um ciclo em $G^\star$.
\end{theorem}

\section{Link/Cut Tree}
\label{sec:linkcuttree}

Para uma introdução sobre link-cut trees, recomendamos o trabalho de~\cite{linkcuttree}.

Biblioteca de link/cut tree:
\begin{itemize}
\item \linkcutCreate($n$): recebe um vértice e devolve um nó de link/cut tree que representa esse vértice.
\item \linkcutAddEdge($F$, $u$, $v$, $w$): Adiciona a aresta~$uv$ com peso~$w$ à floresta~$F$.
\item \linkcutDelEdge($F$, $u$, $v$): Remove a aresta~$uv$ de~$F$.
\item \linkcutQuery($F$, $u$, $v$): Realiza uma consulta de conexidade entre os vértices~$u$ e~$v$.
\item \linkcutMax($F$, $u$, $v$): Retorna o nó com peso máximo no caminho entre~$u$ e~$v$ na floresta~$F$.
\item \linkcutMin($F$, $u$, $v$): Retorna o nó com peso mínimo no caminho entre~$u$ e~$v$ na floresta~$F$.
\end{itemize}

\section{Ideia do algoritmo}
Manteremos $F$ e sua floresta dual~$F^\star$ como definida no Teorema~\ref{teo:MSFdual} usando link-cut trees.

Para implementar \MSFCreate($G_0$), primeiro ordenamos as arestas de~$G_0$ em ordem crescente de peso e em seguida as inserimos sequencialmente em~$G$.
Antes de inserir uma aresta~$uv$, faremos um teste de conexidade entre~$u$ e~$v$, se~$u$ e~$v$ não estiverem conectados, então inserimos~$uv$ em~$F$, caso contrário inserimos $uv^\star$ em~$F^\star$.

\begin{algorithm}[htb]
\caption{\MSFCreate($n$, $G_0$)}
\label{Algo:MSFCreate}
\begin{algorithmic}[1]
\State \varname{lista} $\gets$ \order($G_0$)
\For{$uv$ em \varname{lista}}:
\If \linkcutQuery($F$, $u$, $v$)
\State \linkcutAddEdge($G$.$F^\star$, $e$, $f$, $w$)
\Else
\State \linkcutAddEdge($G$.$F$, $u$, $v$, $w$)
\EndIf
\EndFor
\State \Return $G$
\end{algorithmic}
\end{algorithm}

Como inserimos as arestas em ordem crescente de peso, a floresta~$F$ é de pesos mínimos e logo, pelo Teorema~\ref{teo:MSFdual}, $F^\star$ é de pesos máximos.

\begin{algorithm}[htb]
\caption{\MSFweight($G$)}
\label{Algo:MSFweight}
\begin{algorithmic}[1]
\State \Return $G$.$F$.$w$
\end{algorithmic}
\end{algorithm}

Quando atualizamos o peso de uma aresta~$uv$ com a rotina \MSFupdate{}, cuja implementação pode ser vista no Algoritmo~\ref{Algo:MSFupdate}, precisamos garantir que a floresta~$F$ resultante continuará sendo  de peso mínimo.
Notemos que ou $uv\in F$ ou $uv^\star\in F^\star$, iremos tratar esses casos separadamente.

Se~$uv$ for uma aresta de~$F$ e se existir alguma aresta no corte~$(F, uv)$ com peso menor do que~$w$, então $F$ não será mais de peso mínimo.
Para corrigir isso, precisamos tomar~$xy$ como a aresta de menor peso no corte~$(F, uv)$,
então remover~$uv$ de~$F$ e adicionar~$xy$ a~$F$.
Dessa forma, garantindo que~$F$ seja de peso mínimo.

Para obter a aresta~$xy$, vamos investigar o corte~$(F, uv)$.
Pelo Teorema~\ref{teo:cutset}, o conjunto~$(F, uv)^\star$ forma um ciclo em~$G^\star$.
Como~$uv^\star\in(F, uv)$ e~$uv$ é a única aresta de~$F$ cujo dual está em~$(F, uv)$, então as demais arestas desse corte formam um caminho em~$F^\star$ ligando os vértices incidentes a~$uv^\star$.
Utilizando a rotina \linkcutMin{}  da biblioteca de link/cut trees podemos obter o nó de menor peso nesse percurso.

Note também que, se modificarmos~$F$, então precisaremos atualizar~$F^\star$, removendo $xy^\star$ de~$F^\star$ e adicionando $uv^\star$ com o novo peso~$w$.

Se~$uv$ não for uma aresta de~$F$, então teremos que~$uv^\star$ é uma aresta de~$F^\star$.
O tratamento desse caso é analogo ao caso anterior e é feito entre as linhas~\ref{Algo:MSFupdate:dualinicio} e~\ref{Algo:MSFupdate:dualfim} do Algoritmo~\ref{Algo:MSFupdate}.


\begin{algorithm}[htb]
\caption{\MSFupdate($G$, $u$, $v$, $w$)}
\label{Algo:MSFupdate}
\begin{algorithmic}[1]
\State $e$,$f$ $\gets$ $uv^\star$
\If {$uv$ $\in$ $G$.$F$}

\State \varname{min} $\gets$ \linkcutMin($F^\star$, $e$, $f$)
\If {$\varname{min}$.$w$ > $w$}
\State \linkcutDelEdge($F$, $u$, $v$)
\State \linkcutAddEdge($F$, $u$, $v$, $w$)
\Else
\State $x$,$y$ $\gets$ \varname{min}
\State \linkcutDelEdge($F^\star$, $x$, $y$)
\State \linkcutAddEdge($F^\star$, $e$, $f$, $w$)
\State $x$,$y$ $\gets$ $\varname{min}^\star$
\State \linkcutDelEdge($F$, $u$, $v$)
\State \linkcutAddEdge($F$, $x$, $y$, \varname{min}.$w$)
\EndIf

\Else

\State \varname{max} $\gets$ \linkcutMax($F$, $u$, $v$)\label{Algo:MSFupdate:dualinicio}
\If {\varname{max}.$w$  < $w$}
\State \linkcutDelEdge($F^\star$, $e$, $f$)
\State \linkcutAddEdge($F^\star$, $e$, $f$, $w$)
\Else
\State $x$,$y$ $\gets$ \varname{max}
\State \linkcutDelEdge($F$, $x$, $y$)
\State \linkcutAddEdge($F$, $u$, $v$, $w$)
\State $x$,$y$ $\gets$ $\varname{max}^\star$
\State \linkcutDelEdge($F^\star$, $e$, $f$)
\State \linkcutAddEdge($F^\star$, $x$, $y$, \varname{max}.$w$)\label{Algo:MSFupdate:dualfim}
\EndIf

\EndIf
\end{algorithmic}
\end{algorithm}

%%%%%%%%%%%%%%%%%%%%%%%%%%%%%%%%%%%%%%%%%%%%%%%%
%          Limitante inferior                  %
%%%%%%%%%%%%%%%%%%%%%%%%%%%%%%%%%%%%%%%%%%%%%%%%
\chapter{O limitante inferior de~$\Omega(\lg n)$}
\label{sec:lim}
Nessa capítulo explicaremos o limitante inferior de consumo de tempo amortizado de~$\Omega(\lg n)$ para o problema de conexidade em grafos dinâmicos~\cite{lowerBoundPatrascu}. Esse limitante é incondicional e é válido mesmo para os algoritmos que usam técnicas de aleatorização, no estilo Las Vegas ou Monte Carlo. Além disso, esse limitante é válido mesmo para o problema de conexidade em grafos dinâmicos restrito a florestas ou caminhos. 

Esse resultado é consequência de um limitante inferior para o problema de verificação de soma parcial em $S_k$ (VSP$S_k$), 
que será elaborado na Seção~\ref{sec:lim-verificacao-Sk}.
Para transportar o limitante de um problema para outro, reduziremos o problema VSP$S_k$ ao problema de conexidade em grafos dinâmicos.

O limitante inferior para o problema VSP$S_k$ usa o modelo de computação \textit{cell-probe}. Dessa forma nosso limitante também aplica-se a esse modelo. Portanto, devemos iniciar nossa discussão na próxima seção explicando como esse modelo de computação funciona e como é mensurado o consumo de tempo de um algoritmo nesse modelo. Em seguida, definiremos VSP$S_k$ e seu limitante inferior formalmente e, concluindo esse capítulo, faremos a redução dele para o problema de conexidade em grafos dinâmicos.

\section{O modelo de computação cell-probe}
\label{sec:lim-cell-probe}
No modelo cell-probe, a memória do computador é representada por uma coleção de células. Cada célula é composta por uma quantidade fixa de~$b$ bits e possui um identificador único, que é chamado de \defi{endereço} da célula. Algoritmos nesse modelo podem ler e escrever dados nas células e realizar operações elementares, como aritmética básica, em uma unidade de processamento externa à memória~\cite{Ajtai1988}.

Ao limitar o número de células para $2^b$, garantimos que o endereço de qualquer célula pode ser armazenado em uma única célula, assim aproximando esse modelo abstrato à implementação de memória RAM dos computadores usuais. Lembramos que, nessa implementação de computadores, as operações aritméticas e \textit{bit-wise} são realizadas na CPU, externas à memória e o consumo de tempo real causado por essas operações é muito menor do que o tempo de escrita e leitura da memória. O modelo \textit{cell-probe} representa essa disparidade de consumo de tempo, definindo o consumo de tempo de um algoritmo como sendo proporcional à quantidade de células da memória escritas ou lidas, desconsiderando assim o tempo necessário para a realização de operações que, em um computador usual, seriam realizadas pela CPU.

Muitos resultados sobre algoritmos que usam esse modelo estão em função do parâmetro~$b$, pois esse parâmetro determina quanta informação pode ser armazenada em uma única célula. Outra informação que parametriza o consumo de tempo nesse modelo é a quantidade~$\delta$ de bits necessários para representar cada parâmetro de entrada do algoritmo. É costumeiro separar esses parâmetros, pois em diversas aplicações um tende a ser assintoticamente maior do que o outro.

\section{Verificação de soma parcial em~$S_k$}
\label{sec:lim-verificacao-Sk}

O grupo $S_k$ é o grupo finito formado pelo conjunto de todas as bijeções sobre o conjunto~${[k]:=\{1,2,\ldots,k\}}$ munido da operação de composição de funções~\cite{agozine2010}. Essas bijeções também são chamadas de \defi{permutações}. Podemos visualizar uma permutação desenhando o domínio e contradomínio em duas colunas e desenhando setas relacionando cada elemento do domínio com sua imagem, como feito na Figura~\ref{fig:LIM-exemplo-uma-perm}.

\begin{figure}[htb]
\centering
\begin{tikzpicture}[line cap=round,line join=round,>=triangle 45,x=1cm,y=1cm]
\clip(0,.5) rectangle (3.5,6.5);
\draw (.85,1) node[anchor=center] {1};
\draw (.85,2) node[anchor=center] {2};
\draw (.85,3) node[anchor=center] {3};
\draw (.85,4) node[anchor=center] {4};
\draw (.85,5) node[anchor=center] {5};
\draw (.85,6) node[anchor=center] {6};
\draw (3.15,1) node[anchor=center] {1};
\draw (3.15,2) node[anchor=center] {2};
\draw (3.15,3) node[anchor=center] {3};
\draw (3.15,4) node[anchor=center] {4};
\draw (3.15,5) node[anchor=center] {5};
\draw (3.15,6) node[anchor=center] {6};
\draw [line width=1pt] (1,6) -- (3,5);
\draw [->,line width=1pt] (1,6) -- (3,5);
\draw [->,line width=1pt] (1,4) -- (3,3);
\draw [->,line width=1pt] (1,5) -- (3,6);
\draw [->,line width=1pt] (1,3) -- (3,1);
\draw [->,line width=1pt] (1,2) -- (3,4);
\draw [->,line width=1pt] (1,1) -- (3,2);
\end{tikzpicture}
\caption{Exemplo de representação de uma permutação com $k=6$.}
\label{fig:LIM-exemplo-uma-perm}
\end{figure}

Podemos, com a técnica de visualização empregada na Figura~\ref{fig:LIM-exemplo-uma-perm}, ilustrar também como é feita a composição de uma sequência de permutações ${\phi=(\phi_1, \phi_2, \ldots, \phi_p)}$, que pode ser vista na Figura~\ref{fig:LIM-exemplo-comp}. Algebricamente falando, essa composição é somente a composição de funções. Visualmente, a composição forma~$k$ caminhos direcionados constituídos pelas setas que ilustram cada permutação. Podemos calcular a permutação~$\varphi$ resultante da composição das permutações de~$\phi$ percorrendo, para cada elemento $q\in [k]$, o caminho direcionado iniciado em~$q$ no domínio de $\phi_1$ até chegar ao valor~$r$ no contradomínio de~$\phi_p$. Dessa forma, teremos $\varphi(q)=r$. Também podemos, com essa ideia de percorrer os caminhos direcionados, calcular a composição parcial de~$\phi_1$ até~$\phi_i$, percorrendo o caminho parcialmente até o contradomínio de~$\phi_i$, com $1\leqslant i \leqslant p$.

\begin{figure}[htb]
\centering
\begin{tikzpicture}[line cap=round,line join=round,>=triangle 45,x=1cm,y=1cm]
\clip(0,-.2) rectangle (15,6.5);

\draw (.85,1) node[anchor=center] {1};
\draw (.85,2) node[anchor=center] {2};
\draw (.85,3) node[anchor=center] {3};
\draw (.85,4) node[anchor=center] {4};
\draw (.85,5) node[anchor=center] {5};
\draw (.85,6) node[anchor=center] {6};
\draw [->,line width=1pt] (1,6) -- (3,5);
\draw [->,line width=1pt] (1,4) -- (3,3);
\draw [->,line width=1pt] (1,5) -- (3,6);
\draw [->,line width=1pt] (1,3) -- (3,1);
\draw [->,line width=1pt] (1,2) -- (3,4);
\draw [->,line width=1pt] (1,1) -- (3,2);
\draw [line width=1pt] (1,.5) -- (1,.7);
\draw [line width=1pt] (3,.5) -- (3,.7);
\draw [line width=1pt] (1,.5) -- (3,.5);
\draw (2,.2) node[anchor=center] {$\phi_1$};



\draw (3.15,1) node[anchor=center] {1};
\draw (3.15,2) node[anchor=center] {2};
\draw (3.15,3) node[anchor=center] {3};
\draw (3.15,4) node[anchor=center] {4};
\draw (3.15,5) node[anchor=center] {5};
\draw (3.15,6) node[anchor=center] {6};
\draw [->,line width=1pt] (3.3,6) -- (5.30,4);
\draw [->,line width=1pt] (3.3,4) -- (5.30,3);
\draw [->,line width=1pt] (3.3,5) -- (5.30,6);
\draw [->,line width=1pt] (3.3,3) -- (5.30,1);
\draw [->,line width=1pt] (3.3,2) -- (5.30,5);
\draw [->,line width=1pt] (3.3,1) -- (5.30,2);
\draw [line width=1pt] (3.3,.5) -- (3.3,.7);
\draw [line width=1pt] (5.30,.5) -- (5.30,.7);
\draw [line width=1pt] (3.3,.5) -- (5.30,.5);
\draw (4.3,.2) node[anchor=center] {$\phi_2$};


\draw (5.45,1) node[anchor=center] {1};
\draw (5.45,2) node[anchor=center] {2};
\draw (5.45,3) node[anchor=center] {3};
\draw (5.45,4) node[anchor=center] {4};
\draw (5.45,5) node[anchor=center] {5};
\draw (5.45,6) node[anchor=center] {6};
\draw [->,line width=1pt] (5.6,6) -- (7.6,6);
\draw [->,line width=1pt] (5.6,4) -- (7.6,3);
\draw [->,line width=1pt] (5.6,5) -- (7.6,4);
\draw [->,line width=1pt] (5.6,3) -- (7.6,2);
\draw [->,line width=1pt] (5.6,2) -- (7.6,5);
\draw [->,line width=1pt] (5.6,1) -- (7.6,1);
\draw [line width=1pt] (5.6,.5) -- (5.6,.7);
\draw [line width=1pt] (7.6,.5) -- (7.6,.7);
\draw [line width=1pt] (5.6,.5) -- (7.6,.5);
\draw (6.6,.2) node[anchor=center] {$\phi_3$};



\draw (7.75,1) node[anchor=center] {1};
\draw (7.75,2) node[anchor=center] {2};
\draw (7.75,3) node[anchor=center] {3};
\draw (7.75,4) node[anchor=center] {4};
\draw (7.75,5) node[anchor=center] {5};
\draw (7.75,6) node[anchor=center] {6};
\draw [->,line width=1pt] (7.9,6) -- (9.9,4);
\draw [->,line width=1pt] (7.9,4) -- (9.9,3);
\draw [->,line width=1pt] (7.9,5) -- (9.9,6);
\draw [->,line width=1pt] (7.9,3) -- (9.9,1);
\draw [->,line width=1pt] (7.9,2) -- (9.9,5);
\draw [->,line width=1pt] (7.9,1) -- (9.9,2);
\draw [line width=1pt] (9.9,.5) -- (9.9,.7);
\draw [line width=1pt] (7.9,.5) -- (7.9,.7);
\draw [line width=1pt] (7.9,.5) -- (9.9,.5);
\draw (8.9,.2) node[anchor=center] {$\phi_4$};

\draw (10.05,1) node[anchor=center] {1};
\draw (10.05,2) node[anchor=center] {2};
\draw (10.05,3) node[anchor=center] {3};
\draw (10.05,4) node[anchor=center] {4};
\draw (10.05,5) node[anchor=center] {5};
\draw (10.05,6) node[anchor=center] {6};




\draw (12.05,1) node[anchor=center] {1};
\draw (12.05,2) node[anchor=center] {2};
\draw (12.05,3) node[anchor=center] {3};
\draw (12.05,4) node[anchor=center] {4};
\draw (12.05,5) node[anchor=center] {5};
\draw (12.05,6) node[anchor=center] {6};

\draw (14.35,1) node[anchor=center] {1};
\draw (14.35,2) node[anchor=center] {2};
\draw (14.35,3) node[anchor=center] {3};
\draw (14.35,4) node[anchor=center] {4};
\draw (14.35,5) node[anchor=center] {5};
\draw (14.35,6) node[anchor=center] {6};

\draw [->,line width=1pt] (12.2,6) -- (14.2,4);
\draw [->,line width=1pt] (12.2,5) -- (14.2,1);
\draw [->,line width=1pt] (12.2,4) -- (14.2,2);
\draw [->,line width=1pt] (12.2,3) -- (14.2,6);
\draw [->,line width=1pt] (12.2,2) -- (14.2,5);
\draw [->,line width=1pt] (12.2,1) -- (14.2,3);

\draw [line width=1pt] (12.2,.5) -- (12.2,.7);
\draw [line width=1pt] (14.2,.5) -- (14.2,.7);
\draw [line width=1pt] (12.2,.5) -- (14.2,.5);
\draw (13.2,.2) node[anchor=center] {$\varphi$};
\end{tikzpicture}
\caption{Exemplo de uma composição de permutações adotando $p=4$. Temos~$\varphi = \phi_4\circ \phi_3\circ \phi_2\circ \phi_1$.}
\label{fig:LIM-exemplo-comp}
\end{figure}

Notemos que a substituição de um $\phi_i$ da sequência~$\phi$ pode alterar drasticamente os~$k$ caminhos ilustrados e consequentemente alterar a permutação resultante~$\varphi$. Surge então um novo problema dinâmico, o \defi[problema!de verificação de soma parcial em $S_k$]{problema da verificação de soma parcial em $S_{k}$ (VSP$S_k$)}, que visa manter uma sequência de~$p$ permutações ${\phi=(\phi_1, \phi_2, \ldots, \phi_p)}$ de forma a implementar eficientemente a seguinte biblioteca:
\begin{itemize}
\item \VPSPupdate($\phi$, $i$, $\varphi$): a $i$-ésima coordenada de $\phi$ passa a ser a permutação~$\varphi$ ; e
\item \VPSPverify($\phi$, $i$, $\varphi$): retorna verdadeiro se~$\phi_i\circ \cdots\circ \phi_1 = \varphi$ e falso, caso contrário.
\end{itemize}

Mihai Patrascu e Erik D. Demaine~\cite{lowerBoundPatrascu} provaram o seguinte resultado:

\begin{theorem}
\label{theo:lim}
Os consumos de tempo $t_u$ e $t_q$ das rotinas \VPSPupdate{} e~\VPSPverify, respectivamente, implementadas com qualquer estrutura de dados sob o modelo \textit{cell-probe} para solucionar VSP$S_k$ estão relacionados e limitados por
$$
\min\{t_u,t_q\}\lg \left( \frac{\max\{t_u,t_q\}}{\min\{t_u,t_q\}}\right) = \Omega\left(\frac{\delta}{b}\lg p\right),
$$
onde cada célula possui $b$ bits e são necessários $\delta$ bits para representar cada parâmetro da rotina.  Esse limitante continua válido mesmo se a estrutura de dados utilizar amortização, não determinismo, aleatorização Las Vegas ou Monte Carlo com erro probabilístico~$p^{-\Omega(1)}$.
\end{theorem}

A demonstração completa deste resultado é longa, complexa e foge do escopo desse texto, logo optamos por não detalhá-la aqui e somente tecer alguns comentários sobre ela.

No artigo, os autores demonstram um lema \cite[Lema~5.1]{lowerBoundPatrascu} cuja prova envolve a construção de uma árvore binária que modela o fluxo de células da memória lidas e escritas ao longo de uma sequência de operações da biblioteca dinâmica de VSP$S_k$. Esse lema limita inferiormente a quantidade esperada de leituras feitas em células escritas ao longo dessa sequência, o que implica, como os autores mostram ao fim da Seção~$5.2$ do artigo, em um limitante inferior de
\begin{equation}
    \Omega\left(\frac{\delta}{b}\lg p\right) \label{eq:lim}
\end{equation}
amortizado para cada operação dessa sequência.

Há uma nuance na amortização presente nesse limitante. A sequência de operações é composta por ambas as rotinas~\VPSPupdate{} e~\VPSPverify.
\NEW{Logo} a amortização do custo da sequência por operação significa que pelo menos uma dessas rotinas, mas não necessariamente \textit{ambas}, está limitada inferiormente por~$\Omega(\frac{\delta}{b}\lg p)$. Ou seja, é possível que uma dessas rotinas consuma tempo constante e assim, nesse caso, a outra necessariamente consumirá tempo~$\Omega(\frac{\delta}{b}\lg p)$. Elaboraremos um exemplo envolvendo grafos dinâmicos em que isso ocorre ao final desse capítulo. 


Na Seção 5.5 do artigo, os autores explicitam essa nuance desenvolvendo um método que converte limitantes amortizados em limitantes que relacionam $t_u$ a $t_q$ e, ao aplicar esse método ao limitante~\eqref{eq:lim}, concluem o Teorema~\ref{theo:lim}. Pontuamos que esse método se apoia em uma análise mais fina da sequência modelada pela árvore binária usada pelo lema citado anteriormente, logo discorrer detalhadamente sobre ele também foge do escopo desse trabalho.


\section[Redução do VSP$S_k$ para conexidade em grafos dinâmicos]{Redução do problema de VSP$S_k$ para o de conexidade em grafos dinâmicos}

Para fazer a redução de VSP$S_k$ ao problema de conexidade em grafos dinâmicos, seguiremos o processo descrito na Seção~6.1 de Patrascu e Demaine~\cite{lowerBoundPatrascu} para  converter uma instância de VSP$S_k$ em uma instância do problema de conexidade em grafos dinâmicos. 

A Figura~\ref{fig:LIM-exemplo-comp} nos indica como vamos traduzir um problema no outro. Em essência, vamos converter cada número dessa figura em um vértice e cada seta em uma aresta, obtendo assim um grafo formado por $k$ caminhos disjuntos, cada um de comprimento~$p$, como pode ser visto na Figura~\ref{fig:LIM-convertido}. Formalmente, para a sequência $\phi$ de uma instância de~VSP$S_k$, construiremos um grafo dinâmico $G(\phi)$ cujo conjunto de vértices consiste nos pares~$(x,y)$, com $1\leqslant x \leqslant p+1$ e $1\leqslant y \leqslant k$. Logo $G(\phi)$ terá $n:=k\cdot (p+1)$ vértices. Cada vértice~$(x,y)$ com~$1\leqslant x \leqslant p$ será adjacente ao vértice~$(x+1,\phi_x(y))$. 

Em nossas rotinas, também será necessário calcular o valor $\phi_x(y)$ em $\O{1}$. Para isso, manteremos uma cópia de $\phi$ junto a~$G(\phi)$ e, para manter o pseudocódigo mais limpo, quando for claro, denotaremos essa cópia simplesmente por~$\phi$ em vez da descrição mais carregada~$G(\phi).\phi$.


\begin{figure}[htb]
\centering
\begin{tikzpicture}[line cap=round,line join=round,>=triangle 45,x=1cm,y=1cm]
\clip(0,-.2) rectangle (10.5,6.2);

\draw [fill=black] (1,1) circle (1.75pt);
\draw [fill=black] (1,2) circle (1.75pt);
\draw [fill=black] (1,3) circle (1.75pt);
\draw [fill=black] (1,4) circle (1.75pt);
\draw [fill=black] (1,5) circle (1.75pt);
\draw [fill=black] (1,6) circle (1.75pt);
\draw [line width=1pt] (1,6) -- (3.15,5);
\draw [line width=1pt] (1,4) -- (3.15,3);
\draw [line width=1pt] (1,5) -- (3.15,6);
\draw [line width=1pt] (1,3) -- (3.15,1);
\draw [line width=1pt] (1,2) -- (3.15,4);
\draw [line width=1pt] (1,1) -- (3.15,2);
\draw [line width=1pt] (1,.5) -- (1,.7);
\draw [line width=1pt] (3,.5) -- (3,.7);
\draw [line width=1pt] (1,.5) -- (3,.5);
\draw (2,.2) node[anchor=center] {$\phi_1$};


\draw [fill=black] (3.15,1) circle (1.75pt);
\draw [fill=black] (3.15,2) circle (1.75pt);
\draw [fill=black] (3.15,3) circle (1.75pt);
\draw [fill=black] (3.15,4) circle (1.75pt);
\draw [fill=black] (3.15,5) circle (1.75pt);
\draw [fill=black] (3.15,6) circle (1.75pt);

\draw [line width=1pt] (3.15,6) -- (5.45,4);
\draw [line width=1pt] (3.15,4) -- (5.45,3);
\draw [line width=1pt] (3.15,5) -- (5.45,6);
\draw [line width=1pt] (3.15,3) -- (5.45,1);
\draw [line width=1pt] (3.15,2) -- (5.45,5);
\draw [line width=1pt] (3.15,1) -- (5.45,2);
\draw [line width=1pt] (3.3,.5) -- (3.3,.7);
\draw [line width=1pt] (5.30,.5) -- (5.30,.7);
\draw [line width=1pt] (3.3,.5) -- (5.30,.5);
\draw (4.3,.2) node[anchor=center] {$\phi_2$};

\draw [fill=black] (5.45,1) circle (1.75pt);
\draw [fill=black] (5.45,2) circle (1.75pt);
\draw [fill=black] (5.45,3) circle (1.75pt);
\draw [fill=black] (5.45,4) circle (1.75pt);
\draw [fill=black] (5.45,5) circle (1.75pt);
\draw [fill=black] (5.45,6) circle (1.75pt);


\draw [line width=1pt] (5.45,6) -- (7.75,6);
\draw [line width=1pt] (5.45,4) -- (7.75,3);
\draw [line width=1pt] (5.45,5) -- (7.75,4);
\draw [line width=1pt] (5.45,3) -- (7.75,2);
\draw [line width=1pt] (5.45,2) -- (7.75,5);
\draw [line width=1pt] (5.45,1) -- (7.75,1);
\draw [line width=1pt] (5.6,.5) -- (5.6,.7);
\draw [line width=1pt] (7.6,.5) -- (7.6,.7);
\draw [line width=1pt] (5.6,.5) -- (7.6,.5);
\draw (6.6,.2) node[anchor=center] {$\phi_3$};


\draw [fill=black] (7.75,1) circle (1.75pt);
\draw [fill=black] (7.75,2) circle (1.75pt);
\draw [fill=black] (7.75,3) circle (1.75pt);
\draw [fill=black] (7.75,4) circle (1.75pt);
\draw [fill=black] (7.75,5) circle (1.75pt);
\draw [fill=black] (7.75,6) circle (1.75pt);
\draw [line width=1pt] (7.75,6) -- (10.05,4);
\draw [line width=1pt] (7.75,4) -- (10.05,3);
\draw [line width=1pt] (7.75,5) -- (10.05,6);
\draw [line width=1pt] (7.75,3) -- (10.05,1);
\draw [line width=1pt] (7.75,2) -- (10.05,5);
\draw [line width=1pt] (7.75,1) -- (10.05,2);
\draw [line width=1pt] (9.9,.5) -- (9.9,.7);
\draw [line width=1pt] (7.9,.5) -- (7.9,.7);
\draw [line width=1pt] (7.9,.5) -- (9.9,.5);
\draw (8.9,.2) node[anchor=center] {$\phi_4$};


\draw [fill=black] (10.05,1) circle (1.75pt);
\draw [fill=black] (10.05,2) circle (1.75pt);
\draw [fill=black] (10.05,3) circle (1.75pt);
\draw [fill=black] (10.05,4) circle (1.75pt);
\draw [fill=black] (10.05,5) circle (1.75pt);
\draw [fill=black] (10.05,6) circle (1.75pt);


\draw [fill=black] (12.05,1) circle (1.75pt);
\draw [fill=black] (12.05,2) circle (1.75pt);
\draw [fill=black] (12.05,3) circle (1.75pt);
\draw [fill=black] (12.05,4) circle (1.75pt);
\draw [fill=black] (12.05,5) circle (1.75pt);
\draw [fill=black] (12.05,6) circle (1.75pt);


\draw [fill=black] (14.35,1) circle (1.75pt);
\draw [fill=black] (14.35,2) circle (1.75pt);
\draw [fill=black] (14.35,3) circle (1.75pt);
\draw [fill=black] (14.35,4) circle (1.75pt);
\draw [fill=black] (14.35,5) circle (1.75pt);
\draw [fill=black] (14.35,6) circle (1.75pt);


%\draw [line width=1pt] (12.05,6) -- (14.35,4);
%\draw [line width=1pt] (12.05,5) -- (14.35,1);
%\draw [line width=1pt] (12.05,4) -- (14.35,2);
%\draw [line width=1pt] (12.05,3) -- (14.35,6);
%\draw [line width=1pt] (12.05,2) -- (14.35,5);
%\draw [line width=1pt] (12.05,1) -- (14.35,3);

%\draw [line width=1pt] (12.2,.5) -- (12.2,.7);
%\draw [line width=1pt] (14.2,.5) -- (14.2,.7);
%\draw [line width=1pt] (12.2,.5) -- (14.2,.5);
\draw (13.2,.2) node[anchor=center] {$\phi$};
\end{tikzpicture}
\caption{Instância de VSP$S_k$ da Figura~\ref{fig:LIM-exemplo-comp} convertida em uma instância do problema de conexidade em grafos dinâmicos.}
\label{fig:LIM-convertido}
\end{figure}

Podemos encapsular essa conversão em uma rotina chamada \VPSPconvert($\phi$, $p$, $k$), descrita no Algoritmo~\ref{Algo:VPSPconvert}, que recebe a sequência~$\phi$, seu comprimento $p$ e o tamanho~$k$ do domínio e contradomínio das permutações e retorna o grafo dinâmico~$G(\phi)$.

\begin{algorithm}[htb]
\caption{\VPSPconvert($\phi$, $p$, $k$)}
\label{Algo:VPSPconvert}
\begin{algorithmic}[1]
\State $G(\phi)$ $\gets$ \dymGraphCreate($(p+1)\cdot k)$
\State $G(\phi).\phi$ $\gets$ $\phi$
\For {$x$ $\gets$ 1 até $p+1$}
  \For {$y$ $\gets$ 1 até $k$}
    \State \dymGraphAddEdge($G(\phi)$, $(x,~y)$, $(x+1, ~\phi_x(y))$)
  \EndFor
\EndFor
\State \Return $G(\phi)$
\end{algorithmic}
\end{algorithm}


Com essa conversão feita, podemos implementar a biblioteca de VSP$S_k$ usando a biblioteca de conexidade em grafos dinâmicos. A implementação de \mbox{\VPSPupdate($G(\phi)$, $i$,~$\varphi$)} pode ser vista no Algoritmo~\ref{Algo:VPSPupdate}. Nesse algoritmo, primeiro removemos todas as arestas associadas à permutação~$\phi_i$. Em seguida, inserimos~$k$ novas arestas ligando~$(i,y)$ a~${(i+1,\varphi(y))}$, para cada $y\in[k]$.


\begin{algorithm}[htb]
\caption{\VPSPupdate($G(\phi)$, $i$, $\varphi$)}
\label{Algo:VPSPupdate}
\begin{algorithmic}[1]
\For {$y$ $\gets$ 1 até $k$}
  \State \dymGraphDelEdge($G(\phi)$, $(i,y)$, $(i+1,\phi_i(y)$))
\EndFor
\For {$y$ $\gets$ 1 até $k$}
  \State \dymGraphAddEdge($G(\phi)$, $(i,y)$, $(i+1,\varphi(y)$))
\EndFor
\State $\phi_i$ $\gets$ $\varphi$ 
\end{algorithmic}
\end{algorithm}


Implementamos \VPSPverify($G(\phi)$, $i$, $\varphi$) com~$k$ chamadas à consulta de conexidade em grafos dinâmicos, feita pela rotina \dymGraphQuery{}. Para cada $y\in[k]$, testamos a conexidade entre os vértices~$(1,y)$ e~$(i+1,\varphi(y))$. O teste retorna verdadeiro se e só se existe um caminho entre esses vértices. Mas, nesse grafo, se existe um tal caminho em $G(\phi)$, então~$\varphi(y) = \phi_i\circ\dots\circ\phi_1(y)$. Caso todos os testes de conexidade retornem verdadeiro, então teremos que~$\varphi = \phi_i\circ\dots\circ\phi_1$ e \VPSPverify{} deve retornar verdadeiro. Caso contrário, essa rotina deve retornar falso. 

\begin{algorithm}[htb]
\caption{\VPSPverify($G(\phi)$, $i$, $\varphi$)}
\label{Algo:VPSPverify}
\begin{algorithmic}[1]
\For {$y$ $\gets$ 1 até $k$}
  \If {\textbf{não} \dymGraphQuery($G(\phi)$, $(1,y)$, $(i+1,\varphi(y)$))}
    \State \Return falso
  \EndIf
\EndFor
\State \Return verdadeiro
\end{algorithmic}
\end{algorithm}

Para entender essa rotina melhor, vamos esmiuçar a execução dela em um exemplo. Na Figura~\ref{fig:LIM-exemplo-verify}, vemos um grafo~$G(\phi)$ e uma permutação~$\varphi$ e queremos verificar se a composição de todas as cinco permutações é igual à permutação~$\varphi$, também ilustrada nessa figura. Para tal, realizaremos a chamada \mbox{\VPSPverify($G(\phi)$, $5$, $\varphi$)}. Como podemos ver no Algoritmo~\ref{Algo:VPSPverify}, nessa chamada a rotina \VPSPverify{} testará a conexidade entre os vértices da forma $(1,y)$ e $(6,\varphi(y))$, para cada $y\in [6]$. O primeiro desses testes será entre os vértices $(1,1)$ e $(6,3)$, pois, como pode ser visto na figura, temos que $\varphi(1) = 3$. Como podemos constatar pela figura, existe um caminho ligando esses vértices, logo o teste de conexidade retorna verdadeiro e verificamos assim que
$$
\varphi(1) = 3 = \phi_5\circ \phi_4\circ \phi_3\circ \phi_2\circ \phi_1(1).
$$
Iremos então continuar testando a conexidade entre esses pares de vértices até que ou encontremos um par desconexo, que significa que~$\varphi(y) \neq \phi_5\circ\cdots\circ \phi_1(y)$, ou testamos todos, o que significa que de fato $\varphi$ é igual à composição de todas as cinco permutações de $\phi$. O segundo teste de conexidade é entre os vértices~$(1,2)$ e~$(6,1)$, pois $\varphi(2)=1$, e novamente podemos observar que existe um caminho entre esses vértices, assim a consulta de conexidade retorna verdadeiro e portanto verificamos que $\varphi(2) = \phi_5\circ\cdots\circ \phi_1(2)$. O mesmo ocorre para as duas próximas iterações do laço da rotina~\VPSPverify{}. Como~$\varphi(3)=4$ e~$\varphi(4)=5$, testamos sequencialmente a conexidade entre os vértices~$(1,3)$ e~$(6,4)$ e entre~$(1,4)$ e~$(6,5)$, verificando em ambos os casos que esses vértices estão interligados. Analogamente, como~$\varphi(5)=6$, testamos a conexidade entre $(1,5)$ e~$(6,6)$, que retorna falso, pois não há caminho entre esses vértices. Isso significa que~$\varphi(5)\neq \phi_5\circ\cdots\circ \phi_1(5)$. Podemos verificar essa desigualdade pela Figura~\ref{fig:LIM-exemplo-verify}, acompanhando o caminho iniciado em~$(1,5)$. Podemos constatar que ele liga esse vértice ao vértice~$(6,2)$, o que representa que
$$
 \varphi_5\circ \phi_4\circ \phi_3\circ \phi_2\circ \phi_1(5) = 2 \neq 6 = \varphi(5).
$$
Ao encontrar essa desconexidade, encerramos a nossa busca e retornamos falso, finalizando assim a execução da chamada~\VPSPverify($G(\phi)$, $5$, $\varphi$).

\begin{figure}[htb]
\centering
\scalebox{.85}{
\begin{tikzpicture}[line cap=round,line join=round,>=triangle 45,x=1cm,y=1cm]
\clip(0,-.2) rectangle (30,6.2);

\draw [fill=black] (1,1) circle (1.75pt);
\draw [fill=black] (1,2) circle (1.75pt);
\draw [fill=black] (1,3) circle (1.75pt);
\draw [fill=black] (1,4) circle (1.75pt);
\draw [fill=black] (1,5) circle (1.75pt);
\draw [fill=black] (1,6) circle (1.75pt);
\draw [line width=1pt] (1,6) -- (3.15,5);
\draw [line width=1pt] (1,4) -- (3.15,3);
\draw [line width=1pt] (1,5) -- (3.15,6);
\draw [line width=1pt] (1,3) -- (3.15,1);
\draw [line width=1pt] (1,2) -- (3.15,4);
\draw [line width=1pt] (1,1) -- (3.15,2);
\draw [line width=1pt] (1,.5) -- (1,.7);
\draw [line width=1pt] (3,.5) -- (3,.7);
\draw [line width=1pt] (1,.5) -- (3,.5);
\draw (2,.2) node[anchor=center] {$\phi_1$};


\draw [fill=black] (3.15,1) circle (1.75pt);
\draw [fill=black] (3.15,2) circle (1.75pt);
\draw [fill=black] (3.15,3) circle (1.75pt);
\draw [fill=black] (3.15,4) circle (1.75pt);
\draw [fill=black] (3.15,5) circle (1.75pt);
\draw [fill=black] (3.15,6) circle (1.75pt);

\draw [line width=1pt] (3.15,6) -- (5.45,4);
\draw [line width=1pt] (3.15,4) -- (5.45,3);
\draw [line width=1pt] (3.15,5) -- (5.45,6);
\draw [line width=1pt] (3.15,3) -- (5.45,1);
\draw [line width=1pt] (3.15,2) -- (5.45,5);
\draw [line width=1pt] (3.15,1) -- (5.45,2);
\draw [line width=1pt] (3.3,.5) -- (3.3,.7);
\draw [line width=1pt] (5.30,.5) -- (5.30,.7);
\draw [line width=1pt] (3.3,.5) -- (5.30,.5);
\draw (4.3,.2) node[anchor=center] {$\phi_2$};

\draw [fill=black] (5.45,1) circle (1.75pt);
\draw [fill=black] (5.45,2) circle (1.75pt);
\draw [fill=black] (5.45,3) circle (1.75pt);
\draw [fill=black] (5.45,4) circle (1.75pt);
\draw [fill=black] (5.45,5) circle (1.75pt);
\draw [fill=black] (5.45,6) circle (1.75pt);


\draw [line width=1pt] (5.45,6) -- (7.75,6);
\draw [line width=1pt] (5.45,4) -- (7.75,3);
\draw [line width=1pt] (5.45,5) -- (7.75,4);
\draw [line width=1pt] (5.45,3) -- (7.75,2);
\draw [line width=1pt] (5.45,2) -- (7.75,5);
\draw [line width=1pt] (5.45,1) -- (7.75,1);
\draw [line width=1pt] (5.6,.5) -- (5.6,.7);
\draw [line width=1pt] (7.6,.5) -- (7.6,.7);
\draw [line width=1pt] (5.6,.5) -- (7.6,.5);
\draw (6.6,.2) node[anchor=center] {$\phi_3$};


\draw [fill=black] (7.75,1) circle (1.75pt);
\draw [fill=black] (7.75,2) circle (1.75pt);
\draw [fill=black] (7.75,3) circle (1.75pt);
\draw [fill=black] (7.75,4) circle (1.75pt);
\draw [fill=black] (7.75,5) circle (1.75pt);
\draw [fill=black] (7.75,6) circle (1.75pt);
\draw [line width=1pt] (7.75,6) -- (10.05,4);
\draw [line width=1pt] (7.75,4) -- (10.05,3);
\draw [line width=1pt] (7.75,5) -- (10.05,6);
\draw [line width=1pt] (7.75,3) -- (10.05,1);
\draw [line width=1pt] (7.75,2) -- (10.05,5);
\draw [line width=1pt] (7.75,1) -- (10.05,2);
\draw [line width=1pt] (9.9,.5) -- (9.9,.7);
\draw [line width=1pt] (7.9,.5) -- (7.9,.7);
\draw [line width=1pt] (7.9,.5) -- (9.9,.5);
\draw (8.9,.2) node[anchor=center] {$\phi_4$};


\draw [fill=black] (10.05,1) circle (1.75pt);
\draw [fill=black] (10.05,2) circle (1.75pt);
\draw [fill=black] (10.05,3) circle (1.75pt);
\draw [fill=black] (10.05,4) circle (1.75pt);
\draw [fill=black] (10.05,5) circle (1.75pt);
\draw [fill=black] (10.05,6) circle (1.75pt);
\draw [line width=1pt] (10.05,6) -- (12.35,4);
\draw [line width=1pt] (10.05,4) -- (12.35,6);
\draw [line width=1pt] (10.05,5) -- (12.35,1);
\draw [line width=1pt] (10.05,3) -- (12.35,3);
\draw [line width=1pt] (10.05,2) -- (12.35,5);
\draw [line width=1pt] (10.05,1) -- (12.35,2);
\draw [line width=1pt] (10.05,.5) -- (10.05,.7);
\draw [line width=1pt] (12.35,.5) -- (12.35,.7);
\draw [line width=1pt] (10.05,.5) -- (12.35,.5);
\draw (11,.2) node[anchor=center] {$\phi_5$};


\draw [fill=black] (12.35,1) circle (1.75pt);
\draw [fill=black] (12.35,2) circle (1.75pt);
\draw [fill=black] (12.35,3) circle (1.75pt);
\draw [fill=black] (12.35,4) circle (1.75pt);
\draw [fill=black] (12.35,5) circle (1.75pt);
\draw [fill=black] (12.35,6) circle (1.75pt);

%\draw [fill=black] (16.2,1) circle (1.75pt);
%\draw [fill=black] (16.2,2) circle (1.75pt);
%\draw [fill=black] (16.2,3) circle (1.75pt);
%\draw [fill=black] (16.2,4) circle (1.75pt);
%\draw [fill=black] (16.2,5) circle (1.75pt);
%\draw [fill=black] (16.2,6) circle (1.75pt);

%\draw [fill=black] (14.2,1) circle (1.75pt);
%\draw [fill=black] (14.2,2) circle (1.75pt);
%\draw [fill=black] (14.2,3) circle (1.75pt);
%\draw [fill=black] (14.2,4) circle (1.75pt);
%\draw [fill=black] (14.2,5) circle (1.75pt);
%\draw [fill=black] (14.2,6) circle (1.75pt);


\draw [->,line width=1pt] (14.2,6) -- (16.2,2);
\draw [->,line width=1pt] (14.2,5) -- (16.2,6);
\draw [->,line width=1pt] (14.2,4) -- (16.2,5);
\draw [->,line width=1pt] (14.2,3) -- (16.2,4);
\draw [->,line width=1pt] (14.2,2) -- (16.2,1);
\draw [->,line width=1pt] (14.2,1) -- (16.2,3);

\draw [line width=1pt] (16.2,.5) -- (16.2,.7);
\draw [line width=1pt] (14.2,.5) -- (14.2,.7);
\draw [line width=1pt] (16.2,.5) -- (14.2,.5);
\draw (15.2,.2) node[anchor=center] {$\phi$};

\draw (.8,1) node[anchor=center] {1};
\draw (.8,2) node[anchor=center] {2};
\draw (.8,3) node[anchor=center] {3};
\draw (.8,4) node[anchor=center] {4};
\draw (.8,5) node[anchor=center] {5};
\draw (.8,6) node[anchor=center] {6};

\draw (12.55,1) node[anchor=center] {1};
\draw (12.55,2) node[anchor=center] {2};
\draw (12.55,3) node[anchor=center] {3};
\draw (12.55,4) node[anchor=center] {4};
\draw (12.55,5) node[anchor=center] {5};
\draw (12.55,6) node[anchor=center] {6};

\draw (14,1) node[anchor=center] {1};
\draw (14,2) node[anchor=center] {2};
\draw (14,3) node[anchor=center] {3};
\draw (14,4) node[anchor=center] {4};
\draw (14,5) node[anchor=center] {5};
\draw (14,6) node[anchor=center] {6};


\draw (16.4,1) node[anchor=center] {1};
\draw (16.4,2) node[anchor=center] {2};
\draw (16.4,3) node[anchor=center] {3};
\draw (16.4,4) node[anchor=center] {4};
\draw (16.4,5) node[anchor=center] {5};
\draw (16.4,6) node[anchor=center] {6};
\end{tikzpicture}
}
\caption{Instância de VSP$S_k$ da Figura~\ref{fig:LIM-exemplo-comp} convertida em uma instância do problema de conexidade em grafos dinâmicos.}
\label{fig:LIM-exemplo-verify}
\end{figure}

\section{Limitante inferior para conexidade em grafos dinâmicos}

Com as implementações das rotinas~\VPSPupdate{} e~\VPSPverify{} descritas respectivamente pelos Algoritmos~\ref{Algo:VPSPupdate} e~\ref{Algo:VPSPverify}, podemos transferir o limitante inferior do Teorema~\ref{theo:lim} para o problema de conexidade em grafos dinâmicos. 

Para fazer isso, calcularemos os valores dos parâmetros~$\delta$ e~$b$ usados no enunciado do Teorema~\ref{theo:lim}.
Em particular, definiremos  $\delta$ em termos de $b$. Ambas as rotinas têm como parâmetro a tripla ($G(\phi)$, $i$, $\varphi$). Primeiro notemos que $G(\phi)$ é passado por referência e logo, como comentado na Seção~\ref{sec:lim-cell-probe}, um endereço de célula será armazenado em uma única célula e assim usará até $b$ bits. O inteiro $i$ também usa $b$ bits, já que também assumimos que deva ser possível armazenar os inteiros com que trabalhamos em uma célula. Para implementar a permutação~$\varphi$, mantemos um vetor $V$ de inteiros com comprimento $k$ de forma que $V[i] = \varphi(i)$. Com essa implementação, $\varphi$ precisará de $k\cdot b$
bits, pois cada entrada de $V$ consome $b$ bits. Assim concluímos que $\delta = \Theta(k\cdot b)$.  Substituindo~$\delta$ no resultado do Teorema~\ref{theo:lim}, obtemos que
\begin{equation}
\Omega\left( \frac{\delta}{b}\lg p \right)\implies \Omega\left(\frac{k\cdot b}{b}\lg p\right)\implies \Omega(k\lg p).\label{eq:lim-kp}
\end{equation}

Lembremos que a igualdade~$n=k\cdot (p+1)$ relaciona os valores~$k$ e~$p$ ao número~$n$ de vértices do grafo dinâmico~$G(\phi)$. Para obter um limitante exclusivamente em função de~$n$, escolheremos uma família de instâncias de VSP$S_k$ que permita a substituição de $k$ e~$p$ da Equação~\eqref{eq:lim-kp} por $n$. Especificamente, escolheremos a família em que~$k = p-1$.
Para essa escolha, temos
$$
n =k\cdot (p+1) =(p-1)\cdot  (p+1) =p^2-1.
$$
Dessa forma, teremos:
$$
\Omega(k \lg p) = \Omega(k\cdot  2\cdot \lg p) = \Omega(k \lg p^2) = \Omega(k \lg n).\nonumber 
$$

Para concluirmos, note que $\Omega(k\lg n)$ limita inferiormente o consumo de tempo das rotinas \VPSPupdate{} e \VPSPverify{}. O algoritmo \VPSPupdate{} faz~$k$ remoções e inserções de arestas e \VPSPverify{} faz~$k$ consultas de conexidade. Logo se denotarmos por $t_m$ a soma do consumo de tempo de uma execução de \dymGraphAddEdge{} e \dymGraphDelEdge{} e $t_c$ o consumo de tempo de \dymGraphQuery{}, então deduzimos que $t_u = k t_m$ e $t_q = k t_c$. Substituindo esses valores no Teorema~\ref{theo:lim}, teremos:
\begin{align*}
\min\{t_u,t_q\}\lg \left( \frac{\max\{t_u,t_q\}}{\min\{t_u,t_q\}}\right) &= \Omega\left(\frac{\delta}{b}\lg p\right)&\implies\\
\min\{k t_m,k t_c\}\lg \left( \frac{\max\{k t_m,k t_c\}}{\min\{k t_m,k t_c\}}\right) &= \Omega(k \lg n)&\implies\\
k\min\{ t_m, t_c\}\lg \left( \frac{\max\{ t_m, t_c\}}{\min\{ t_m,t_c\}}\right) &= \Omega(k \lg n).
\end{align*}
Portanto obtemos o seguinte limitante amortizado: 
$$
\min\{ t_m, t_c\}\lg \left( \frac{\max\{ t_m, t_c\}}{\min\{ t_m,t_c\}}\right) = \Omega(\lg n)
$$
para cada uma das chamadas das operações de conexidade em grafos dinâmicos.

Ressaltamos novamente que esse limitante implica que pelo menos um entre~$t_m$ e~$t_c$ é~$\Omega(\lg n)$, mas não necessariamente ambos.


Como exemplo, em $2015$, Kejlberg-Rasmussen \textit{et al. }\cite{kejlbergrasmussen_et_al} apresentaram uma estrutura de dados que permite fazer consulta de conexidade em grafos dinâmicos em tempo constante e, respeitando o limite inferior elaborado aqui, possui consumo de tempo para adição e remoção de arestas de $\O{\sqrt{\frac{n(\lg \lg n)^2}{\lg n}}}$.




\chapter{Estudos experimentais}
\label{sec:conclusao}

Nessa seção, comentaremos as principais avaliações empíricas já realizadas sobre algoritmos que resolvem o problema de conectividade em grafos dinâmicos.
Iniciamos com uma revisão histórica dessas avaliações já realizadas, destacando os avanços e as limitações identificadas ao longo do tempo.
Em seguida, comentaremos o desenrolar da nossa própria análise empírica.

Em $1997$, Alberts, Cattaneo e Italiano~\cite{EmpiricalStudy1997} fizeram um estudo experimental envolvendo um algoritmo simples baseado em esparsificação proposto por Eppstein et al.~\cite{Eppstein1992SparsificationaTF} e o algoritmo de Henzinger e King~\cite{HenzingerKing}, sem a melhoria de Henzinger e Thorup~\cite{HenzingerThorup}.

O algoritmo de Henzinger e King, assim como o de Holm, de Lichtenberg e Thorup~\cite{poly_log} estudado no Capítulo~\ref{sec:connDG}, associa um nível entre $0$ e $\ceil{\lg n}$ a cada aresta.
Nesse experimento, os autores propuseram heuristicamente truncar essa estrutura de níveis, mantendo assim somente os níveis entre~$k$ e~$\ceil{\lg n}$, onde $k$ é um parâmetro pré-definido.
Nessa simplificação, no nível~$k$, em vez de fazer a busca por aresta substituta proposta por Henzinger e King, o algoritmo faz uma busca exaustiva. Dessa forma, essa simplificação possui consumo de tempo assintótico pior do que o algoritmo original de Henzinger e King, porém ainda assim ela se saiu bem nos experimentos com grafos aleatórios.

Para grafos não aleatórios, o algoritmo baseado em esparsificação se saiu melhor para instâncias com poucas operações de atualização, enquanto que o algoritmo original de Henzinger e King se saiu melhor com mais operações de atualização.
As implementações desenvolvidas para estes experimentos foram feitas em C++, usando a plataforma Leda~\cite{LEDA}.

Em $2002$, Raj Iyer, David Karger, Hariharan Rahul e Mikkel Thorup~\cite{EmpiricalStudy2002} usaram como base o estudo de Alberts et al. para comparar o então recente algoritmo de Holm, de Lichtenberg e Thorup~\cite{poly_log} com o algoritmo de Henzinger e King, considerando algumas variantes destes dois algoritmos no seu estudo.
Entre outras coisas, os autores mostraram que uma das variantes de Henzinger e King considerada tem consumo de tempo $\O{\lg^2 n}$ por operação de atualização.

Os autores também propõem duas heurísticas para o algoritmo de Holm, de Lichtenberg e Thorup. Essas heurísticas não invalidam as análises do consumo de tempo do algoritmo original.

A primeira heurística usa a ideia do algoritmo de Henzinger e King de fazer um sorteio aleatório das primeiras arestas que são testadas como possíveis substitutas, em vez de percorrer sequencialmente as arestas candidatas e ir fazendo os rebaixamentos.

A segunda heurística é inspirada na heurística analisada no estudo de Alberts e outros, de truncar o número de níveis usados.

O resultado do estudo experimental em relação a estes algoritmos é que a versão do algoritmo de Holm, de Lichtenberg e Thorup com as duas heurísticas implementadas se sai melhor que o algoritmo original de Henzinger e King.
As implementações desenvolvidas para estes experimentos também foram feitas em C++ usando a plataforma Leda~\cite{LEDA}.


Em $2019$, David Fernández-Baca e Lei Liu~\cite{xp-Phylogeny} realizam uma avaliação de heurísticas envolvendo o algoritmo de Holm, de Lichtenberg e Thorup focada em solucionar problemas de biologia computacional com esse algoritmo.
Esse estudo reinforça a eficiência das heuristicas aplicadas a esse algoritmo.


Mais recentemente, Chen et al.~\cite{QC22} apresentaram um outro estudo experimental, envolvendo duas heurísticas que eles propuseram e supostamente o algoritmo de Henzinger e King, entre outros.
A nossa intenção inicial era incluir o algoritmo de Holm, de Lichtenberg e Thorup nesse estudo experimental. Este foi o principal motivo que nos levou a implementar o algoritmo deles em Python~3, que é a linguagem usada neste estudo experimental.
No entanto, durante a fase de testes usando como base o estudo experimental de Chen e outros, notamos que a implementação do algoritmo de Henzinger e King incluída no estudo tratava-se de sua versão simplificada com níveis truncados, usada no estudo de Alberts e outros, para a qual a análise original não se aplica.
Na verdade, a implementação de Chen e outros desta simplificação também executa a escolha aleatória das arestas candidatas a substitutas de uma maneira pouco eficiente, o que resulta em uma implementação com consumo de tempo muito pior que o consumo do algoritmo original de Henzinger e King.
Ou seja, não é de fato uma comparação entre as heurísticas deles e o algoritmo de Henzinger e King, como é dito no artigo.
Após percebermos estes problemas, desistimos de estender este estudo experimental.
Ademais, nesse meio tempo, também encontramos o trabalho de Iyer, Karger, Rahul e Thorup~\cite{EmpiricalStudy2002} que já apresenta um excelente estudo comparativo entre o algoritmo de Henzinger e King e o algoritmo de Holm, de Lichtenberg e Thorup. 
De qualquer modo, produzimos uma implementação do algoritmo de Holm, de Lichtenberg e Thorup em Python3, que pode ser acessada em~\cite{github}.



%%%%%%%%%%%%%%%%%%%%%%%%%%%% APÊNDICES E ANEXOS %%%%%%%%%%%%%%%%%%%%%%%%%%%%%%%%

% Um apêndice é algum conteúdo adicional de sua autoria que faz parte e
% colabora com a ideia geral do texto mas que, por alguma razão, não precisa
% fazer parte da sequência do discurso; por exemplo, a demonstração de um
% teorema intermediário, as perguntas usadas em uma pesquisa qualitativa etc.
%
% Um anexo é um documento que não faz parte da tese (em geral, nem é de sua
% autoria) mas é relevante para o conteúdo; por exemplo, a especificação do
% padrão técnico ou a legislação que o trabalho discute, um artigo de jornal
% apresentando a percepção do público sobre o tema da tese etc.
%
% Os comandos appendix e annex reiniciam a numeração de capítulos e passam
% a numerá-los com letras. "annex" não faz parte de nenhuma classe padrão,
% foi criado para este modelo. Se o trabalho não tiver apêndices ou anexos,
% remova estas linhas.
%
% Diferentemente de \mainmatter, \backmatter etc., \appendix e \annex não
% forçam o início de uma nova página. Em geral isso não é importante, pois
% o comando seguinte costuma ser "\chapter", mas pode causar problemas com
% a formatação dos cabeçalhos. Assim, vamos forçar uma nova página antes
% de cada um deles.

%%%% Apêndices %%%%

%\makeatletter
%\if@openright\cleardoublepage\else\clearpage\fi
%\makeatother

%\pagestyle{appendix}

%\appendix


% \addappheadtotoc acrescenta a palavra "Apêndice" ao sumário; se
% só há apêndices, sem anexos, provavelmente não é necessário.
%\addappheadtotoc

%\chapter{Caso decremental do problema MSF}

Nesse caso particular, o grafo inicial possui um conjunto de arestas não vazio e a única modificação permitida é a remoção de arestas. Como o grafo inicial não é vazio, precisaremos de um algoritmo de pré-processamento que lerá um arquivo que descreve esse grafo inicial e o constrói e, após essa construção, leremos um segundo arquivo que lista quais e em qual ordem as arestas devem ser removidas.

Podemos adaptar o algoritmo \HDT{} que resolve o problema de conexidade em grafos dinâmicos para solucionar o caso decremental do problema MSF.
As modificações visam manter o seguinte invariante adicional junto aos invariantes comentados na Seção~\ref{sec:fatia-em-niveis}.
\begin{enumerate}[label=(\roman*)]
    \setcounter{enumi}{3}
    \item Cada $F_{\leqslant i}$ é uma MSF de $G_{\leqslant i}$.\label{invar:MSF}
    \item Para $e$ e $f$ arestas em uma mesma componente conexa das florestas $F_i$ com pesos $w(e)$ e $w(f)$ e níveis $l(e)$ e $l(f)$, então\label{invar:MSF:mono}
	    $$w(e) \leq w(f) \iff l(e) \leq l(f).$$
\end{enumerate}

O intuito da invariante~\ref{invar:MSF} é garantir que $F_{\lceil \log n \rceil}$ seja uma MSF do grafo dinâmico, assim podemos responder a consultas usando essa floresta. A invariante~\ref{invar:MSF:mono} nos auxiliará a manter a invariante~\ref{invar:MSF}.

\section{Pré-processamento}


O algoritmo de pré-processamento que constrói nosso grafo inicial é simples.
Criaremos um grafo dinâmico $G$ vazio e em seguida leremos o arquivo que descreve sua configuração inicial, armazenaremos as arestas contidas nesse arquivo em uma fila de prioridade de mínimo cuja chave é o peso das arestas. 
Após o término dessa leitura, inseriremos cada aresta em $G$ seguindo a ordem de extração da fila de prioridade.

Notemos que o grafo obtido após esse algoritmo mantém as invariantes adicionais. Como as arestas estão sendo inseridas em ordem crescente de peso em $F_{\lceil \log n \rceil}$, teremos que essa floresta será de pesos mínimos, logo satisfaz a invariante~\ref{invar:MSF} e como somente essa floresta contém arestas, a invariante~\ref{invar:MSF:mono} também está satisfeita.


\section{Tratando remoções de arestas}

\textbf{RESUMO DE MODIFICAÇÕES:}
\begin{itemize}
	\item Substituir campos $res$ e $cres$ (que serviam para obter o conjunto de arestas reservas) por $minW$ e $nte$;
	\item Substituir cada hash $R_i[v]$ por uma fila $nte$;
	\item Modificar laço da linha de~\dymGraphReplaceMSF{} como descrito.
\end{itemize}

Após remover uma aresta $uv$ de nível~$i$, começamos o processo de busca por uma aresta substituta.
Relembramos que essa busca é encapsulado pela rotina~\dymGraphReplace.
A rotina modificada está descrita no Algoritmo~\ref{Algo:dymGraphReplaceMSF}.

Para manter o invariante~\ref{invar:MSF}, precisamos garantir que, caso exista alguma aresta substituta, a de menor peso seja escolhida.
Para tal, percorreremos as arestas reservas em ordem crescente de peso.
Dessa forma, a primeira aresta reserva que for descoberta como uma substituta, será a de menor peso.

Já argumentamos que as aresta substitutas, se existirem, estão em algum nível $\geq i$.
A invariante~\ref{invar:MSF:mono} garante que, se continuarmos realizando essa busca em ordem crescente de nível, no primeiro nível que houver uma substituta, terá uma substituta de menor peso.

Assim, para cada nível, percorremos as arestas reservas em ordem crescente de peso usando a rotina~\treapGetEdgeMinWeight($\node$), descrita no Algoritmo~\ref{Algo:treapGetEdgeMinWeight}, e que retorna a aresta de menor peso na subárvore enraizada em~$\node$.


Para manter as arestas reservas em ordem crescente de peso, basta armazená-las em uma fila, pois no Algoritmo de pré-processamento elas já são inseridas em ordem crescente.
Como no algoritmo de pré-processamento as arestas são inseridas em ordem crescente, então as arestas reservas serão inseridas nas filas de $R_i$ em ordem crescente.


Não é necessário esforço adicional para manter a invariante~\ref{invar:MSF:mono}, ao realizar a buscar em cada nível em ordem crescente de peso, as arestas de menor peso serão testadas primeiro.
Dessa forma, caso não sejam substitutas, serão rebaixadas antes das arestas reservas de maior peso.

Note que as invariantes~\ref{invar:MSF} e~\ref{invar:MSF:mono} podem ser quebradas se permitimos a inserção de novas arestas.

\begin{algorithm}
\caption{\dymGraphReplaceMSF($G$, $u$, $v$, $niv$)}
\label{Algo:dymGraphReplaceMSF}
\begin{algorithmic}[1]
\For {$i$ $\gets$ $niv$ até $\lceil \lg n \rceil$}\label{Algo:dymGraphReplace:linha:primeira}
\State $T_v$ $\gets$  \treapGetRoot($F_i[v,v]$)
\State $T_u$ $\gets$  \treapGetRoot($F_i[u,u]$)
\If {\treapGetSize($T_v$) < \treapGetSize($T_u$)}\Comment{Garantimos que $|T_v|\geqslant |T_u|$}
\State $u$ $\leftrightarrow$ $v$
\State $T_u \leftrightarrow T_v$
\EndIf
\For {$xy$ em $T_u$ com nível = $i$}\label{Algo:dymGraphReplace:linha:moveTu}\Comment{Move $T_u$ para o nível $i-1$}
\State \nivel$[x,y]$ $\gets$ $i-1$
\State \dymForestAddEdge($G$.$F_{i-1}$, $x$, $y$) 
\EndFor
\While {$xy$ $\gets$ \treapGetEdgeMinWeight($T_u$)}\Comment{Procura substituta para $uv$}
\If {$y \in T_v$}
\For {$j \gets i$ até $\lceil \lg n \rceil$}\label{Algo:dymGraphReplace:linha:inseresub}
\State \dymForestAddEdge($G$.$F_j$, $x$, $y$)
\EndFor
\State \Return
\Else
\State \nivel$[x,y]$ $\gets$ $i-1$
\State \graphAdd($G$.$R_{i-1}$, $x$, $y$)
\EndIf
\EndWhile
\EndFor\label{Algo:dymGraphReplace:linha:ultima}
\end{algorithmic}
\end{algorithm}


\begin{algorithm}
\caption{\treapGetEdgeMinWeight($\node$)}
\label{Algo:treapGetEdgeMinWeight}
\begin{algorithmic}[1]
\If {$\node$ $=$ \Nil{} ou $\node$.$minW$ $=0$}
\State \Return \Nil
\EndIf
\If {$\node.weight$ $=$ $\node$.$minW$}
	\State  \Return get($\node.nte$)\Comment{Retorna o primeiro nó da fila $nte$}
	\State [...] \Comment{Atualiza campo $minW$ até a raiz}
\EndIf

\If {$\node.esq.minW$ $\leq$ $\node.dir$.$minW$}
\State  \Return \treapGetEdgeMinWeight($\node.esq$)
\EndIf
\State \Return \treapGetEdgeMinWeight($\node.dir$)

\end{algorithmic}
\end{algorithm}

%%!TeX root=../tese.tex
%("dica" para o editor de texto: este arquivo é parte de um documento maior)
% para saber mais: https://tex.stackexchange.com/q/78101

\chapter{Código-fonte e pseudocódigo}
\label{ap:pseudocode}

Com a \textit{package} \textsf{listings}, programas podem ser inseridos
diretamente no arquivo, como feito no caso do Programa~\ref{prog:java},
ou importados de um arquivo externo com o comando
\textsf{\textbackslash{}lstinputlisting}, como no caso
do Programa~\ref{prog:mdcinput}.

% O exemplo foi copiado da documentação de algorithmicx
\begin{program}
  \lstinputlisting[
    language=pseudocode,
    style=pseudocode,
    style=wider,
    functions={},
    specialidentifiers={},
  ]
  {conteudo/euclid.psc}

  \caption{Máximo divisor comum (arquivo importado).\label{prog:mdcinput}}
\end{program}

Trechos de código curtos (menores que uma página) podem ou não ser
incluídos como \textit{floats}; trechos longos necessariamente incluem
quebras de página e, portanto, não podem ser \textit{floats}. Com
\textit{floats}, a legenda e as linhas separadoras são colocadas pelo
comando \textsf{\textbackslash{}begin\{program\}}; sem eles, utilize o
ambiente \textsf{programruledcaption} (atenção para a colocação do
comando \textsf{\textbackslash{}label\{\}}, dentro da legenda), como
no Programa~\ref{prog:mdc}\footnote{\textsf{listings} oferece alguns
recursos próprios para a definição de \textit{floats} e legendas, mas
neste modelo não os utilizamos.}:

\begin{programruledcaption}{Máximo divisor comum (em português).\label{prog:mdc}}
  \begin{lstlisting}[
    language={[brazilian]pseudocode},
    style=pseudocode,
    style=wider,
    functions={},
    specialidentifiers={},
  ]
      funcao euclides(a, b) // O máximo divisor comum de \textbf{a} e \textbf{b}
          r := a $\bmod$ b
	  enquanto r != 0 // Atingimos a resposta se \textbf{r} é zero
              a := b
              b := r
              r := a $\bmod$ b
          fim
	  devolva b // O máximo divisor comum é \textbf{b}
      fim
  \end{lstlisting}
\end{programruledcaption}

Além do suporte às várias linguagens incluídas em \textsf{listings},
este modelo traz uma extensão para permitir o uso de pseudocódigo,
útil para a descrição de algoritmos em alto nível. Ela oferece
diversos recursos:

\begin{itemize}

    \item Comentários seguem o padrão de C++ (\lstinline{//} e
          \lstinline{/* ... */}), mas o delimitador é impresso
          como ``$\triangleright$''.

    \item ``:='', ``<>'', ``<='', ``>='' e ``!='' são substituídos
          pelo símbolo matemático adequado.

    \item É possível acrescentar palavras-chave além de ``if'', ``and''
          etc. com a opção ``\textsf{morekeywords=\{pchave1,\linebreak[0]{}pchave2\}}''
          (para um trecho de código específico) ou com o comando
          \textsf{\textbackslash{}lstset\{morekeywords=\linebreak[0]{}\{pchave1,pchave2\}\}}
          (como comando de configuração geral).

    \item É possível usar pequenos trechos de código, como nomes de variáveis,
          dentro de um parágrafo normal com \textsf{\textbackslash{}lstinline\{blah\}}.

    \item ``\$\dots\$'' ativa o modo matemático em qualquer lugar.

    \item Outros comandos \LaTeX{} funcionam apenas em comentários; fora, a
          linguagem simula alguns pré-definidos (\textsf{\textbackslash{}textit\{\}},
          \textsf{\textbackslash{}texttt\{\}} etc.).

    \item O comando \textsf{\textbackslash{}label} também funciona em
          comentários; a referência correspondente (\textsf{\textbackslash{}ref})
          indica o número da linha de código. Se quiser usá-lo numa linha sem
          comentários, use \lstinline{///}~\textsf{\textbackslash{}label\{blah\}};
          ``\lstinline{///}'' funciona como \lstinline{//}, permitindo
          a inserção de comandos \LaTeX{}, mas não imprime o delimitador
          (\ensuremath{\triangleright}).

    \item Para suspender a formatação automática, use \textsf{\textbackslash{}noparse\{blah\}}.

    \item Para forçar a formatação de um texto como função, identificador,
          palavra-chave ou comentário, use \textsf{\textbackslash{}func\{blah\}},
          \textsf{\textbackslash{}id\{blah\}}, \textsf{\textbackslash{}kw\{blah\}} ou
          \textsf{\textbackslash{}comment\{blah\}}.

    \item Palavras-chave dentro de comentários não são formatadas
          automaticamente; se necessário, use \textsf{\textbackslash{}func\{\}},
          \textsf{\textbackslash{}id\{\}} etc. ou comandos \LaTeX{} padrão.

    \item As palavras ``Program'', ``Procedure'' e ``Function'' têm formatação
          especial e fazem a palavra seguinte ser formatada como função.
          Funções em outros lugares \emph{não} são detectadas automaticamente;
          use \textsf{\textbackslash{}func\{\}}, a opção ``\textsf{functions=\{func1,func2\}}''
          ou o comando ``\textsf{\textbackslash{}lstset\{functions=\{func1,func2\}\}}''
          para que elas sejam detectadas.

    \item Além de funções, palavras-chave, strings, comentários e
          identificadores, há ``\textsf{specialidentifiers}''. Você pode
          usá-los com \textsf{\textbackslash{}specialid\{blah\}}, com a opção
          ``\textsf{specialidentifiers=\{id1,id2\}}'' ou com o comando
          ``\textsf{\textbackslash{}lstset\{specialidentifiers=\{id1,id2\}\}}''.

\end{itemize}



%\par

%%%% Anexos %%%%

%\makeatletter
%\if@openright\cleardoublepage\else\clearpage\fi
%\makeatother

%\pagestyle{appendix} % repete o anterior, caso você não use apêndices

%\annex

% \addappheadtotoc acrescenta a palavra "Anexo" ao sumário; se
% só há anexos, sem apêndices, provavelmente não é necessário.
%\addappheadtotoc

%%!TeX root=../tese.tex
%("dica" para o editor de texto: este arquivo é parte de um documento maior)
% para saber mais: https://tex.stackexchange.com/q/78101

\chapter[Perguntas Frequentes sobre o Modelo]{Perguntas Frequentes sobre o Modelo\footnote{Esta
seção não é de fato um anexo, mas sim um apêndice; ela foi definida desta
forma apenas para servir como exemplo de anexo.}}

\begin{itemize}

\item \textbf{Não consigo decorar tantos comandos!}\\
Use a colinha que é distribuída juntamente com este modelo (\url{gitlab.com/ccsl-usp/modelo-latex/raw/master/pre-compilados/colinha.pdf?inline=false}).

\item \textbf{Por que tantos arquivos?}\\
O preâmbulo \LaTeX{} deste modelo é muito longo; as partes que normalmente não precisam ser modificadas foram colocadas no diretório \texttt{extras}, juntamente com alguns arquivos acessórios. Já os arquivos de conteúdo (capítulos, anexos etc.) foram divididos de maneira que seja fácil para você atualizar o modelo (copiando os novos arquivos ou com um sistema de controle de versões) sem que alterações no conteúdo de exemplo (este texto que você está lendo) causem conflitos com o seu próprio texto.\looseness=-1

\item \textbf{As figuras e tabelas são colocadas em lugares ruins.}\\
Veja a discussão a respeito na Seção~\ref{sec:limitations}.

\item \textbf{Estou tendo problemas com caracteres acentuados.}\\
Versões modernas de \LaTeX{} usam UTF-8, mas arquivos antigos podem usar outras codificações (como ISO-8859-1, também conhecido como latin1 ou Windows-1252). Nesses casos, use \textsf{\textbackslash{}usepackage[latin1]\{inputenc\}} no preâmbulo do documento. Você também pode representar os caracteres acentuados usando comandos \LaTeX{}: \textsf{\textbackslash\textquotesingle{}a} para á, \textsf{\textbackslash{}c\{c\}} para cedilha etc., independentemente da codificação usada no texto\footnote{Você pode consultar os comandos desse tipo mais comuns em \url{en.wikibooks.org/wiki/LaTeX/Special_Characters}. Observe que a dica sobre o pingo do i \emph{não} é mais válida atualmente; basta usar \textsf{\textbackslash\textquotesingle{}i}.}.

\item \textbf{Existe algo específico para citações de páginas web?}\\
Biblatex define o tipo ``online'', que deve ser usado para materiais com título, autor etc., como uma postagem ou comentário em um blog, um gráfico ou mesmo uma mensagem de email para uma lista de discussão. Bibtex\index{bibtex}, por padrão, não tem um tipo específico para isso; com ele, normalmente usa-se o campo ``howpublished'' para especificar que se trata de um recurso \textit{online}. Se o que você está citando não é algo determinado com título, autor etc. mas sim um sítio (como uma empresa ou um produto), pode ser mais adequado colocar a referência apenas como nota de rodapé e não na lista de referências; nesses casos, algumas pessoas acrescentam uma segunda lista de referências especificamente para recursos \textit{online} (biblatex\index{biblatex} permite criar múltiplas bibliografias). Já artigos disponíveis \textit{online} mas que fazem parte de uma publicação de formato tradicional (mesmo que apenas \textit{online}), como os anais de um congresso, devem ser citados por seu tipo verdadeiro e apenas incluir o campo ``url'' (não é nem necessário usar o comando \textsf{\textbackslash{}url\{\}}), aceito por todos os tipos de documento do bibtex/biblatex.

\item \textbf{Aparece uma folha em branco entre os capítulos.}\\
Essa característica foi colocada propositalmente, dado que todo capítulo deve (ou deveria) começar em uma página de numeração ímpar (lado direito do documento). Se quiser mudar esse comportamento, acrescente ``openany'' como opção da classe, i.e., \textsf{\textbackslash{}documentclass[openany,\dots]\{book\}}.

\item \textbf{É possível resumir o nome das seções/capítulos que aparece no topo das páginas e no sumário?}\\
Sim, usando a sintaxe \textsf{\textbackslash{}section[mini-titulo]\{titulo enorme\}}. Isso é especialmente útil nas legendas (\textit{captions}\index{Legendas}) das figuras e tabelas, que muitas vezes são demasiadamente longas para a lista de figuras/tabelas.

\item \textbf{Existe algum programa para gerenciar referências em formato bibtex?}\\
Sim, há vários. Uma opção bem comum é o JabRef; outra é usar Zotero\index{Zotero} ou Mendeley\index{Mendeley} e exportar os dados deles no formato .bib.

\item \textbf{Posso usar pacotes \LaTeX{} adicionais aos sugeridos?}\\
Com certeza! Você pode modificar os arquivos o quanto desejar, o modelo serve só como uma ajuda inicial para o seu trabalho.

\item \textbf{Como faço para usar o Makefile (comando make) no Windows?}\\
Lembre-se que a ferramenta recomendada para compilação do documento é o \textsf{latexmk}, então você não precisa do \textsf{make}. Mas, se quiser usá-lo, você pode instalar o MSYS2 (\url{www.msys2.org}) ou o Windows Subsystem for Linux (procure as versões de Linux disponíveis na Microsoft Store). Se você pretende usar algum dos editores sugeridos, é possível deixar a compilação a cargo deles, também dispensando o \textsf{make}.\looseness=-1

\item \textbf{Como eu faço para...}\\
Leia os comentários dos arquivos ``tese.tex'' e outros que compõem este modelo, além do tutorial (Capítulo \ref{chap:tutorial}) e dos exemplos do Capítulo \ref{chap:exemplos}; é provável que haja uma dica neles ou, pelo menos, a indicação da \textit{package} relacionada ao que você precisa.

\end{itemize}

%\par


%%%%%%%%%%%%%%% SEÇÕES FINAIS (BIBLIOGRAFIA E ÍNDICE REMISSIVO) %%%%%%%%%%%%%%%%

% O comando backmatter desabilita a numeração de capítulos.
\backmatter

\pagestyle{backmatter}

% Espaço adicional no sumário antes das referências / índice remissivo
\addtocontents{toc}{\vspace{2\baselineskip plus .5\baselineskip minus .5\baselineskip}}

% A bibliografia é obrigatória
%\bibliographystyle{plain}
%\bibliography{bibliografia}
\printbibliography[
  title=\refname\label{bibliografia}, % "Referências", recomendado pela ABNT
  %title=\bibname\label{bibliografia}, % "Bibliografia"
  heading=bibintoc, % Inclui a bibliografia no sumário
]

\printindex % imprime o índice remissivo no documento (opcional)

\end{document}
